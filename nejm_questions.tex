\documentclass[11pt]{article}
\usepackage[utf8]{inputenc}
\usepackage[T1]{fontenc}
\usepackage{lmodern}
\usepackage{textgreek}
\usepackage{fullpage}
\usepackage{graphicx}
\usepackage{xcolor}
\usepackage{fancyvrb}

\title{NEJM Medical Challenge Questions}
\author{}
\date{}

\begin{document}
\maketitle
\tableofcontents
\newpage

\section*{Question 1 (ID: 20250724)}
\textbf{Date: }July 24,2025
\vspace{6pt}

A 50-year-old woman presented to the emergency department with a 10-day history of a throbbing headache that had suddenly started after she had felt a “pop” in her back. The headache worsened with upright positioning and improved with lying flat. Physical examination, including fundoscopy, was normal. Magnetic resonance imaging of the brain showed diffuse pachymeningeal enhancement (arrow; upper left image, axial view, T1 sequence with contrast). It also showed protrusion of the cerebellar tonsils (arrow) into the foramen magnum, flattening of the pons with effacement of the prepontine cistern (arrowhead), and enlargement of the pituitary gland owing to increased blood flow (asterisk; right image, T1 sequence without contrast). Which of the following is the most likely diagnosis?
\vspace{12pt}

\textbf{Options:}
\begin{enumerate}
\item[A.] Chiari-Type 1 Malformation
\item[B.] Leptomeningeal carcinomatosis
\item[C.] Neurosarcoidosis
\item[D.] Spontaneous Intracranial Hypotension
\item[E.] Subdural hygroma
\end{enumerate}

\textbf{Image:}
\begin{center}
\includegraphics[width=0.95\textwidth,height=0.50\textheight,width=0.90\textwidth,keepaspectratio]{images/nejm_20250724.jpg}
\end{center}
\vspace{12pt}
\newpage

\section*{Question 2 (ID: 20250731)}
\textbf{Date: }July 31,2025
\vspace{6pt}

A 56-year-old woman with rheumatoid arthritis presented to the dermatology clinic with a 2-month history of an extremely itchy rash. She had been treated with tofacitinib and methotrexate for the previous 4 years. On physical examination, large patches of erythematous skin with overlying scales, crusting, and deep fissures were seen on the chest, upper arms, neck, abdomen, buttocks, and thighs. Similar findings were noted on the hands, with sparing of the nails. Which of the following is the most likely diagnosis?
\vspace{12pt}

\textbf{Options:}
\begin{enumerate}
\item[A.] Atopic Dermatitis
\item[B.] Crusted Scabies
\item[C.] Cutaneous T-cell lymphoma
\item[D.] Erythroderma
\item[E.] Psoriasis
\end{enumerate}

\textbf{Image:}
\begin{center}
\includegraphics[width=0.95\textwidth,height=0.50\textheight,width=0.90\textwidth,keepaspectratio]{images/nejm_20250731.jpg}
\end{center}
\vspace{12pt}
\newpage

\section*{Question 3 (ID: 20250807)}
\textbf{Date: }August 07,2025
\vspace{6pt}

A previously healthy 47-year-old woman presented to the neurology clinic with a 1-month history of worsening headaches associated with blurry vision, galactorrhea, and irregular menstrual cycles. Physical examination was notable for reduced visual acuity of 20/50 in both eyes, with normal visual fields and extraocular movements. Magnetic resonance imaging (MRI) of the head showed a mass protruding from the sella turcica into the suprasellar cistern. Which of the following is the most likely diagnosis?
\vspace{12pt}

\textbf{Options:}
\begin{enumerate}
\item[A.] Craniopharyngioma
\item[B.] Meningioma
\item[C.] Pituitary macroadenoma
\item[D.] Pituitary metastasis
\item[E.] Optic chiasmic astrocytoma
\end{enumerate}

\textbf{Image:}
\begin{center}
\includegraphics[width=0.74\textwidth,height=0.50\textheight,width=0.90\textwidth,keepaspectratio]{images/nejm_20250807.jpg}
\end{center}
\vspace{12pt}
\newpage

\section*{Question 4 (ID: 20250814)}
\textbf{Date: }August 14,2025
\vspace{6pt}

A 79-year-old man with heart failure with reduced ejection fraction and atrial fibrillation presented to the emergency department with a 2-week history of shortness of breath. On physical examination, there was an irregular heart rhythm, crackles in both lungs, and pitting edema in both legs up to the mid-thigh. The jugular venous pressure was estimated to be 17 cm of water (normal value, < 4 cm of water). When firm, sustained pressure was applied to the center of the abdomen, there was a sustained increase in the jugular venous pressure of more than 3 cm of water for more than 10 seconds while pressure was applied. What is the mechanism of this physical exam finding?
\vspace{12pt}

\textbf{Options:}
\begin{enumerate}
\item[A.] Decreased compliance of the pericardium
\item[B.] Impaired filling or contraction of the right ventricle
\item[C.] Left-to-right shunting
\item[D.] Severe tricuspid regurgitation
\item[E.] Ventricular interdependence
\end{enumerate}

\textbf{Image:}
\begin{center}
\includegraphics[width=0.95\textwidth,height=0.50\textheight,width=0.90\textwidth,keepaspectratio]{images/nejm_20250814.jpg}
\end{center}
\vspace{12pt}
\newpage

\section*{Question 5 (ID: 20250821)}
\textbf{Date: }August 21,2025
\vspace{6pt}

A 54-year-old woman with infiltrating ductal carcinoma of the right breast that had been treated with lumpectomy, locoregional radiation, and chemotherapy presented to the emergency department with a 4-week history of cough, fever, and dyspnea. On physical examination, an occasional cough was noted. Lung auscultation was normal. Computerized tomography (CT) of the chest is shown. What is the most likely diagnosis?
\vspace{12pt}

\textbf{Options:}
\begin{enumerate}
\item[A.] Drug-induced pneumonitis
\item[B.] Idiopathic pulmonary fibrosis
\item[C.] Lymphangitis carcinomatosis
\item[D.] Radiation pneumonitis
\item[E.] Viral pneumonia
\end{enumerate}

\textbf{Image:}
\begin{center}
\includegraphics[width=0.95\textwidth,height=0.50\textheight,width=0.90\textwidth,keepaspectratio]{images/nejm_20250821.jpg}
\end{center}
\vspace{12pt}
\newpage

\section*{Question 6 (ID: 20250828)}
\textbf{Date: }August 28,2025
\vspace{6pt}

A 71-year-old man presented with a 2-year history of a gradually darkening line on his left thumbnail. Which of the following is the most likely diagnosis?
\vspace{12pt}

\textbf{Options:}
\begin{enumerate}
\item[A.] Fungal melanonychia
\item[B.] Nail lentigo
\item[C.] Nail melanocytic nevus
\item[D.] Subungual hematoma
\item[E.] Subungual melanoma in situ
\end{enumerate}

\textbf{Image:}
\begin{center}
\includegraphics[width=0.95\textwidth,height=0.50\textheight,width=0.90\textwidth,keepaspectratio]{images/nejm_20250828.jpg}
\end{center}
\vspace{12pt}
\newpage

\section*{Question 7 (ID: 20250904)}
\textbf{Date: }September 04,2025
\vspace{6pt}

A previously healthy 42-year-old man presented with a 20-day history of an expanding, asymptomatic rash on his trunk. The initial lesion had been a red spot on the patient’s left side. Ten days after that spot had first appeared, smaller lesions had developed elsewhere. The patient reported no viral prodrome. The skin examination - including the initial lesion that had appeared - is shown. What is the diagnosis?
\vspace{12pt}

\textbf{Options:}
\begin{enumerate}
\item[A.] Nummular eczema
\item[B.] Pityriasis rosea
\item[C.] Secondary syphilis
\item[D.] Tinea corporis
\item[E.] Tinea versicolor
\end{enumerate}

\textbf{Image:}
\begin{center}
\includegraphics[width=0.95\textwidth,height=0.50\textheight,width=0.90\textwidth,keepaspectratio]{images/nejm_20250904.jpg}
\end{center}
\vspace{12pt}
\newpage

\section*{Question 8 (ID: 20250911)}
\textbf{Date: }September 11,2025
\vspace{6pt}

A 44-year-old man presented with a 4-day history of an itchy rash and a 2-day history of fever and malaise. The rash had first appeared on the scalp and then spread across the body within 24 hours. What is the most likely diagnosis?
\vspace{12pt}

\textbf{Options:}
\begin{enumerate}
\item[A.] Cutaneous t-cell lymphoma
\item[B.] Dengue fever
\item[C.] Disseminated gonococcal infection
\item[D.] Primary varicella infection
\item[E.] Small pox
\end{enumerate}

\textbf{Image:}
\begin{center}
\includegraphics[width=0.95\textwidth,height=0.50\textheight,width=0.90\textwidth,keepaspectratio]{images/nejm_20250911.jpg}
\end{center}
\vspace{12pt}
\newpage

\section*{Question 9 (ID: 20250918)}
\textbf{Date: }September 18,2025
\vspace{6pt}

A 15-year-old girl with acne vulgaris presented to the dermatology clinic with a 2-week history of a progressively worsening painful, blistering rash on both hands. One week before the onset of the rash, she had started treatment for her acne. What is the most likely diagnosis?
\vspace{12pt}

\textbf{Options:}
\begin{enumerate}
\item[A.] Actinic prurigo
\item[B.] Contact dermatitis
\item[C.] Drug-induced photosensitivity
\item[D.] Phytophotodermatitis
\item[E.] Solar urticaria
\end{enumerate}

\textbf{Image:}
\begin{center}
\includegraphics[width=0.95\textwidth,height=0.50\textheight,width=0.90\textwidth,keepaspectratio]{images/nejm_20250918.jpg}
\end{center}
\vspace{12pt}
\newpage

\section*{Question 10 (ID: 20250925)}
\textbf{Date: }September 25,2025
\vspace{6pt}

A 52-year-old woman with hypertension presented to the hospital with a 1-month history of fatigue, as well as 6 days of vomiting and 1 day of confusion. Over the past 20 years, she had not received regular treatment for hypertension. Physical examination was notable for pallor of the conjunctiva and oral mucosa, as well as for powdery, crystalline deposits on the arms, legs, trunk, and scalp. Which of the following is the most likely diagnosis?
\vspace{12pt}

\textbf{Options:}
\begin{enumerate}
\item[A.] Contact dermatitis
\item[B.] Pityriasis versicolor
\item[C.] Psoriasis
\item[D.] Uremic frost
\item[E.] Xerosis cutis
\end{enumerate}

\textbf{Image:}
\begin{center}
\includegraphics[width=0.89\textwidth,height=0.50\textheight,width=0.90\textwidth,keepaspectratio]{images/nejm_20250925.jpg}
\end{center}
\vspace{12pt}
\newpage

\section*{Question 11 (ID: 20251002)}
\textbf{Date: }October 02,2025
\vspace{6pt}

A 21-year-old woman who had presented to the pulmonary clinic with a 7-day history of sore throat and cough was noted to have a vertebral abnormality on chest radiograph. She reported no history of back pain. Computed tomography of the chest is shown. What is the most likely underlying etiology?
\vspace{12pt}

\textbf{Options:}
\begin{enumerate}
\item[A.] Congenital abnormality
\item[B.] Decreased bone mineralization
\item[C.] Increased bone turnover
\item[D.] Malignancy
\item[E.] Prior trauma
\end{enumerate}

\textbf{Image:}
\begin{center}
\includegraphics[width=0.95\textwidth,height=0.50\textheight,width=0.90\textwidth,keepaspectratio]{images/nejm_20251002.jpg}
\end{center}
\vspace{12pt}
\newpage

\section*{Question 12 (ID: 20251009)}
\textbf{Date: }October 09,2025
\vspace{6pt}

A 23-year-old man with autism spectrum disorder and chronic constipation presented to the emergency department with a 1-week history of abdominal pain on his left side, nausea, and vomiting. Physical examination was notable for abdominal distention and mild tenderness to palpation on the left side of the abdomen. Computed tomography of the abdomen and pelvis is shown. What is the most likely diagnosis?
\vspace{12pt}

\textbf{Options:}
\begin{enumerate}
\item[A.] Hirschsprung’s disease
\item[B.] Simple fecal impaction
\item[C.] Stercoral colitis
\item[D.] Toxic megacolon
\item[E.] Ulcerative colitis
\end{enumerate}

\textbf{Image:}
\begin{center}
\includegraphics[width=0.95\textwidth,height=0.50\textheight,width=0.90\textwidth,keepaspectratio]{images/nejm_20251009.jpg}
\end{center}
\vspace{12pt}
\newpage

\section*{Question 13 (ID: 20251016)}
\textbf{Date: }October 16,2025
\vspace{6pt}

A 34-year-old man presented to the emergency department a week after returning from a safari in Zimbabwe with a 4-day history of fever and generalized weakness. On physical examination, a crusted, tender lesion on the crown of the head was noted. The parietal scalp was shaved with the patient’s permission, revealing a 4 cm by 4 cm ulceration. A peripheral-blood smear is also shown. What is the most likely diagnosis?
\vspace{12pt}

\textbf{Options:}
\begin{enumerate}
\item[A.] Cutaneous leishmaniasis
\item[B.] Loaiasis
\item[C.] Mansonellosis
\item[D.] Onchocerciasis
\item[E.] Trypanosomiasis
\end{enumerate}

\textbf{Image:}
\begin{center}
\includegraphics[width=0.95\textwidth,height=0.50\textheight,width=0.90\textwidth,keepaspectratio]{images/nejm_20251016.jpg}
\end{center}
\vspace{12pt}
\newpage

\section*{Question 14 (ID: 20251023)}
\textbf{Date: }October 23,2025
\vspace{6pt}

A 30-year-old man presented to the ophthalmology clinic with a 3-day history of recurrent, painful bumps on his left lower eyelid. He reported annual recurrences of similar lesions at the same site at the same site since adolescence, often triggered by psychological stress or sun exposure and with spontaneous resolution. Visual acuity and findings from a slit-lamp examination were normal. Which of the following is the most likely diagnosis?
\vspace{12pt}

\textbf{Options:}
\begin{enumerate}
\item[A.] Atopic blepharitis
\item[B.] Chalazion
\item[C.] Dacryoadenitis
\item[D.] Herpes simplex blepharitis
\item[E.] Hordeolum
\end{enumerate}

\textbf{Image:}
\begin{center}
\includegraphics[width=0.95\textwidth,height=0.50\textheight,width=0.90\textwidth,keepaspectratio]{images/nejm_20251023.jpg}
\end{center}
\vspace{12pt}
\newpage

\section*{Question 15 (ID: 20251030)}
\textbf{Date: }October 30,2025
\vspace{6pt}

A 1-week-old full-term baby boy was brought back to the hospital with discharge from his navel. Physical examination showed pink tissue at the umbilicus that seeped yellow liquid. Abdominal ultrasonography was performed. What is the underlying pathophysiology?
\vspace{12pt}

\textbf{Options:}
\begin{enumerate}
\item[A.] Failure of the allantois to regress
\item[B.] Failure of the cloacal membrane to develop properly
\item[C.] Failure of the regression between the yolk sac and midgut
\item[D.] Incomplete closure of the abdominal wall and subsequent tissue protrusion
\item[E.] Persistence of the vitelline duct
\end{enumerate}

\textbf{Image:}
\begin{center}
\includegraphics[width=0.5\textwidth,height=0.50\textheight,width=0.90\textwidth,keepaspectratio]{images/nejm_20251030.jpg}
\end{center}
\vspace{12pt}
\newpage

\section*{Question 16 (ID: 20251106)}
\textbf{Date: }November 06,2025
\vspace{6pt}

A previously healthy 8-year-old boy was brought to clinic with a 7-day history of an itchy rash on his cheeks and a lacy rash on his chest and arms. He reported no recent fever, runny nose, or malaise. What is the most likely etiology of the rash?
\vspace{12pt}

\textbf{Options:}
\begin{enumerate}
\item[A.] Group A streptococcus
\item[B.] Human herpesvirus 6
\item[C.] Measles
\item[D.] Parvovirus B19
\item[E.] Rubella
\end{enumerate}

\textbf{Image:}
\begin{center}
\includegraphics[width=0.95\textwidth,height=0.50\textheight,width=0.90\textwidth,keepaspectratio]{images/nejm_20251106.jpg}
\end{center}
\vspace{12pt}
\newpage

\section*{Question 17 (ID: 20251113)}
\textbf{Date: }November 13,2025
\vspace{6pt}

A 75-year-old woman with a history of infiltrating ductal carcinoma of the breast (previously treated with bilateral radical mastectomy and ongoing adjuvant hormonal therapy) presented to the dermatology clinic with a 1-year history of rash on the chest. Physical examination and histopathology of a skin-biopsy sample from the left chest-wall scar are shown. What is the most likely diagnosis?
\vspace{12pt}

\textbf{Options:}
\begin{enumerate}
\item[A.] Cellulitis
\item[B.] Cutaneous metastases
\item[C.] Eosinophilic fasciitis
\item[D.] Erysipeleas
\item[E.] Post-mastectomy scar granuloma
\end{enumerate}

\textbf{Image:}
\begin{center}
\includegraphics[width=0.95\textwidth,height=0.50\textheight,width=0.90\textwidth,keepaspectratio]{images/nejm_20251113.jpg}
\end{center}
\vspace{12pt}
\newpage

\section*{Question 18 (ID: 20251120)}
\textbf{Date: }November 20,2025
\vspace{6pt}

A 76-year-old man with coronary artery disease and heart failure with reduced ejection fraction presented to clinic with an 8-month history of progressive breast enlargement and tenderness. Which of the following is most likely to identify the underlying etiology of this finding?
\vspace{12pt}

\textbf{Options:}
\begin{enumerate}
\item[A.] Computed tomography of the chest, abdomen, and pelvis
\item[B.] Measurement of serum testosterone level
\item[C.] Medication review
\item[D.] Screening for substance use
\item[E.] Testicular examination
\end{enumerate}

\textbf{Image:}
\begin{center}
\includegraphics[width=0.95\textwidth,height=0.50\textheight,width=0.90\textwidth,keepaspectratio]{images/nejm_20251120.jpg}
\end{center}
\vspace{12pt}
\newpage

\section*{Question 19 (ID: 20251127)}
\textbf{Date: }November 27,2025
\vspace{6pt}

A 66-year-old man with chronic obstructive pulmonary disease presented to the emergency department with a 2-week history of shortness of breath and cough and 5 days of left flank pain. Two days before presentation, he had noted the appearance and rapid expansion of a mass on his left side. Computed tomography of the chest is shown. What is the underlying etiology?
\vspace{12pt}

\textbf{Options:}
\begin{enumerate}
\item[A.] Autoimmune pleuritis
\item[B.] Empyema necessitans
\item[C.] Hematoma
\item[D.] Mesothelioma
\item[E.] Soft tissue sarcoma of the chest wall
\end{enumerate}

\textbf{Image:}
\begin{center}
\includegraphics[width=0.95\textwidth,height=0.50\textheight,width=0.90\textwidth,keepaspectratio]{images/nejm_20251127.jpg}
\end{center}
\vspace{12pt}
\newpage

\section*{Question 20 (ID: 20251204)}
\textbf{Date: }December 04,2025
\vspace{6pt}

A 26-year-old woman presented with pain in her left elbow after falling on an outstretched hand while ice-skating. Physical examination revealed left lateral elbow swelling and tenderness to palpation, a valgus deformity, and limited range of motion. Which of the following findings are present on the imaging?
\vspace{12pt}

\textbf{Options:}
\begin{enumerate}
\item[A.] Fracture of the shaft of the ulna
\item[B.] Fracture of the shaft of the ulna and dislocation of the radial head
\item[C.] Fracture of the shaft of the ulna, fracture of the olecranon, and dislocation of the radial head
\item[D.] Fracture of the shaft and coronoid process of the ulna and dislocation of the radial head
\item[E.] Fracture of the shaft of the radius
\end{enumerate}

\textbf{Image:}
\begin{center}
\includegraphics[width=0.95\textwidth,height=0.50\textheight,width=0.90\textwidth,keepaspectratio]{images/nejm_20251204.jpg}
\end{center}
\vspace{12pt}
\newpage

\section*{Question 21 (ID: 20220210)}
\textbf{Date: }February 10,2022
\vspace{6pt}

A 21-year-old woman with a history of surgically corrected Hirschsprung’s disease presented with 6-week history of abdominal pain and constipation. Her blood pressure was 162/108 mm Hg. She had an elevated creatinine and CT showed large subcapsular urinomas and hydronephrosis caused by distal ureteral obstruction from a rectal mass. What is the likely mechanism of her hypertension?
\vspace{12pt}

\textbf{Options:}
\begin{enumerate}
\item[A.] Atherosclerotic renal artery stenosis
\item[B.] Fibromuscular dysplasia
\item[C.] Medullary sponge kidney
\item[D.] Page kidney
\item[E.] Polycystic kidney disease
\end{enumerate}

\textbf{Image:}
\begin{center}
\includegraphics[width=0.95\textwidth,height=0.50\textheight,width=0.90\textwidth,keepaspectratio]{images/nejm_20220210.jpg}
\end{center}
\vspace{12pt}
\newpage

\section*{Question 22 (ID: 20220217)}
\textbf{Date: }February 17,2022
\vspace{6pt}

A 50-year-old man with a history of several months of productive cough presented to the emergency department with stupor and hypercarbic respiratory failure. He was intubated. Chest radiograph several days later revealed complete collapse of the left lung. On bronchoscopy, several casts, as shown, were retrieved. Follow-up CT of the chest showed only bibasilar atelectasis. What is the etiology of his casts?
\vspace{12pt}

\textbf{Options:}
\begin{enumerate}
\item[A.] Asthma
\item[B.] Bronchiectasis
\item[C.] Cystic fibrosis
\item[D.] Plastic bronchitis
\item[E.] Tuberculosis
\end{enumerate}

\textbf{Image:}
\begin{center}
\includegraphics[width=0.95\textwidth,height=0.50\textheight,width=0.90\textwidth,keepaspectratio]{images/nejm_20220217.jpg}
\end{center}
\vspace{12pt}
\newpage

\section*{Question 23 (ID: 20220224)}
\textbf{Date: }February 24,2022
\vspace{6pt}

A 31-year-old man presented with 10 years of progressive fingertip and toe enlargement and intermittent aches of his distal forearms and lower legs. Physical exam was notable for digital clubbing, proximal nail fold hyperpigmentation, and forearm and lower leg tenderness of palpation. Plain radiographs of the ankles showed increased periosteal bone formation at the distal tibia and fibula. Serum laboratory tests, echocardiogram, and a whole-body PET-CT were normal. What is the most likely diagnosis?
\vspace{12pt}

\textbf{Options:}
\begin{enumerate}
\item[A.] Acromegaly
\item[B.] Osteopetrosis
\item[C.] Paget’s disease
\item[D.] Primary hypertrophic osteoarthropathy
\item[E.] Scleromyxedema
\end{enumerate}

\textbf{Image:}
\begin{center}
\includegraphics[width=0.95\textwidth,height=0.50\textheight,width=0.90\textwidth,keepaspectratio]{images/nejm_20220224.jpg}
\end{center}
\vspace{12pt}
\newpage

\section*{Question 24 (ID: 20220303)}
\textbf{Date: }March 03,2022
\vspace{6pt}

A 48-year-old woman who lived in rural India presented with diarrhea, weight loss, and a rash. Physical examination showed well-demarcated areas of hyperpigmentation on the sun-exposed areas of her neck, chest, and forearms. What is the most likely diagnosis?
\vspace{12pt}

\textbf{Options:}
\begin{enumerate}
\item[A.] Cobalamin (vitamin B12) deficiency
\item[B.] Pellagra
\item[C.] Phytophotodermatitis
\item[D.] Porphyria cutanea tarda
\item[E.] Zinc deficiency
\end{enumerate}

\textbf{Image:}
\begin{center}
\includegraphics[width=0.95\textwidth,height=0.50\textheight,width=0.90\textwidth,keepaspectratio]{images/nejm_20220303.jpg}
\end{center}
\vspace{12pt}
\newpage

\section*{Question 25 (ID: 20220310)}
\textbf{Date: }March 10,2022
\vspace{6pt}

A 42-year-old man with human immunodeficiency virus (HIV) infection presented with painful skin ulcerations and systemic symptoms. Physical exam showed large ulcers with overlying keratosis and crusting scattered across his scalp, face, perineum, and limbs. Laboratory studies showed a CD4 count of 399 cells/mm3. Skin biopsy showed superficial dermal infiltration of lymphocytes, histiocytes, and plasma cells in a lichenoid pattern with psoriasiform epidermal hyperplasia. What is the most likely diagnosis?
\vspace{12pt}

\textbf{Options:}
\begin{enumerate}
\item[A.] Bacillary angiomatosis
\item[B.] Disseminated histoplasmosis
\item[C.] Lupus vulgaris (tuberculosis luposa)
\item[D.] Malignant syphilis
\item[E.] Norwegian scabies
\end{enumerate}

\textbf{Image:}
\begin{center}
\includegraphics[width=0.71\textwidth,height=0.50\textheight,width=0.90\textwidth,keepaspectratio]{images/nejm_20220310.jpg}
\end{center}
\vspace{12pt}
\newpage

\section*{Question 26 (ID: 20220317)}
\textbf{Date: }March 17,2022
\vspace{6pt}

A 37-year-old man presented early in life with nephrolithiasis and pyelonephritis. Two weeks before this encounter, he was treated for a urinary tract infection. Microscopic examination of a spun urine specimen is shown, demonstrating two crystal types. What are these crystals, demonstrated in panels A and B, respectively?
\vspace{12pt}

\textbf{Options:}
\begin{enumerate}
\item[A.] Calcium oxalate and triple phosphate (struvite) crystals
\item[B.] Cholesterol and acyclovir crystals
\item[C.] Cystine and triple phosphate (struvite) crystals
\item[D.] Sulfonamide and cystine crystals
\item[E.] Uric acid and calcium oxalate crystals
\end{enumerate}

\textbf{Image:}
\begin{center}
\includegraphics[width=0.95\textwidth,height=0.50\textheight,width=0.90\textwidth,keepaspectratio]{images/nejm_20220317.jpg}
\end{center}
\vspace{12pt}
\newpage

\section*{Question 27 (ID: 20220324)}
\textbf{Date: }March 24,2022
\vspace{6pt}

A 63-year-old man developed purplish discoloration of his face after he underwent stenting and balloon dilation of the left common carotid artery. Cholesterol embolization syndrome was diagnosed. What is the name of this physical exam finding?
\vspace{12pt}

\textbf{Options:}
\begin{enumerate}
\item[A.] Cutis marmorata
\item[B.] Harlequin syndrome
\item[C.] Livedo racemosa
\item[D.] Livedo reticularis
\item[E.] Telangiectasia
\end{enumerate}

\textbf{Image:}
\begin{center}
\includegraphics[width=0.73\textwidth,height=0.50\textheight,width=0.90\textwidth,keepaspectratio]{images/nejm_20220324.jpg}
\end{center}
\vspace{12pt}
\newpage

\section*{Question 28 (ID: 20220331)}
\textbf{Date: }March 31,2022
\vspace{6pt}

A 53-year-old man presented with several hours of right eye itching after gardening near a horse and sheep farm. A physical exam was performed. What is the most likely diagnosis?
\vspace{12pt}

\textbf{Options:}
\begin{enumerate}
\item[A.] Allergic conjunctivitis
\item[B.] Anterior uveitis
\item[C.] Bacterial conjunctivitis
\item[D.] Ophthalmomyiasis
\item[E.] Superficial punctate keratitis
\end{enumerate}

\textbf{Image:}
\begin{center}
\includegraphics[width=0.83\textwidth,height=0.50\textheight,width=0.90\textwidth,keepaspectratio]{images/nejm_20220331.jpg}
\end{center}
\vspace{12pt}
\newpage

\section*{Question 29 (ID: 20220407)}
\textbf{Date: }April 07,2022
\vspace{6pt}

A 30-year-old man with Crohn’s disease presented to the dermatology clinic with multiple painful ulcers in his groin and perineal area, as well as an ongoing Crohn’s disease flare with fatigue, abdominal pain, and bloody diarrhea. Physical examination of the skin showed deep, linear erosions across the perineum and inguinal folds, on the scrotum and penile shaft, and in the intergluteal cleft. Which of the following is the most likely etiology underlying the genital ulcers?
\vspace{12pt}

\textbf{Options:}
\begin{enumerate}
\item[A.] Ecthyma gangrenosum
\item[B.] Knife-cut ulcers from Crohn’s disease
\item[C.] Lichen sclerosus
\item[D.] Linear IgA disease
\item[E.] Pemphigus foliaceus
\end{enumerate}

\textbf{Image:}
\begin{center}
\includegraphics[width=0.95\textwidth,height=0.50\textheight,width=0.90\textwidth,keepaspectratio]{images/nejm_20220407.jpg}
\end{center}
\vspace{12pt}
\newpage

\section*{Question 30 (ID: 20220414)}
\textbf{Date: }April 14,2022
\vspace{6pt}

A 64-year-old man with metastatic lung adenocarcinoma who had recently started high-dose steroids for malignant spinal cord compression developed an itchy rash and diarrhea. He had a 3-year history of intermittent peripheral eosinophilia of unknown cause. What is the most likely diagnosis?
\vspace{12pt}

\textbf{Options:}
\begin{enumerate}
\item[A.] Cutaneous larva migrans
\item[B.] Cutaneous schistosomiasis
\item[C.] Larva currens from Strongyloides
\item[D.] Lichen striatus
\item[E.] Scabies
\end{enumerate}

\textbf{Image:}
\begin{center}
\includegraphics[width=0.95\textwidth,height=0.50\textheight,width=0.90\textwidth,keepaspectratio]{images/nejm_20220414.jpg}
\end{center}
\vspace{12pt}
\newpage

\section*{Question 31 (ID: 20220421)}
\textbf{Date: }April 21,2022
\vspace{6pt}

A 56-year-old man with sickle hemoglobin C disease and proliferative sickle cell retinopathy presented for a routine eye examination. He was asymptomatic. What is seen on this examination of the anterior segment of the eye?
\vspace{12pt}

\textbf{Options:}
\begin{enumerate}
\item[A.] Bilateral acute depigmentation of the iris (BADI)
\item[B.] Cornea verticillata
\item[C.] Iris atrophy
\item[D.] Mutton-fat keratic precipitates
\item[E.] Sunflower cataract
\end{enumerate}

\textbf{Image:}
\begin{center}
\includegraphics[width=0.95\textwidth,height=0.50\textheight,width=0.90\textwidth,keepaspectratio]{images/nejm_20220421.jpg}
\end{center}
\vspace{12pt}
\newpage

\section*{Question 32 (ID: 20220428)}
\textbf{Date: }April 28,2022
\vspace{6pt}

A 53-year-old landscaper presented to the dermatology clinic with a 4-month history of red, raised, itchy skin lesions on his left lower back and buttock. Physical examination showed numerous verrucous nodules and plaques with overlying crusting and surrounding erythema on the left lower back and buttock. Grocott-Gomori methenamine silver staining showed broad-based budding organisms. Chest imaging showed no abnormalities. Which of the following is the best treatment?
\vspace{12pt}

\textbf{Options:}
\begin{enumerate}
\item[A.] Clarithromycin and amikacin
\item[B.] Itraconazole
\item[C.] Penicillin
\item[D.] Rifampicin, isoniazid, pyrazinamide, and ethambutol
\item[E.] Trimethoprim-sulfamethoxazole
\end{enumerate}

\textbf{Image:}
\begin{center}
\includegraphics[width=0.95\textwidth,height=0.50\textheight,width=0.90\textwidth,keepaspectratio]{images/nejm_20220428.jpg}
\end{center}
\vspace{12pt}
\newpage

\section*{Question 33 (ID: 20220505)}
\textbf{Date: }May 05,2022
\vspace{6pt}

A 74-year-old man with coronary artery disease presented to the emergency department with 3 days of anorexia and weakness. Computed tomography of the chest was performed, and blood cultures were drawn. What is the most likely diagnosis?
\vspace{12pt}

\textbf{Options:}
\begin{enumerate}
\item[A.] Emphysematous aortitis due to Clostridium septicum
\item[B.] Granulomatosis with polyangiitis
\item[C.] Pyogenic aortitis due to Salmonella enteritidis
\item[D.] Syphilitic aortitis due to Treponema pallidum
\item[E.] Tuberculous aortitis
\end{enumerate}

\textbf{Image:}
\begin{center}
\includegraphics[width=0.95\textwidth,height=0.50\textheight,width=0.90\textwidth,keepaspectratio]{images/nejm_20220505.jpg}
\end{center}
\vspace{12pt}
\newpage

\section*{Question 34 (ID: 20220512)}
\textbf{Date: }May 12,2022
\vspace{6pt}

A 73-year-old man presented with 1-month history of fatigue and diffuse bone pain. Laboratory studies showed pancytopenia and a whole-body MRI showed osteonecrosis of the right humerus and right femur. A bone marrow biopsy showed macrophages with a “wrinkled tissue paper” appearance in the cytoplasm. What is the diagnosis?
\vspace{12pt}

\textbf{Options:}
\begin{enumerate}
\item[A.] Chronic lymphocytic leukemia
\item[B.] Gaucher’s disease
\item[C.] Mucopolysaccharidosis type 1
\item[D.] Niemann-Pick disease
\item[E.] Non-Hodgkin’s lymphoma
\end{enumerate}

\textbf{Image:}
\begin{center}
\includegraphics[width=0.95\textwidth,height=0.50\textheight,width=0.90\textwidth,keepaspectratio]{images/nejm_20220512.jpg}
\end{center}
\vspace{12pt}
\newpage

\section*{Question 35 (ID: 20220519)}
\textbf{Date: }May 19,2022
\vspace{6pt}

A 40-year-old man presented with a 2-week history of scaly foot rash, joint swelling, penile rash, knee swelling, tongue changes, red eyes, and low back pain. He had had diarrhea and urethral discharge 7 days before the onset of symptoms. On examination, yellow pustules with hyperkeratosis were seen on his plantar feet. What is the name of this physical examination sign?
\vspace{12pt}

\textbf{Options:}
\begin{enumerate}
\item[A.] Keratoderma blenorrhagicum
\item[B.] Palmoplantar keratoderma
\item[C.] Pitted keratolysis
\item[D.] Porokeratosis
\item[E.] Pustular psoriasis
\end{enumerate}

\textbf{Image:}
\begin{center}
\includegraphics[width=0.75\textwidth,height=0.50\textheight,width=0.90\textwidth,keepaspectratio]{images/nejm_20220519.jpg}
\end{center}
\vspace{12pt}
\newpage

\section*{Question 36 (ID: 20220526)}
\textbf{Date: }May 26,2022
\vspace{6pt}

A 27-year-old man with a history of obesity presented to the dermatology clinic with an asymptomatic rash on his back, arms, and hands that had developed 1 week earlier. On physical examination, scattered pink-yellow papules were present on the upper back, extensor surfaces of the upper arms, and dorsa of the hands. A fasting blood sample was grossly lipemic. Which of the following is the most likely diagnosis?
\vspace{12pt}

\textbf{Options:}
\begin{enumerate}
\item[A.] Eruptive xanthomas
\item[B.] Generalized eruptive histiocytoma
\item[C.] Granuloma annulare
\item[D.] Molluscum contagiosum
\item[E.] Sebaceous hyperplasia
\end{enumerate}

\textbf{Image:}
\begin{center}
\includegraphics[width=0.95\textwidth,height=0.50\textheight,width=0.90\textwidth,keepaspectratio]{images/nejm_20220526.jpg}
\end{center}
\vspace{12pt}
\newpage

\section*{Question 37 (ID: 20220602)}
\textbf{Date: }June 02,2022
\vspace{6pt}

A 4-year-old boy was brought to the orthopedic clinic with a 2-day history of pain in the right hip and limping. There had been no preceding trauma or fever. Physical exam showed normal hip range of motion and an antalgic gait favoring the right side. A radiograph of the pelvis was performed. What is the most likely diagnosis?
\vspace{12pt}

\textbf{Options:}
\begin{enumerate}
\item[A.] Legg-Calvé-Perthes disease
\item[B.] Juvenile idiopathic arthritis
\item[C.] Osteoid osteoma
\item[D.] Septic arthritis
\item[E.] Slipped capital femoral epiphysis
\end{enumerate}

\textbf{Image:}
\begin{center}
\includegraphics[width=0.95\textwidth,height=0.50\textheight,width=0.90\textwidth,keepaspectratio]{images/nejm_20220602.jpg}
\end{center}
\vspace{12pt}
\newpage

\section*{Question 38 (ID: 20220609)}
\textbf{Date: }June 09,2022
\vspace{6pt}

A 67-year-old man presented with a 1-year history of bilateral eye swelling and right eye protrusion and a 2-week history of blurry vision. There was no swelling of the parotid or submandibular glands. MRI of the orbits showed bilateral lacrimal gland and lateral rectus muscle swelling as well as mass lesions surrounding the optic nerves. He was initiated on daily prednisone with improvement in his symptoms. What is the diagnosis?
\vspace{12pt}

\textbf{Options:}
\begin{enumerate}
\item[A.] IgG4-related ophthalmic disease
\item[B.] Non-Hodgkin's lymphoma
\item[C.] Sjogren’s syndrome
\item[D.] Tuberculosis-related dacryoadenitis
\item[E.] Thyroid eye disease
\end{enumerate}

\textbf{Image:}
\begin{center}
\includegraphics[width=0.95\textwidth,height=0.50\textheight,width=0.90\textwidth,keepaspectratio]{images/nejm_20220609.jpg}
\end{center}
\vspace{12pt}
\newpage

\section*{Question 39 (ID: 20220616)}
\textbf{Date: }June 16,2022
\vspace{6pt}

A 3-year-old girl was brought to the emergency department with a 2-month history of a white pupil and a 1-day history of redness and pain in the right eye. An eye examination showed leukocoria, as well as iris neovascularization and a white, nodular mass in the posterior chamber. The left eye was normal. B-scan ultrasonography showed calcification of the mass and vitreous seeding in the affected eye. Which of the following is the most likely diagnosis?
\vspace{12pt}

\textbf{Options:}
\begin{enumerate}
\item[A.] Cataract
\item[B.] Coats' disease
\item[C.] Ocular toxocariasis
\item[D.] Retinal detachment
\item[E.] Retinoblastoma
\end{enumerate}

\textbf{Image:}
\begin{center}
\includegraphics[width=0.95\textwidth,height=0.50\textheight,width=0.90\textwidth,keepaspectratio]{images/nejm_20220616.jpg}
\end{center}
\vspace{12pt}
\newpage

\section*{Question 40 (ID: 20220623)}
\textbf{Date: }June 23,2022
\vspace{6pt}

An 8-month-old girl presented with strabismus and developmental delay. On examination, she was noted to have macrocephaly, intermittent exotropia in the left eye, and hypotonia. Fundoscopic exam results are shown in the image. MRI of the brain revealed inadequate myelination diffusely and a thin corpus callosum. What is the diagnosis?
\vspace{12pt}

\textbf{Options:}
\begin{enumerate}
\item[A.] Myotonic dystrophy
\item[B.] Pompe disease
\item[C.] Prader-Willi syndrome
\item[D.] Spinal muscular atrophy
\item[E.] Tay-Sachs disease
\end{enumerate}

\textbf{Image:}
\begin{center}
\includegraphics[width=0.83\textwidth,height=0.50\textheight,width=0.90\textwidth,keepaspectratio]{images/nejm_20220623.jpg}
\end{center}
\vspace{12pt}
\newpage

\section*{Question 41 (ID: 20220630)}
\textbf{Date: }June 30,2022
\vspace{6pt}

A 28-year-old woman with a history of ulcerative colitis and pulmonary embolism presented with a 3-day history of dry cough and pleuritic chest pain. A chest radiograph was obtained. What is the name of this imaging finding?
\vspace{12pt}

\textbf{Options:}
\begin{enumerate}
\item[A.] Fleischner sign
\item[B.] Hampton’s hump
\item[C.] Palla sign
\item[D.] Spine sign
\item[E.] Westermark sign
\end{enumerate}

\textbf{Image:}
\begin{center}
\includegraphics[width=0.9\textwidth,height=0.50\textheight,width=0.90\textwidth,keepaspectratio]{images/nejm_20220630.jpg}
\end{center}
\vspace{12pt}
\newpage

\section*{Question 42 (ID: 20220707)}
\textbf{Date: }July 07,2022
\vspace{6pt}

A 30-year-old man was admitted to the hospital with 2 weeks of rash and fever that had started one month after taking a course of trimethoprim-sulfamethoxazole for the treatment of folliculitis. Physical examination was notable for fever, a diffuse morbilliform rash, submandibular lymphadenopathy, and facial erythema with periorbital sparing. Four days after admission, facial edema developed. Labs showed elevated absolute eosinophils and aminotransferase levels. What’s the most likely diagnosis?
\vspace{12pt}

\textbf{Options:}
\begin{enumerate}
\item[A.] Acute generalized exanthematous pustulosis
\item[B.] Drug reaction with eosinophilia and systemic symptoms
\item[C.] Erythroderma
\item[D.] Hypereosinophilic syndrome
\item[E.] Stevens-Johnson syndrome/toxic epidermal necrolysis
\end{enumerate}

\textbf{Image:}
\begin{center}
\includegraphics[width=0.95\textwidth,height=0.50\textheight,width=0.90\textwidth,keepaspectratio]{images/nejm_20220707.jpg}
\end{center}
\vspace{12pt}
\newpage

\section*{Question 43 (ID: 20220714)}
\textbf{Date: }July 14,2022
\vspace{6pt}

A 26-year-old woman with recurrent tonsillitis presents with progressive sore throat and pain with swallowing. She was febrile and had a white-cell count of 19,450 cells per cubic millimeter. CT scan of the neck is shown. What is the diagnosis?
\vspace{12pt}

\textbf{Options:}
\begin{enumerate}
\item[A.] Epiglottitis
\item[B.] Peritonsillar abscess
\item[C.] Retropharyngeal abscess
\item[D.] Septic thrombophlebitis of the internal jugular vein (Lemierre’s syndrome)
\item[E.] Submandibular space infection (Ludwig’s angina)
\end{enumerate}

\textbf{Image:}
\begin{center}
\includegraphics[width=0.95\textwidth,height=0.50\textheight,width=0.90\textwidth,keepaspectratio]{images/nejm_20220714.jpg}
\end{center}
\vspace{12pt}
\newpage

\section*{Question 44 (ID: 20220721)}
\textbf{Date: }July 21,2022
\vspace{6pt}

A 23-year-old man presented with a 1-month history of double vision and right eyelid drooping that worsened at the end of the day. On examination, manual raising of the ptotic right eyelid resulted in dropping of the left eyelid. What is the most likely diagnosis?
\vspace{12pt}

\textbf{Options:}
\begin{enumerate}
\item[A.] Botulism
\item[B.] Horner’s syndrome
\item[C.] Miller Fisher syndrome
\item[D.] Multiple sclerosis
\item[E.] Myasthenia gravis
\end{enumerate}

\textbf{Image:}
\begin{center}
\includegraphics[width=0.95\textwidth,height=0.50\textheight,width=0.90\textwidth,keepaspectratio]{images/nejm_20220721.jpg}
\end{center}
\vspace{12pt}
\newpage

\section*{Question 45 (ID: 20220728)}
\textbf{Date: }July 28,2022
\vspace{6pt}

A 68-year-old woman living in Japan presented to the clinic with a 5-day history of fevers, myalgias, and petechial rash spreading on her arms, trunk, palms, and soles. On exam, there was also an eschar on the right medial ankle. Laboratory studies showed thrombocytopenia, elevated inflammatory markers, and elevated liver transaminases. Which of the following is the most appropriate treatment?
\vspace{12pt}

\textbf{Options:}
\begin{enumerate}
\item[A.] Antihistamine
\item[B.] Glucocorticoid
\item[C.] Observation
\item[D.] Tetracycline
\item[E.] Valacyclovir
\end{enumerate}

\textbf{Image:}
\begin{center}
\includegraphics[width=0.95\textwidth,height=0.50\textheight,width=0.90\textwidth,keepaspectratio]{images/nejm_20220728.jpg}
\end{center}
\vspace{12pt}
\newpage

\section*{Question 46 (ID: 20220804)}
\textbf{Date: }August 04,2022
\vspace{6pt}

A 71-year-old man presented with an 8-year history of enlarging nodules on his nose. On physical examination, he had painless, violaceous, indurated nodules on his nose, ears, fingers, and toes. Computed tomography of the chest revealed hilar and mediastinal lymphadenopathy. Skin biopsy showed non-caseating granulomas. What is the most likely diagnosis?
\vspace{12pt}

\textbf{Options:}
\begin{enumerate}
\item[A.] Entomophthoramycosis
\item[B.] Leprosy
\item[C.] Lupus pernio
\item[D.] Lupus vulgaris
\item[E.] Rhinophyma
\end{enumerate}

\textbf{Image:}
\begin{center}
\includegraphics[width=0.95\textwidth,height=0.50\textheight,width=0.90\textwidth,keepaspectratio]{images/nejm_20220804.jpg}
\end{center}
\vspace{12pt}
\newpage

\section*{Question 47 (ID: 20220811)}
\textbf{Date: }August 11,2022
\vspace{6pt}

An 86-year-old woman presented with 1 day of nausea, vomiting, and right-sided abdominal pain. Computed tomography of the abdomen and pelvis revealed a distended gallbladder with a thickened wall outside the gallbladder fossa. There was also swirling of the cystic artery and duct but no dilatation of the common bile duct. Which of the following is most likely diagnosis?
\vspace{12pt}

\textbf{Options:}
\begin{enumerate}
\item[A.] Acalculous cholecystitis
\item[B.] Cholangiocarcinoma
\item[C.] Cholangitis
\item[D.] Ectopic gallbladder
\item[E.] Gallbladder volvulus
\end{enumerate}

\textbf{Image:}
\begin{center}
\includegraphics[width=0.95\textwidth,height=0.50\textheight,width=0.90\textwidth,keepaspectratio]{images/nejm_20220811.jpg}
\end{center}
\vspace{12pt}
\newpage

\section*{Question 48 (ID: 20220818)}
\textbf{Date: }August 18,2022
\vspace{6pt}

A 13-year-old girl presented with new onset blurry vision in the left eye associated with painful eye movements. Three weeks earlier, weakness and paresthesias had developed in her left leg that spontaneously resolved. When a light was swung from the normal right eye (Panel A) to the affected left eye, the left pupil dilated (Panel B) - a finding known as a relative afferent pupillary defect. Fundoscopic examination of both eyes was normal. Examination of the brain and whole spine by means of magnetic resonance imaging showed multiple, oblong hyperintense lesions on T2-weighted images. What is the most likely etiology?
\vspace{12pt}

\textbf{Options:}
\begin{enumerate}
\item[A.] Acute disseminated encephalomyelitis
\item[B.] Leber hereditary optic neuropathy
\item[C.] Mononeuritis multiplex
\item[D.] Multiple sclerosis
\item[E.] Retinal detachment
\end{enumerate}

\textbf{Image:}
\begin{center}
\includegraphics[width=0.95\textwidth,height=0.50\textheight,width=0.90\textwidth,keepaspectratio]{images/nejm_20220818.jpg}
\end{center}
\vspace{12pt}
\newpage

\section*{Question 49 (ID: 20220825)}
\textbf{Date: }August 25,2022
\vspace{6pt}

A 32-year-old man presented with a 6-week history of tingling in his arms and legs and a 2-week history of inability to walk. A positive Romberg test, sensory ataxia, impaired proprioception and vibratory sensation, and preserved nociception were noted. Magnetic resonance imaging of the whole spine showed hyperintensity in the posterior spinal cord from C1 to T12 and hyperintense lesions in the dorsal column on T2-weighted images. A vitamin B12 level was 107 pg per ml (reference value, >231) without macrocytic anemia. Toxicity of which of the following is the most likely cause of this presentation?
\vspace{12pt}

\textbf{Options:}
\begin{enumerate}
\item[A.] Copper
\item[B.] Nitrous oxide
\item[C.] Organophosphates
\item[D.] Tetrodotoxin (pufferfish poisoning)
\item[E.] Thallium
\end{enumerate}

\textbf{Image:}
\begin{center}
\includegraphics[width=0.95\textwidth,height=0.50\textheight,width=0.90\textwidth,keepaspectratio]{images/nejm_20220825.jpg}
\end{center}
\vspace{12pt}
\newpage

\section*{Question 50 (ID: 20220901)}
\textbf{Date: }September 01,2022
\vspace{6pt}

A 59-year-old woman was admitted to the hospital with unheralded syncope. She had lost a son to sudden cardiac death when he was 29 years of age. On hospital day 2, the patient developed sustained monomorphic ventricular tachycardia treated with synchronized cardioversion and intravenous amiodarone. An electrocardiogram was performed after the event (Panel A). A transesophageal echocardiogram showed a severely dilated right ventricle (Panel B). Cardiac magnetic resonance imaging showed a right ventricular ejection fraction of 27\% with regional akinesis. What is the most likely diagnosis?
\vspace{12pt}

\textbf{Options:}
\begin{enumerate}
\item[A.] Arrhythmogenic right ventricular cardiomyopathy
\item[B.] Brugada Syndrome
\item[C.] Cardiac sarcoidosis
\item[D.] Inferior Myocardial Infarction
\item[E.] Pulmonary Embolism
\end{enumerate}

\textbf{Image:}
\begin{center}
\includegraphics[width=0.95\textwidth,height=0.50\textheight,width=0.90\textwidth,keepaspectratio]{images/nejm_20220901.jpg}
\end{center}
\vspace{12pt}
\newpage

\section*{Question 51 (ID: 20220908)}
\textbf{Date: }September 08,2022
\vspace{6pt}

A 79-year-old woman with a history of prior stroke was referred for an abnormal X-ray finding along the left heart border, first noticed 6 years prior. In the absence of symptoms, the patient was monitored with serial radiographs which showed a gradual increase in size of the finding. Which of the following is the most likely diagnosis?
\vspace{12pt}

\textbf{Options:}
\begin{enumerate}
\item[A.] Giant Coronary Aneurysm
\item[B.] Left Atrial Appendage Thrombus
\item[C.] Mediastinal Hematoma
\item[D.] Pericardial Cyst
\item[E.] Teratoma
\end{enumerate}

\textbf{Image:}
\begin{center}
\includegraphics[width=0.83\textwidth,height=0.50\textheight,width=0.90\textwidth,keepaspectratio]{images/nejm_20220908.jpg}
\end{center}
\vspace{12pt}
\newpage

\section*{Question 52 (ID: 20220915)}
\textbf{Date: }September 15,2022
\vspace{6pt}

A 58-year-old man with coronary artery disease presented to the emergency department with a 1-day history of intermittent chest pain at rest. An ECG was obtained on arrival, at which time the patient reported no chest pain. Which of the following is the most likely site of the culprit lesion?
\vspace{12pt}

\textbf{Options:}
\begin{enumerate}
\item[A.] Right coronary artery
\item[B.] Left circumflex artery
\item[C.] Left coronary artery
\item[D.] Posterior descending artery
\item[E.] Left anterior descending artery
\end{enumerate}

\textbf{Image:}
\begin{center}
\includegraphics[width=0.95\textwidth,height=0.50\textheight,width=0.90\textwidth,keepaspectratio]{images/nejm_20220915.jpg}
\end{center}
\vspace{12pt}
\newpage

\section*{Question 53 (ID: 20220922)}
\textbf{Date: }September 22,2022
\vspace{6pt}

A 40-year-old man presented to the dermatology clinic with a 1-year history of painless swelling of his earlobes. He also reported nasal congestion and intermittent nosebleeds. Which of the following is the most likely diagnosis?
\vspace{12pt}

\textbf{Options:}
\begin{enumerate}
\item[A.] Cauliflower ear
\item[B.] Eosinophilic Granulomatosis and Polyangiitis
\item[C.] Lepromatous Leprosy
\item[D.] Malignant Otitis Externa
\item[E.] Relapsing Polychondritis
\end{enumerate}

\textbf{Image:}
\begin{center}
\includegraphics[width=0.62\textwidth,height=0.50\textheight,width=0.90\textwidth,keepaspectratio]{images/nejm_20220922.jpg}
\end{center}
\vspace{12pt}
\newpage

\section*{Question 54 (ID: 20220929)}
\textbf{Date: }September 29,2022
\vspace{6pt}

A 24-year-old woman presented with a 2-month history of rash on her shins. The rash was asymptomatic and was more prominent in the left leg. It had first appeared during the winter, at which time she had been sitting by an electric heater that was closer to her left leg. Exam was without palpable purpura, subcutaneous nodules, or areas of gangrene. What is the diagnosis?
\vspace{12pt}

\textbf{Options:}
\begin{enumerate}
\item[A.] Cryoglobulinemia
\item[B.] Cutaneous polyarteritis nodosa
\item[C.] Erythema ab igne
\item[D.] Livedo reticularis
\item[E.] Livedoid vasculitis
\end{enumerate}

\textbf{Image:}
\begin{center}
\includegraphics[width=0.84\textwidth,height=0.50\textheight,width=0.90\textwidth,keepaspectratio]{images/nejm_20220929.jpg}
\end{center}
\vspace{12pt}
\newpage

\section*{Question 55 (ID: 20221006)}
\textbf{Date: }October 06,2022
\vspace{6pt}

A 34-year-old woman presented with a 3-year history of changes in skin color. Examination was notable for hypopigmented macules on a background of hyperpigmentation, creating a raindrop-like appearance on the chest and back. Neighbors who drank water from the same well had similar skin changes. What is the most likely diagnosis?
\vspace{12pt}

\textbf{Options:}
\begin{enumerate}
\item[A.] Arsenic toxicity
\item[B.] Cadmium toxicity
\item[C.] Copper toxicity
\item[D.] Lead toxicity
\item[E.] Mercury toxicity
\end{enumerate}

\textbf{Image:}
\begin{center}
\includegraphics[width=0.9\textwidth,height=0.50\textheight,width=0.90\textwidth,keepaspectratio]{images/nejm_20221006.jpg}
\end{center}
\vspace{12pt}
\newpage

\section*{Question 56 (ID: 20221013)}
\textbf{Date: }October 13,2022
\vspace{6pt}

A 19-year-old woman presented with a 6-month history of a slowly growing asymptomatic dark spot on her left palm. Examination was notable for a nonscaling, nonpalpable brown patch (left). Dermatoscopy revealed pigmented spicules, and palmar skin scrapings were prepared (right). What is the most likely diagnosis?
\vspace{12pt}

\textbf{Options:}
\begin{enumerate}
\item[A.] Acral lentiginous melanoma
\item[B.] Lichen planus
\item[C.] Secondary syphilis
\item[D.] Tinea nigra
\item[E.] Tinea versicolor
\end{enumerate}

\textbf{Image:}
\begin{center}
\includegraphics[width=0.95\textwidth,height=0.50\textheight,width=0.90\textwidth,keepaspectratio]{images/nejm_20221013.jpg}
\end{center}
\vspace{12pt}
\newpage

\section*{Question 57 (ID: 20221020)}
\textbf{Date: }October 20,2022
\vspace{6pt}

A 37-year-old man presented to the emergency department with a 1-week history of pain and swelling in the left upper arm that had started after blunt trauma during soccer practice. The heart rate was 120 beats per minute, the blood pressure 96/54 mm Hg, and the body temperature 37.9°C. Examination of the left upper arm was notable for swelling, tenderness, and crepitus. The overlying skin was red and warm to the touch. An initial radiograph is shown. Which of the following factors is most associated with increased mortality in this diagnosis?
\vspace{12pt}

\textbf{Options:}
\begin{enumerate}
\item[A.] Delay in surgical intervention >24 hours
\item[B.] Involvement of the upper extremity
\item[C.] Negative Blood Cultures
\item[D.] Patient age <60 years
\item[E.] Presence of erythema
\end{enumerate}

\textbf{Image:}
\begin{center}
\includegraphics[width=0.59\textwidth,height=0.50\textheight,width=0.90\textwidth,keepaspectratio]{images/nejm_20221020.jpg}
\end{center}
\vspace{12pt}
\newpage

\section*{Question 58 (ID: 20221027)}
\textbf{Date: }October 27,2022
\vspace{6pt}

A 75-year-old man presented to the emergency department with a 6-month history of epigastric pain, watery diarrhea, and weight loss. Esophagogastroduodenoscopy revealed severe esophagitis, antral erosions, and duodenal ulcers (left). The gastric pH was below 2, and a fasting serum gastrin level was above 1000 pg per milliliter (reference range, 13-115). Cross-sectional imaging identified a single tumor along the distal duodenum that was resected (right). Histopathological analysis showed a well-differentiated neuroendocrine tumor that was positive for gastrin on immunohistochemical staining. The diagnosis in this case is most closely associated with which of the following syndromes?
\vspace{12pt}

\textbf{Options:}
\begin{enumerate}
\item[A.] Li-Fraumeni syndrome
\item[B.] Multiple endocrine neoplasia type 1
\item[C.] Multiple endocrine neoplasia type 2A
\item[D.] Multiple endocrine neoplasia type 2B
\item[E.] Von Hippel-Lindau syndrome
\end{enumerate}

\textbf{Image:}
\begin{center}
\includegraphics[width=0.95\textwidth,height=0.50\textheight,width=0.90\textwidth,keepaspectratio]{images/nejm_20221027.jpg}
\end{center}
\vspace{12pt}
\newpage

\section*{Question 59 (ID: 20221103)}
\textbf{Date: }November 03,2022
\vspace{6pt}

A 5-year-old boy from China with a history of IgA vasculitis at 2 years of age presented with a 1-month history of progressively painful skin ulcerations on his extremities. Two months prior, he had had an episode of nasal pustules that had not responded to antimicrobial therapy or surgical débridement, but had ultimately improved with 2 days of systemic glucocorticoid therapy. At the current presentation, laboratory and radiographic studies showed no signs of underlying systemic disease, such as cancer, inflammatory bowel disease, or autoimmune conditions. Biopsy samples from the lesions showed a dense neutrophilic infiltrate with negative tissue cultures. What is the diagnosis?
\vspace{12pt}

\textbf{Options:}
\begin{enumerate}
\item[A.] Cutaneous leishmaniasis
\item[B.] Cutaneous polyarteritis nodosa
\item[C.] Cutaneous tuberculosis
\item[D.] Ecthyma gangrenosum
\item[E.] Pyoderma gangrenosum
\end{enumerate}

\textbf{Image:}
\begin{center}
\includegraphics[width=0.95\textwidth,height=0.50\textheight,width=0.90\textwidth,keepaspectratio]{images/nejm_20221103.jpg}
\end{center}
\vspace{12pt}
\newpage

\section*{Question 60 (ID: 20221110)}
\textbf{Date: }November 10,2022
\vspace{6pt}

A previously healthy 25-year-old man presented to the general medicine clinic with a 3-month history of worsening fatigue, nausea, and dry cough following administration of a folk remedy for tinea cruris. Laboratory studies showed a hemoglobin level of 6.5 g per deciliter (reference range, 12.0 to 16.0), and normal renal function. Chest and abdominal radiographs were obtained. Which of the following is the appropriate treatment for this condition?
\vspace{12pt}

\textbf{Options:}
\begin{enumerate}
\item[A.] Chelation Therapy
\item[B.] Exchange Transfusion
\item[C.] Hemodialysis
\item[D.] Plasma Exchange Therapy
\item[E.] Therapeutic Phlebotomy
\end{enumerate}

\textbf{Image:}
\begin{center}
\includegraphics[width=0.95\textwidth,height=0.50\textheight,width=0.90\textwidth,keepaspectratio]{images/nejm_20221110.jpg}
\end{center}
\vspace{12pt}
\newpage

\section*{Question 61 (ID: 20221117)}
\textbf{Date: }November 17,2022
\vspace{6pt}

A 45-year-old woman presented with a 6-week history of a painless upper-lip lesion. The lesion had rapidly increased in size over the previous 3 weeks and bled when lightly touched. Examination showed an erythematous, round, smooth, pedunculated mass that was friable and non-tender. What is the most likely diagnosis?
\vspace{12pt}

\textbf{Options:}
\begin{enumerate}
\item[A.] Arteriovenous malformation
\item[B.] Cherry hemangioma
\item[C.] Malignant melanoma
\item[D.] Pyogenic granuloma
\item[E.] Strawberry hemangioma
\end{enumerate}

\textbf{Image:}
\begin{center}
\includegraphics[width=0.74\textwidth,height=0.50\textheight,width=0.90\textwidth,keepaspectratio]{images/nejm_20221117.jpg}
\end{center}
\vspace{12pt}
\newpage

\section*{Question 62 (ID: 20221124)}
\textbf{Date: }November 24,2022
\vspace{6pt}

A 70-year-old woman with depression and irritable bowel syndrome presented to the emergency department with a 3-day history of altered mental status, shortness of breath, nausea, and vomiting. She had recently been self-medicating worsening irritable bowel symptoms. Physical examination was notable for a respiratory rate of 22 breaths per minute, confusion, and mild, diffuse abdominal pain. Laboratory testing showed primary anion-gap metabolic acidosis and primary respiratory alkalosis. Toxicity of which of the following is the most likely cause of her acute symptoms and the findings seen in this non-contrast abdominal radiograph?
\vspace{12pt}

\textbf{Options:}
\begin{enumerate}
\item[A.] Bismuth Salicylate
\item[B.] Calcium Carbonate
\item[C.] Ferrous Sulfate
\item[D.] Magnesium hydroxide
\item[E.] Simethicone
\end{enumerate}

\textbf{Image:}
\begin{center}
\includegraphics[width=0.67\textwidth,height=0.50\textheight,width=0.90\textwidth,keepaspectratio]{images/nejm_20221124.jpg}
\end{center}
\vspace{12pt}
\newpage

\section*{Question 63 (ID: 20221201)}
\textbf{Date: }December 01,2022
\vspace{6pt}

A 26-year-old man presented to the outpatient clinic with a 1-month history of pain and swelling in the scrotum and low-grade fevers. On examination, there was swelling and tenderness of the right side of the scrotum. Laboratory studies showed peripheral eosinophilia. An ultrasound of the scrotum showed echogenic, linear structures moving within the lymphatic channels (arrowhead) adjacent to the epididymal head and testis (asterisk) - a finding known as “filarial dance sign.” What vector is responsible for transmitting the nematode causing this disease?
\vspace{12pt}

\textbf{Options:}
\begin{enumerate}
\item[A.] Aquatic snail
\item[B.] Blackfly
\item[C.] Mosquito
\item[D.] Sandfly
\item[E.] Tsetse fly
\end{enumerate}

\textbf{Image:}
\begin{center}
\includegraphics[width=0.95\textwidth,height=0.50\textheight,width=0.90\textwidth,keepaspectratio]{images/nejm_20221201.jpg}
\end{center}
\vspace{12pt}
\newpage

\section*{Question 64 (ID: 20221208)}
\textbf{Date: }December 08,2022
\vspace{6pt}

An 81-year-old woman with a history of hepatitis C virus-related cirrhosis and hepatocellular carcinoma, who was treated 9 days earlier with transarterial chemoembolization with the use of doxorubicin-eluting beads, presented with painful, progressively worsening skin lesions over the abdomen. A skin biopsy revealed epidermal necrosis and occlusion of small vessels in the reticular dermis. What is the most likely cause of the epidermal necrosis?
\vspace{12pt}

\textbf{Options:}
\begin{enumerate}
\item[A.] Air Embolism
\item[B.] Bacterial Embolism
\item[C.] Drug-Eluting Bead Embolism
\item[D.] Cholesterol Embolism
\item[E.] Tumor Embolism
\end{enumerate}

\textbf{Image:}
\begin{center}
\includegraphics[width=0.95\textwidth,height=0.50\textheight,width=0.90\textwidth,keepaspectratio]{images/nejm_20221208.jpg}
\end{center}
\vspace{12pt}
\newpage

\section*{Question 65 (ID: 20221215)}
\textbf{Date: }December 15,2022
\vspace{6pt}

A 58-year-old woman with a history of metastatic uterine leiomyosarcoma presented with a 1-month history of a nonpruritic rash on her arms. On examination, hyperpigmented plaques were observed on the dorsa of both hands at the sites of previous intravenous access. The darkened skin extended up the arms in a linear pattern along the network of superficial veins; the lesions were palpable but non-tender. What is the most likely diagnosis?
\vspace{12pt}

\textbf{Options:}
\begin{enumerate}
\item[A.] Bleomycin-induced flagellate erythema
\item[B.] Erythema ab igne
\item[C.] Phlegmasia cerulea dolens
\item[D.] Serpentine supravenous hyperpigmentation
\item[E.] Superficial thrombophlebitis
\end{enumerate}

\textbf{Image:}
\begin{center}
\includegraphics[width=0.95\textwidth,height=0.50\textheight,width=0.90\textwidth,keepaspectratio]{images/nejm_20221215.jpg}
\end{center}
\vspace{12pt}
\newpage

\section*{Question 66 (ID: 20221222)}
\textbf{Date: }December 22,2022
\vspace{6pt}

An 18-year-old man presented with a 2-day history of fever, vomiting, and diarrhea. Three weeks before presentation, he had fallen into a canal. Physical exam was notable for the findings shown in the image. Laboratory studies showed acute kidney injury and elevations in aminotransferase and total bilirubin levels. What is the most likely diagnosis?
\vspace{12pt}

\textbf{Options:}
\begin{enumerate}
\item[A.] Acute hemorrhagic conjunctivitis
\item[B.] Ebola hemorrhagic fever
\item[C.] Giardiasis
\item[D.] Leptospirosis
\item[E.] Subarachnoid hemorrhage
\end{enumerate}

\textbf{Image:}
\begin{center}
\includegraphics[width=0.95\textwidth,height=0.50\textheight,width=0.90\textwidth,keepaspectratio]{images/nejm_20221222.jpg}
\end{center}
\vspace{12pt}
\newpage

\section*{Question 67 (ID: 20221229)}
\textbf{Date: }December 29,2022
\vspace{6pt}

A 34-year-old man was admitted to the hospital with diabetic ketoacidosis. He reported having had blurry vision, increasing shoe size, and a change in his facial features over the past several years. On exam, he was found to have bitemporal hemianopsia. Which of the following is the most likely cause of his presentation?
\vspace{12pt}

\textbf{Options:}
\begin{enumerate}
\item[A.] Anabolic steroid use
\item[B.] Beckwith-Wiedemann syndrome
\item[C.] Nelson’s syndrome
\item[D.] Pachydermoperiostosis
\item[E.] Somatotropic pituitary adenoma
\end{enumerate}

\textbf{Image:}
\begin{center}
\includegraphics[width=0.95\textwidth,height=0.50\textheight,width=0.90\textwidth,keepaspectratio]{images/nejm_20221229.jpg}
\end{center}
\vspace{12pt}
\newpage

\section*{Question 68 (ID: 20230105)}
\textbf{Date: }January 05,2023
\vspace{6pt}

A 17-year-old girl presented with a 1-month history of pruritic genital lesions. Three months before presentation, she had unprotected sexual intercourse with multiple male partners. On exam, multiple smooth, grayish-white papules and plaques were noted on the vulva and upper inner thighs. The lesions had a smooth, moist appearance. What is the most likely diagnosis?
\vspace{12pt}

\textbf{Options:}
\begin{enumerate}
\item[A.] Condylomata acuminata
\item[B.] Condylomata lata
\item[C.] Extraintestinal Crohn’s disease
\item[D.] Herpes vegetans
\end{enumerate}

\textbf{Image:}
\begin{center}
\includegraphics[width=0.95\textwidth,height=0.50\textheight,width=0.90\textwidth,keepaspectratio]{images/nejm_20230105.jpg}
\end{center}
\vspace{12pt}
\newpage

\section*{Question 69 (ID: 20230112)}
\textbf{Date: }January 12,2023
\vspace{6pt}

A previously healthy 5-year-old boy was brought to the surgery clinic with a 2-day history of intermittent abdominal pain. On palpation of the abdomen there was pain in the periumbilical region, but no rebound or guarding. An ultrasound was normal, and a computed tomography of the abdomen was performed (Panels A,B). Based on the CT findings, what is the most likely etiology of the patient’s symptoms?
\vspace{12pt}

\textbf{Options:}
\begin{enumerate}
\item[A.] Crohn’s disease
\item[B.] Colocolonic intussusception
\item[C.] Colonic pseudo-obstruction
\item[D.] Ileocolic intussusception
\item[E.] Sigmoid volvulus
\end{enumerate}

\textbf{Image:}
\begin{center}
\includegraphics[width=0.95\textwidth,height=0.50\textheight,width=0.90\textwidth,keepaspectratio]{images/nejm_20230112.jpg}
\end{center}
\vspace{12pt}
\newpage

\section*{Question 70 (ID: 20230119)}
\textbf{Date: }January 19,2023
\vspace{6pt}

A 37-year-old primigravid woman presented at 30 weeks’ gestation with a 4-month history of pruritic pustules on her torso, arms, and legs. On examination, there were papules, nodules, and follicular pustules with surrounding erythema, with no lesions on the palms and soles. A skin biopsy showed subepidermal pustules and perifollicular neutrophilic infiltration. Gram’s stain, periodic acid-Schiff stain, and Grocott methenamine silver stain were all negative. What is the most likely diagnosis?
\vspace{12pt}

\textbf{Options:}
\begin{enumerate}
\item[A.] Atopic eruption of pregnancy
\item[B.] Disseminated herpes zoster
\item[C.] Pemphigoid gestationis
\item[D.] Polymorphic eruption of pregnancy
\item[E.] Secondary syphilis
\end{enumerate}

\textbf{Image:}
\begin{center}
\includegraphics[width=0.95\textwidth,height=0.50\textheight,width=0.90\textwidth,keepaspectratio]{images/nejm_20230119.jpg}
\end{center}
\vspace{12pt}
\newpage

\section*{Question 71 (ID: 20230126)}
\textbf{Date: }January 26,2023
\vspace{6pt}

A previously healthy 14-year-old boy who lived on a farm presented with a 1-month history of episodic headaches associated with vomiting. Magnetic resonance imaging of the head was performed. Which of the following is the most likely diagnosis?
\vspace{12pt}

\textbf{Options:}
\begin{enumerate}
\item[A.] Arachnoid cysts
\item[B.] Colloid cyst
\item[C.] Cystic echinococcosis
\item[D.] Neurocysticercosis
\item[E.] Pineal cyst
\end{enumerate}

\textbf{Image:}
\begin{center}
\includegraphics[width=0.95\textwidth,height=0.50\textheight,width=0.90\textwidth,keepaspectratio]{images/nejm_20230126.jpg}
\end{center}
\vspace{12pt}
\newpage

\section*{Question 72 (ID: 20230202)}
\textbf{Date: }February 02,2023
\vspace{6pt}

A previously healthy 42-year-old woman presented to the dental clinic with a 6-month history of swelling and pain on the right side of her chin (left). She reported no history of chin trauma, tooth pain, or fevers, but did recall injuring her right lateral incisor 10 years prior. Palpation of the lesion caused pain and serosanguinous fluid drainage. Intra-oral examination showed discoloration of the right lateral mandibular incisor (right). What is the most likely diagnosis?
\vspace{12pt}

\textbf{Options:}
\begin{enumerate}
\item[A.] Odontogenic cutaneous fistula
\item[B.] Osteosarcoma
\item[C.] Pilar cyst
\item[D.] Sebaceous cyst
\item[E.] Squamous cell carcinoma
\end{enumerate}

\textbf{Image:}
\begin{center}
\includegraphics[width=0.95\textwidth,height=0.50\textheight,width=0.90\textwidth,keepaspectratio]{images/nejm_20230202.jpg}
\end{center}
\vspace{12pt}
\newpage

\section*{Question 73 (ID: 20230209)}
\textbf{Date: }February 09,2023
\vspace{6pt}

A 77-year-old man with metastatic lung adenocarcinoma presented to the emergency department with a 2-week history of dyspnea. Imaging studies showed a pleural effusion on the right side, previously known liver metastases and perihepatic fluid, and new intrahepatic dilatation of the biliary ducts. A chest tube was placed, and the color of the drained pleural fluid was olive brown. What additional pleural-fluid test is most likely to reveal the diagnosis?
\vspace{12pt}

\textbf{Options:}
\begin{enumerate}
\item[A.] Bacterial gram stain and culture
\item[B.] Bilirubin
\item[C.] Cytology
\item[D.] Fungal stain and culture
\item[E.] Triglycerides
\end{enumerate}

\textbf{Image:}
\begin{center}
\includegraphics[width=0.95\textwidth,height=0.50\textheight,width=0.90\textwidth,keepaspectratio]{images/nejm_20230209.jpg}
\end{center}
\vspace{12pt}
\newpage

\section*{Question 74 (ID: 20230216)}
\textbf{Date: }February 16,2023
\vspace{6pt}

An 80-year-old woman was admitted to the intensive care unit with severe second- and third-degree burns. On laboratory studies, she was noted to have a platelet count of 930,000 per cubic millimeter. Manual analysis of a subsequent peripheral-blood smear identified a platelet count of 115,000 per cubic millimeter. What is the cause for this discrepancy?
\vspace{12pt}

\textbf{Options:}
\begin{enumerate}
\item[A.] Pseudothrombocytopenia due to giant platelets
\item[B.] Pseudothrombocytopenia due to plate clumping associated with EDTA
\item[C.] Pseudothrombocytosis due to cryoglobulin-related interference by cryoprecipitates
\item[D.] Pseudothrombocytosis due to presence of microorganisms
\item[E.] Pseudothrombocytosis due to red cell fragments
\end{enumerate}

\textbf{Image:}
\begin{center}
\includegraphics[width=0.95\textwidth,height=0.50\textheight,width=0.90\textwidth,keepaspectratio]{images/nejm_20230216.jpg}
\end{center}
\vspace{12pt}
\newpage

\section*{Question 75 (ID: 20230223)}
\textbf{Date: }February 23,2023
\vspace{6pt}

A 52-year-old woman with end-stage kidney disease that was being managed with peritoneal dialysis presented with a 1-month history of bloody dialysate. She had had 3 episodes of bacterial peritonitis in the past 12 years. Physical examination and laboratory studies were unremarkable. Computed tomography of the abdomen was performed. What is the most likely diagnosis?
\vspace{12pt}

\textbf{Options:}
\begin{enumerate}
\item[A.] Calciphylaxis
\item[B.] Encapsulating peritoneal sclerosis
\item[C.] Fungal peritonitis
\item[D.] Peritoneal carcinomatosis
\item[E.] Tuberculous peritonitis
\end{enumerate}

\textbf{Image:}
\begin{center}
\includegraphics[width=0.95\textwidth,height=0.50\textheight,width=0.90\textwidth,keepaspectratio]{images/nejm_20230223.jpg}
\end{center}
\vspace{12pt}
\newpage

\section*{Question 76 (ID: 20230302)}
\textbf{Date: }March 02,2023
\vspace{6pt}

A 62-year-old man undergoing abdominal ultrasonography for the evaluation of gallstones was found to have a retroperitoneal mass. A physical examination and the results of routine laboratory studies were normal. A computed tomographic (CT) urogram was completed (left), and a subsequent positron-emission tomography-CT showed no hypermetabolic activity. He underwent stent placement in both ureters, and a core-biopsy of the perinephric soft tissue was obtained. Hematoxylin and eosin staining of the specimen is shown (right). What is the diagnosis?
\vspace{12pt}

\textbf{Options:}
\begin{enumerate}
\item[A.] Erdheim-Chester disease
\item[B.] IgG4-related disease
\item[C.] Langerhans cell histiocytosis
\item[D.] Lymphoma
\item[E.] Sarcoidosis
\end{enumerate}

\textbf{Image:}
\begin{center}
\includegraphics[width=0.95\textwidth,height=0.50\textheight,width=0.90\textwidth,keepaspectratio]{images/nejm_20230302.jpg}
\end{center}
\vspace{12pt}
\newpage

\section*{Question 77 (ID: 20230309)}
\textbf{Date: }March 09,2023
\vspace{6pt}

A 30-year-old woman presented with a 3-month history of progressive skin lesions on her face. On physical examination, there were hyperpigmented, yellow, papillomatous papules and plaques on the face (left). Hyperpigmented velvety plaques were also seen on the skin of the neck, groin and axillae. She also reported new-onset heartburn, fatigue, and unintentional 15-kg weight loss. A skin biopsy of the facial lesion showed epidermal papillomatosis, acanthosis, hyperkeratosis, and negative staining for human papillomavirus. A computed tomography (CT) of the  abdomen and pelvis was performed (right). What is the most likely diagnosis?
\vspace{12pt}

\textbf{Options:}
\begin{enumerate}
\item[A.] Erythrasma
\item[B.] Ichthyosis hystrix
\item[C.] Malignant acanthosis nigricans
\item[D.] Pellagra (Vitamin B3 deficiency)
\item[E.] Pemphigus vegetans
\end{enumerate}

\textbf{Image:}
\begin{center}
\includegraphics[width=0.95\textwidth,height=0.50\textheight,width=0.90\textwidth,keepaspectratio]{images/nejm_20230309.jpg}
\end{center}
\vspace{12pt}
\newpage

\section*{Question 78 (ID: 20230316)}
\textbf{Date: }March 16,2023
\vspace{6pt}

A 12-year-old boy with a history of obesity presented with a 2-week history of limping and dull pain in the right hip. The symptoms had begun after he had slipped and fallen at school. On physical exam, active and passive ranges of motion of the right hip were limited by pain. Radiographs of the pelvis were performed (frog-leg lateral view is shown). What is the most likely diagnosis?
\vspace{12pt}

\textbf{Options:}
\begin{enumerate}
\item[A.] Chondrosarcoma
\item[B.] Legg-Calve-Perthes disease
\item[C.] Osgood-Schlatter disease
\item[D.] Osteoarthritis
\item[E.] Slipped capital femoral epiphysis
\end{enumerate}

\textbf{Image:}
\begin{center}
\includegraphics[width=0.95\textwidth,height=0.50\textheight,width=0.90\textwidth,keepaspectratio]{images/nejm_20230316.jpg}
\end{center}
\vspace{12pt}
\newpage

\section*{Question 79 (ID: 20230323)}
\textbf{Date: }March 23,2023
\vspace{6pt}

A baby girl was noted to have vesicular skin lesions immediately after spontaneous vaginal delivery at 32 weeks’ gestation. Her 37-year-old mother had premature rupture of membranes 12 days before the delivery. On physical examination, clusters of vesicles were seen on the infant’s torso (left), periumbilical region (right), and pharyngeal, nasal, and conjunctival mucosa. No lesions were seen on the mother’s genitals or the placenta. What is the most likely diagnosis?
\vspace{12pt}

\textbf{Options:}
\begin{enumerate}
\item[A.] Bullous impetigo
\item[B.] Epidermolysis bullosa
\item[C.] Molluscum contagiosum
\item[D.] Neonatal herpes simplex virus
\item[E.] Transient pustular melanosis
\end{enumerate}

\textbf{Image:}
\begin{center}
\includegraphics[width=0.95\textwidth,height=0.50\textheight,width=0.90\textwidth,keepaspectratio]{images/nejm_20230323.jpg}
\end{center}
\vspace{12pt}
\newpage

\section*{Question 80 (ID: 20230330)}
\textbf{Date: }March 30,2023
\vspace{6pt}

A pregnant 33-year-old woman at 35 weeks’ gestation presented with redness on both palms without pain and pruritus. She was otherwise asymptomatic. The skin changes had first appeared during the second trimester of pregnancy. Skin examination was notable for mottled, blanching erythema across the palms and fingers of both hands. What is the diagnosis?
\vspace{12pt}

\textbf{Options:}
\begin{enumerate}
\item[A.] Atopic eruption of pregnancy
\item[B.] Intrahepatic cholestasis of pregnancy
\item[C.] Palmar erythema of pregnancy
\item[D.] Polymorphic eruption of pregnancy
\item[E.] Tinea manuum
\end{enumerate}

\textbf{Image:}
\begin{center}
\includegraphics[width=0.95\textwidth,height=0.50\textheight,width=0.90\textwidth,keepaspectratio]{images/nejm_20230330.jpg}
\end{center}
\vspace{12pt}
\newpage

\section*{Question 81 (ID: 20230406)}
\textbf{Date: }April 06,2023
\vspace{6pt}

An 87-year-old man with chronic kidney disease, hypertension, and atrial fibrillation presented to the emergency department with a 3-day history of constipation and lower abdominal pain. Physical examination was notable for abdominal distention and tenderness to palpation of the right lower quadrant without rebound or guarding. A plain radiograph of the abdomen is shown. What is the most likely diagnosis?
\vspace{12pt}

\textbf{Options:}
\begin{enumerate}
\item[A.] Cecal volvulus
\item[B.] Constipation
\item[C.] Hiatal hernia
\item[D.] Ogilvie’s Syndrome (acute colonic pseudo-obstruction)
\item[E.] Small bowel obstruction
\end{enumerate}

\textbf{Image:}
\begin{center}
\includegraphics[width=0.67\textwidth,height=0.50\textheight,width=0.90\textwidth,keepaspectratio]{images/nejm_20230406.jpg}
\end{center}
\vspace{12pt}
\newpage

\section*{Question 82 (ID: 20230413)}
\textbf{Date: }April 13,2023
\vspace{6pt}

A 34-year-old woman with tobacco use disorder was referred to the pulmonary clinic for worsening dyspnea and dry cough. She had chronic hypoxemic respiratory failure of unknown cause, requiring supplemental oxygen. Computed tomography of the chest showed diffuse ground-glass opacities and areas of peripheral consolidation but no basilar reticulations or honeycombing. Bronchoalveolar lavage was not diagnostic, so surgical lung biopsy was performed. Histopathological analysis showed extensive alveolar filling with pigment-laden macrophages. What is the most likely diagnosis?
\vspace{12pt}

\textbf{Options:}
\begin{enumerate}
\item[A.] Desquamative interstitial pneumonia
\item[B.] Diffuse alveolar hemorrhage
\item[C.] Idiopathic pulmonary fibrosis
\item[D.] Lipoid pneumonia
\item[E.] Pulmonary alveolar proteinosis
\end{enumerate}

\textbf{Image:}
\begin{center}
\includegraphics[width=0.95\textwidth,height=0.50\textheight,width=0.90\textwidth,keepaspectratio]{images/nejm_20230413.jpg}
\end{center}
\vspace{12pt}
\newpage

\section*{Question 83 (ID: 20230420)}
\textbf{Date: }April 20,2023
\vspace{6pt}

A 60-year-old left-handed woman presented to the emergency department with pain in her left forearm. The arm was swollen and tender, especially with passive pronation and supination. The overlying skin was intact, and the results of neurovascular examination were normal. Radiographs of the left forearm were performed (upper image, anteroposterior view; lower image, lateral view). The findings should raise concern for which contributory factor?
\vspace{12pt}

\textbf{Options:}
\begin{enumerate}
\item[A.] Assault
\item[C.] Malignancy
\item[D.] Osteoporosis
\item[E.] Repetitive strain
\end{enumerate}

\textbf{Image:}
\begin{center}
\includegraphics[width=0.95\textwidth,height=0.50\textheight,width=0.90\textwidth,keepaspectratio]{images/nejm_20230420.jpg}
\end{center}
\vspace{12pt}
\newpage

\section*{Question 84 (ID: 20230427)}
\textbf{Date: }April 27,2023
\vspace{6pt}

A 75-year-old woman presented to the infectious diseases clinic with a 10-day history of painful lesions on the right hand and forearm. An aspiration of the hand lesion was performed. Bacterial culture of the aspirate grew filamentous, gram-positive, acid-fast branching rods. Which of the following activities likely preceded development of the lesions?
\vspace{12pt}

\textbf{Options:}
\begin{enumerate}
\item[A.] Cleaning a fish tank
\item[B.] Gardening
\item[C.] Getting bitten by a sandfly
\item[D.] Handling feline waste
\item[E.] Petting an armadillo
\end{enumerate}

\textbf{Image:}
\begin{center}
\includegraphics[width=0.95\textwidth,height=0.50\textheight,width=0.90\textwidth,keepaspectratio]{images/nejm_20230427.jpg}
\end{center}
\vspace{12pt}
\newpage

\section*{Question 85 (ID: 20230504)}
\textbf{Date: }May 04,2023
\vspace{6pt}

A 66-year-old man presented with an 8-month history of painful skin tightening and swelling of his arms and legs. The symptoms had progressed on glucocorticoids and methotrexate. On physical exam, there was symmetric, woody induration of the trunk, arms, and legs, sparing the hands and feet. The involved skin had a dimpled appearance. Elevation of the arms resulted in visible indentions along the course of superficial veins. Elbow contractures were also noted. What is the most likely diagnosis?
\vspace{12pt}

\textbf{Options:}
\begin{enumerate}
\item[A.] Eosinophilic fasciitis
\item[B.] Myxedema
\item[C.] Scleredema
\item[D.] Scleroderma
\item[E.] Scleromyxedema
\end{enumerate}

\textbf{Image:}
\begin{center}
\includegraphics[width=0.89\textwidth,height=0.50\textheight,width=0.90\textwidth,keepaspectratio]{images/nejm_20230504.jpg}
\end{center}
\vspace{12pt}
\newpage

\section*{Question 86 (ID: 20230511)}
\textbf{Date: }May 11,2023
\vspace{6pt}

A 55-year-old man who had been admitted to the intensive care unit had a sudden-onset vesicular rash appear across his trunk and arms, as shown. He had undergone a small-bowel resection; his post-operative course had been complicated by sepsis, fevers, and respiratory failure. On exam, he was wrapped tightly in blankets. The vesicles did not rupture with light palpation but broke when more pressure was applied. What is the diagnosis?
\vspace{12pt}

\textbf{Options:}
\begin{enumerate}
\item[A.] Drug eruption
\item[B.] Dyshidrotic eczema
\item[C.] Herpes zoster due to virus reactivation
\item[D.] Miliaria crystallina
\item[E.] Pityrosporum folliculitis
\end{enumerate}

\textbf{Image:}
\begin{center}
\includegraphics[width=0.95\textwidth,height=0.50\textheight,width=0.90\textwidth,keepaspectratio]{images/nejm_20230511.jpg}
\end{center}
\vspace{12pt}
\newpage

\section*{Question 87 (ID: 20230518)}
\textbf{Date: }May 18,2023
\vspace{6pt}

A 44-year-old man with a history of human immunodeficiency virus infection and intermittent adherence to antiretroviral therapy presented to the infectious disease clinic with a 1-month history of fevers and pruritic skin lesions. Physical examination was notable for blackish-brown lamellated plaques on the limbs and scalp. Laboratory studies were notable for a CD4 cell count of 86 per cubic millimeter (reference range, 414 to 1123) and a rapid plasmin reagin titer of 1:32. A skin biopsy of the left forearm showed diffuse dermal lymphocytes and histiocytes admixed with a plasma-cell infiltrate.  What is the most likely diagnosis?
\vspace{12pt}

\textbf{Options:}
\begin{enumerate}
\item[A.] Erythema annulare centrifugum
\item[B.] Erythema gyratum repens
\item[C.] Erythema migrans (Lyme disease)
\item[D.] Lupus vulgaris (cutaneous tuberculosis)
\item[E.] Malignant syphilis
\end{enumerate}

\textbf{Image:}
\begin{center}
\includegraphics[width=0.95\textwidth,height=0.50\textheight,width=0.90\textwidth,keepaspectratio]{images/nejm_20230518.jpg}
\end{center}
\vspace{12pt}
\newpage

\section*{Question 88 (ID: 20230525)}
\textbf{Date: }May 25,2023
\vspace{6pt}

A 50-year-old man with a history of mitral-valve prolapse and of Hodgkin’s lymphoma 20 years earlier presented to the emergency department with a 5-day history of dyspnea. Physical examination showed jugular venous distention, a holosystolic murmur at the cardiac apex, and diminished breath sounds in the lung bases. A radiograph of the chest was performed. What is the best next step in regard to the right lung opacity?
\vspace{12pt}

\textbf{Options:}
\begin{enumerate}
\item[A.] Collect a sputum culture and then initiate empiric antibiotics
\item[B.] Diurese the patient and then obtain a repeat chest radiograph
\item[C.] Obtain a computed tomography scan of chest
\item[D.] Perform a bronchoscopy
\item[E.] Perform a thoracentesis
\end{enumerate}

\textbf{Image:}
\begin{center}
\includegraphics[width=0.95\textwidth,height=0.50\textheight,width=0.90\textwidth,keepaspectratio]{images/nejm_20230525.jpg}
\end{center}
\vspace{12pt}
\newpage

\section*{Question 89 (ID: 20230601)}
\textbf{Date: }June 01,2023
\vspace{6pt}

A 35-year-old man with IgA nephropathy presented with confusion, blurry vision, and seizures. Two weeks before presentation, he had started receiving cyclosporine. Physical examination was notable for a blood pressure of 160/80 mm Hg, drowsiness, and decreased visual acuity. A fundoscopic examinations was normal. T2-weighted magnetic resonance imaging (MRI) with fluid-attenuated inversion recovery sequencing of the head was performed. What is the most likely diagnosis?
\vspace{12pt}

\textbf{Options:}
\begin{enumerate}
\item[A.] Acute demyelinating encephalomyelitis
\item[B.] Methanol ingestion
\item[C.] Multifocal ischemic infarcts
\item[D.] Posterior reversible encephalopathy syndrome
\item[E.] West Nile virus encephalitis
\end{enumerate}

\textbf{Image:}
\begin{center}
\includegraphics[width=0.95\textwidth,height=0.50\textheight,width=0.90\textwidth,keepaspectratio]{images/nejm_20230601.jpg}
\end{center}
\vspace{12pt}
\newpage

\section*{Question 90 (ID: 20230608)}
\textbf{Date: }June 08,2023
\vspace{6pt}

A 32-year-old man presented with 1-month history of enlarging necrotic chest wounds. He reported a 3-year history of daily use of fentanyl by injection into his neck and arm veins. CT of the chest showed osteomyelitis of the clavicles and manubrium, in addition to soft-tissue ulceration and inflammation. Adulteration of fentanyl by which of the following substances is most likely to have contributed to the development of these superinfected wounds?
\vspace{12pt}

\textbf{Options:}
\begin{enumerate}
\item[A.] Cocaine
\item[B.] Levamisole
\item[C.] Methamphetamine
\item[D.] Talcum powder
\item[E.] Xylazine
\end{enumerate}

\textbf{Image:}
\begin{center}
\includegraphics[width=0.95\textwidth,height=0.50\textheight,width=0.90\textwidth,keepaspectratio]{images/nejm_20230608.jpg}
\end{center}
\vspace{12pt}
\newpage

\section*{Question 91 (ID: 20230615)}
\textbf{Date: }June 15,2023
\vspace{6pt}

A 67-year-old man with chronic lymphocytic leukemia (CLL) presented with a painful rash. On physical examination, there was purple discoloration of the ear along with livedoid skin changes on the cheek and purpura on both calves. A skin biopsy showed leukocytoclastic vasculitis. Laboratory testing was notable for a low complement 4 level. What is the most likely diagnosis?
\vspace{12pt}

\textbf{Options:}
\begin{enumerate}
\item[A.] Behçet’s disease
\item[B.] Cryoglobulinemic vasculitis
\item[C.] Granulomatosis with polyangiitis
\item[D.] Leukemia cutis
\item[E.] Microscopic polyangiitis
\end{enumerate}

\textbf{Image:}
\begin{center}
\includegraphics[width=0.93\textwidth,height=0.50\textheight,width=0.90\textwidth,keepaspectratio]{images/nejm_20230615.jpg}
\end{center}
\vspace{12pt}
\newpage

\section*{Question 92 (ID: 20230622)}
\textbf{Date: }June 22,2023
\vspace{6pt}

A 39-year-old man with human immunodeficiency virus infection that was being treated with antiretroviral therapy presented to the dermatology clinic with a 1-year history of recurrent, painful penile ulcers. Approximately once per month, erosions would appear ulcerate, heal spontaneously and then recur. Laboratory testing showed a CD4 cell count of 494 per cubic millimeter (reference range, 414 to 1123) and an HIV viral load of 450 copies per milliliter (reference range, <20). Biopsies of the lesions revealed epidermal necrosis, pseudoepitheliomatous epidermal hyperplasia, and a dense infiltrate of inflammatory cells in the dermis and subcutaneous tissue. Next-generation sequencing (NGS) of the tissue was performed. What is the most likely diagnosis?
\vspace{12pt}

\textbf{Options:}
\begin{enumerate}
\item[A.] Chancroid
\item[B.] Condyloma acuminata
\item[C.] Condyloma lata
\item[D.] Herpes vegetans
\item[E.] Pemphigus vegetans
\end{enumerate}

\textbf{Image:}
\begin{center}
\includegraphics[width=0.93\textwidth,height=0.50\textheight,width=0.90\textwidth,keepaspectratio]{images/nejm_20230622.jpg}
\end{center}
\vspace{12pt}
\newpage

\section*{Question 93 (ID: 20230629)}
\textbf{Date: }June 29,2023
\vspace{6pt}

A 64-year-old man who reported current use of tobacco presented with a 2-week history of tongue discoloration. He reported no associated dysgeusia or tongue pain. Approximately 21 days before presentation, he had completed a course of clindamycin to treat a periodontal infection. After the tongue discoloration began, a course of fluconazole was prescribed to treat possible oral candidiasis. However, the tongue changes had persisted. Which of the following is the appropriate next step in management?
\vspace{12pt}

\textbf{Options:}
\begin{enumerate}
\item[A.] Apply topical budesonide
\item[B.] Counsel patient to stop smoking
\item[C.] Perform a biopsy
\item[D.] Prescribe nystatin
\item[E.] Screen for high risk sexual behaviors
\end{enumerate}

\textbf{Image:}
\begin{center}
\includegraphics[width=0.8\textwidth,height=0.50\textheight,width=0.90\textwidth,keepaspectratio]{images/nejm_20230629.jpg}
\end{center}
\vspace{12pt}
\newpage

\section*{Question 94 (ID: 20230706)}
\textbf{Date: }July 06,2023
\vspace{6pt}

A 38-year-old man presented with a 9-month history of a mildly itchy rash in his groin. The rash had been previously diagnosed as tinea cruris, but it had not improved with topical antifungal treatment. On physical examination, well-circumscribed, reddish-brown plaques were visualized in the inguinal folds when the patient elevated his genitals (left image). No scaling or satellite lesions were present. A potassium hydroxide preparation of skin scrapings was negative. Under a Wood’s lamp, the rash showed coral-red fluorescence (right image). What is the most likely causative organism?
\vspace{12pt}

\textbf{Options:}
\begin{enumerate}
\item[A.] Candida albicans
\item[B.] Corynebacterium minutissimum
\item[C.] Malassezia furfur
\item[D.] Pseudomonas aeruginosa
\item[E.] Trichophyton mentagrophytes
\end{enumerate}

\textbf{Image:}
\begin{center}
\includegraphics[width=0.95\textwidth,height=0.50\textheight,width=0.90\textwidth,keepaspectratio]{images/nejm_20230706.jpg}
\end{center}
\vspace{12pt}
\newpage

\section*{Question 95 (ID: 20230713)}
\textbf{Date: }July 13,2023
\vspace{6pt}

A 78-year-old man with chronic obstructive pulmonary disease (COPD) presented with a 2-month history of dysphonia. For the past 10 years, he had used an inhaled glucocorticoid daily to manage his COPD. Fiberoptic laryngoscopy revealed white plaques on both vocal cords. A biopsy showed hyperkeratinized stratified squamous epithelium and threadlike filaments that stained with Grocott-Gomori methenamine silver stain. What is the most likely diagnosis?
\vspace{12pt}

\textbf{Options:}
\begin{enumerate}
\item[A.] Laryngeal amyloidosis
\item[B.] Laryngeal candidiasis
\item[C.] Laryngeal papillomatosis
\item[D.] Leukoplakia
\item[E.] Vocal-cord dysfunction
\end{enumerate}

\textbf{Image:}
\begin{center}
\includegraphics[width=0.95\textwidth,height=0.50\textheight,width=0.90\textwidth,keepaspectratio]{images/nejm_20230713.jpg}
\end{center}
\vspace{12pt}
\newpage

\section*{Question 96 (ID: 20230720)}
\textbf{Date: }July 20,2023
\vspace{6pt}

A 26-year-old man from Somalia presented with a 5-month history of dry cough, night sweats, and unintentional weight loss of 18 kg. During this period, epigastric pain and postprandial vomiting had also developed. His BMI was 11. On examination, he was cachectic with abdominal distention and diffuse tenderness to palpation. On the basis of chest imaging and sputum studies, a diagnosis of pulmonary tuberculosis was made, and intravenous antituberculous treatment was initiated. However, he continued to have postprandial vomiting. Contrast-enhanced CT of the abdomen was obtained. What is the cause of his abdominal symptoms?
\vspace{12pt}

\textbf{Options:}
\begin{enumerate}
\item[A.] Chronic mesenteric ischemia
\item[B.] Gastric tuberculosis
\item[C.] Gastrointestinal lymphoma
\item[D.] Intussusception
\item[E.] Superior mesenteric artery syndrome
\end{enumerate}

\textbf{Image:}
\begin{center}
\includegraphics[width=0.95\textwidth,height=0.50\textheight,width=0.90\textwidth,keepaspectratio]{images/nejm_20230720.jpg}
\end{center}
\vspace{12pt}
\newpage

\section*{Question 97 (ID: 20230727)}
\textbf{Date: }July 27,2023
\vspace{6pt}

A 13-year-old boy from Mali was referred to the pediatric urology clinic with a 3-month history of gross hematuria. He reported no fevers, flank pain, or dysuria. A physical examination was normal. Laboratory studies showed normal kidney function and an absolute eosinophil count of 2660 per cubic millimeter (reference range, 40-200). A urinalysis showed hematuria and pyuria, and a urine culture was negative. Microsopic examination of the urine is shown. What is the most likely diagnosis?
\vspace{12pt}

\textbf{Options:}
\begin{enumerate}
\item[A.] Balantidium coli
\item[B.] Schistosoma hematobium
\item[C.] Schistosoma mansoni
\item[D.] Strongyloides stercoralis
\item[E.] Trichomonas vaginalis
\end{enumerate}

\textbf{Image:}
\begin{center}
\includegraphics[width=0.83\textwidth,height=0.50\textheight,width=0.90\textwidth,keepaspectratio]{images/nejm_20230727.jpg}
\end{center}
\vspace{12pt}
\newpage

\section*{Question 98 (ID: 20230803)}
\textbf{Date: }August 03,2023
\vspace{6pt}

A 95-year-old woman presented with a 4-week history of dyspnea and dry cough. She had not previously reported these symptoms to her doctor. For the past 6 months, she had been taking nitrofurantoin daily to prevent recurrent urinary tract infections. Her oxygen saturation was 83\% on room air. Physical examination showed inspiratory crackles in the upper lung fields but no jugular venous distention or edema. Laboratory studies showed neutrophilic leukocytosis but no eosinophilia or elevations in aminotransferase levels. A sputum culture and viral respiratory panel were negative. Chest radiograph and computed tomography of the chest are shown. What is the most important element of the management of this condition?
\vspace{12pt}

\textbf{Options:}
\begin{enumerate}
\item[A.] Antibiotics
\item[B.] Cessation of nitrofurantoin
\item[C.] Glucocorticoids
\item[D.] Intravenous loop diuretics
\item[E.] Scheduled nebulized bronchodilators
\end{enumerate}

\textbf{Image:}
\begin{center}
\includegraphics[width=0.95\textwidth,height=0.50\textheight,width=0.90\textwidth,keepaspectratio]{images/nejm_20230803.jpg}
\end{center}
\vspace{12pt}
\newpage

\section*{Question 99 (ID: 20230810)}
\textbf{Date: }August 10,2023
\vspace{6pt}

A 3-year-old girl presented to the emergency department with a 1-day history of dark urine and jaundice after the development of an upper respiratory tract infection 1 week earlier. The physical examination was notable for pharyngeal erythema and exudates, conjunctival pallor, and scleral icterus. There was no hepatosplenomegaly, rash, or acrocyanosis. Laboratory studies showed a new anemia and findings consistent with hemolysis. A direct antiglobulin test was positive for C3d and weakly positive for IgG. A subsequent peripheral blood smear is shown. What is the most likely diagnosis?
\vspace{12pt}

\textbf{Options:}
\begin{enumerate}
\item[A.] Cold agglutinin syndrome
\item[B.] G6PD Deficiency
\item[C.] Hemolytic uremic syndrome
\item[D.] Hereditary spherocytosis
\item[E.] Warm autoimmune hemolytic anemia
\end{enumerate}

\textbf{Image:}
\begin{center}
\includegraphics[width=0.95\textwidth,height=0.50\textheight,width=0.90\textwidth,keepaspectratio]{images/nejm_20230810.jpg}
\end{center}
\vspace{12pt}
\newpage

\section*{Question 100 (ID: 20230817)}
\textbf{Date: }August 17,2023
\vspace{6pt}

A 30-year-old male with recently diagnosed aplastic anemia presented to the emergency department with a 2-day history of fever, nonpruritic rash, and ankle and knee pain. Ten days prior, he had completed a course of horse anti-thymocyte globulin as a treatment for his aplastic anemia. Laboratory studies were notable for worsening neutropenia, low C3 and C4 levels, a CRP of 114 mg per liter (reference value < 5), and negative blood cultures. What type of hypersensitivity reaction is most likely responsible for this patient’s clinical presentation?
\vspace{12pt}

\textbf{Options:}
\begin{enumerate}
\item[A.] Type I hypersensitivity reaction
\item[B.] Type II hypersensitivity reaction
\item[C.] Type III hypersensitivity reaction
\item[D.] Type IV hypersensitivity reaction
\item[E.] This is not a hypersensitivity reaction
\end{enumerate}

\textbf{Image:}
\begin{center}
\includegraphics[width=0.95\textwidth,height=0.50\textheight,width=0.90\textwidth,keepaspectratio]{images/nejm_20230817.jpg}
\end{center}
\vspace{12pt}
\newpage

\section*{Question 101 (ID: 20230824)}
\textbf{Date: }August 24,2023
\vspace{6pt}

A 73-year-old woman presented with a 2-day history of left flank pain. Findings from a physical examination were normal. Laboratory testing was notable for a platelet count of 652,000 per cubic millimeter (reference range, 150,000 to 400,000) with an otherwise normal complete blood count. No previous platelet count was available. Computed tomography of the abdomen with intravenous contrast material showed a filling defect in the left renal artery and a perfusion defect in the left renal cortex, consistent with an occlusive thrombosis of the left renal artery and associated renal infarction. Further testing revealed no evidence of cardioembolic disease, renal artery injury, or inherited thrombophilia. Genetic testing was positive for a genetic variant. Which of the following variants is most likely to be associated with the above findings?
\vspace{12pt}

\textbf{Options:}
\begin{enumerate}
\item[A.] BCR-ABL1
\end{enumerate}

\textbf{Image:}
\begin{center}
\includegraphics[width=0.95\textwidth,height=0.50\textheight,width=0.90\textwidth,keepaspectratio]{images/nejm_20230824.jpg}
\end{center}
\vspace{12pt}
\newpage

\section*{Question 102 (ID: 20230831)}
\textbf{Date: }August 31,2023
\vspace{6pt}

A 53-year-old man who had been admitted to the hospital after a fall was noted to have an abnormal indentation of the lower eyelids. He had a history of corneal transplantation in both eyes. He had recently experienced progressive visual impairment, which had led to his fall. Ophthalmologic examination was notable for a deflection of the lower eyelids when he was looking down, owing to dome-shaped eyes, and decreased visual acuity. What is the most likely diagnosis?
\vspace{12pt}

\textbf{Options:}
\begin{enumerate}
\item[A.] Astigmatism
\item[B.] Corneal ulcer
\item[C.] Keratoconus
\item[D.] Keratoglobus
\item[E.] Pellucid marginal degeneration
\end{enumerate}

\textbf{Image:}
\begin{center}
\includegraphics[width=0.95\textwidth,height=0.50\textheight,width=0.90\textwidth,keepaspectratio]{images/nejm_20230831.jpg}
\end{center}
\vspace{12pt}
\newpage

\section*{Question 103 (ID: 20230907)}
\textbf{Date: }September 07,2023
\vspace{6pt}

A 34-year-old man presented to the emergency department with a 10-day history of shortness of breath that worsened when he bent forward or laid supine and abated when he sat upright. His respiratory rate was 36 breath per minute, and his oxygen saturation was 98\% when sitting upright and 88\% when supine in room air. On physical examination, the lungs were clear and accessory inspiratory muscles were engaged. When the patient laid supine, the abdominal wall paradoxically moved inward during inspiration (left) and outward during expiration (right). Which of the following is not a test used to confirm the cause of the patient’s dyspnea?
\vspace{12pt}

\textbf{Options:}
\begin{enumerate}
\item[A.] Chest radiography
\item[B.] Measurement of inspiratory pressures
\item[C.] Polysomnography
\item[D.] Pulmonary function tests
\item[E.] Ultrasonography of the diaphragm
\end{enumerate}

\textbf{Image:}
\begin{center}
\includegraphics[width=0.95\textwidth,height=0.50\textheight,width=0.90\textwidth,keepaspectratio]{images/nejm_20230907.jpg}
\end{center}
\vspace{12pt}
\newpage

\section*{Question 104 (ID: 20230914)}
\textbf{Date: }September 14,2023
\vspace{6pt}

A 51-year-old man from the Democratic Republic of Congo presented to the emergency department with a 1-week history of small-volume hemoptysis. Sixteen months before presentation, computed tomography (CT) of the chest had shown a right upper lung cavitation that was a sequela of treated pulmonary tuberculosis (top). At the current presentation, a repeat CT of the chest showed thickening of the wall of the right upper lung cavitation and a new intracavitary mass (bottom). Which of the following agents may be used in the management of this condition?
\vspace{12pt}

\textbf{Options:}
\begin{enumerate}
\item[A.] Azithromycin
\item[B.] Rifampin
\item[C.] Rituximab
\item[D.] Trimethoprim-sulfamethoxazole
\item[E.] Voriconazole
\end{enumerate}

\textbf{Image:}
\begin{center}
\includegraphics[width=0.68\textwidth,height=0.50\textheight,width=0.90\textwidth,keepaspectratio]{images/nejm_20230914.jpg}
\end{center}
\vspace{12pt}
\newpage

\section*{Question 105 (ID: 20230921)}
\textbf{Date: }September 21,2023
\vspace{6pt}

A 54-year-old woman with asthma and allergic rhinitis presented with a 3-month history of productive cough and dyspnea. She also reported fevers, chills, night sweats, and an unintentional 9-kg (20-lb) weight loss. On lung examination, there was expiratory wheezing and diffuse crackles. Laboratory testing showed pronounced hypereosinophilia. A computed tomography of the chest showed upper lobe-predominant peripheral and subpleural consolidations that spared the perihilar region (shown). Serum testing for IgE against Aspergillus fumigatus, antibodies against coccidioides, and antineutrophil cytoplasmic antibodies was negative. Bronchoscopy with bronchoalveolar lavage was notable for 74\% eosinophils in the cell count (reference value, <2) and negative tests for infectious diseases. What is the most likely diagnosis?
\vspace{12pt}

\textbf{Options:}
\begin{enumerate}
\item[A.] Allergic bronchopulmonary aspergillosis
\item[B.] Chronic eosinophilic pneumonia
\item[C.] Cryptogenic organizing pneumonia
\item[D.] Drug-Induced eosinophilic pneumonia
\item[E.] Pulmonary tuberculosis
\end{enumerate}

\textbf{Image:}
\begin{center}
\includegraphics[width=0.95\textwidth,height=0.50\textheight,width=0.90\textwidth,keepaspectratio]{images/nejm_20230921.jpg}
\end{center}
\vspace{12pt}
\newpage

\section*{Question 106 (ID: 20230928)}
\textbf{Date: }September 28,2023
\vspace{6pt}

A 55-year-old man presented with 10 years of progressive handwriting impairment and rapid, slurred speech. In his thirties, he had worked as a welder without access to personal protective equipment. Neurologic examination was notable for reduced facial expression, blepharospasm, and cluttered, dysarthric speech. Postural reflexes were mildly impaired. MRI imaging of the head showed a nonenhancing, T1-weighted, hyperintense signal in the basal ganglia on both sides. Ceruloplasmin and iron levels were normal. What treatment should be administered?
\vspace{12pt}

\textbf{Options:}
\begin{enumerate}
\item[A.] Carbidopa-levodopa
\item[B.] Chelation therapy
\item[C.] Cholinesterase inhibitors
\item[D.] Manganese repletion
\item[E.] Phlebotomy
\end{enumerate}

\textbf{Image:}
\begin{center}
\includegraphics[width=0.66\textwidth,height=0.50\textheight,width=0.90\textwidth,keepaspectratio]{images/nejm_20230928.jpg}
\end{center}
\vspace{12pt}
\newpage

\section*{Question 107 (ID: 20231005)}
\textbf{Date: }October 05,2023
\vspace{6pt}

A 72-year-old man presented to the emergency department with a 2-day history of an itchy rash on his back. On physical examination, edematous, flagellate plaques and linear patches were present across the patient’s entire back and upper buttocks. There was no adenopathy, dermographism, or mucosal involvement. What substance had the patient most likely handled and ingested before subsequently developing this rash?
\vspace{12pt}

\textbf{Options:}
\begin{enumerate}
\item[A.] Phenytoin
\item[B.] Shellfish
\item[C.] Shiitake mushrooms
\item[D.] Toxicodendron radicans (poison ivy)
\item[E.] Warfarin
\end{enumerate}

\textbf{Image:}
\begin{center}
\includegraphics[width=0.57\textwidth,height=0.50\textheight,width=0.90\textwidth,keepaspectratio]{images/nejm_20231005.jpg}
\end{center}
\vspace{12pt}
\newpage

\section*{Question 108 (ID: 20231012)}
\textbf{Date: }October 12,2023
\vspace{6pt}

A 44-year-old man with diabetes and end-stage kidney disease presented with a 2-week history of pain and blurry vision in his left eye, fevers, and back pain. An ophthalmologic examination was notable for conjunctival injection and corneal clouding in the left eye. Visual acuity was 20/60 in the right eye and was reduced to minimal light perception in the left eye. Slit-lamp examination revealed a hazy, edematous cornea and a small hypopyon in the left eye. Cultures of vitreous fluid and blood grew methicillin-sensitive Staphylococcus aureus. Which of the following is the best next step in management in addition to systemic antimicrobial treatment?
\vspace{12pt}

\textbf{Options:}
\begin{enumerate}
\item[A.] Intravitreal antimicrobial injections
\item[B.] Intravitreal steroid injection
\item[C.] Removal of intraocular lens
\item[D.] Topical antimicrobial drops
\item[E.] Topicals steroid drops
\end{enumerate}

\textbf{Image:}
\begin{center}
\includegraphics[width=0.95\textwidth,height=0.50\textheight,width=0.90\textwidth,keepaspectratio]{images/nejm_20231012.jpg}
\end{center}
\vspace{12pt}
\newpage

\section*{Question 109 (ID: 20231019)}
\textbf{Date: }October 19,2023
\vspace{6pt}

An 8-year-old girl was brought to the dermatology clinic with a 3-day history of painful lip crusting, oral ulcers, rash, and genital pain. Ten days before presentation, a fever and cough had developed. She had no history of medication use. Physical examination was notable for conjunctivitis in both eyes and ulcers in the oropharynx. There were scattered vesicular lesions across her face, and her swollen, bleeding, exudative lips made it difficult for her to open her mouth (left). Targetoid lesions with vesicles were scattered across her arms and legs (right), and multiple small ulcers were found in the vulvar and perianal regions. Rales were present in both lung bases. Imaging of the chest showed infiltrates in both lungs. What is the most appropriate next step?
\vspace{12pt}

\textbf{Options:}
\begin{enumerate}
\item[A.] Administration of systemic glucocorticoids
\item[B.] Allergy/immunology consultation
\item[C.] Skin biopsy
\item[D.] Testing for bacterial and viral respiratory pathogens
\item[E.] Testing for herpes simplex virus
\end{enumerate}

\textbf{Image:}
\begin{center}
\includegraphics[width=0.95\textwidth,height=0.50\textheight,width=0.90\textwidth,keepaspectratio]{images/nejm_20231019.jpg}
\end{center}
\vspace{12pt}
\newpage

\section*{Question 110 (ID: 20231026)}
\textbf{Date: }October 26,2023
\vspace{6pt}

An 83-year-old woman with a history of type 2 diabetes mellitus presented with a 4-month history of a pruritic rash on her back. Physical examination showed a linear array of crateriform lesions containing crusted material on an erythematous base. A skin biopsy showed a cup-shaped ulceration with transepidermal elimination of basophilic collagen and with cellular debris. What is the most likely diagnosis?
\vspace{12pt}

\textbf{Options:}
\begin{enumerate}
\item[A.] Dermatofibroma
\item[B.] Folliculitis
\item[C.] Keratoacanthoma
\item[D.] Reactive perforating collagenosis
\item[E.] Prurigo nodularis
\end{enumerate}

\textbf{Image:}
\begin{center}
\includegraphics[width=0.95\textwidth,height=0.50\textheight,width=0.90\textwidth,keepaspectratio]{images/nejm_20231026.jpg}
\end{center}
\vspace{12pt}
\newpage

\section*{Question 111 (ID: 20231102)}
\textbf{Date: }November 02,2023
\vspace{6pt}

A 38-year-old man with end-stage renal disease who was undergoing hemodialysis presented to the hospital with several years of progressive difficulty in walking. Four years before presentation, acute pain and swelling in both knees had developed after he had stepped off an auto rickshaw. At that time, he had opted for conservative management of his injuries. On physical examination at the current presentation, there was a soft-tissue depression proximal to the patella on both sides (left). A palpable suprapatellar gap was also present on both sides at the site of the expected quadriceps tendon insertion (right). The patient was unable to extend his knees and walked with flexed knees. What is the most likely diagnosis?
\vspace{12pt}

\textbf{Options:}
\begin{enumerate}
\item[A.] Femoral nerve injuries
\item[B.] Patellar stress fractures
\item[C.] Patellar tendon ruptures
\item[D.] Quadriceps strains
\item[E.] Quadriceps tendon ruptures
\end{enumerate}

\textbf{Image:}
\begin{center}
\includegraphics[width=0.95\textwidth,height=0.50\textheight,width=0.90\textwidth,keepaspectratio]{images/nejm_20231102.jpg}
\end{center}
\vspace{12pt}
\newpage

\section*{Question 112 (ID: 20231109)}
\textbf{Date: }November 09,2023
\vspace{6pt}

A 60-year-old woman with hypertension and chronic kidney disease of unknown cause was referred to the emergency department because of a serum creatinine level of 7.8 mg per deciliter (reference range, 0.5 to 0.9). She had been feeling well before presentation. A urinalysis showed 2+ protein and more than 180 red cells per high-power field with no urinary casts. On the basis of a positive indirect fluorescent antibody assay for anti-glomerular basement membrane (GBM) antibodies and a kidney biopsy showing crescentic glomerulonephritis (left) with strong linear GBM staining for IgG on immunofluorescence (right), a diagnosis of anti-GBM glomerulonephritis was made. Which additional diagnostic test is required to look for extra-renal involvement?
\vspace{12pt}

\textbf{Options:}
\begin{enumerate}
\item[A.] Chest imaging
\item[B.] Echocardiogram
\item[C.] Ophthalmologic exam
\item[D.] Otorhinolaryngology evaluation
\item[E.] Skin biopsy
\end{enumerate}

\textbf{Image:}
\begin{center}
\includegraphics[width=0.95\textwidth,height=0.50\textheight,width=0.90\textwidth,keepaspectratio]{images/nejm_20231109.jpg}
\end{center}
\vspace{12pt}
\newpage

\section*{Question 113 (ID: 20231116)}
\textbf{Date: }November 16,2023
\vspace{6pt}

A 53-year-old man presented with a 3-year history of an itchy rash, Raynaud’s phenomenon, dysphagia, and a burning sensation in his hands. Physical examination was notable for firm, greasy papules across his forehead that led to the formation of glabellar grooves (left). There were waxy papules on his hands, with associated skin thickening and finger flexion contractures (right). Similar skin changes were seen on his nose, lips, ears, trunk, and feet. There was no telangiectasia or calcinosis. Sensory neuropathy was present in his hands, arms, and face. Tests of thyroid function were normal. Serum protein electrophoresis with immunofixation identified an IgG-monoclonal gammopathy, and a bone marrow biopsy was normal. What is the most likely diagnosis?
\vspace{12pt}

\textbf{Options:}
\begin{enumerate}
\item[A.] Light chain (AL) amyloidosis
\item[B.] Multiple myeloma
\item[C.] Scleredema
\item[D.] Scleromyxedema
\item[E.] Systemic sclerosis
\end{enumerate}

\textbf{Image:}
\begin{center}
\includegraphics[width=0.95\textwidth,height=0.50\textheight,width=0.90\textwidth,keepaspectratio]{images/nejm_20231116.jpg}
\end{center}
\vspace{12pt}
\newpage

\section*{Question 114 (ID: 20231123)}
\textbf{Date: }November 23,2023
\vspace{6pt}

A 69-year-old woman with a history of asbestos exposure presented to the emergency department with a 3-year history of dyspnea on exertion. Physical examination showed signs of volume overload. A chest radiograph showed circumferential calcification of the pericardium and pleural effusions. Simultaneous left and right heart catheterization showed ventricular interdependence and discordance of the pressure tracings (right ventricular pressure, solid arrow; left ventricular pressure, dotted arrow). What is the diagnosis?
\vspace{12pt}

\textbf{Options:}
\begin{enumerate}
\item[A.] Cardiac Tamponade
\item[B.] Constrictive Pericarditis
\item[C.] Effusive-Constrictive Pericarditis
\item[D.] Primary Pericardial Mesothelioma
\item[E.] Restrictive Cardiomyopathy
\end{enumerate}

\textbf{Image:}
\begin{center}
\includegraphics[width=0.94\textwidth,height=0.50\textheight,width=0.90\textwidth,keepaspectratio]{images/nejm_20231123.jpg}
\end{center}
\vspace{12pt}
\newpage

\section*{Question 115 (ID: 20231130)}
\textbf{Date: }November 30,2023
\vspace{6pt}

A 53-year-old woman presented with a 3-month history of worsening vascular skin lesions and a 1-month history of fever. On physical examination, diffuse telangiectasis, hyperpigmented plaques, and several ulcerated nodules (arrows) were observed on the skin across the chest and abdomen (left) and the legs. No palpable lymphadenopathy or hepatosplenomegaly was noted. Laboratory studies were notable for a lactate dehydrogenase level of 35664 U per liter (reference range, 120 to 250). A deep skin biopsy specimen from the abdomen showed intravascular aggregation of round, atypical lymphocytes (right, hematoxylin and eosin staining). Subsequent immunohistochemical staining was positive for CD20, PAX-5, and MUM-1 in the neoplastic cells. Which of the following is the most likely diagnosis?
\vspace{12pt}

\textbf{Options:}
\begin{enumerate}
\item[A.] Chronic lymphocytic leukemia
\item[B.] Cutaneous small vessel vasculitis
\item[C.] Idiopathic multicentric Castleman
\item[D.] Intralymphatic histiocytosis
\item[E.] Intravascular lymphoma
\end{enumerate}

\textbf{Image:}
\begin{center}
\includegraphics[width=0.95\textwidth,height=0.50\textheight,width=0.90\textwidth,keepaspectratio]{images/nejm_20231130.jpg}
\end{center}
\vspace{12pt}
\newpage

\section*{Question 116 (ID: 20231207)}
\textbf{Date: }December 07,2023
\vspace{6pt}

A previously healthy 26-year-old man presented to the emergency department with a 5-day history of an asymptomatic rash, sore throat, fevers, chills, and malaise. On physical examination, scattered, erythematous papules and macules could be seen across the upper chest and anterior neck (left). In the mouth, palatal petechiae, buccal mucosal ulcerations, and pharyngeal erythema were observed (right). The patient reported having had condomless sex with a new partner 2 weeks before presentation. A diagnosis of acute Human Immunodeficiency Virus (HIV) infection was made. Which of the following is the first to become positive after a patient is infected with HIV?
\vspace{12pt}

\textbf{Options:}
\begin{enumerate}
\item[A.] HIV IgG antibody
\item[B.] HIV IgM antibody
\item[C.] HIV nucleic acid test
\item[D.] p24 antigen
\item[E.] Western Blot
\end{enumerate}

\textbf{Image:}
\begin{center}
\includegraphics[width=0.95\textwidth,height=0.50\textheight,width=0.90\textwidth,keepaspectratio]{images/nejm_20231207.jpg}
\end{center}
\vspace{12pt}
\newpage

\section*{Question 117 (ID: 20231214)}
\textbf{Date: }December 14,2023
\vspace{6pt}

A 49-year-old woman with a history of Graves’ disease presented to the endocrinology clinic with a 3-year history of skin changes on her arms and legs and a 1-year history of thyrotoxicosis symptoms. She had intermittently adhered to carbimazole therapy since her diagnosis 10 years earlier. On physical examination, violaceous, nonpitting, indurated nodules were observed on the dorsa of her hands, distal forearms, and shins (left). A goiter (right, arrow), proptosis, and lid lag were also noted. Laboratory testing showed an undetectable thyrotropin level and elevated total and free thyroxine and triiodothyronine levels. Testing for which value would you most expect to be positive in this patient?
\vspace{12pt}

\textbf{Options:}
\begin{enumerate}
\item[A.] C-peptide
\item[B.] Glutamic acid decarboxylase antibodies
\item[C.] Thyroglobulin antibodies
\item[D.] Thyroid peroxidase antibodies
\item[E.] Thyrotropin receptor antibodies
\end{enumerate}

\textbf{Image:}
\begin{center}
\includegraphics[width=0.95\textwidth,height=0.50\textheight,width=0.90\textwidth,keepaspectratio]{images/nejm_20231214.jpg}
\end{center}
\vspace{12pt}
\newpage

\section*{Question 118 (ID: 20231221)}
\textbf{Date: }December 21,2023
\vspace{6pt}

A 25-year-old woman with severe obesity presented to the emergency department with a 1-week history of blurred vision, headaches, and transient visual obscurations. Neurologic examination showed optic disk swelling and retinal hemorrhages in both eyes. An MRI and MRV of the head showed flattened posterior globes, an empty sella, and stenoses of the transverse sinuses without obstruction or thromboses. Which abnormality would most likely be notable from a lumbar puncture?
\vspace{12pt}

\textbf{Options:}
\begin{enumerate}
\item[A.] Elevated opening pressure
\item[B.] Elevated white blood cell count
\item[C.] Oligoclonal bands
\item[D.] Positive anti-aquaporin 4 antibody
\item[E.] Xanthochromia
\end{enumerate}

\textbf{Image:}
\begin{center}
\includegraphics[width=0.95\textwidth,height=0.50\textheight,width=0.90\textwidth,keepaspectratio]{images/nejm_20231221.jpg}
\end{center}
\vspace{12pt}
\newpage

\section*{Question 119 (ID: 20231228)}
\textbf{Date: }December 28,2023
\vspace{6pt}

A 43-year-old woman presented to the dermatology clinic with an 8-year history of yellow-brown spots on her shins. The lesions had been asymptomatic, and she had not sought care for them until they had grown in size. She had no history of diabetes mellitus, hypertension, or thyroid disease. On physical examination, atrophic yellow-brown plaques with telangiectasias and irregular violaceous borders were observed on both shins. A skin biopsy of the right shin was performed. Histopathological analysis showed several layers of necrobiosis within the dermis, perivascular inflammatory-cell infiltrates, collagen degeneration, and findings consistent with granulomatous dermatitis. Which of the following is the most likely diagnosis?
\vspace{12pt}

\textbf{Options:}
\begin{enumerate}
\item[A.] Cutaneous sarcoidosis
\item[B.] Granuloma annulare
\item[C.] Necrobiosis lipoidica
\item[D.] Pigmented purpuric dermatosis
\item[E.] Stasis purpuric dermatosis
\end{enumerate}

\textbf{Image:}
\begin{center}
\includegraphics[width=0.64\textwidth,height=0.50\textheight,width=0.90\textwidth,keepaspectratio]{images/nejm_20231228.jpg}
\end{center}
\vspace{12pt}
\newpage

\section*{Question 120 (ID: 20240104)}
\textbf{Date: }January 04,2024
\vspace{6pt}

A 26-year-old man presented to the dermatology clinic with a 1-week history of an asymptomatic rash on his hands and feet. He also reported having had a fever - which had resolved 6 days before presentation - but no joint pains or oral lesions. Physical examination was notable for scattered, partially blanchable macules around the wrists and ankles that merged into erythematous patches on both the ventral (left) and dorsal (right) surfaces of the hands and feet. Serum B19 IgM antibody test and polymerase-chain-reaction assay were positive for parvovirus, and a diagnosis of papular-purpuric “gloves and socks” syndrome in the context of parvovirus B19 infection was made. What is the more classic appearance and distribution of the rash associated with this virus?
\vspace{12pt}

\textbf{Options:}
\begin{enumerate}
\item[A.] Purpura on the buttocks and lower extremities
\item[B.] “Sandpaper” rash that is accentuated in flexor creases
\item[C.] “Slapped cheek” pattern on the face
\item[D.] Transient salmon-pink maculopapular rash on the trunk
\item[E.] Vesicles on an erythematous base in a dermatomal distribution
\end{enumerate}

\textbf{Image:}
\begin{center}
\includegraphics[width=0.95\textwidth,height=0.50\textheight,width=0.90\textwidth,keepaspectratio]{images/nejm_20240104.jpg}
\end{center}
\vspace{12pt}
\newpage

\section*{Question 121 (ID: 20240111)}
\textbf{Date: }January 11,2024
\vspace{6pt}

A 55-year-old woman presented to the dermatology clinic with a 1-year history of skin darkening on her face. Two years before presentation, she had started applying a skin-lightening cream containing hydroquinone to her face daily to treat melasma. On physical examination, bluish-brown patches with background erythema and telangiectasias were observed on the cheeks, nasal bridge, and perioral region, with lesser involvement on the forehead (left). Dermoscopy of the affected areas revealed hyperchromic, pinpoint macules (middle). A skin-biopsy sample from the left cheek showed extracellular deposition of yellow-brown, banana-shaped bodies in the dermis (right, hematoxylin and eosin stain). What is the most likely diagnosis?
\vspace{12pt}

\textbf{Options:}
\begin{enumerate}
\item[A.] Contact dermatitis
\item[B.] Eczematous drug eruption
\item[C.] Exogenous ochronosis
\item[D.] Lichen planus pigmentosus
\item[E.] Solar lentigenes
\end{enumerate}

\textbf{Image:}
\begin{center}
\includegraphics[width=0.95\textwidth,height=0.50\textheight,width=0.90\textwidth,keepaspectratio]{images/nejm_20240111.jpg}
\end{center}
\vspace{12pt}
\newpage

\section*{Question 122 (ID: 20240118)}
\textbf{Date: }January 18,2024
\vspace{6pt}

A 9-year-old boy who had recently emigrated from Brazil presented to the emergency department with a 3-week history of neck swelling, fevers, and weight loss. On physical examination, there was fixed, tender lymphadenopathy in the posterior auricular, submandibular, and occipital chains. Laboratory testing was notable for peripheral eosinophilia. Tests for cryptococcus, histoplasmosis, and human immunodeficiency virus were negative. CT scan of the neck showed hyperattenuating cervical lymphadenopathy on both sides. Lymph node biopsy results are shown. What is the most likely diagnosis?
\vspace{12pt}

\textbf{Options:}
\begin{enumerate}
\item[A.] Actinomycosis
\item[B.] Blastomycosis
\item[C.] Coccidiomycosis
\item[D.] Hodgkin’s Lymphoma
\item[E.] Paracoccidiomycosis
\end{enumerate}

\textbf{Image:}
\begin{center}
\includegraphics[width=0.95\textwidth,height=0.50\textheight,width=0.90\textwidth,keepaspectratio]{images/nejm_20240118.jpg}
\end{center}
\vspace{12pt}
\newpage

\section*{Question 123 (ID: 20240125)}
\textbf{Date: }January 25,2024
\vspace{6pt}

A 66-year-old man with a history of hypertension, diabetes mellitus, and ischemic stroke was transferred to a tertiary hospital after a cardiac arrest. For 6 months before presentation, he had recurrent exertional angina but had not sought evaluation. On the morning of the cardiac arrest, he had woken up with chest pain, lost consciousness, and regained consciousness after brief cardiopulmonary resuscitation by his family. On transfer to the tertiary hospital, findings from a physical examination and a transthoracic echocardiogram were normal. Coronary angiography revealed 50\% stenosis in the middle left anterior descending (LAD) coronary artery during diastole (upper left) with complete occlusion during systole (upper right) and sluggish distal flow. Which of the following is NOT a recommended therapy for this condition?
\vspace{12pt}

\textbf{Options:}
\begin{enumerate}
\item[A.] Beta blockers
\item[B.] Calcium channel blockers
\item[C.] Coronary artery bypass grafting
\item[D.] Ivabradine
\item[E.] Nitroglycerin
\end{enumerate}

\textbf{Image:}
\begin{center}
\includegraphics[width=0.95\textwidth,height=0.50\textheight,width=0.90\textwidth,keepaspectratio]{images/nejm_20240125.jpg}
\end{center}
\vspace{12pt}
\newpage

\section*{Question 124 (ID: 20240201)}
\textbf{Date: }February 01,2024
\vspace{6pt}

A 44-year-old man presented to the emergency department with a 3-day history of vision loss and pain in the left eye. The symptoms had started after he had passed out for 3 hours in a position that put pressure on his left eye; before losing consciousness, he had taken insomnia medications and consumed alcohol. An anterior segment examination showed hemorrhagic chemosis and a fixed, mid-dilated pupil (left). The intraocular pressure in the left eye was normal. Funduscopy showed diffuse retinal whitening, a finding consistent with infarction, and optical coherence tomography revealed full-thickness retinal edema. Magnetic resonance imaging of the orbit showed engorgement of the extraocular muscles and orbital tissue (right). A diagnosis of ischemic retinopathy and choroidopathy owing to prolonged orbital compression was made. Which of the following is LEAST likely to be found on physical examination in this patient?
\vspace{12pt}

\textbf{Options:}
\begin{enumerate}
\item[A.] Absence of light perception in the left eye
\item[B.] A relative afferent pupillary defect
\item[C.] Complete ophthalmoplegia of the left eye
\item[D.] Proptosis
\item[E.] Vertical nystagmus
\end{enumerate}

\textbf{Image:}
\begin{center}
\includegraphics[width=0.95\textwidth,height=0.50\textheight,width=0.90\textwidth,keepaspectratio]{images/nejm_20240201.jpg}
\end{center}
\vspace{12pt}
\newpage

\section*{Question 125 (ID: 20240208)}
\textbf{Date: }February 08,2024
\vspace{6pt}

An otherwise healthy 42-year-old woman presented with a 10-day history of a rash in her axillae and on her groin and abdomen. One and a half weeks before the onset of the rash, she had started taking dexketoprofen (a nonsteroidal antiinflammatory drug [NSAID]) at a dose of 25 mg per day to treat knee pain. She reported no fevers, mucosal lesions, or symptoms other than mild pruritus. On physical examination, symmetric patches of reddish-purple skin with peeling borders were present in the cervical and axillary regions (top-left), the
abdominal and inguinal regions (right), and the intertriginous area of the back (bottom-left). What drug class is the most common trigger of this rash?
\vspace{12pt}

\textbf{Options:}
\begin{enumerate}
\item[A.] Angiotensin receptor blockers
\item[B.] Beta lactam antibiotics
\item[C.] Estrogens
\item[D.] Fluoroquinolones
\item[E.] Sulfonamides
\end{enumerate}

\textbf{Image:}
\begin{center}
\includegraphics[width=0.95\textwidth,height=0.50\textheight,width=0.90\textwidth,keepaspectratio]{images/nejm_20240208.jpg}
\end{center}
\vspace{12pt}
\newpage

\section*{Question 126 (ID: 20240215)}
\textbf{Date: }February 15,2024
\vspace{6pt}

A 63-year-old man with a history of follicular lymphoma presented with 2 weeks of fatigue and 3 days of dyspnea. On physical examination, he had decreased breath sounds at the base of both lungs. A chest radiograph revealed pleural effusions, which were greater on the right side than on the left. Thoracentesis of the right side resulted in the removal of 1 liter of milky, yellow fluid. A chylothorax was suspected. Which of the following pleural fluid tests is most commonly used to diagnose a chylothorax?
\vspace{12pt}

\textbf{Options:}
\begin{enumerate}
\item[A.] Cholesterol
\item[B.] Chylomicrons
\item[C.] Cytology
\item[D.] Lactate dehydrogenase
\item[E.] Triglyceride
\end{enumerate}

\textbf{Image:}
\begin{center}
\includegraphics[width=0.95\textwidth,height=0.50\textheight,width=0.90\textwidth,keepaspectratio]{images/nejm_20240215.jpg}
\end{center}
\vspace{12pt}
\newpage

\section*{Question 127 (ID: 20240222)}
\textbf{Date: }February 22,2024
\vspace{6pt}

A 28-year-old woman presented to the hospital with a 6-month history of dry cough. She was a lifetime nonsmoker and reported no fevers, joint aches, eye pain, or rashes. On physical examination, auscultation of both lower lungs revealed fine crackles. High-resolution computed tomography (CT) of the chest showed mediastinal lymphadenopathy and diffuse ground-glass opacities (left, axial view). Also visible were areas of superimposed interlobular and intralobular septal thickening, a pattern known as crazy paving (left, box). A subsequent transbronchial lung biopsy showed multiple noncaseating granulomas (middle, inset showing granuloma; hematoxylin and eosin stain). Bronchoalveolar lavage cultures, histopathological analysis, and molecular testing were negative for infectious organisms, including Mycobacterium tuberculosis. Which of the following is the most likely diagnosis?
\vspace{12pt}

\textbf{Options:}
\begin{enumerate}
\item[A.] Foreign body granulomatosis
\item[B.] Granulomatosis with polyangiitis
\item[C.] Pulmonary alveolar proteinosis
\item[D.] Pulmonary Langerhans cell histiocytosis
\item[E.] Pulmonary Sarcoidosis
\end{enumerate}

\textbf{Image:}
\begin{center}
\includegraphics[width=0.95\textwidth,height=0.50\textheight,width=0.90\textwidth,keepaspectratio]{images/nejm_20240222.jpg}
\end{center}
\vspace{12pt}
\newpage

\section*{Question 128 (ID: 20240229)}
\textbf{Date: }February 29,2024
\vspace{6pt}

A 40-year-old man was referred to the otorhinolaryngology clinic with a 1-month history of a sore throat. He reported no upper respiratory symptoms, fever, rash, or genital lesions. The physical examination was notable for nonulcerated white plaques that formed a butterfly shape across the posterior oropharynx, upper uvula, and tonsils. No lymphadenopathy or skin or genital lesions were present. Testing for the human immunodeficiency virus was negative. A biopsy of the plaques showed dense lymphoplasmacytic infiltration. Which of the following tests is most likely to reveal the diagnosis?
\vspace{12pt}

\textbf{Options:}
\begin{enumerate}
\item[A.] Congo red stain of the biopsy specimen
\item[B.] Flow cytometry of the biopsy specimen
\item[C.] Grocott’s methenamine silver stain of the biopsy specimen
\item[D.] Serum protein electrophoresis
\item[E.] Treponema pallidum hemagglutination assay
\end{enumerate}

\textbf{Image:}
\begin{center}
\includegraphics[width=0.95\textwidth,height=0.50\textheight,width=0.90\textwidth,keepaspectratio]{images/nejm_20240229.jpg}
\end{center}
\vspace{12pt}
\newpage

\section*{Question 129 (ID: 20240307)}
\textbf{Date: }March 07,2024
\vspace{6pt}

A 76-year-old man presented to the dermatology clinic with a 2-day history of blood-filled blisters on the tongue. He reported no prior trauma or other bleeding symptoms. Physical examination was notable for hemorrhagic bullae on the tongue and gingiva, and purpura on the arms and legs. Which of the following lab abnormalities is most likely present?
\vspace{12pt}

\textbf{Options:}
\begin{enumerate}
\item[A.] Anemia
\item[B.] Elevated PT and PTT
\item[C.] Leukopenia
\item[D.] Thrombocytopenia
\item[E.] Vitamin C deficiency
\end{enumerate}

\textbf{Image:}
\begin{center}
\includegraphics[width=0.95\textwidth,height=0.50\textheight,width=0.90\textwidth,keepaspectratio]{images/nejm_20240307.jpg}
\end{center}
\vspace{12pt}
\newpage

\section*{Question 130 (ID: 20240314)}
\textbf{Date: }March 14,2024
\vspace{6pt}

An 82-year-old man presented to the emergency department with a 3-year history of progressive generalized weakness. Four months before presentation, a left adrenal mass had been identified incidentally on computed tomography (CT) that had been performed to evaluate an episode of chest pain. During the month preceding presentation, he had lost 8 kg of weight and had become unable to sit up in bed. Physical examination was unremarkable. Laboratory studies were notable for a white-cell count of 2700 per cubic millimeter (reference range, 3700 to 10,500), a normal adrenal axis, and negative testing for HIV. Owing to concern for cancer, positron-emission tomography-CT of the whole body was performed and showed an adrenal mass on each side with fluorodeoxyglucose (FDG) uptake (left image shows a coronal view), a 17-cm-long spleen without FDG uptake (asterisk in left image), and no other abnormal findings. Subsequent biopsy of the left adrenal mass showed necrotizing granulomatous inflammation with intracellular fungal organisms (middle, arrows; hematoxylin and eosin staining) that stained positive with Grocott’s methenamine
silver (right). The organism causing this disease is predominantly found in what geographic regions?
\vspace{12pt}

\textbf{Options:}
\begin{enumerate}
\item[A.] Central and mideastern United States \& Central America
\item[B.] Central and South America
\item[C.] Southeast Asia, southern China, \& eastern India
\item[D.] Southwestern United States
\item[E.] Ubiquitously in soil throughout the globe
\end{enumerate}

\textbf{Image:}
\begin{center}
\includegraphics[width=0.95\textwidth,height=0.50\textheight,width=0.90\textwidth,keepaspectratio]{images/nejm_20240314.jpg}
\end{center}
\vspace{12pt}
\newpage

\section*{Question 131 (ID: 20240321)}
\textbf{Date: }March 21,2024
\vspace{6pt}

A healthy 14-year-old boy presented with a 6-hour history of hoarseness and difficulty swallowing after accidentally swallowing a coin. On physical examination, the patient was breathing comfortably without stridor or drooling. Radiographs of the chest (left, posteroanterior view) and neck (right, lateral view) showed a rounded radiopaque foreign object situated vertically within the subglottis. Which of the following is NOT an indication for urgent removal for this diagnosis?
\vspace{12pt}

\textbf{Options:}
\begin{enumerate}
\item[A.] Evidence of near-complete esophageal obstruction
\item[B.] Ingestion of an object less than 5cm in length
\item[C.] Ingestion of sharp object
\item[D.] Ingestion of a high-powered magnet or magnets
\item[E.] Signs of airway compromise
\end{enumerate}

\textbf{Image:}
\begin{center}
\includegraphics[width=0.95\textwidth,height=0.50\textheight,width=0.90\textwidth,keepaspectratio]{images/nejm_20240321.jpg}
\end{center}
\vspace{12pt}
\newpage

\section*{Question 132 (ID: 20240328)}
\textbf{Date: }March 28,2024
\vspace{6pt}

A 55-year-old woman with a 29-pack-year history of smoking presented with a 1.5-year history of pain in her fingers, wrists, hips, knees, and ankles. On physical examination, clubbing and slight thickening of the skin of the fingers and toes were noted (left image, right foot). There was mild tenderness on palpation of all her affected joints but no erythema or swelling. A cardiopulmonary examination was normal. Radiographs of the forearms, hands, femurs, tibia, and feet showed symmetric periostitis of the tubular bones (right image, right ulna and radius, arrows). Which of the following is the most appropriate next test in her evaluation?
\vspace{12pt}

\textbf{Options:}
\begin{enumerate}
\item[B.] Anti-CCP and rheumatoid factor
\item[C.] Bone scan
\item[D.] Calcium, phosphorus, and PTH levels
\item[E.] Chest imaging
\end{enumerate}

\textbf{Image:}
\begin{center}
\includegraphics[width=0.95\textwidth,height=0.50\textheight,width=0.90\textwidth,keepaspectratio]{images/nejm_20240328.jpg}
\end{center}
\vspace{12pt}
\newpage

\section*{Question 133 (ID: 20240404)}
\textbf{Date: }April 04,2024
\vspace{6pt}

A 35-year-old man presented to the dermatology clinic with a 4-day history of anxiety, insomnia, and resting tremor of the hands and feet. One month before presentation, the patient had begun taking multidrug therapy for leprosy. On physical examination, bluish discoloration of the lips and tongue was observed. What drug is the most likely cause of this presentation?
\vspace{12pt}

\textbf{Options:}
\begin{enumerate}
\item[A.] Clofazimine
\item[B.] Dapsone
\item[C.] Rifampin
\item[D.] Thalidomide
\item[E.] Salicylates
\end{enumerate}

\textbf{Image:}
\begin{center}
\includegraphics[width=0.95\textwidth,height=0.50\textheight,width=0.90\textwidth,keepaspectratio]{images/nejm_20240404.jpg}
\end{center}
\vspace{12pt}
\newpage

\section*{Question 134 (ID: 20240411)}
\textbf{Date: }April 11,2024
\vspace{6pt}

A 35-year-old man presented to his primary care physician for evaluation of previously unknown asymptomatic hypertension. The blood pressure was 146/89 mm Hg in the left arm, 146/99 mm Hg in the right arm, 104/83 mm Hg in the left leg, and 109/90 mm Hg in the right leg. Radiograph of the chest is shown. Which of the following findings would not be typical of this condition?
\vspace{12pt}

\textbf{Options:}
\begin{enumerate}
\item[A.] Collateral arterial circulation
\item[B.] Continuous murmur
\item[C.] Left ventricular hypertrophy
\item[D.] Radial-femoral delay
\item[E.] Rib notching
\end{enumerate}

\textbf{Image:}
\begin{center}
\includegraphics[width=0.95\textwidth,height=0.50\textheight,width=0.90\textwidth,keepaspectratio]{images/nejm_20240411.jpg}
\end{center}
\vspace{12pt}
\newpage

\section*{Question 135 (ID: 20240418)}
\textbf{Date: }April 18,2024
\vspace{6pt}

A 57-year-old woman presented to the emergency department with a 3-day history of shortness of breath and dizziness. The physical examination was notable for pallor. Laboratory studies showed a hemoglobin of 4.4 g per deciliter (reference range, 11.6 to 15.5), an elevated reticulocyte count, an elevated lactate dehydrogenase level, and a low haptoglobin level. The results of hemoglobin electrophoresis and glucose-6-phosphate dehydrogenase testing were normal, and methemoglobin and direct antiglobulin tests were negative. A peripheral blood smear (left, Giemsa staining) showed poikilocytosis, nucleated red cells (black arrows), and polychromatic cells (white arrows). The peripheral blood smear also showed bite cells (left, red arrows), blister cells (left, asterisks), and erythrocyte inclusions (middle, Giemsa staining). The erythrocyte inclusions were identified as Heinz bodies on the basis of positive staining with methyl violet (right). Which of the following is the most likely etiology of this patient’s hemolytic anemia?
\vspace{12pt}

\textbf{Options:}
\begin{enumerate}
\item[A.] Immune-mediated
\item[B.] Infection
\item[C.] Intrinsic membrane defect
\item[D.] Oxidative Injury
\item[E.] Thrombotic microangiopathy
\end{enumerate}

\textbf{Image:}
\begin{center}
\includegraphics[width=0.95\textwidth,height=0.50\textheight,width=0.90\textwidth,keepaspectratio]{images/nejm_20240418.jpg}
\end{center}
\vspace{12pt}
\newpage

\section*{Question 136 (ID: 20240425)}
\textbf{Date: }April 25,2024
\vspace{6pt}

A 63-year-old woman who had presented to the ophthalmology clinic for evaluation of cataracts was found to have white-yellow rings in both eyes. Over the past few years, she had noticed mild worsening of her vision. She reported no history of keratitis or ocular trauma. A lipid panel had been normal at a health maintenance visit 6 weeks before presentation. On ophthalmologic examination, peripheral opacities were observed in the lens of each eye, a finding consistent with age-related cataracts. A funduscopic examination was normal. Visual acuity was 20/30 in each eye. On slit-lamp examination, two concentric white-yellow rings were seen in each cornea. No corneal thinning or inflammation was apparent. Deposition of what substance is responsible for this finding?
\vspace{12pt}

\textbf{Options:}
\begin{enumerate}
\item[A.] Calcium deposition
\item[B.] Copper deposition
\item[C.] Crystal deposition
\item[D.] Iron deposition
\item[E.] Lipid deposition
\end{enumerate}

\textbf{Image:}
\begin{center}
\includegraphics[width=0.95\textwidth,height=0.50\textheight,width=0.90\textwidth,keepaspectratio]{images/nejm_20240425.jpg}
\end{center}
\vspace{12pt}
\newpage

\section*{Question 137 (ID: 20240502)}
\textbf{Date: }May 02,2024
\vspace{6pt}

A 62-year-old man presented to the hospital with a 1-month history of muscle aches and weakness in the anterior thighs and the lower posterior aspect of both legs and weight loss of 10 kg. On physical examination, there was numbness of the anterior thighs and posterior lower legs but no skin changes or abdominal tenderness. Laboratory tests showed elevated levels of inflammatory markers. Findings on computed tomography of the chest, abdomen, and pelvis were unremarkable. Tests for antineutrophil cytoplasmic antibodies were negative. Abdominal angiography was performed, shown in the left and right images. What other condition is associated with this patient’s diagnosis?
\vspace{12pt}

\textbf{Options:}
\begin{enumerate}
\item[A.] Chronic lymphocytic leukemia
\item[B.] Hepatitis B
\item[D.] Systemic lupus erythematosus
\item[E.] Tuberculosis
\end{enumerate}

\textbf{Image:}
\begin{center}
\includegraphics[width=0.95\textwidth,height=0.50\textheight,width=0.90\textwidth,keepaspectratio]{images/nejm_20240502.jpg}
\end{center}
\vspace{12pt}
\newpage

\section*{Question 138 (ID: 20240509)}
\textbf{Date: }May 09,2024
\vspace{6pt}

A 96-year-old woman presented to the emergency department with a 1-day history of pleuritic chest pain 4 days after a single-chamber transvenous pacemaker had been implanted. A chest radiograph and computed tomographic scan of the chest showed the tip of the right ventricular lead in the left pleural space. What rhythm does the electrocardiogram show?
\vspace{12pt}

\textbf{Options:}
\begin{enumerate}
\item[A.] Atrial fibrillation
\item[B.] Complete heart block
\item[C.] Second degree heart block
\item[D.] Ventricular-paced rhythm
\item[E.] Wandering atrial pacemaker
\end{enumerate}

\textbf{Image:}
\begin{center}
\includegraphics[width=0.89\textwidth,height=0.50\textheight,width=0.90\textwidth,keepaspectratio]{images/nejm_20240509.jpg}
\end{center}
\vspace{12pt}
\newpage

\section*{Question 139 (ID: 20240516)}
\textbf{Date: }May 16,2024
\vspace{6pt}

A 79-year-old woman presented to the emergency department with a 3-day history of hematemesis, melena, and epigastric pain. Approximately 1 day before the onset of symptoms, nonbloody vomiting had developed after she had eaten unrefrigerated food. She was taking aspirin daily for coronary artery disease. Her heart rate was 102 beats per minute, and her blood pressure was 89/66 mmHg. Physical examination was notable for pallor and diaphoresis. Her hemoglobin level was 9.1 g per deciliter (reference range, 11 to 15), a decrease from a baseline level of 12 g per deciliter 6 months previously. Computed tomography of the chest was performed to evaluate for esophageal rupture and showed only a mucosal tear and soft-tissue swelling at the gastroesophageal junction. An upper endoscopy was performed. A diagnosis of Mallory-Weiss syndrome was made. Which of the following would be the LEAST appropriate next step in management for this diagnosis?
\vspace{12pt}

\textbf{Options:}
\begin{enumerate}
\item[A.] Acid suppression
\item[B.] Anti-emetics
\item[C.] Endoscopic therapy for bleeding
\item[D.] Fluid resuscitation
\item[E.] Tranexamic acid
\end{enumerate}

\textbf{Image:}
\begin{center}
\includegraphics[width=0.95\textwidth,height=0.50\textheight,width=0.90\textwidth,keepaspectratio]{images/nejm_20240516.jpg}
\end{center}
\vspace{12pt}
\newpage

\section*{Question 140 (ID: 20240523)}
\textbf{Date: }May 23,2024
\vspace{6pt}

A 70-year-old woman with diabetes mellitus presented to the emergency department with a 3-day history of epigastric pain and vomiting. An electrocardiogram showed normal sinus rhythm with pathologic Q waves and T-wave inversions in the anterior leads. A high-sensitivity troponin test showed a troponin level of 686 ng per liter (reference value, <11). What finding is shown in these images from the patient’s left ventriculography and transthoracic echocardiography?
\vspace{12pt}

\textbf{Options:}
\begin{enumerate}
\item[A.] Mural thrombus
\item[B.] Rhabdomyoma
\item[C.] Ruptured papillary muscle
\item[D.] Sarcoma
\item[E.] Vegetation
\end{enumerate}

\textbf{Image:}
\begin{center}
\includegraphics[width=0.95\textwidth,height=0.50\textheight,width=0.90\textwidth,keepaspectratio]{images/nejm_20240523.jpg}
\end{center}
\vspace{12pt}
\newpage

\section*{Question 141 (ID: 20240530)}
\textbf{Date: }May 30,2024
\vspace{6pt}

A 3-year-old boy who had been born prematurely was brought to the emergency department with a 1-day history of intermittent abdominal pain, nausea, and vomiting. He had had no bloody stools or contact with persons known to be sick. His vital signs were normal. An abdominal examination was notable for hypoactive bowel sounds and tenderness in the right lower quadrant. Ultrasonography of the right lower quadrant is shown in the left image. A longitudinal view of the same lesion region is shown in the right image. What is this patient’s diagnosis?
\vspace{12pt}

\textbf{Options:}
\begin{enumerate}
\item[A.] Acute gastroenteritis
\item[B.] Appendicitis
\item[C.] Enteric duplication cyst
\item[D.] Ileoileal intussusception
\item[E.] Meckel’s diverticulum
\end{enumerate}

\textbf{Image:}
\begin{center}
\includegraphics[width=0.95\textwidth,height=0.50\textheight,width=0.90\textwidth,keepaspectratio]{images/nejm_20240530.jpg}
\end{center}
\vspace{12pt}
\newpage

\section*{Question 142 (ID: 20240606)}
\textbf{Date: }June 06,2024
\vspace{6pt}

A 59-year-old veterinarian presented to the dermatology clinic with a 1-year history of a painful rash on his right hand (left) and right index finger (middle). He had no other symptoms. Biopsy of the lesion on the dorsum of the hand revealed pseudoepitheliomatous hyperplasia and granulomas in the dermis (right). Metagenomic next-generation sequencing of the tissue identified Mycobacterium tuberculosis, and a diagnosis of tuberculosis verrucosa cutis was made. Which of the following tests is most likely to support this diagnosis?
\vspace{12pt}

\textbf{Options:}
\begin{enumerate}
\item[A.] HPV genotyping
\item[B.] Interferon-γ release assay
\item[C.] KOH staining
\item[D.] Rapid plasma reagin
\item[E.] Tissue culture
\end{enumerate}

\textbf{Image:}
\begin{center}
\includegraphics[width=0.95\textwidth,height=0.50\textheight,width=0.90\textwidth,keepaspectratio]{images/nejm_20240606.jpg}
\end{center}
\vspace{12pt}
\newpage

\section*{Question 143 (ID: 20240613)}
\textbf{Date: }June 13,2024
\vspace{6pt}

A 65-year-old man with type 2 diabetes was admitted to the hospital with hyperosmolar hyperglycemia state. Two weeks before admission, the patient’s insulin dose had been increased owing to inadequate glycemic control. The physical exam was notable for confusion and sarcopenia. There were also rubbery, subcutaneous masses on either side of the umbilicus where the patient had been repeatedly administering insulin injections. Which of the following should be done to prevent this finding?
\vspace{12pt}

\textbf{Options:}
\begin{enumerate}
\item[A.] Cool site before injecting
\item[B.] Discontinue the use of irritating cleansing solutions at injection sites
\item[C.] Massage the injection site after injecting
\item[D.] Rotating insulin injection sites
\item[E.] Use smaller needles
\end{enumerate}

\textbf{Image:}
\begin{center}
\includegraphics[width=0.95\textwidth,height=0.50\textheight,width=0.90\textwidth,keepaspectratio]{images/nejm_20240613.jpg}
\end{center}
\vspace{12pt}
\newpage

\section*{Question 144 (ID: 20240620)}
\textbf{Date: }June 20,2024
\vspace{6pt}

A 78-year-old man presented to primary care clinic with a 5-day history of constipation and flank bulge. On physical examination, scabbed lesions were seen along the T12 dermatome. There was also an outpouching of the left lower abdominal wall without any underlying masses or fascial defects. Computed tomography of the abdomen was notable only for a protrusion of the left lower abdominal wall. What is the diagnosis?
\vspace{12pt}

\textbf{Options:}
\begin{enumerate}
\item[A.] Abdominal wall hernia
\item[B.] Diastasis recti
\item[C.] Lower thoracic disc herniation
\item[D.] Postherpetic abdominal pseudohernia
\item[E.] Seroma
\end{enumerate}

\textbf{Image:}
\begin{center}
\includegraphics[width=0.95\textwidth,height=0.50\textheight,width=0.90\textwidth,keepaspectratio]{images/nejm_20240620.jpg}
\end{center}
\vspace{12pt}
\newpage

\section*{Question 145 (ID: 20051013)}
\textbf{Date: }October 13,2005
\vspace{6pt}

What is the diagnosis?
\vspace{12pt}

\textbf{Options:}
\begin{enumerate}
\item[A.] Left facial palsy
\item[B.] Cavernous sinus thrombosis
\item[C.] Orbital lymphoma
\item[D.] Herpes zoster ophthalmicus
\item[E.] Orbtial fracture
\end{enumerate}

\textbf{Image:}
\begin{center}
\includegraphics[width=0.95\textwidth,height=0.50\textheight,width=0.90\textwidth,keepaspectratio]{images/nejm_20051013.jpg}
\end{center}
\vspace{12pt}
\newpage

\section*{Question 146 (ID: 20051020)}
\textbf{Date: }October 20,2005
\vspace{6pt}

What process is illustrated in the radiograph?
\vspace{12pt}

\textbf{Options:}
\begin{enumerate}
\item[A.] Paget's disease
\item[B.] Osteopetrosis
\item[C.] Hyperparathyroidism
\item[D.] Bone marrow hyperplasia
\item[E.] Acromegaly
\end{enumerate}

\textbf{Image:}
\begin{center}
\includegraphics[width=0.95\textwidth,height=0.50\textheight,width=0.90\textwidth,keepaspectratio]{images/nejm_20051020.jpg}
\end{center}
\vspace{12pt}
\newpage

\section*{Question 147 (ID: 20051027)}
\textbf{Date: }October 27,2005
\vspace{6pt}

The pictured patient suffered a traumatic head injury. What underlying diagnosis is suggested by the examination findings?
\vspace{12pt}

\textbf{Options:}
\begin{enumerate}
\item[A.] Skull fracture
\item[B.] Infratentorial hemorrhage
\item[C.] Intraorbital foreign body
\item[D.] Uncal herniation
\item[E.] Carotid dissection
\end{enumerate}

\textbf{Image:}
\begin{center}
\includegraphics[width=0.95\textwidth,height=0.50\textheight,width=0.90\textwidth,keepaspectratio]{images/nejm_20051027.jpg}
\end{center}
\vspace{12pt}
\newpage

\section*{Question 148 (ID: 20051103)}
\textbf{Date: }November 03,2005
\vspace{6pt}

This radiograph was taken 12 minutes after infusion of intravenous contrast. What diagnosis is suggested?
\vspace{12pt}

\textbf{Options:}
\begin{enumerate}
\item[A.] Left renal artery stenosis
\item[B.] Right staghorn calculus
\item[C.] Left hypernephroma
\item[D.] Bladder carcinoma
\item[E.] Right ureterovesical calculus
\end{enumerate}

\textbf{Image:}
\begin{center}
\includegraphics[width=0.66\textwidth,height=0.50\textheight,width=0.90\textwidth,keepaspectratio]{images/nejm_20051103.jpg}
\end{center}
\vspace{12pt}
\newpage

\section*{Question 149 (ID: 20051110)}
\textbf{Date: }November 10,2005
\vspace{6pt}

What is the most important diagnosis to exclude in this 81-year-old woman?
\vspace{12pt}

\textbf{Options:}
\begin{enumerate}
\item[A.] Lentigo maligna
\item[B.] Basal cell carcinoma
\item[C.] Varicella zoster
\item[D.] Systemic sclerosis
\item[E.] Actinic keratosis
\end{enumerate}

\textbf{Image:}
\begin{center}
\includegraphics[width=0.95\textwidth,height=0.50\textheight,width=0.90\textwidth,keepaspectratio]{images/nejm_20051110.jpg}
\end{center}
\vspace{12pt}
\newpage

\section*{Question 150 (ID: 20051117)}
\textbf{Date: }November 17,2005
\vspace{6pt}

What is the diagnosis?
\vspace{12pt}

\textbf{Options:}
\begin{enumerate}
\item[A.] Budd chiari syndrome
\item[B.] Cutaneous larva migrans
\item[C.] Umbilical hernia
\item[D.] Portal hypertension
\item[E.] Gastric carcinoma
\end{enumerate}

\textbf{Image:}
\begin{center}
\includegraphics[width=0.95\textwidth,height=0.50\textheight,width=0.90\textwidth,keepaspectratio]{images/nejm_20051117.jpg}
\end{center}
\vspace{12pt}
\newpage

\section*{Question 151 (ID: 20051124)}
\textbf{Date: }November 24,2005
\vspace{6pt}

This 61-year-old man is receiving care for an arrhythmia. What is the cause of his appearance?
\vspace{12pt}

\textbf{Options:}
\begin{enumerate}
\item[A.] Procainemide
\item[B.] Bretylium
\item[C.] Amiodarone
\item[D.] Sotalol
\item[E.] Hydralazine
\end{enumerate}

\textbf{Image:}
\begin{center}
\includegraphics[width=0.6\textwidth,height=0.50\textheight,width=0.90\textwidth,keepaspectratio]{images/nejm_20051124.jpg}
\end{center}
\vspace{12pt}
\newpage

\section*{Question 152 (ID: 20051201)}
\textbf{Date: }December 01,2005
\vspace{6pt}

What is the diagnosis?
\vspace{12pt}

\textbf{Options:}
\begin{enumerate}
\item[B.] Rheumatoid arthritis
\item[C.] Gonococcal arthritis
\item[D.] Leprosy
\item[E.] Psoriasis
\end{enumerate}

\textbf{Image:}
\begin{center}
\includegraphics[width=0.95\textwidth,height=0.50\textheight,width=0.90\textwidth,keepaspectratio]{images/nejm_20051201.jpg}
\end{center}
\vspace{12pt}
\newpage

\section*{Question 153 (ID: 20051208)}
\textbf{Date: }December 08,2005
\vspace{6pt}

This urine sample was photographed under light microscopy. What type of crystal is illustrated?
\vspace{12pt}

\textbf{Options:}
\begin{enumerate}
\item[B.] Oxalate
\item[C.] Cysteine
\item[D.] Ethylene glycol
\item[E.] Triphosphate
\end{enumerate}

\textbf{Image:}
\begin{center}
\includegraphics[width=0.95\textwidth,height=0.50\textheight,width=0.90\textwidth,keepaspectratio]{images/nejm_20051208.jpg}
\end{center}
\vspace{12pt}
\newpage

\section*{Question 154 (ID: 20051215)}
\textbf{Date: }December 15,2005
\vspace{6pt}

What is the diagnosis?
\vspace{12pt}

\textbf{Options:}
\begin{enumerate}
\item[A.] Pterygium
\item[B.] Retroorbital hematoma
\item[C.] Retained suture
\item[D.] Loa loa
\item[E.] Toxocariasis
\end{enumerate}

\textbf{Image:}
\begin{center}
\includegraphics[width=0.95\textwidth,height=0.50\textheight,width=0.90\textwidth,keepaspectratio]{images/nejm_20051215.jpg}
\end{center}
\vspace{12pt}
\newpage

\section*{Question 155 (ID: 20051222)}
\textbf{Date: }December 22,2005
\vspace{6pt}

What endocrinopathy is most frequently associated with this sign?
\vspace{12pt}

\textbf{Options:}
\begin{enumerate}
\item[A.] Addison's disease
\item[B.] Insulin resistance
\item[C.] Growth hormone excess
\item[D.] Glucagonoma
\item[E.] Diabetes insipidus
\end{enumerate}

\textbf{Image:}
\begin{center}
\includegraphics[width=0.95\textwidth,height=0.50\textheight,width=0.90\textwidth,keepaspectratio]{images/nejm_20051222.jpg}
\end{center}
\vspace{12pt}
\newpage

\section*{Question 156 (ID: 20051229)}
\textbf{Date: }December 29,2005
\vspace{6pt}

What is the diagnosis?
\vspace{12pt}

\textbf{Options:}
\begin{enumerate}
\item[A.] Aspergillosis
\item[B.] Adrenal insufficiency
\item[C.] Oral leukoplakia
\item[D.] Pellagra
\item[E.] Lingua villosa nigra
\end{enumerate}

\textbf{Image:}
\begin{center}
\includegraphics[width=0.62\textwidth,height=0.50\textheight,width=0.90\textwidth,keepaspectratio]{images/nejm_20051229.jpg}
\end{center}
\vspace{12pt}
\newpage

\section*{Question 157 (ID: 20060105)}
\textbf{Date: }January 05,2006
\vspace{6pt}

A 37-year-old woman with idiopathic T-cell deficiency underwent computed tomography of the abdomen. What diagnosis is most likely to account for the findings?
\vspace{12pt}

\textbf{Options:}
\begin{enumerate}
\item[A.] Histoplasmosis
\item[B.] Miliary tuberculosis
\item[C.] Amyloidosis
\item[D.] Lymphoma
\item[E.] Portal hypertension
\end{enumerate}

\textbf{Image:}
\begin{center}
\includegraphics[width=0.95\textwidth,height=0.50\textheight,width=0.90\textwidth,keepaspectratio]{images/nejm_20060105.jpg}
\end{center}
\vspace{12pt}
\newpage

\section*{Question 158 (ID: 20060112)}
\textbf{Date: }January 12,2006
\vspace{6pt}

What is most likely to account for the findings on this abdominal radiograph?
\vspace{12pt}

\textbf{Options:}
\begin{enumerate}
\item[A.] Schistosomiasis
\item[B.] Chronic laxative use
\item[C.] Hyperparathyroidism
\item[D.] Ischemic colitis
\item[E.] Ingestion of a heavy metal
\end{enumerate}

\textbf{Image:}
\begin{center}
\includegraphics[width=0.95\textwidth,height=0.50\textheight,width=0.90\textwidth,keepaspectratio]{images/nejm_20060112.jpg}
\end{center}
\vspace{12pt}
\newpage

\section*{Question 159 (ID: 20060119)}
\textbf{Date: }January 19,2006
\vspace{6pt}

What is the most likely etiology of the findings on this computed tomogram of the chest?
\vspace{12pt}

\textbf{Options:}
\begin{enumerate}
\item[A.] Aspergillus fumigatus
\item[B.] Bronchiolitis obliterans
\item[C.] Amiodarone
\item[D.] Tobacco
\item[E.] Pneumoconiosis
\end{enumerate}

\textbf{Image:}
\begin{center}
\includegraphics[width=0.95\textwidth,height=0.50\textheight,width=0.90\textwidth,keepaspectratio]{images/nejm_20060119.jpg}
\end{center}
\vspace{12pt}
\newpage

\section*{Question 160 (ID: 20060126)}
\textbf{Date: }January 26,2006
\vspace{6pt}

This 6-year-old boy presented with fever and rash that did not improve despite treatment with cephalexin. What diagnosis is suggested?
\vspace{12pt}

\textbf{Options:}
\begin{enumerate}
\item[A.] Bullous pemphigoid
\item[B.] Staphylococcal scalded skin
\item[C.] Stevens-Johnson syndrome
\item[D.] Herpes simplex infection
\item[E.] Kawasaki disease
\end{enumerate}

\textbf{Image:}
\begin{center}
\includegraphics[width=0.95\textwidth,height=0.50\textheight,width=0.90\textwidth,keepaspectratio]{images/nejm_20060126.jpg}
\end{center}
\vspace{12pt}
\newpage

\section*{Question 161 (ID: 20060202)}
\textbf{Date: }February 02,2006
\vspace{6pt}

This patient underwent computed tomography of the pelvis having presented with lower extremity edema. What diagnosis is suggested?
\vspace{12pt}

\textbf{Options:}
\begin{enumerate}
\item[A.] Uterine fibroid
\item[B.] Ovarian cyst
\item[C.] Urinary retention
\item[D.] Teratoma
\item[E.] Aortic aneurysm
\end{enumerate}

\textbf{Image:}
\begin{center}
\includegraphics[width=0.95\textwidth,height=0.50\textheight,width=0.90\textwidth,keepaspectratio]{images/nejm_20060202.jpg}
\end{center}
\vspace{12pt}
\newpage

\section*{Question 162 (ID: 20060209)}
\textbf{Date: }February 09,2006
\vspace{6pt}

This 16-year-old boy presented with malaise and fever. What is the diagnosis?
\vspace{12pt}

\textbf{Options:}
\begin{enumerate}
\item[A.] Scarlet fever
\item[B.] Infectious mononucleosis
\item[C.] Measles
\item[E.] Coxsackie viral infection
\end{enumerate}

\textbf{Image:}
\begin{center}
\includegraphics[width=0.95\textwidth,height=0.50\textheight,width=0.90\textwidth,keepaspectratio]{images/nejm_20060209.jpg}
\end{center}
\vspace{12pt}
\newpage

\section*{Question 163 (ID: 20060216)}
\textbf{Date: }February 16,2006
\vspace{6pt}

This 75-year-old man presented with cough and hoarseness. What is the most likely cause of his appearance?
\vspace{12pt}

\textbf{Options:}
\begin{enumerate}
\item[A.] Superior vena cava syndrome
\item[B.] Mitral stenosis
\item[C.] Chronic obstructive airways disease
\item[D.] Angioedema
\item[E.] Systemic lupus erythematosus
\end{enumerate}

\textbf{Image:}
\begin{center}
\includegraphics[width=0.59\textwidth,height=0.50\textheight,width=0.90\textwidth,keepaspectratio]{images/nejm_20060216.jpg}
\end{center}
\vspace{12pt}
\newpage

\section*{Question 164 (ID: 20060223)}
\textbf{Date: }February 23,2006
\vspace{6pt}

What nebulized medication is most likely to have caused this patient's anisocoria?
\vspace{12pt}

\textbf{Options:}
\begin{enumerate}
\item[A.] Racemic epinephrine
\item[B.] Ipatropium
\item[C.] Ribavirin
\item[D.] Salbutamol
\item[E.] Theophylline
\end{enumerate}

\textbf{Image:}
\begin{center}
\includegraphics[width=0.95\textwidth,height=0.50\textheight,width=0.90\textwidth,keepaspectratio]{images/nejm_20060223.jpg}
\end{center}
\vspace{12pt}
\newpage

\section*{Question 165 (ID: 20060302)}
\textbf{Date: }March 02,2006
\vspace{6pt}

A 31-year-old woman presented with fever following a trip to Brazil. What diagnosis is suggested by the findings on her blood smear?
\vspace{12pt}

\textbf{Options:}
\begin{enumerate}
\item[A.] Paracoccidioides brasiliensis
\item[B.] Trypanozoma cruzi
\item[C.] Rickettsia typhi
\item[D.] Plasmodium vivax
\item[E.] Leishmania donovani
\end{enumerate}

\textbf{Image:}
\begin{center}
\includegraphics[width=0.95\textwidth,height=0.50\textheight,width=0.90\textwidth,keepaspectratio]{images/nejm_20060302.jpg}
\end{center}
\vspace{12pt}
\newpage

\section*{Question 166 (ID: 20060309)}
\textbf{Date: }March 09,2006
\vspace{6pt}

What diagnosis explains this 15-year-old patient's chest pain?
\vspace{12pt}

\textbf{Options:}
\begin{enumerate}
\item[A.] Aortic coarctation
\item[B.] Metastatic carcinoma
\item[C.] Pneumomediastinum
\item[D.] Acute severe asthma
\item[E.] Pulmonary hypertension
\end{enumerate}

\textbf{Image:}
\begin{center}
\includegraphics[width=0.67\textwidth,height=0.50\textheight,width=0.90\textwidth,keepaspectratio]{images/nejm_20060309.jpg}
\end{center}
\vspace{12pt}
\newpage

\section*{Question 167 (ID: 20060316)}
\textbf{Date: }March 16,2006
\vspace{6pt}

Twelve hours after urgent coronary angiography, the appearance of this patient's feet had changed. What is the most likely explanation for the finding?
\vspace{12pt}

\textbf{Options:}
\begin{enumerate}
\item[A.] Endocarditis
\item[B.] Raynaud phenomenon
\item[C.] Heparin-induced thrombocytopenia
\item[D.] Contrast allergy
\item[E.] Cholesterol emboli
\end{enumerate}

\textbf{Image:}
\begin{center}
\includegraphics[width=0.67\textwidth,height=0.50\textheight,width=0.90\textwidth,keepaspectratio]{images/nejm_20060316.jpg}
\end{center}
\vspace{12pt}
\newpage

\section*{Question 168 (ID: 20060323)}
\textbf{Date: }March 23,2006
\vspace{6pt}

A 55-year-old man underwent colonoscopy after complaining of crampy lower abdominal pain. Mobile 1-cm worms were noted in cecum. What is the most likely diagnosis?
\vspace{12pt}

\textbf{Options:}
\begin{enumerate}
\item[A.] Trichuris trichiura
\item[B.] Enterobius vermicularis
\item[C.] Ancylostoma duodenale
\item[D.] Ascaris lumbricoides
\item[E.] Necator americanus
\end{enumerate}

\textbf{Image:}
\begin{center}
\includegraphics[width=0.93\textwidth,height=0.50\textheight,width=0.90\textwidth,keepaspectratio]{images/nejm_20060323.jpg}
\end{center}
\vspace{12pt}
\newpage

\section*{Question 169 (ID: 20060330)}
\textbf{Date: }March 30,2006
\vspace{6pt}

This abnormality was found during auditory screening of a five-year-old girl at school. What is the most likely diagnosis?
\vspace{12pt}

\textbf{Options:}
\begin{enumerate}
\item[A.] Serous otitis media
\item[B.] Auditory polyp
\item[C.] Mastoiditis
\item[D.] Glomus jugalare
\item[E.] Cholesteatoma
\end{enumerate}

\textbf{Image:}
\begin{center}
\includegraphics[width=0.83\textwidth,height=0.50\textheight,width=0.90\textwidth,keepaspectratio]{images/nejm_20060330.jpg}
\end{center}
\vspace{12pt}
\newpage

\section*{Question 170 (ID: 20060406)}
\textbf{Date: }April 06,2006
\vspace{6pt}

This 36-year-old man has undergone renal transplantation and parathyroidectomy. What is the most likely cause of the changes in his hands?
\vspace{12pt}

\textbf{Options:}
\begin{enumerate}
\item[A.] Pseudoclubbing
\item[B.] Albright's hereditary osteodystrophy
\item[C.] Hypertrophic osteoarthropathy
\item[D.] Yellow nail syndrome
\item[E.] Psoriasis
\end{enumerate}

\textbf{Image:}
\begin{center}
\includegraphics[width=0.95\textwidth,height=0.50\textheight,width=0.90\textwidth,keepaspectratio]{images/nejm_20060406.jpg}
\end{center}
\vspace{12pt}
\newpage

\section*{Question 171 (ID: 20060413)}
\textbf{Date: }April 13,2006
\vspace{6pt}

This 46-year-old woman developed pruritus and similar papular lesions over her axillae, groin, and buttocks. What is the most likely diagnosis?
\vspace{12pt}

\textbf{Options:}
\begin{enumerate}
\item[A.] Pilonidal sinus
\item[B.] Psoriasis
\item[C.] Impetigo
\item[D.] Dermatomyositis
\item[E.] Scabies
\end{enumerate}

\textbf{Image:}
\begin{center}
\includegraphics[width=0.95\textwidth,height=0.50\textheight,width=0.90\textwidth,keepaspectratio]{images/nejm_20060413.jpg}
\end{center}
\vspace{12pt}
\newpage

\section*{Question 172 (ID: 20060420)}
\textbf{Date: }April 20,2006
\vspace{6pt}

This 63-year-old man with a 10-year history of progressive neck enlargement noted a sudden increase in the size of the swelling on the left side. What is the most likely diagnosis?
\vspace{12pt}

\textbf{Options:}
\begin{enumerate}
\item[A.] Superior vena cava syndrome
\item[B.] Scrofula
\item[C.] Bleeding into a thyroid nodule
\item[D.] Non-hodgkins lymphoma
\item[E.] Follicular thyroid carcinoma
\end{enumerate}

\textbf{Image:}
\begin{center}
\includegraphics[width=0.57\textwidth,height=0.50\textheight,width=0.90\textwidth,keepaspectratio]{images/nejm_20060420.jpg}
\end{center}
\vspace{12pt}
\newpage

\section*{Question 173 (ID: 20060427)}
\textbf{Date: }April 27,2006
\vspace{6pt}

A 39-year-old Zambian man who has tested positive for the human immunodeficiency virus presented with a three-week history of a draining neck mass. What is the most likely diagnosis?
\vspace{12pt}

\textbf{Options:}
\begin{enumerate}
\item[A.] Actinomycosis
\item[B.] Scrofula
\item[C.] Kaposi sarcoma
\item[D.] Burkitt lymphoma
\item[E.] Nocardiosis
\end{enumerate}

\textbf{Image:}
\begin{center}
\includegraphics[width=0.95\textwidth,height=0.50\textheight,width=0.90\textwidth,keepaspectratio]{images/nejm_20060427.jpg}
\end{center}
\vspace{12pt}
\newpage

\section*{Question 174 (ID: 20060504)}
\textbf{Date: }May 04,2006
\vspace{6pt}

What is the most likely cause of the abnormality on the chest radiograph?
\vspace{12pt}

\textbf{Options:}
\begin{enumerate}
\item[A.] Aspiration pneumonia
\item[B.] Pneumocystis jiroveci pneumonia
\item[C.] Acute respiratory distress syndrome
\item[D.] Hemothorax
\item[E.] Reexapansion pulmonary edema
\end{enumerate}

\textbf{Image:}
\begin{center}
\includegraphics[width=0.79\textwidth,height=0.50\textheight,width=0.90\textwidth,keepaspectratio]{images/nejm_20060504.jpg}
\end{center}
\vspace{12pt}
\newpage

\section*{Question 175 (ID: 20060511)}
\textbf{Date: }May 11,2006
\vspace{6pt}

This 61-year-old man presented with abdominal pain. Basophilic stippling was evident on a blood smear. What is the most likely diagnosis?
\vspace{12pt}

\textbf{Options:}
\begin{enumerate}
\item[A.] Acute myelogenous leukemia
\item[B.] Chronic lead poisoning
\item[C.] Beta-thallasemia
\item[D.] Megaloblastic anemia
\item[E.] Sickle cell anemia
\end{enumerate}

\textbf{Image:}
\begin{center}
\includegraphics[width=0.95\textwidth,height=0.50\textheight,width=0.90\textwidth,keepaspectratio]{images/nejm_20060511.jpg}
\end{center}
\vspace{12pt}
\newpage

\section*{Question 176 (ID: 20060518)}
\textbf{Date: }May 18,2006
\vspace{6pt}

What diagnosis explains the combination of findings on this lateral chest radiograph?
\vspace{12pt}

\textbf{Options:}
\begin{enumerate}
\item[A.] Syphilis
\item[B.] Dressler's syndrome
\item[C.] Turner's syndrome
\item[D.] Rheumatic heart disease
\item[E.] Tertiary hyperparathyroidism
\end{enumerate}

\textbf{Image:}
\begin{center}
\includegraphics[width=0.95\textwidth,height=0.50\textheight,width=0.90\textwidth,keepaspectratio]{images/nejm_20060518.jpg}
\end{center}
\vspace{12pt}
\newpage

\section*{Question 177 (ID: 20060525)}
\textbf{Date: }May 25,2006
\vspace{6pt}

This 9-year-old boy presented with a two-day history of right shoulder pain after an upper respiratory tract infection. What is the cause of the abnormality demonstrated?
\vspace{12pt}

\textbf{Options:}
\begin{enumerate}
\item[A.] Palsy of the long thoracic nerve
\item[B.] Subcutaneous emphysema
\item[C.] Neuralgic amyotrophy
\item[D.] Pleural prolapse
\item[E.] Scapular subluxation
\end{enumerate}

\textbf{Image:}
\begin{center}
\includegraphics[width=0.69\textwidth,height=0.50\textheight,width=0.90\textwidth,keepaspectratio]{images/nejm_20060525.jpg}
\end{center}
\vspace{12pt}
\newpage

\section*{Question 178 (ID: 20060601)}
\textbf{Date: }June 01,2006
\vspace{6pt}

What diagnosis is suggested by this radiograph taken following a barium swallow?
\vspace{12pt}

\textbf{Options:}
\begin{enumerate}
\item[A.] Achalasia
\item[B.] Esophageal carcinoma
\item[C.] Zenker's diverticulum
\item[D.] Esophageal stricture
\item[E.] Diffuse esophageal spasm
\end{enumerate}

\textbf{Image:}
\begin{center}
\includegraphics[width=0.37\textwidth,height=0.50\textheight,width=0.90\textwidth,keepaspectratio]{images/nejm_20060601.jpg}
\end{center}
\vspace{12pt}
\newpage

\section*{Question 179 (ID: 20060608)}
\textbf{Date: }June 08,2006
\vspace{6pt}

A 70-year-old man presented with weight loss and hemoptysis. Multiple painless cutaneous nodules had developed over several weeks. What is the most likely diagnosis?
\vspace{12pt}

\textbf{Options:}
\begin{enumerate}
\item[A.] Neurofibromatosis
\item[B.] Acute myelogenous leukemia
\item[C.] Disseminated leishmaniasis
\item[D.] Acquired immunodeficiency syndrome
\item[E.] Small-cell lung cancer
\end{enumerate}

\textbf{Image:}
\begin{center}
\includegraphics[width=0.95\textwidth,height=0.50\textheight,width=0.90\textwidth,keepaspectratio]{images/nejm_20060608.jpg}
\end{center}
\vspace{12pt}
\newpage

\section*{Question 180 (ID: 20060615)}
\textbf{Date: }June 15,2006
\vspace{6pt}

This 68-year-old woman presented with hair growth and a sore furrowed tongue. What is the most likely diagnosis?
\vspace{12pt}

\textbf{Options:}
\begin{enumerate}
\item[A.] Cushing's syndrome
\item[B.] Prophyria cutanea tarda
\item[C.] Paraneoplastic disorder
\item[D.] Acromegaly
\item[E.] Polycystic ovarian syndrome
\end{enumerate}

\textbf{Image:}
\begin{center}
\includegraphics[width=0.95\textwidth,height=0.50\textheight,width=0.90\textwidth,keepaspectratio]{images/nejm_20060615.jpg}
\end{center}
\vspace{12pt}
\newpage

\section*{Question 181 (ID: 20060622)}
\textbf{Date: }June 22,2006
\vspace{6pt}

A 55-year-old kidney-transplant recipient presented with headache and fever. The cerebrospinal fluid contained 84 percent neutrophils. What is the most likely diagnosis?
\vspace{12pt}

\textbf{Options:}
\begin{enumerate}
\item[A.] Nocardia asteroides infection
\item[B.] Cerebral toxoplasmosis
\item[C.] Listeria moncytogenes infection
\item[D.] Miliary tuberculosis
\item[E.] Cryptococcus neoformans infection
\end{enumerate}

\textbf{Image:}
\begin{center}
\includegraphics[width=0.83\textwidth,height=0.50\textheight,width=0.90\textwidth,keepaspectratio]{images/nejm_20060622.jpg}
\end{center}
\vspace{12pt}
\newpage

\section*{Question 182 (ID: 20060629)}
\textbf{Date: }June 29,2006
\vspace{6pt}

What diagnosis is suggested?
\vspace{12pt}

\textbf{Options:}
\begin{enumerate}
\item[A.] Parotitis
\item[B.] Otitis externa
\item[C.] Herpes zoster
\item[D.] Frey's syndrome
\item[E.] Parotid adenoma
\end{enumerate}

\textbf{Image:}
\begin{center}
\includegraphics[width=0.63\textwidth,height=0.50\textheight,width=0.90\textwidth,keepaspectratio]{images/nejm_20060629.jpg}
\end{center}
\vspace{12pt}
\newpage

\section*{Question 183 (ID: 20060706)}
\textbf{Date: }July 06,2006
\vspace{6pt}

This patient presented with an S4 gallop, an elevated jugular venous pressure, and bilateral pitting edema. What diagnosis is most likely?
\vspace{12pt}

\textbf{Options:}
\begin{enumerate}
\item[A.] Congestive heart failure
\item[B.] Constrictive pericarditis
\item[C.] Subacute bacterial endocarditis
\item[D.] Systemic amyloidosis
\item[E.] Scleroderma
\end{enumerate}

\textbf{Image:}
\begin{center}
\includegraphics[width=0.77\textwidth,height=0.50\textheight,width=0.90\textwidth,keepaspectratio]{images/nejm_20060706.jpg}
\end{center}
\vspace{12pt}
\newpage

\section*{Question 184 (ID: 20060713)}
\textbf{Date: }July 13,2006
\vspace{6pt}

Treatment with which antihypertensive is most likely to cause this appearance?
\vspace{12pt}

\textbf{Options:}
\begin{enumerate}
\item[A.] Beta-blocker
\item[B.] Diuretic
\item[C.] Alpha-blocker
\item[D.] Angiotensin converting-enzyme inhibitor
\item[E.] Calcium-channel blocker
\end{enumerate}

\textbf{Image:}
\begin{center}
\includegraphics[width=0.95\textwidth,height=0.50\textheight,width=0.90\textwidth,keepaspectratio]{images/nejm_20060713.jpg}
\end{center}
\vspace{12pt}
\newpage

\section*{Question 185 (ID: 20060720)}
\textbf{Date: }July 20,2006
\vspace{6pt}

This appearance most typically follows treatment with which medication?
\vspace{12pt}

\textbf{Options:}
\begin{enumerate}
\item[A.] Atypical antipsychotic
\item[B.] Protease inhibitor
\item[C.] Glucocorticoid
\item[D.] Thiazolidinedione
\item[E.] Barbiturate
\end{enumerate}

\textbf{Image:}
\begin{center}
\includegraphics[width=0.56\textwidth,height=0.50\textheight,width=0.90\textwidth,keepaspectratio]{images/nejm_20060720.jpg}
\end{center}
\vspace{12pt}
\newpage

\section*{Question 186 (ID: 20060727)}
\textbf{Date: }July 27,2006
\vspace{6pt}

The pupils of this female smoker were unresponsive to light and accommodation. She had noted weight loss and postural hypotension and was found to have an irregular centimeter-sized nodule in the right middle lobe. A test for which one of the following would most likely lead to the diagnosis?
\vspace{12pt}

\textbf{Options:}
\begin{enumerate}
\item[A.] Calcium-receptor antibody
\item[B.] Parathyroid hormone related protein
\item[C.] Serum ceruloplasmin
\item[D.] Treponema pallidum haemagglutination assay
\item[E.] Anti-Hu antibody
\end{enumerate}

\textbf{Image:}
\begin{center}
\includegraphics[width=0.95\textwidth,height=0.50\textheight,width=0.90\textwidth,keepaspectratio]{images/nejm_20060727.jpg}
\end{center}
\vspace{12pt}
\newpage

\section*{Question 187 (ID: 20060803)}
\textbf{Date: }August 03,2006
\vspace{6pt}

What is the diagnosis?
\vspace{12pt}

\textbf{Options:}
\begin{enumerate}
\item[A.] Elephantiasis
\item[B.] Hydrocele
\item[C.] Inguinal hernia
\item[D.] Non-seminomatous germ-cell tumor
\item[E.] Testicular torsion
\end{enumerate}

\textbf{Image:}
\begin{center}
\includegraphics[width=0.95\textwidth,height=0.50\textheight,width=0.90\textwidth,keepaspectratio]{images/nejm_20060803.jpg}
\end{center}
\vspace{12pt}
\newpage

\section*{Question 188 (ID: 20060810)}
\textbf{Date: }August 10,2006
\vspace{6pt}

A 25-year-old bodybuilder presented with a tender, erythematous mass in the left upper arm. What is the most likely diagnosis?
\vspace{12pt}

\textbf{Options:}
\begin{enumerate}
\item[A.] Ruputred biceps tendon
\item[B.] Abscess
\item[C.] Axillary vein thrombosis
\item[D.] Glenohumeral joint dislocation
\item[E.] Osteosarcoma
\end{enumerate}

\textbf{Image:}
\begin{center}
\includegraphics[width=0.64\textwidth,height=0.50\textheight,width=0.90\textwidth,keepaspectratio]{images/nejm_20060810.jpg}
\end{center}
\vspace{12pt}
\newpage

\section*{Question 189 (ID: 20060817)}
\textbf{Date: }August 17,2006
\vspace{6pt}

This patient's appearance is a consequence of what surgery?
\vspace{12pt}

\textbf{Options:}
\begin{enumerate}
\item[A.] Gastric bypass
\item[B.] Adrenalectomy
\item[C.] Liposuction
\item[D.] Pancreatectomy
\item[E.] Small bowel transplant
\end{enumerate}

\textbf{Image:}
\begin{center}
\includegraphics[width=0.82\textwidth,height=0.50\textheight,width=0.90\textwidth,keepaspectratio]{images/nejm_20060817.jpg}
\end{center}
\vspace{12pt}
\newpage

\section*{Question 190 (ID: 20060824)}
\textbf{Date: }August 24,2006
\vspace{6pt}

This lesion developed after the patient consumed raw meat from a sick goat. What is the most likely diagnosis?
\vspace{12pt}

\textbf{Options:}
\begin{enumerate}
\item[A.] Brucellosis
\item[B.] Anthrax
\item[C.] Herpes simplex
\item[E.] Pasteurella multocida
\end{enumerate}

\textbf{Image:}
\begin{center}
\includegraphics[width=0.74\textwidth,height=0.50\textheight,width=0.90\textwidth,keepaspectratio]{images/nejm_20060824.jpg}
\end{center}
\vspace{12pt}
\newpage

\section*{Question 191 (ID: 20060831)}
\textbf{Date: }August 31,2006
\vspace{6pt}

This 38-year-old woman developed recurrent right-sided chest pain synchronously with her menses. What is the most likely diagnosis?
\vspace{12pt}

\textbf{Options:}
\begin{enumerate}
\item[A.] Recurrent pulmonary emboli
\item[B.] Alpha-1 antitrypsin deficiency
\item[C.] Meig's syndrome
\item[D.] Thoracic endometriosis
\item[E.] Lymphangiomyomatosis
\end{enumerate}

\textbf{Image:}
\begin{center}
\includegraphics[width=0.7\textwidth,height=0.50\textheight,width=0.90\textwidth,keepaspectratio]{images/nejm_20060831.jpg}
\end{center}
\vspace{12pt}
\newpage

\section*{Question 192 (ID: 20060907)}
\textbf{Date: }September 07,2006
\vspace{6pt}

What is the diagnosis?
\vspace{12pt}

\textbf{Options:}
\begin{enumerate}
\item[A.] Cardiac tamponade
\item[B.] Tension hydrothorax
\item[C.] Diaphragmatic eventeration
\item[D.] Pulmonary hydatid disease
\item[E.] Lymphangiomyomatosis
\end{enumerate}

\textbf{Image:}
\begin{center}
\includegraphics[width=0.83\textwidth,height=0.50\textheight,width=0.90\textwidth,keepaspectratio]{images/nejm_20060907.jpg}
\end{center}
\vspace{12pt}
\newpage

\section*{Question 193 (ID: 20060914)}
\textbf{Date: }September 14,2006
\vspace{6pt}

What is the diagnosis?
\vspace{12pt}

\textbf{Options:}
\begin{enumerate}
\item[A.] Cytomegalovirus retinitis
\item[B.] Roth spots
\item[C.] Central retinal vein occlusion
\item[D.] Hypertensive retinopathy
\item[E.] Papilledema
\end{enumerate}

\textbf{Image:}
\begin{center}
\includegraphics[width=0.95\textwidth,height=0.50\textheight,width=0.90\textwidth,keepaspectratio]{images/nejm_20060914.jpg}
\end{center}
\vspace{12pt}
\newpage

\section*{Question 194 (ID: 20060921)}
\textbf{Date: }September 21,2006
\vspace{6pt}

This plantar lesion was associated with inguinal lymphadenopathy. What is the most likely diagnosis?
\vspace{12pt}

\textbf{Options:}
\begin{enumerate}
\item[A.] Pyogenic granuloma
\item[B.] Plantar wart
\item[C.] Malignant melanoma
\item[D.] Syphilis
\item[E.] Cat scratch disease
\end{enumerate}

\textbf{Image:}
\begin{center}
\includegraphics[width=0.95\textwidth,height=0.50\textheight,width=0.90\textwidth,keepaspectratio]{images/nejm_20060921.jpg}
\end{center}
\vspace{12pt}
\newpage

\section*{Question 195 (ID: 20060928)}
\textbf{Date: }September 28,2006
\vspace{6pt}

This patient presented with discoloration of his palms and soles. He reported a normal diet and had normal serum creatinine, thyroxine, and bilirubin concentrations. What is the diagnosis?
\vspace{12pt}

\textbf{Options:}
\begin{enumerate}
\item[A.] Diabetes mellitus
\item[B.] Vitamin B12 deficiency
\item[C.] Amiodarone exposure
\item[D.] Hyperlipidemia type III
\item[E.] Syphilis
\end{enumerate}

\textbf{Image:}
\begin{center}
\includegraphics[width=0.46\textwidth,height=0.50\textheight,width=0.90\textwidth,keepaspectratio]{images/nejm_20060928.jpg}
\end{center}
\vspace{12pt}
\newpage

\section*{Question 196 (ID: 20061005)}
\textbf{Date: }October 05,2006
\vspace{6pt}

What is the diagnosis?
\vspace{12pt}

\textbf{Options:}
\begin{enumerate}
\item[A.] Rocky mountain spotted fever
\item[B.] Hand foot and mouth disease
\item[C.] Infective endocarditis
\item[D.] Psoriasis
\item[E.] Secondary syphilis
\end{enumerate}

\textbf{Image:}
\begin{center}
\includegraphics[width=0.95\textwidth,height=0.50\textheight,width=0.90\textwidth,keepaspectratio]{images/nejm_20061005.jpg}
\end{center}
\vspace{12pt}
\newpage

\section*{Question 197 (ID: 20061012)}
\textbf{Date: }October 12,2006
\vspace{6pt}

An elderly woman presented with abdominal pain and vomiting for three days. A computed tomogram of the abdomen was obtained. What is the diagnosis?
\vspace{12pt}

\textbf{Options:}
\begin{enumerate}
\item[A.] Cecal volvulus
\item[B.] Intussusception
\item[C.] Mesenteric ischemia
\item[D.] Obturator hernia
\item[E.] Ovarian cancer
\end{enumerate}

\textbf{Image:}
\begin{center}
\includegraphics[width=0.88\textwidth,height=0.50\textheight,width=0.90\textwidth,keepaspectratio]{images/nejm_20061012.jpg}
\end{center}
\vspace{12pt}
\newpage

\section*{Question 198 (ID: 20061019)}
\textbf{Date: }October 19,2006
\vspace{6pt}

This patient presented with lower extremity edema. What is the diagnosis?
\vspace{12pt}

\textbf{Options:}
\begin{enumerate}
\item[A.] Cardiac amyloidosis
\item[B.] Mitral stenosis
\item[C.] Constrictive pericarditis
\item[D.] Left ventricular aneurysm
\item[E.] Mediastinal lymphoma
\end{enumerate}

\textbf{Image:}
\begin{center}
\includegraphics[width=0.84\textwidth,height=0.50\textheight,width=0.90\textwidth,keepaspectratio]{images/nejm_20061019.jpg}
\end{center}
\vspace{12pt}
\newpage

\section*{Question 199 (ID: 20061026)}
\textbf{Date: }October 26,2006
\vspace{6pt}

These lesions were neither pruritic nor painful. What is the diagnosis?
\vspace{12pt}

\textbf{Options:}
\begin{enumerate}
\item[A.] Pyoderma gangenosus
\item[B.] Phlegmasia cerulea dolens
\item[C.] Pretibial myxedema
\item[D.] Necrobiosis lipoidica diabeticorum
\item[E.] Erythema nodosum
\end{enumerate}

\textbf{Image:}
\begin{center}
\includegraphics[width=0.95\textwidth,height=0.50\textheight,width=0.90\textwidth,keepaspectratio]{images/nejm_20061026.jpg}
\end{center}
\vspace{12pt}
\newpage

\section*{Question 200 (ID: 20061102)}
\textbf{Date: }November 02,2006
\vspace{6pt}

What accounts for this patient's hand pain?
\vspace{12pt}

\textbf{Options:}
\begin{enumerate}
\item[A.] Ulnar artery occlusion
\item[B.] Wrist fracture
\item[C.] Radial artery thrombosis
\item[D.] Carpal tunnel syndrome
\item[E.] Reflex sympathetic dystrophy
\end{enumerate}

\textbf{Image:}
\begin{center}
\includegraphics[width=0.83\textwidth,height=0.50\textheight,width=0.90\textwidth,keepaspectratio]{images/nejm_20061102.jpg}
\end{center}
\vspace{12pt}
\newpage

\section*{Question 201 (ID: 20061109)}
\textbf{Date: }November 09,2006
\vspace{6pt}

Treatment with what antibiotic is most likely to have resulted in this patient's skin changes?
\vspace{12pt}

\textbf{Options:}
\begin{enumerate}
\item[A.] Rifampin
\item[B.] Chloramphenicol
\item[C.] Nitrofurantoin
\item[D.] Minocycline
\item[E.] Trimethoprim
\end{enumerate}

\textbf{Image:}
\begin{center}
\includegraphics[width=0.95\textwidth,height=0.50\textheight,width=0.90\textwidth,keepaspectratio]{images/nejm_20061109.jpg}
\end{center}
\vspace{12pt}
\newpage

\section*{Question 202 (ID: 20061116)}
\textbf{Date: }November 16,2006
\vspace{6pt}

What is the diagnosis?
\vspace{12pt}

\textbf{Options:}
\begin{enumerate}
\item[A.] Pericardial hydatid cysts
\item[B.] Diaphragmatic rupture
\item[C.] Empyema thoracis
\item[D.] Cystic fibrosis
\item[E.] Pericardial metastases
\end{enumerate}

\textbf{Image:}
\begin{center}
\includegraphics[width=0.83\textwidth,height=0.50\textheight,width=0.90\textwidth,keepaspectratio]{images/nejm_20061116.jpg}
\end{center}
\vspace{12pt}
\newpage

\section*{Question 203 (ID: 20061123)}
\textbf{Date: }November 23,2006
\vspace{6pt}

This patient presented with severe jaw pain while being treated for osteoporosis. What is the diagnosis?
\vspace{12pt}

\textbf{Options:}
\begin{enumerate}
\item[A.] Submandibular abscess
\item[B.] Osteonecrosis
\item[C.] Behcet's disease
\item[D.] Accessory tooth
\item[E.] Calciphylaxis
\end{enumerate}

\textbf{Image:}
\begin{center}
\includegraphics[width=0.95\textwidth,height=0.50\textheight,width=0.90\textwidth,keepaspectratio]{images/nejm_20061123.jpg}
\end{center}
\vspace{12pt}
\newpage

\section*{Question 204 (ID: 20061130)}
\textbf{Date: }November 30,2006
\vspace{6pt}

What is the diagnosis?
\vspace{12pt}

\textbf{Options:}
\begin{enumerate}
\item[A.] Fabry's disease
\item[B.] Hereditary hemorrhagic telangiectasia
\item[C.] Peutz-Jegher's syndrome
\item[D.] Roseola
\item[E.] Discoid lupus
\end{enumerate}

\textbf{Image:}
\begin{center}
\includegraphics[width=0.58\textwidth,height=0.50\textheight,width=0.90\textwidth,keepaspectratio]{images/nejm_20061130.jpg}
\end{center}
\vspace{12pt}
\newpage

\section*{Question 205 (ID: 20061207)}
\textbf{Date: }December 07,2006
\vspace{6pt}

This 19-year-old man presented with a 10-month history of Raynaud's phenomenon, fever, abdominal pain, and hypertension. What diagnosis is suggested by the findings on his angiogram?
\vspace{12pt}

\textbf{Options:}
\begin{enumerate}
\item[A.] Takayasu's arteritis
\item[B.] Wegener's granulomatosis
\item[C.] Paraganglioma
\item[D.] Systemic lupus erythematosus
\item[E.] Polyarteritis nodosa
\end{enumerate}

\textbf{Image:}
\begin{center}
\includegraphics[width=0.52\textwidth,height=0.50\textheight,width=0.90\textwidth,keepaspectratio]{images/nejm_20061207.jpg}
\end{center}
\vspace{12pt}
\newpage

\section*{Question 206 (ID: 20061214)}
\textbf{Date: }December 14,2006
\vspace{6pt}

What is the diagnosis?
\vspace{12pt}

\textbf{Options:}
\begin{enumerate}
\item[A.] Pulmonary hydatid disease
\item[B.] Bullous emphysema
\item[C.] Pulmonary silicosis
\item[D.] Multiple bacterial abscesses
\item[E.] Aspergillosis
\end{enumerate}

\textbf{Image:}
\begin{center}
\includegraphics[width=0.73\textwidth,height=0.50\textheight,width=0.90\textwidth,keepaspectratio]{images/nejm_20061214.jpg}
\end{center}
\vspace{12pt}
\newpage

\section*{Question 207 (ID: 20061221)}
\textbf{Date: }December 21,2006
\vspace{6pt}

This 84-cm tall 40-year-old man underwent whole-body dual-energy x-ray absorptiometry. What is the diagnosis?
\vspace{12pt}

\textbf{Options:}
\begin{enumerate}
\item[A.] Pseudohypoparathyroidism
\item[B.] X-linked hypophosphatemia
\item[C.] Osteogenesis imperfecta
\item[D.] Achondroplasia
\item[E.] Ehlers Danlos syndrome
\end{enumerate}

\textbf{Image:}
\begin{center}
\includegraphics[width=0.67\textwidth,height=0.50\textheight,width=0.90\textwidth,keepaspectratio]{images/nejm_20061221.jpg}
\end{center}
\vspace{12pt}
\newpage

\section*{Question 208 (ID: 20061228)}
\textbf{Date: }December 28,2006
\vspace{6pt}

What is the diagnosis?
\vspace{12pt}

\textbf{Options:}
\begin{enumerate}
\item[A.] Prolapse of orbital fat
\item[B.] Pinguecula
\item[C.] Graves' disease
\item[D.] Neurofibromatosis
\item[E.] Wilson's disease
\end{enumerate}

\textbf{Image:}
\begin{center}
\includegraphics[width=0.95\textwidth,height=0.50\textheight,width=0.90\textwidth,keepaspectratio]{images/nejm_20061228.jpg}
\end{center}
\vspace{12pt}
\newpage

\section*{Question 209 (ID: 20070104)}
\textbf{Date: }January 04,2007
\vspace{6pt}

This patient developed difficulty swallowing following a dental procedure. What is the diagnosis?
\vspace{12pt}

\textbf{Options:}
\begin{enumerate}
\item[A.] Spondylolisthesis
\item[B.] Prevertebral air
\item[C.] Pharyngeal diverticulum
\item[D.] Pharyngeal foreign body
\item[E.] Periodontal abscess
\end{enumerate}

\textbf{Image:}
\begin{center}
\includegraphics[width=0.63\textwidth,height=0.50\textheight,width=0.90\textwidth,keepaspectratio]{images/nejm_20070104.jpg}
\end{center}
\vspace{12pt}
\newpage

\section*{Question 210 (ID: 20070111)}
\textbf{Date: }January 11,2007
\vspace{6pt}

What is the diagnosis?
\vspace{12pt}

\textbf{Options:}
\begin{enumerate}
\item[A.] Cholesterol emboli
\item[B.] Acute arterial insufficiency
\item[C.] Deep venous thrombosis
\item[D.] Frost bite
\item[E.] Thromboangiitis obliterans
\end{enumerate}

\textbf{Image:}
\begin{center}
\includegraphics[width=0.61\textwidth,height=0.50\textheight,width=0.90\textwidth,keepaspectratio]{images/nejm_20070111.jpg}
\end{center}
\vspace{12pt}
\newpage

\section*{Question 211 (ID: 20070118)}
\textbf{Date: }January 18,2007
\vspace{6pt}

What is the diagnosis?
\vspace{12pt}

\textbf{Options:}
\begin{enumerate}
\item[A.] Retained surgical clamp
\item[B.] Uterine perforation
\item[C.] Nephrostomy tube
\item[D.] Hermaphoditism
\item[E.] Appendicitis
\end{enumerate}

\textbf{Image:}
\begin{center}
\includegraphics[width=0.69\textwidth,height=0.50\textheight,width=0.90\textwidth,keepaspectratio]{images/nejm_20070118.jpg}
\end{center}
\vspace{12pt}
\newpage

\section*{Question 212 (ID: 20070125)}
\textbf{Date: }January 25,2007
\vspace{6pt}

What is the diagnosis?
\vspace{12pt}

\textbf{Options:}
\begin{enumerate}
\item[A.] Diphtheria
\item[B.] Secondary syphilis
\item[C.] Oral leukoplakia
\item[D.] Candida
\item[E.] Ludwig's angina
\end{enumerate}

\textbf{Image:}
\begin{center}
\includegraphics[width=0.95\textwidth,height=0.50\textheight,width=0.90\textwidth,keepaspectratio]{images/nejm_20070125.jpg}
\end{center}
\vspace{12pt}
\newpage

\section*{Question 213 (ID: 20070201)}
\textbf{Date: }February 01,2007
\vspace{6pt}

What is the diagnosis?
\vspace{12pt}

\textbf{Options:}
\begin{enumerate}
\item[A.] Tinea barbae
\item[B.] Herpes simplex infection
\item[C.] Eczema
\item[D.] Mycosis fungoides
\item[E.] Impetigo
\end{enumerate}

\textbf{Image:}
\begin{center}
\includegraphics[width=0.95\textwidth,height=0.50\textheight,width=0.90\textwidth,keepaspectratio]{images/nejm_20070201.jpg}
\end{center}
\vspace{12pt}
\newpage

\section*{Question 214 (ID: 20070208)}
\textbf{Date: }February 08,2007
\vspace{6pt}

What is the diagnosis?
\vspace{12pt}

\textbf{Options:}
\begin{enumerate}
\item[A.] Digitalis intoxication
\item[B.] Dextrocardia
\item[C.] Pacemaker malfunction
\item[D.] Electrical alternans
\item[E.] Cardiac allograft
\end{enumerate}

\textbf{Image:}
\begin{center}
\includegraphics[width=0.95\textwidth,height=0.50\textheight,width=0.90\textwidth,keepaspectratio]{images/nejm_20070208.jpg}
\end{center}
\vspace{12pt}
\newpage

\section*{Question 215 (ID: 20070215)}
\textbf{Date: }February 15,2007
\vspace{6pt}

This patient presented with loss of vision. What is the diagnosis?
\vspace{12pt}

\textbf{Options:}
\begin{enumerate}
\item[A.] Central retinal artery occlusion
\item[B.] Diabetic retinopathy
\item[C.] Tay-Sach's disease
\item[D.] Hypertensive retinopathy
\item[E.] Papilledema
\end{enumerate}

\textbf{Image:}
\begin{center}
\includegraphics[width=0.95\textwidth,height=0.50\textheight,width=0.90\textwidth,keepaspectratio]{images/nejm_20070215.jpg}
\end{center}
\vspace{12pt}
\newpage

\section*{Question 216 (ID: 20070222)}
\textbf{Date: }February 22,2007
\vspace{6pt}

Which one of the following organs is enlarged?
\vspace{12pt}

\textbf{Options:}
\begin{enumerate}
\item[C.] Stomach
\item[D.] Spleen
\item[E.] Gall bladder
\end{enumerate}

\textbf{Image:}
\begin{center}
\includegraphics[width=0.87\textwidth,height=0.50\textheight,width=0.90\textwidth,keepaspectratio]{images/nejm_20070222.jpg}
\end{center}
\vspace{12pt}
\newpage

\section*{Question 217 (ID: 20070308)}
\textbf{Date: }March 08,2007
\vspace{6pt}

This patient's serum was found to be discolored four hours following a surgical procedure. What is the most likely cause?
\vspace{12pt}

\textbf{Options:}
\begin{enumerate}
\item[A.] Fluorescent dye
\item[B.] Methemoglobinemia
\item[C.] Propofol
\item[D.] Pseudomonal sepsis
\item[E.] Ethylene glycol
\end{enumerate}

\textbf{Image:}
\begin{center}
\includegraphics[width=0.3\textwidth,height=0.50\textheight,width=0.90\textwidth,keepaspectratio]{images/nejm_20070308.jpg}
\end{center}
\vspace{12pt}
\newpage

\section*{Question 218 (ID: 20070315)}
\textbf{Date: }March 15,2007
\vspace{6pt}

What is the diagnosis?
\vspace{12pt}

\textbf{Options:}
\begin{enumerate}
\item[A.] Carpal tunnel syndrome
\item[B.] Rheumatoid Arthritis
\item[C.] Scleroderma
\item[D.] Diabetic peripheral neuropathy
\item[E.] Dupuyten's contracture
\end{enumerate}

\textbf{Image:}
\begin{center}
\includegraphics[width=0.95\textwidth,height=0.50\textheight,width=0.90\textwidth,keepaspectratio]{images/nejm_20070315.jpg}
\end{center}
\vspace{12pt}
\newpage

\section*{Question 219 (ID: 20070322)}
\textbf{Date: }March 22,2007
\vspace{6pt}

What is the diagnosis?
\vspace{12pt}

\textbf{Options:}
\begin{enumerate}
\item[A.] Chalazion
\item[B.] Papilloma
\item[C.] Pterygium
\item[D.] Pinguecula
\item[E.] Coloboma
\end{enumerate}

\textbf{Image:}
\begin{center}
\includegraphics[width=0.95\textwidth,height=0.50\textheight,width=0.90\textwidth,keepaspectratio]{images/nejm_20070322.jpg}
\end{center}
\vspace{12pt}
\newpage

\section*{Question 220 (ID: 20070329)}
\textbf{Date: }March 29,2007
\vspace{6pt}

What is the diagnosis?
\vspace{12pt}

\textbf{Options:}
\begin{enumerate}
\item[A.] Subcutaneous metastases
\item[B.] Filariasis
\item[C.] Caput Medusae
\item[D.] Neurofibromatosis
\item[E.] Hepatocellular carcinoma
\end{enumerate}

\textbf{Image:}
\begin{center}
\includegraphics[width=0.72\textwidth,height=0.50\textheight,width=0.90\textwidth,keepaspectratio]{images/nejm_20070329.jpg}
\end{center}
\vspace{12pt}
\newpage

\section*{Question 221 (ID: 20070405)}
\textbf{Date: }April 05,2007
\vspace{6pt}

This 9-kg liver was removed at the time of liver transplantation. What is the diagnosis?
\vspace{12pt}

\textbf{Options:}
\begin{enumerate}
\item[A.] Macronodular cirrhosis
\item[B.] Hepatocellular carcinoma
\item[C.] Echinococcosis
\item[D.] Polycystic liver disease
\item[E.] Trophoblastic tumor
\end{enumerate}

\textbf{Image:}
\begin{center}
\includegraphics[width=0.95\textwidth,height=0.50\textheight,width=0.90\textwidth,keepaspectratio]{images/nejm_20070405.jpg}
\end{center}
\vspace{12pt}
\newpage

\section*{Question 222 (ID: 20070412)}
\textbf{Date: }April 12,2007
\vspace{6pt}

What is the diagnosis?
\vspace{12pt}

\textbf{Options:}
\begin{enumerate}
\item[A.] Intracranial hemorrhage
\item[B.] Osteoma
\item[C.] Neurocysticercosis
\item[D.] Arachnoid cyst
\item[E.] Meningioma
\end{enumerate}

\textbf{Image:}
\begin{center}
\includegraphics[width=0.7\textwidth,height=0.50\textheight,width=0.90\textwidth,keepaspectratio]{images/nejm_20070412.jpg}
\end{center}
\vspace{12pt}
\newpage

\section*{Question 223 (ID: 20070426)}
\textbf{Date: }April 26,2007
\vspace{6pt}

In what vessel is this patient's dialysis catheter?
\vspace{12pt}

\textbf{Options:}
\begin{enumerate}
\item[A.] Left axillary vein
\item[B.] Left internal mammary vein
\item[C.] Left-sided superior vena cava
\item[D.] Azygous vein
\item[E.] Highest intercostal vein
\end{enumerate}

\textbf{Image:}
\begin{center}
\includegraphics[width=0.75\textwidth,height=0.50\textheight,width=0.90\textwidth,keepaspectratio]{images/nejm_20070426.jpg}
\end{center}
\vspace{12pt}
\newpage

\section*{Question 224 (ID: 20070503)}
\textbf{Date: }May 03,2007
\vspace{6pt}

What is the diagnosis?
\vspace{12pt}

\textbf{Options:}
\begin{enumerate}
\item[A.] Central retinal vein occlusion
\item[B.] Profilerative diabetic retinopathy
\item[C.] Hypertensive retinopathy
\item[D.] Chorioretinitis
\item[E.] Papilledema
\end{enumerate}

\textbf{Image:}
\begin{center}
\includegraphics[width=0.95\textwidth,height=0.50\textheight,width=0.90\textwidth,keepaspectratio]{images/nejm_20070503.jpg}
\end{center}
\vspace{12pt}
\newpage

\section*{Question 225 (ID: 20070510)}
\textbf{Date: }May 10,2007
\vspace{6pt}

What is the diagnosis?
\vspace{12pt}

\textbf{Options:}
\begin{enumerate}
\item[A.] Pulmonary embolism
\item[B.] Blalock-Taussig shunt
\item[C.] Azygous vein
\item[D.] Interlobar mesothelioma
\item[E.] Cervical rib
\end{enumerate}

\textbf{Image:}
\begin{center}
\includegraphics[width=0.95\textwidth,height=0.50\textheight,width=0.90\textwidth,keepaspectratio]{images/nejm_20070510.jpg}
\end{center}
\vspace{12pt}
\newpage

\section*{Question 226 (ID: 20070517)}
\textbf{Date: }May 17,2007
\vspace{6pt}

These lesions developed over 1 to 2 years. What is the diagnosis?
\vspace{12pt}

\textbf{Options:}
\begin{enumerate}
\item[A.] Melanocytic nevi
\item[B.] Seborrheic keratoses
\item[C.] Pemphigus erythematosus
\item[D.] Bowenoid papulosis
\item[E.] Guttate psoriasis
\end{enumerate}

\textbf{Image:}
\begin{center}
\includegraphics[width=0.95\textwidth,height=0.50\textheight,width=0.90\textwidth,keepaspectratio]{images/nejm_20070517.jpg}
\end{center}
\vspace{12pt}
\newpage

\section*{Question 227 (ID: 20070524)}
\textbf{Date: }May 24,2007
\vspace{6pt}

What is the diagnosis?
\vspace{12pt}

\textbf{Options:}
\begin{enumerate}
\item[A.] Nocardiosis
\item[B.] Adenocarcinoma
\item[C.] Ventricular rupture
\item[D.] Alpha-1 antitrypsin deficiency
\item[E.] Pulmonary hemosiderosis
\end{enumerate}

\textbf{Image:}
\begin{center}
\includegraphics[width=0.79\textwidth,height=0.50\textheight,width=0.90\textwidth,keepaspectratio]{images/nejm_20070524.jpg}
\end{center}
\vspace{12pt}
\newpage

\section*{Question 228 (ID: 20070531)}
\textbf{Date: }May 31,2007
\vspace{6pt}

These lesions appeared spontaneously in a patient with untreated multiple myeloma. Coagulation studies were normal and platelet count was 80,000 per cubic millimeter. What is the diagnosis?
\vspace{12pt}

\textbf{Options:}
\begin{enumerate}
\item[A.] Orbital fracture
\item[B.] Disseminated intravascular coagulation
\item[C.] Tuberous sclerosis
\item[D.] Amyloid purpura
\item[E.] Horner's syndrome
\end{enumerate}

\textbf{Image:}
\begin{center}
\includegraphics[width=0.95\textwidth,height=0.50\textheight,width=0.90\textwidth,keepaspectratio]{images/nejm_20070531.jpg}
\end{center}
\vspace{12pt}
\newpage

\section*{Question 229 (ID: 20070607)}
\textbf{Date: }June 07,2007
\vspace{6pt}

What is the diagnosis?
\vspace{12pt}

\textbf{Options:}
\begin{enumerate}
\item[A.] Hypertriglyceridemia
\item[B.] Hypertensive retinopathy
\item[C.] Optic atrophy
\item[D.] Central retinal artery occlusion
\item[E.] Cytomegalovirus retinitis
\end{enumerate}

\textbf{Image:}
\begin{center}
\includegraphics[width=0.95\textwidth,height=0.50\textheight,width=0.90\textwidth,keepaspectratio]{images/nejm_20070607.jpg}
\end{center}
\vspace{12pt}
\newpage

\section*{Question 230 (ID: 20070614)}
\textbf{Date: }June 14,2007
\vspace{6pt}

These lesions appeared in a recent immigrant from Pakistan. The lesions were neither pruritic nor hypoesthetic. What is the diagnosis?
\vspace{12pt}

\textbf{Options:}
\begin{enumerate}
\item[A.] Granuloma annulare
\item[B.] Scrofula
\item[C.] Leprosy
\item[D.] Pityriasis rosea
\item[E.] Tinea corporis
\end{enumerate}

\textbf{Image:}
\begin{center}
\includegraphics[width=0.58\textwidth,height=0.50\textheight,width=0.90\textwidth,keepaspectratio]{images/nejm_20070614.jpg}
\end{center}
\vspace{12pt}
\newpage

\section*{Question 231 (ID: 20070621)}
\textbf{Date: }June 21,2007
\vspace{6pt}

This patient presented with hypertension, proteinuria, and renal failure. What is the most likely diagnosis?
\vspace{12pt}

\textbf{Options:}
\begin{enumerate}
\item[A.] Mycosis fungoides
\item[B.] Wegener's granulomatosis
\item[C.] Invasive aspergillosis
\item[D.] Sarcoidosis
\item[E.] Polychondritis
\end{enumerate}

\textbf{Image:}
\begin{center}
\includegraphics[width=0.58\textwidth,height=0.50\textheight,width=0.90\textwidth,keepaspectratio]{images/nejm_20070621.jpg}
\end{center}
\vspace{12pt}
\newpage

\section*{Question 232 (ID: 20070628)}
\textbf{Date: }June 28,2007
\vspace{6pt}

What is the diagnosis?
\vspace{12pt}

\textbf{Options:}
\begin{enumerate}
\item[A.] Deep venous thrombosis
\item[B.] Necrotizing fasciitis
\item[C.] Chemical burn
\item[D.] Scleroderma
\item[E.] Digital ischemia
\end{enumerate}

\textbf{Image:}
\begin{center}
\includegraphics[width=0.81\textwidth,height=0.50\textheight,width=0.90\textwidth,keepaspectratio]{images/nejm_20070628.jpg}
\end{center}
\vspace{12pt}
\newpage

\section*{Question 233 (ID: 20070705)}
\textbf{Date: }July 05,2007
\vspace{6pt}

This patient developed progressive thickening of the skin following failure of a transplanted kidney. What is the most likely diagnosis?
\vspace{12pt}

\textbf{Options:}
\begin{enumerate}
\item[A.] Eosinophilic fasciitis
\item[B.] Myxedema
\item[C.] Nephrogenic fibrosing dermopathy
\item[D.] Scleroderma
\item[E.] Vitiligo
\end{enumerate}

\textbf{Image:}
\begin{center}
\includegraphics[width=0.95\textwidth,height=0.50\textheight,width=0.90\textwidth,keepaspectratio]{images/nejm_20070705.jpg}
\end{center}
\vspace{12pt}
\newpage

\section*{Question 234 (ID: 20070712)}
\textbf{Date: }July 12,2007
\vspace{6pt}

This patient developed jaundice and an enlarging neck mass. What is the diagnosis?
\vspace{12pt}

\textbf{Options:}
\begin{enumerate}
\item[A.] Medullary carcinoma of the thyroid
\item[B.] Peripancreatic malignancy
\item[C.] Scrofula
\item[D.] Hodgkin's lymphoma
\item[E.] Pancoast tumor
\end{enumerate}

\textbf{Image:}
\begin{center}
\includegraphics[width=0.63\textwidth,height=0.50\textheight,width=0.90\textwidth,keepaspectratio]{images/nejm_20070712.jpg}
\end{center}
\vspace{12pt}
\newpage

\section*{Question 235 (ID: 20070719)}
\textbf{Date: }July 19,2007
\vspace{6pt}

This patient presented with chest pain. What is the diagnosis?
\vspace{12pt}

\textbf{Options:}
\begin{enumerate}
\item[A.] Acute pulmonary embolism
\item[B.] Hypertrophic cardiomyopathy
\item[C.] Ascending aortic aneurysm
\item[D.] Coarctation of the aorta
\item[E.] Lymphoma
\end{enumerate}

\textbf{Image:}
\begin{center}
\includegraphics[width=0.95\textwidth,height=0.50\textheight,width=0.90\textwidth,keepaspectratio]{images/nejm_20070719.jpg}
\end{center}
\vspace{12pt}
\newpage

\section*{Question 236 (ID: 20070726)}
\textbf{Date: }July 26,2007
\vspace{6pt}

This patient presented with acute onset of erythema. Nearly a year earlier she was treated for invasive breast cancer. What is the diagnosis?
\vspace{12pt}

\textbf{Options:}
\begin{enumerate}
\item[A.] Erythrasma
\item[B.] Recurrent breast cancer
\item[C.] Radiation dermatitis
\item[D.] Cellulitis
\item[E.] Lymphedema
\end{enumerate}

\textbf{Image:}
\begin{center}
\includegraphics[width=0.95\textwidth,height=0.50\textheight,width=0.90\textwidth,keepaspectratio]{images/nejm_20070726.jpg}
\end{center}
\vspace{12pt}
\newpage

\section*{Question 237 (ID: 20070809)}
\textbf{Date: }August 09,2007
\vspace{6pt}

What is the diagnosis?
\vspace{12pt}

\textbf{Options:}
\begin{enumerate}
\item[A.] Molluscum contagiosum
\item[B.] Pearly penile papules
\item[C.] Secondary syphilis
\item[D.] Obstruction of smegma-producing gland
\item[E.] Condyloma accuminatum
\end{enumerate}

\textbf{Image:}
\begin{center}
\includegraphics[width=0.78\textwidth,height=0.50\textheight,width=0.90\textwidth,keepaspectratio]{images/nejm_20070809.jpg}
\end{center}
\vspace{12pt}
\newpage

\section*{Question 238 (ID: 20070816)}
\textbf{Date: }August 16,2007
\vspace{6pt}

This 12-year-old boy presented with abdominal pain. What is the diagnosis?
\vspace{12pt}

\textbf{Options:}
\begin{enumerate}
\item[A.] Cowden syndrome
\item[B.] Cronkhite-Canada syndrome
\item[C.] Osler-Weber-Rendu syndrome
\item[D.] Peutz-Jeghers syndrome
\item[E.] VonWillebrand syndrome
\end{enumerate}

\textbf{Image:}
\begin{center}
\includegraphics[width=0.95\textwidth,height=0.50\textheight,width=0.90\textwidth,keepaspectratio]{images/nejm_20070816.jpg}
\end{center}
\vspace{12pt}
\newpage

\section*{Question 239 (ID: 20070823)}
\textbf{Date: }August 23,2007
\vspace{6pt}

Which one of the following is the most likely diagnosis?
\vspace{12pt}

\textbf{Options:}
\begin{enumerate}
\item[A.] Chronic renal failure
\item[B.] Iron deficiency
\item[C.] Graves' disease
\item[D.] Chemotherapy treatment
\item[E.] Psoriasis
\end{enumerate}

\textbf{Image:}
\begin{center}
\includegraphics[width=0.95\textwidth,height=0.50\textheight,width=0.90\textwidth,keepaspectratio]{images/nejm_20070823.jpg}
\end{center}
\vspace{12pt}
\newpage

\section*{Question 240 (ID: 20070830)}
\textbf{Date: }August 30,2007
\vspace{6pt}

What is the diagnosis?
\vspace{12pt}

\textbf{Options:}
\begin{enumerate}
\item[A.] Cytomegalovirus retinitis
\item[B.] Lipemia retinalis
\item[C.] Central retinal vein occlusion
\item[D.] Diabetic retinopathy
\item[E.] Retinitis pigmentosa
\end{enumerate}

\textbf{Image:}
\begin{center}
\includegraphics[width=0.95\textwidth,height=0.50\textheight,width=0.90\textwidth,keepaspectratio]{images/nejm_20070830.jpg}
\end{center}
\vspace{12pt}
\newpage

\section*{Question 241 (ID: 20070906)}
\textbf{Date: }September 06,2007
\vspace{6pt}

What is the diagnosis?
\vspace{12pt}

\textbf{Options:}
\begin{enumerate}
\item[A.] Cutis laxa
\item[B.] Turner's syndrome
\item[C.] Ehlers-Danlos syndrome
\item[D.] Pseudoxanthoma elasticum
\item[E.] Marfan syndrome
\end{enumerate}

\textbf{Image:}
\begin{center}
\includegraphics[width=0.95\textwidth,height=0.50\textheight,width=0.90\textwidth,keepaspectratio]{images/nejm_20070906.jpg}
\end{center}
\vspace{12pt}
\newpage

\section*{Question 242 (ID: 20070913)}
\textbf{Date: }September 13,2007
\vspace{6pt}

What is the diagnosis?
\vspace{12pt}

\textbf{Options:}
\begin{enumerate}
\item[A.] Herpetic glossitis
\item[B.] Aphthous ulceration
\item[C.] Pemphigoid
\item[D.] Scurvy
\item[E.] Oral candidiasis
\end{enumerate}

\textbf{Image:}
\begin{center}
\includegraphics[width=0.63\textwidth,height=0.50\textheight,width=0.90\textwidth,keepaspectratio]{images/nejm_20070913.jpg}
\end{center}
\vspace{12pt}
\newpage

\section*{Question 243 (ID: 20070920)}
\textbf{Date: }September 20,2007
\vspace{6pt}

What is the diagnosis?
\vspace{12pt}

\textbf{Options:}
\begin{enumerate}
\item[A.] Scleroderma
\item[B.] Thromboangiitis obliterans
\item[C.] Calciphylaxis
\item[D.] Hypoparathyroidism
\item[E.] Osteomyelitis
\end{enumerate}

\textbf{Image:}
\begin{center}
\includegraphics[width=0.95\textwidth,height=0.50\textheight,width=0.90\textwidth,keepaspectratio]{images/nejm_20070920.jpg}
\end{center}
\vspace{12pt}
\newpage

\section*{Question 244 (ID: 20070927)}
\textbf{Date: }September 27,2007
\vspace{6pt}

What is the diagnosis?
\vspace{12pt}

\textbf{Options:}
\begin{enumerate}
\item[A.] Rheumatoid arthritis
\item[B.] Amyloidosis
\item[C.] Cirrhosis
\item[D.] Psoriasis
\item[E.] Endocarditis
\end{enumerate}

\textbf{Image:}
\begin{center}
\includegraphics[width=0.82\textwidth,height=0.50\textheight,width=0.90\textwidth,keepaspectratio]{images/nejm_20070927.jpg}
\end{center}
\vspace{12pt}
\newpage

\section*{Question 245 (ID: 20071004)}
\textbf{Date: }October 04,2007
\vspace{6pt}

What is the diagnosis?
\vspace{12pt}

\textbf{Options:}
\begin{enumerate}
\item[A.] Pancreatic pseudocyst
\item[B.] Duodenal torsion
\item[C.] Cholangiocarcinoma
\item[D.] Gallbladder lipomatosis
\item[E.] Emphysematous cholecystitis
\end{enumerate}

\textbf{Image:}
\begin{center}
\includegraphics[width=0.95\textwidth,height=0.50\textheight,width=0.90\textwidth,keepaspectratio]{images/nejm_20071004.jpg}
\end{center}
\vspace{12pt}
\newpage

\section*{Question 246 (ID: 20071011)}
\textbf{Date: }October 11,2007
\vspace{6pt}

What is the diagnosis?
\vspace{12pt}

\textbf{Options:}
\begin{enumerate}
\item[A.] Cholangiocarcinoma
\item[B.] Cirrhosis
\item[C.] Echinococcosis
\item[D.] Hepatocellular carcinoma
\item[E.] Tuberculosis
\end{enumerate}

\textbf{Image:}
\begin{center}
\includegraphics[width=0.95\textwidth,height=0.50\textheight,width=0.90\textwidth,keepaspectratio]{images/nejm_20071011.jpg}
\end{center}
\vspace{12pt}
\newpage

\section*{Question 247 (ID: 20071018)}
\textbf{Date: }October 18,2007
\vspace{6pt}

What is the diagnosis?
\vspace{12pt}

\textbf{Options:}
\begin{enumerate}
\item[A.] Graves' ophthalmopathy
\item[B.] Retinal detachment
\item[C.] Choroidal melanoma
\item[D.] Ocular implant
\item[E.] Angle closure glaucoma
\end{enumerate}

\textbf{Image:}
\begin{center}
\includegraphics[width=0.83\textwidth,height=0.50\textheight,width=0.90\textwidth,keepaspectratio]{images/nejm_20071018.jpg}
\end{center}
\vspace{12pt}
\newpage

\section*{Question 248 (ID: 20071025)}
\textbf{Date: }October 25,2007
\vspace{6pt}

This patient presented with a 10-day history of left foot discoloration that resolved with elevation. What is the most likely diagnosis?
\vspace{12pt}

\textbf{Options:}
\begin{enumerate}
\item[A.] Angioneurotic edema
\item[B.] Arterial insufficiency
\item[C.] Erysipelas
\item[D.] Peripheral microembolization
\item[E.] Phlegmasia cerulea dolens
\end{enumerate}

\textbf{Image:}
\begin{center}
\includegraphics[width=0.95\textwidth,height=0.50\textheight,width=0.90\textwidth,keepaspectratio]{images/nejm_20071025.jpg}
\end{center}
\vspace{12pt}
\newpage

\section*{Question 249 (ID: 20071101)}
\textbf{Date: }November 01,2007
\vspace{6pt}

Which one of the following drugs of abuse is most typically associated with the illustrated complication?
\vspace{12pt}

\textbf{Options:}
\begin{enumerate}
\item[A.] Ketamine
\item[B.] Heroin
\item[C.] Cocaine
\item[D.] Phencyclidine
\item[E.] Mescaline
\end{enumerate}

\textbf{Image:}
\begin{center}
\includegraphics[width=0.78\textwidth,height=0.50\textheight,width=0.90\textwidth,keepaspectratio]{images/nejm_20071101.jpg}
\end{center}
\vspace{12pt}
\newpage

\section*{Question 250 (ID: 20071108)}
\textbf{Date: }November 08,2007
\vspace{6pt}

This 81-year-old woman presented with swelling and pain in the left foot a week after treatment for acute bronchitis. Which one of the following treatments is most likely to have been contributory?
\vspace{12pt}

\textbf{Options:}
\begin{enumerate}
\item[A.] Beta-adrenergic agonist
\item[B.] Cephalosporin antibiotic
\item[C.] Leukotriene receptor antagonist
\item[D.] Guaifenesin expectorant
\item[E.] Fluoroquinolone antibiotic
\end{enumerate}

\textbf{Image:}
\begin{center}
\includegraphics[width=0.79\textwidth,height=0.50\textheight,width=0.90\textwidth,keepaspectratio]{images/nejm_20071108.jpg}
\end{center}
\vspace{12pt}
\newpage

\section*{Question 251 (ID: 20071115)}
\textbf{Date: }November 15,2007
\vspace{6pt}

What is the diagnosis?
\vspace{12pt}

\textbf{Options:}
\begin{enumerate}
\item[A.] Bezoar
\item[B.] Pericardial effusion
\item[C.] Pneumatosis coli
\item[D.] Gastric carcinoma
\item[E.] Pheochromocytoma
\end{enumerate}

\textbf{Image:}
\begin{center}
\includegraphics[width=0.95\textwidth,height=0.50\textheight,width=0.90\textwidth,keepaspectratio]{images/nejm_20071115.jpg}
\end{center}
\vspace{12pt}
\newpage

\section*{Question 252 (ID: 20071122)}
\textbf{Date: }November 22,2007
\vspace{6pt}

What diagnosis is suggested by the findings on this anteroposterior radiograph of the right tibia?
\vspace{12pt}

\textbf{Options:}
\begin{enumerate}
\item[A.] Ewing's sarcoma
\item[B.] Marrow hyperplasia
\item[C.] Osteomalacia
\item[D.] Polyostotic fibrous dysplasia
\item[E.] Osteopetrosis
\end{enumerate}

\textbf{Image:}
\begin{center}
\includegraphics[width=0.82\textwidth,height=0.50\textheight,width=0.90\textwidth,keepaspectratio]{images/nejm_20071122.jpg}
\end{center}
\vspace{12pt}
\newpage

\section*{Question 253 (ID: 20071129)}
\textbf{Date: }November 29,2007
\vspace{6pt}

Which one of following is typically associated with this finding?
\vspace{12pt}

\textbf{Options:}
\begin{enumerate}
\item[A.] Bulimia
\item[B.] Recent endotracheal intubation
\item[C.] Acquired immunodeficiency syndrome
\item[D.] Treatment with a sulfonylurea
\item[E.] Candida albicans infection
\end{enumerate}

\textbf{Image:}
\begin{center}
\includegraphics[width=0.59\textwidth,height=0.50\textheight,width=0.90\textwidth,keepaspectratio]{images/nejm_20071129.jpg}
\end{center}
\vspace{12pt}
\newpage

\section*{Question 254 (ID: 20071206)}
\textbf{Date: }December 06,2007
\vspace{6pt}

This patient is most likely to have presented with which one of the following findings?
\vspace{12pt}

\textbf{Options:}
\begin{enumerate}
\item[A.] Amnesia
\item[B.] Ataxia
\item[C.] Dysphagia
\item[D.] Hemianopia
\item[E.] Hemiparesis
\end{enumerate}

\textbf{Image:}
\begin{center}
\includegraphics[width=0.83\textwidth,height=0.50\textheight,width=0.90\textwidth,keepaspectratio]{images/nejm_20071206.jpg}
\end{center}
\vspace{12pt}
\newpage

\section*{Question 255 (ID: 20071213)}
\textbf{Date: }December 13,2007
\vspace{6pt}

What is the most likely diagnosis in this 36-year-old woman with chronic renal failure?
\vspace{12pt}

\textbf{Options:}
\begin{enumerate}
\item[A.] Lipoma
\item[B.] Osteitis fibrosa
\item[C.] Osteosarcoma
\item[D.] Fistula
\item[E.] Calcinosis
\end{enumerate}

\textbf{Image:}
\begin{center}
\includegraphics[width=0.74\textwidth,height=0.50\textheight,width=0.90\textwidth,keepaspectratio]{images/nejm_20071213.jpg}
\end{center}
\vspace{12pt}
\newpage

\section*{Question 256 (ID: 20071220)}
\textbf{Date: }December 20,2007
\vspace{6pt}

Which cardiac valves have been replaced?
\vspace{12pt}

\textbf{Options:}
\begin{enumerate}
\item[A.] Aortic and mitral valves
\item[B.] Aortic and tricuspid valves
\item[C.] Mitral and tricuspid valves
\item[D.] Pulmonary and mitral valves
\item[E.] Pulmonary and tricuspid valves
\end{enumerate}

\textbf{Image:}
\begin{center}
\includegraphics[width=0.75\textwidth,height=0.50\textheight,width=0.90\textwidth,keepaspectratio]{images/nejm_20071220.jpg}
\end{center}
\vspace{12pt}
\newpage

\section*{Question 257 (ID: 20071227)}
\textbf{Date: }December 27,2007
\vspace{6pt}

What is the diagnosis?
\vspace{12pt}

\textbf{Options:}
\begin{enumerate}
\item[A.] Rhinophyma
\item[B.] Leishmaniasis
\item[C.] Lupus pernio
\item[D.] Wegener's granulomatosis
\item[E.] Basal cell carcinoma
\end{enumerate}

\textbf{Image:}
\begin{center}
\includegraphics[width=0.89\textwidth,height=0.50\textheight,width=0.90\textwidth,keepaspectratio]{images/nejm_20071227.jpg}
\end{center}
\vspace{12pt}
\newpage

\section*{Question 258 (ID: 20080103)}
\textbf{Date: }January 03,2008
\vspace{6pt}

What diagnosis is suggested by the finding on the sole of this patient's foot?
\vspace{12pt}

\textbf{Options:}
\begin{enumerate}
\item[A.] Chemical burn
\item[B.] Pemphigus
\item[C.] Thrombotic vasculopathy
\item[D.] Radiation dermatitis
\item[E.] Frostbite
\end{enumerate}

\textbf{Image:}
\begin{center}
\includegraphics[width=0.95\textwidth,height=0.50\textheight,width=0.90\textwidth,keepaspectratio]{images/nejm_20080103.jpg}
\end{center}
\vspace{12pt}
\newpage

\section*{Question 259 (ID: 20080110)}
\textbf{Date: }January 10,2008
\vspace{6pt}

This patient presented with fatigue, fever, anorexia, and weight loss. What is the most likely diagnosis?
\vspace{12pt}

\textbf{Options:}
\begin{enumerate}
\item[A.] Leukemia
\item[B.] Scurvy
\item[C.] Acquired immunodeficiency syndrome
\item[D.] Sarcoidosis
\item[E.] Pellagra
\end{enumerate}

\textbf{Image:}
\begin{center}
\includegraphics[width=0.95\textwidth,height=0.50\textheight,width=0.90\textwidth,keepaspectratio]{images/nejm_20080110.jpg}
\end{center}
\vspace{12pt}
\newpage

\section*{Question 260 (ID: 20080117)}
\textbf{Date: }January 17,2008
\vspace{6pt}

What is the diagnosis?
\vspace{12pt}

\textbf{Options:}
\begin{enumerate}
\item[A.] Cutaneous leishmaniasis
\item[B.] Cutaneous larva migrans
\item[C.] Epidermoid cyst
\item[D.] Furuncular myiasis
\item[E.] Tungiasis
\end{enumerate}

\textbf{Image:}
\begin{center}
\includegraphics[width=0.95\textwidth,height=0.50\textheight,width=0.90\textwidth,keepaspectratio]{images/nejm_20080117.jpg}
\end{center}
\vspace{12pt}
\newpage

\section*{Question 261 (ID: 20080124)}
\textbf{Date: }January 24,2008
\vspace{6pt}

What is the diagnosis?
\vspace{12pt}

\textbf{Options:}
\begin{enumerate}
\item[A.] Dental abscess
\item[B.] Neurofibromatosis type 1
\item[C.] Cleft jaw
\item[D.] Hemiatrophy syndrome
\item[E.] Mandibular fracture
\end{enumerate}

\textbf{Image:}
\begin{center}
\includegraphics[width=0.65\textwidth,height=0.50\textheight,width=0.90\textwidth,keepaspectratio]{images/nejm_20080124.jpg}
\end{center}
\vspace{12pt}
\newpage

\section*{Question 262 (ID: 20080131)}
\textbf{Date: }January 31,2008
\vspace{6pt}

This patient presented with jaw pain and was found to have an elevated alkaline phosphatase and a normal serum creatinine. Which one of the following tests would confirm the diagnosis?
\vspace{12pt}

\textbf{Options:}
\begin{enumerate}
\item[A.] Bone scan
\item[B.] Insulin-like growth factor-1 level
\item[C.] Serum calcium
\item[D.] Abdominal ultrasound
\item[E.] Testing the function of the facial nerve
\end{enumerate}

\textbf{Image:}
\begin{center}
\includegraphics[width=0.57\textwidth,height=0.50\textheight,width=0.90\textwidth,keepaspectratio]{images/nejm_20080131.jpg}
\end{center}
\vspace{12pt}
\newpage

\section*{Question 263 (ID: 20080207)}
\textbf{Date: }February 07,2008
\vspace{6pt}

This 43-year-old patient presented with bilateral pain, swelling, and stiffness in the hands and feet. Her chest radiograph was abnormal. What is the most likely diagnosis?
\vspace{12pt}

\textbf{Options:}
\begin{enumerate}
\item[A.] Miliary tuberculosis
\item[B.] Psoriasis
\item[C.] Syphilis
\item[D.] Reiter syndrome
\item[E.] Sarcoidosis
\end{enumerate}

\textbf{Image:}
\begin{center}
\includegraphics[width=0.7\textwidth,height=0.50\textheight,width=0.90\textwidth,keepaspectratio]{images/nejm_20080207.jpg}
\end{center}
\vspace{12pt}
\newpage

\section*{Question 264 (ID: 20080214)}
\textbf{Date: }February 14,2008
\vspace{6pt}

This patient presented with transient, painless visual obscuration in the left eye. What is the diagnosis?
\vspace{12pt}

\textbf{Options:}
\begin{enumerate}
\item[A.] Papilledema
\item[B.] Hypertensive retinopathy
\item[C.] Cholesterol emboli
\item[D.] Temporal arteritis
\item[E.] Diabetic retinopathy
\end{enumerate}

\textbf{Image:}
\begin{center}
\includegraphics[width=0.63\textwidth,height=0.50\textheight,width=0.90\textwidth,keepaspectratio]{images/nejm_20080214.jpg}
\end{center}
\vspace{12pt}
\newpage

\section*{Question 265 (ID: 20080221)}
\textbf{Date: }February 21,2008
\vspace{6pt}

What is the most likely diagnosis?
\vspace{12pt}

\textbf{Options:}
\begin{enumerate}
\item[A.] Chronic venous insufficiency
\item[B.] Reiter syndrome
\item[C.] Gunshot wound
\item[D.] Chronic renal failure
\item[E.] Pseudohypoparathyroidism
\end{enumerate}

\textbf{Image:}
\begin{center}
\includegraphics[width=0.31\textwidth,height=0.50\textheight,width=0.90\textwidth,keepaspectratio]{images/nejm_20080221.jpg}
\end{center}
\vspace{12pt}
\newpage

\section*{Question 266 (ID: 20080228)}
\textbf{Date: }February 28,2008
\vspace{6pt}

What is the most likely diagnosis?
\vspace{12pt}

\textbf{Options:}
\begin{enumerate}
\item[A.] Cisplatin overdose
\item[B.] Lesch-Nyhan syndrome
\item[C.] Rhabdomyolysis
\item[D.] Primary hyperparathyroidism
\item[E.] Ethylene glycol poisoning
\end{enumerate}

\textbf{Image:}
\begin{center}
\includegraphics[width=0.95\textwidth,height=0.50\textheight,width=0.90\textwidth,keepaspectratio]{images/nejm_20080228.jpg}
\end{center}
\vspace{12pt}
\newpage

\section*{Question 267 (ID: 20070419)}
\textbf{Date: }April 19,2007
\vspace{6pt}

What is the diagnosis?
\vspace{12pt}

\textbf{Options:}
\begin{enumerate}
\item[A.] Syphilis
\item[B.] Phenytoin exposure
\item[C.] Marfan syndrome
\item[D.] Torus palatinus
\item[E.] Kaposi sarcoma
\end{enumerate}

\textbf{Image:}
\begin{center}
\includegraphics[width=0.95\textwidth,height=0.50\textheight,width=0.90\textwidth,keepaspectratio]{images/nejm_20070419.jpg}
\end{center}
\vspace{12pt}
\newpage

\section*{Question 268 (ID: 20080306)}
\textbf{Date: }March 06,2008
\vspace{6pt}

Treatment with which one of the following medications is associated with this clinical finding?
\vspace{12pt}

\textbf{Options:}
\begin{enumerate}
\item[A.] Erlotinib
\item[B.] Leflunomide
\item[C.] Methotrexate
\item[D.] Pegvisomant
\item[E.] Psoralen
\end{enumerate}

\textbf{Image:}
\begin{center}
\includegraphics[width=0.95\textwidth,height=0.50\textheight,width=0.90\textwidth,keepaspectratio]{images/nejm_20080306.jpg}
\end{center}
\vspace{12pt}
\newpage

\section*{Question 269 (ID: 20080313)}
\textbf{Date: }March 13,2008
\vspace{6pt}

What is the most likely diagnosis?
\vspace{12pt}

\textbf{Options:}
\begin{enumerate}
\item[A.] Paget's disease
\item[B.] Meningioma
\item[C.] Neurocysticercosis
\item[D.] Pneumocephalus
\item[E.] Hyperparathyroidism
\end{enumerate}

\textbf{Image:}
\begin{center}
\includegraphics[width=0.68\textwidth,height=0.50\textheight,width=0.90\textwidth,keepaspectratio]{images/nejm_20080313.jpg}
\end{center}
\vspace{12pt}
\newpage

\section*{Question 270 (ID: 20080320)}
\textbf{Date: }March 20,2008
\vspace{6pt}

What is the diagnosis?
\vspace{12pt}

\textbf{Options:}
\begin{enumerate}
\item[A.] Small-bowel obstruction
\item[B.] Echinococcosis
\item[C.] Mesenteric ischemia
\item[D.] Pancreatic pseudocysts
\item[E.] Cecal volvulus
\end{enumerate}

\textbf{Image:}
\begin{center}
\includegraphics[width=0.95\textwidth,height=0.50\textheight,width=0.90\textwidth,keepaspectratio]{images/nejm_20080320.jpg}
\end{center}
\vspace{12pt}
\newpage

\section*{Question 271 (ID: 20080327)}
\textbf{Date: }March 27,2008
\vspace{6pt}

What is the diagnosis?
\vspace{12pt}

\textbf{Options:}
\begin{enumerate}
\item[A.] Behçet's syndrome
\item[B.] Lichen simplex chronicus
\item[C.] Condyloma acuminatum
\item[D.] Lichen sclerosus
\item[E.] Vestibular papillomatosis
\end{enumerate}

\textbf{Image:}
\begin{center}
\includegraphics[width=0.85\textwidth,height=0.50\textheight,width=0.90\textwidth,keepaspectratio]{images/nejm_20080327.jpg}
\end{center}
\vspace{12pt}
\newpage

\section*{Question 272 (ID: 20080403)}
\textbf{Date: }April 03,2008
\vspace{6pt}

A patient with this tomogram would be most likely to present with which one of the following signs?
\vspace{12pt}

\textbf{Options:}
\begin{enumerate}
\item[A.] Uniocular blindness
\item[B.] Hemiplegia
\item[C.] Alexia without agraphia
\item[D.] Hemiballismus
\item[E.] Internuclear ophthalmoplegia
\end{enumerate}

\textbf{Image:}
\begin{center}
\includegraphics[width=0.86\textwidth,height=0.50\textheight,width=0.90\textwidth,keepaspectratio]{images/nejm_20080403.jpg}
\end{center}
\vspace{12pt}
\newpage

\section*{Question 273 (ID: 20080410)}
\textbf{Date: }April 10,2008
\vspace{6pt}

This rash appeared following treatment for leukemia. What is the diagnosis?
\vspace{12pt}

\textbf{Options:}
\begin{enumerate}
\item[A.] Cryoglobulinemia
\item[B.] Leukemia cutis
\item[C.] Herpes zoster
\item[D.] Graft-versus-host disease
\item[E.] Urticaria pigmentosa
\end{enumerate}

\textbf{Image:}
\begin{center}
\includegraphics[width=0.95\textwidth,height=0.50\textheight,width=0.90\textwidth,keepaspectratio]{images/nejm_20080410.jpg}
\end{center}
\vspace{12pt}
\newpage

\section*{Question 274 (ID: 20080417)}
\textbf{Date: }April 17,2008
\vspace{6pt}

What is the diagnosis?
\vspace{12pt}

\textbf{Options:}
\begin{enumerate}
\item[A.] Babesiosis
\item[B.] Iron deficiency anemia
\item[C.] Hereditary spherocytosis
\item[D.] Malaria
\item[E.] Sideroblastic anemia
\end{enumerate}

\textbf{Image:}
\begin{center}
\includegraphics[width=0.73\textwidth,height=0.50\textheight,width=0.90\textwidth,keepaspectratio]{images/nejm_20080417.jpg}
\end{center}
\vspace{12pt}
\newpage

\section*{Question 275 (ID: 20080424)}
\textbf{Date: }April 24,2008
\vspace{6pt}

What is the diagnosis?
\vspace{12pt}

\textbf{Options:}
\begin{enumerate}
\item[A.] Central retinal artery occlusion
\item[B.] Diabetic papillopathy
\item[C.] Ocular toxoplasmosis
\item[D.] Optic neuritis
\item[E.] Malignant hypertension
\end{enumerate}

\textbf{Image:}
\begin{center}
\includegraphics[width=0.95\textwidth,height=0.50\textheight,width=0.90\textwidth,keepaspectratio]{images/nejm_20080424.jpg}
\end{center}
\vspace{12pt}
\newpage

\section*{Question 276 (ID: 20080501)}
\textbf{Date: }May 01,2008
\vspace{6pt}

Which structure is most dilated?
\vspace{12pt}

\textbf{Options:}
\begin{enumerate}
\item[B.] Left atrium
\item[C.] Left ventricle
\item[D.] Right atrium
\item[E.] Right ventricle
\end{enumerate}

\textbf{Image:}
\begin{center}
\includegraphics[width=0.95\textwidth,height=0.50\textheight,width=0.90\textwidth,keepaspectratio]{images/nejm_20080501.jpg}
\end{center}
\vspace{12pt}
\newpage

\section*{Question 277 (ID: 20080508)}
\textbf{Date: }May 08,2008
\vspace{6pt}

What is the diagnosis?
\vspace{12pt}

\textbf{Options:}
\begin{enumerate}
\item[A.] Dermatopathia pigmentosa reticularis
\item[B.] Lichen planus
\item[C.] Psoriasis
\item[D.] Rubella
\item[E.] Keratoderma blennorrhagicum
\end{enumerate}

\textbf{Image:}
\begin{center}
\includegraphics[width=0.57\textwidth,height=0.50\textheight,width=0.90\textwidth,keepaspectratio]{images/nejm_20080508.jpg}
\end{center}
\vspace{12pt}
\newpage

\section*{Question 278 (ID: 20080515)}
\textbf{Date: }May 15,2008
\vspace{6pt}

What term is used to describe this finding?
\vspace{12pt}

\textbf{Options:}
\begin{enumerate}
\item[A.] Hyphema
\item[B.] Hypopyon
\item[C.] Iridocyclitis
\item[D.] Iridodonesis
\item[E.] Synechia
\end{enumerate}

\textbf{Image:}
\begin{center}
\includegraphics[width=0.95\textwidth,height=0.50\textheight,width=0.90\textwidth,keepaspectratio]{images/nejm_20080515.jpg}
\end{center}
\vspace{12pt}
\newpage

\section*{Question 279 (ID: 20080522)}
\textbf{Date: }May 22,2008
\vspace{6pt}

What is the underlying diagnosis?
\vspace{12pt}

\textbf{Options:}
\begin{enumerate}
\item[A.] Addison's disease
\item[B.] Marfan syndrome
\item[C.] Syphilis
\item[D.] Takayasu's arteritis
\item[E.] Osteoporosis
\end{enumerate}

\textbf{Image:}
\begin{center}
\includegraphics[width=0.74\textwidth,height=0.50\textheight,width=0.90\textwidth,keepaspectratio]{images/nejm_20080522.jpg}
\end{center}
\vspace{12pt}
\newpage

\section*{Question 280 (ID: 20080529)}
\textbf{Date: }May 29,2008
\vspace{6pt}

What is the diagnosis?
\vspace{12pt}

\textbf{Options:}
\begin{enumerate}
\item[A.] Ludwig's angina
\item[B.] Glossopharyngeal nerve palsy
\item[C.] Pharyngeal gonorrhea
\item[D.] Bilateral peritonsillar abscesses
\item[E.] Infectious mononucleosis
\end{enumerate}

\textbf{Image:}
\begin{center}
\includegraphics[width=0.95\textwidth,height=0.50\textheight,width=0.90\textwidth,keepaspectratio]{images/nejm_20080529.jpg}
\end{center}
\vspace{12pt}
\newpage

\section*{Question 281 (ID: 20080605)}
\textbf{Date: }June 05,2008
\vspace{6pt}

This smoker presented with a four-year history of blanching of the fingers on exposure to cold. What is the diagnosis?
\vspace{12pt}

\textbf{Options:}
\begin{enumerate}
\item[A.] Thromboangiitis obliterans
\item[B.] Marantic endocarditis
\item[C.] Kawasaki disease
\item[D.] Brachial entrapment syndrome
\item[E.] Takayasu's arteritis
\end{enumerate}

\textbf{Image:}
\begin{center}
\includegraphics[width=0.92\textwidth,height=0.50\textheight,width=0.90\textwidth,keepaspectratio]{images/nejm_20080605.jpg}
\end{center}
\vspace{12pt}
\newpage

\section*{Question 282 (ID: 20080612)}
\textbf{Date: }June 12,2008
\vspace{6pt}

Which one of the following patterns of visual disturbance would be predicted to be demonstrable on examination of this patient?
\vspace{12pt}

\textbf{Options:}
\begin{enumerate}
\item[A.] Inferior hemifield loss
\item[B.] Temporal quadrantanopsia
\item[C.] Uniocular blindness
\item[D.] Macular sparing hemianopia
\item[E.] Peripheral ring scotoma
\end{enumerate}

\textbf{Image:}
\begin{center}
\includegraphics[width=0.83\textwidth,height=0.50\textheight,width=0.90\textwidth,keepaspectratio]{images/nejm_20080612.jpg}
\end{center}
\vspace{12pt}
\newpage

\section*{Question 283 (ID: 20080619)}
\textbf{Date: }June 19,2008
\vspace{6pt}

What is the diagnosis?
\vspace{12pt}

\textbf{Options:}
\begin{enumerate}
\item[A.] Osgood-Schlatter disease
\item[B.] Cystinosis
\item[C.] Systemic sclerosis
\item[D.] Multiple myeloma
\item[E.] Paget's disease
\end{enumerate}

\textbf{Image:}
\begin{center}
\includegraphics[width=0.95\textwidth,height=0.50\textheight,width=0.90\textwidth,keepaspectratio]{images/nejm_20080619.jpg}
\end{center}
\vspace{12pt}
\newpage

\section*{Question 284 (ID: 20080626)}
\textbf{Date: }June 26,2008
\vspace{6pt}

What is the most likely diagnosis?
\vspace{12pt}

\textbf{Options:}
\begin{enumerate}
\item[A.] Renal tubular acidosis
\item[B.] Primary hypoparathyroidism
\item[C.] Familial hypocalciuric hypercalcemia
\item[D.] Salicylate overdose
\item[E.] Paget's disease
\end{enumerate}

\textbf{Image:}
\begin{center}
\includegraphics[width=0.95\textwidth,height=0.50\textheight,width=0.90\textwidth,keepaspectratio]{images/nejm_20080626.jpg}
\end{center}
\vspace{12pt}
\newpage

\section*{Question 285 (ID: 20080703)}
\textbf{Date: }July 03,2008
\vspace{6pt}

What is the diagnosis?
\vspace{12pt}

\textbf{Options:}
\begin{enumerate}
\item[A.] Flail chest
\item[B.] Pectus arcuatum
\item[C.] Pectus carinatum
\item[D.] Pectus excavatum
\item[E.] Spondylocostal dysostosis
\end{enumerate}

\textbf{Image:}
\begin{center}
\includegraphics[width=0.95\textwidth,height=0.50\textheight,width=0.90\textwidth,keepaspectratio]{images/nejm_20080703.jpg}
\end{center}
\vspace{12pt}
\newpage

\section*{Question 286 (ID: 20080710)}
\textbf{Date: }July 10,2008
\vspace{6pt}

These lesions became more evident after the skin was illuminated with Wood's light. What diagnosis is suggested?
\vspace{12pt}

\textbf{Options:}
\begin{enumerate}
\item[A.] Pityriasis rosea
\item[B.] Melanoma
\item[C.] Tuberous sclerosis
\item[D.] Psoriasis
\item[E.] Vitiligo
\end{enumerate}

\textbf{Image:}
\begin{center}
\includegraphics[width=0.95\textwidth,height=0.50\textheight,width=0.90\textwidth,keepaspectratio]{images/nejm_20080710.jpg}
\end{center}
\vspace{12pt}
\newpage

\section*{Question 287 (ID: 20080717)}
\textbf{Date: }July 17,2008
\vspace{6pt}

What is the diagnosis?
\vspace{12pt}

\textbf{Options:}
\begin{enumerate}
\item[A.] Ankylosing spondylitis
\item[B.] Castleman disease
\item[C.] Metastatic prostate cancer
\item[D.] Osteopetrosis
\item[E.] Primary hyperparathyroidism
\end{enumerate}

\textbf{Image:}
\begin{center}
\includegraphics[width=0.94\textwidth,height=0.50\textheight,width=0.90\textwidth,keepaspectratio]{images/nejm_20080717.jpg}
\end{center}
\vspace{12pt}
\newpage

\section*{Question 288 (ID: 20080724)}
\textbf{Date: }July 24,2008
\vspace{6pt}

What is the most likely diagnosis?
\vspace{12pt}

\textbf{Options:}
\begin{enumerate}
\item[A.] Amyloidosis
\item[B.] Celiac disease
\item[C.] Hypothyroidism
\item[D.] Kawasaki disease
\item[E.] Type 2 diabetes
\end{enumerate}

\textbf{Image:}
\begin{center}
\includegraphics[width=0.95\textwidth,height=0.50\textheight,width=0.90\textwidth,keepaspectratio]{images/nejm_20080724.jpg}
\end{center}
\vspace{12pt}
\newpage

\section*{Question 289 (ID: 20080731)}
\textbf{Date: }July 31,2008
\vspace{6pt}

This patient had left knee pain. What is the diagnosis?
\vspace{12pt}

\textbf{Options:}
\begin{enumerate}
\item[A.] Acanthosis nigricans
\item[B.] Erythema ab igne
\item[C.] Lymphangitis
\item[D.] Mycosis fungoides
\item[E.] Livedo reticularis
\end{enumerate}

\textbf{Image:}
\begin{center}
\includegraphics[width=0.9\textwidth,height=0.50\textheight,width=0.90\textwidth,keepaspectratio]{images/nejm_20080731.jpg}
\end{center}
\vspace{12pt}
\newpage

\section*{Question 290 (ID: 20080807)}
\textbf{Date: }August 07,2008
\vspace{6pt}

This 35-year-old pet shop worker developed progressively spreading nodular lesions. He was afebrile. What is the most likely causative organism?
\vspace{12pt}

\textbf{Options:}
\begin{enumerate}
\item[A.] Acinetobacter baumannii
\item[B.] Erysipelothrix rhusiopathiae
\item[C.] Mycobacterium marinum
\item[D.] Pasteurella multocida
\item[E.] Staphylococcus epidermidis
\end{enumerate}

\textbf{Image:}
\begin{center}
\includegraphics[width=0.95\textwidth,height=0.50\textheight,width=0.90\textwidth,keepaspectratio]{images/nejm_20080807.jpg}
\end{center}
\vspace{12pt}
\newpage

\section*{Question 291 (ID: 20080814)}
\textbf{Date: }August 14,2008
\vspace{6pt}

What is the diagnosis?
\vspace{12pt}

\textbf{Options:}
\begin{enumerate}
\item[A.] Psoriatic arthropathy
\item[B.] Reflex sympathetic dystrophy
\item[C.] Osteoarthritis
\item[E.] Rheumatoid arthritis
\end{enumerate}

\textbf{Image:}
\begin{center}
\includegraphics[width=0.68\textwidth,height=0.50\textheight,width=0.90\textwidth,keepaspectratio]{images/nejm_20080814.jpg}
\end{center}
\vspace{12pt}
\newpage

\section*{Question 292 (ID: 20080821)}
\textbf{Date: }August 21,2008
\vspace{6pt}

What is the diagnosis?
\vspace{12pt}

\textbf{Options:}
\begin{enumerate}
\item[A.] Basal-cell carcinoma
\item[B.] Granuloma fissuratum
\item[C.] Melanoma
\item[D.] Seborrheic keratosis
\item[E.] Squamous-cell carcinoma
\end{enumerate}

\textbf{Image:}
\begin{center}
\includegraphics[width=0.6\textwidth,height=0.50\textheight,width=0.90\textwidth,keepaspectratio]{images/nejm_20080821.jpg}
\end{center}
\vspace{12pt}
\newpage

\section*{Question 293 (ID: 20080828)}
\textbf{Date: }August 28,2008
\vspace{6pt}

This 6-day-old infant's mother had developed intensely pruritic lesions at 33 weeks of gestation. What is the diagnosis?
\vspace{12pt}

\textbf{Options:}
\begin{enumerate}
\item[A.] Congenital syphilis
\item[B.] Epidermolysis bullosa acquisita
\item[C.] Neonatal gonorrhea
\item[D.] Herpes gestationis
\item[E.] Pruritic plaques of pregnancy
\end{enumerate}

\textbf{Image:}
\begin{center}
\includegraphics[width=0.6\textwidth,height=0.50\textheight,width=0.90\textwidth,keepaspectratio]{images/nejm_20080828.jpg}
\end{center}
\vspace{12pt}
\newpage

\section*{Question 294 (ID: 20080904)}
\textbf{Date: }September 04,2008
\vspace{6pt}

This worm was identified in an endotracheal aspirate of a patient with pulmonary infiltrates. What is the infecting organism?
\vspace{12pt}

\textbf{Options:}
\begin{enumerate}
\item[A.] Ascaris lumbricoides
\item[B.] Clonorchis sinensis
\item[C.] Paragonimus westermani
\item[D.] Strongyloides stercoralis
\item[E.] Toxoplasma gondii
\end{enumerate}

\textbf{Image:}
\begin{center}
\includegraphics[width=0.95\textwidth,height=0.50\textheight,width=0.90\textwidth,keepaspectratio]{images/nejm_20080904.jpg}
\end{center}
\vspace{12pt}
\newpage

\section*{Question 295 (ID: 20080911)}
\textbf{Date: }September 11,2008
\vspace{6pt}

What is the diagnosis?
\vspace{12pt}

\textbf{Options:}
\begin{enumerate}
\item[A.] Aspiration pneumonia
\item[B.] Sarcoidosis
\item[C.] Silicosis
\item[D.] Idiopathic pulmonary fibrosis
\item[E.] Lymphangioleiomyomatosis
\end{enumerate}

\textbf{Image:}
\begin{center}
\includegraphics[width=0.95\textwidth,height=0.50\textheight,width=0.90\textwidth,keepaspectratio]{images/nejm_20080911.jpg}
\end{center}
\vspace{12pt}
\newpage

\section*{Question 296 (ID: 20080918)}
\textbf{Date: }September 18,2008
\vspace{6pt}

What is the diagnosis?
\vspace{12pt}

\textbf{Options:}
\begin{enumerate}
\item[A.] Condylomata lata
\item[B.] Neurofibromatosis type 1
\item[C.] Keloid
\item[D.] von Hippel-Lindau disease
\item[E.] Tuberous sclerosis
\end{enumerate}

\textbf{Image:}
\begin{center}
\includegraphics[width=0.69\textwidth,height=0.50\textheight,width=0.90\textwidth,keepaspectratio]{images/nejm_20080918.jpg}
\end{center}
\vspace{12pt}
\newpage

\section*{Question 297 (ID: 20080925)}
\textbf{Date: }September 25,2008
\vspace{6pt}

What is the diagnosis?
\vspace{12pt}

\textbf{Options:}
\begin{enumerate}
\item[A.] Asbestosis
\item[B.] Colonic interposition surgery
\item[C.] Suppurative mediastinitis
\item[D.] Pneumopericardium
\item[E.] Thoracic aortic aneurysm
\end{enumerate}

\textbf{Image:}
\begin{center}
\includegraphics[width=0.95\textwidth,height=0.50\textheight,width=0.90\textwidth,keepaspectratio]{images/nejm_20080925.jpg}
\end{center}
\vspace{12pt}
\newpage

\section*{Question 298 (ID: 20081002)}
\textbf{Date: }October 02,2008
\vspace{6pt}

This 8-year-old girl had a 2-year history of abdominal pain. What is the diagnosis?
\vspace{12pt}

\textbf{Options:}
\begin{enumerate}
\item[A.] Crohn's disease
\item[B.] Enteric duplication cyst
\item[C.] Pancreatic pseudocyst
\item[D.] Intestinal malrotation
\item[E.] Wilms' tumor
\end{enumerate}

\textbf{Image:}
\begin{center}
\includegraphics[width=0.95\textwidth,height=0.50\textheight,width=0.90\textwidth,keepaspectratio]{images/nejm_20081002.jpg}
\end{center}
\vspace{12pt}
\newpage

\section*{Question 299 (ID: 20081009)}
\textbf{Date: }October 09,2008
\vspace{6pt}

What is the diagnosis?
\vspace{12pt}

\textbf{Options:}
\begin{enumerate}
\item[A.] Amelanotic melanoma
\item[B.] Cicatricial pemphigoid
\item[C.] Keratoacanthoma
\item[D.] Hidrocystoma
\item[E.] Papular mucinosis
\end{enumerate}

\textbf{Image:}
\begin{center}
\includegraphics[width=0.8\textwidth,height=0.50\textheight,width=0.90\textwidth,keepaspectratio]{images/nejm_20081009.jpg}
\end{center}
\vspace{12pt}
\newpage

\section*{Question 300 (ID: 20081016)}
\textbf{Date: }October 16,2008
\vspace{6pt}

What diagnosis is suggested by this corneal photograph?
\vspace{12pt}

\textbf{Options:}
\begin{enumerate}
\item[A.] Anterior uveitis
\item[B.] Chlamydia trachomatis infection
\item[C.] Cytomegalovirus retinitis
\item[D.] Herpes simplex virus infection
\item[E.] Toxocariasis
\end{enumerate}

\textbf{Image:}
\begin{center}
\includegraphics[width=0.95\textwidth,height=0.50\textheight,width=0.90\textwidth,keepaspectratio]{images/nejm_20081016.jpg}
\end{center}
\vspace{12pt}
\newpage

\section*{Question 301 (ID: 20081023)}
\textbf{Date: }October 23,2008
\vspace{6pt}

What is the diagnosis?
\vspace{12pt}

\textbf{Options:}
\begin{enumerate}
\item[A.] Chronic paronychia
\item[B.] Dermatomyositis
\item[C.] Selenium deficiency
\item[D.] Rheumatoid arthritis
\item[E.] Psoriasis
\end{enumerate}

\textbf{Image:}
\begin{center}
\includegraphics[width=0.51\textwidth,height=0.50\textheight,width=0.90\textwidth,keepaspectratio]{images/nejm_20081023.jpg}
\end{center}
\vspace{12pt}
\newpage

\section*{Question 302 (ID: 20081030)}
\textbf{Date: }October 30,2008
\vspace{6pt}

This man had cervical adenopathy. What is the most likely diagnosis?
\vspace{12pt}

\textbf{Options:}
\begin{enumerate}
\item[A.] Arteriovenous malformation
\item[B.] Cellulitis
\item[C.] Graves' disease
\item[D.] Lymphoma
\item[E.] Orbital fracture
\end{enumerate}

\textbf{Image:}
\begin{center}
\includegraphics[width=0.95\textwidth,height=0.50\textheight,width=0.90\textwidth,keepaspectratio]{images/nejm_20081030.jpg}
\end{center}
\vspace{12pt}
\newpage

\section*{Question 303 (ID: 20081106)}
\textbf{Date: }November 06,2008
\vspace{6pt}

What is the diagnosis?
\vspace{12pt}

\textbf{Options:}
\begin{enumerate}
\item[A.] Angioid streaks
\item[B.] Glaucoma
\item[C.] Macular degeneration
\item[D.] Ocular myiasis
\item[E.] Pseudoxanthoma elasticum
\end{enumerate}

\textbf{Image:}
\begin{center}
\includegraphics[width=0.95\textwidth,height=0.50\textheight,width=0.90\textwidth,keepaspectratio]{images/nejm_20081106.jpg}
\end{center}
\vspace{12pt}
\newpage

\section*{Question 304 (ID: 20081113)}
\textbf{Date: }November 13,2008
\vspace{6pt}

What diagnosis is suggested by this barium swallow?
\vspace{12pt}

\textbf{Options:}
\begin{enumerate}
\item[A.] Ingested foreign body
\item[B.] Esophageal diverticula
\item[C.] Diffuse esophageal spasm
\item[D.] Gastric linitis plastica
\item[E.] Esophageal carcinoma
\end{enumerate}

\textbf{Image:}
\begin{center}
\includegraphics[width=0.95\textwidth,height=0.50\textheight,width=0.90\textwidth,keepaspectratio]{images/nejm_20081113.jpg}
\end{center}
\vspace{12pt}
\newpage

\section*{Question 305 (ID: 20081127)}
\textbf{Date: }November 27,2008
\vspace{6pt}

This patient with chronic alcoholism presented with dysarthria and horizontal nystagmus. What is the diagnosis?
\vspace{12pt}

\textbf{Options:}
\begin{enumerate}
\item[A.] Brainstem glioma
\item[B.] Central pontine myelinolysis
\item[C.] Neurosarcoidosis
\item[D.] Pontine stroke
\item[E.] Tabes dorsalis
\end{enumerate}

\textbf{Image:}
\begin{center}
\includegraphics[width=0.9\textwidth,height=0.50\textheight,width=0.90\textwidth,keepaspectratio]{images/nejm_20081127.jpg}
\end{center}
\vspace{12pt}
\newpage

\section*{Question 306 (ID: 20081204)}
\textbf{Date: }December 04,2008
\vspace{6pt}

This patient recently returned from Brazil. What is the diagnosis?
\vspace{12pt}

\textbf{Options:}
\begin{enumerate}
\item[A.] Dracunculiasis
\item[B.] Ingrown toenail
\item[C.] Leptospirosis
\item[D.] Scabies
\item[E.] Tungiasis
\end{enumerate}

\textbf{Image:}
\begin{center}
\includegraphics[width=0.95\textwidth,height=0.50\textheight,width=0.90\textwidth,keepaspectratio]{images/nejm_20081204.jpg}
\end{center}
\vspace{12pt}
\newpage

\section*{Question 307 (ID: 20081211)}
\textbf{Date: }December 11,2008
\vspace{6pt}

What is the most likely diagnosis?
\vspace{12pt}

\textbf{Options:}
\begin{enumerate}
\item[A.] Alport syndrome
\item[B.] Angelman syndrome
\item[C.] Fragile X syndrome
\item[D.] Turner syndrome
\item[E.] Williams syndrome
\end{enumerate}

\textbf{Image:}
\begin{center}
\includegraphics[width=0.82\textwidth,height=0.50\textheight,width=0.90\textwidth,keepaspectratio]{images/nejm_20081211.jpg}
\end{center}
\vspace{12pt}
\newpage

\section*{Question 308 (ID: 20081218)}
\textbf{Date: }December 18,2008
\vspace{6pt}

What endocrine diagnosis is affecting the identical twin on the right side of the image?
\vspace{12pt}

\textbf{Options:}
\begin{enumerate}
\item[A.] Acromegaly
\item[B.] Addison's disease
\item[C.] Cushing's syndrome
\item[D.] Hypogonadism
\item[E.] Hypothyroidism
\end{enumerate}

\textbf{Image:}
\begin{center}
\includegraphics[width=0.95\textwidth,height=0.50\textheight,width=0.90\textwidth,keepaspectratio]{images/nejm_20081218.jpg}
\end{center}
\vspace{12pt}
\newpage

\section*{Question 309 (ID: 20081225)}
\textbf{Date: }December 25,2008
\vspace{6pt}

What is the diagnosis?
\vspace{12pt}

\textbf{Options:}
\begin{enumerate}
\item[A.] Neuropathic arthropathy
\item[B.] Rheumatoid arthritis
\item[C.] Hemophilia
\item[D.] Chondrocalcinosis
\item[E.] Rickets
\end{enumerate}

\textbf{Image:}
\begin{center}
\includegraphics[width=0.59\textwidth,height=0.50\textheight,width=0.90\textwidth,keepaspectratio]{images/nejm_20081225.jpg}
\end{center}
\vspace{12pt}
\newpage

\section*{Question 310 (ID: 20090101)}
\textbf{Date: }January 01,2009
\vspace{6pt}

What diagnosis is suggested by this fundus photograph from a 5-month-old girl?
\vspace{12pt}

\textbf{Options:}
\begin{enumerate}
\item[A.] Exudative retinitis
\item[B.] Primary infantile glaucoma
\item[C.] Retinopathy of prematurity
\item[D.] Retinoblastoma
\item[E.] Shaken-baby syndrome
\end{enumerate}

\textbf{Image:}
\begin{center}
\includegraphics[width=0.95\textwidth,height=0.50\textheight,width=0.90\textwidth,keepaspectratio]{images/nejm_20090101.jpg}
\end{center}
\vspace{12pt}
\newpage

\section*{Question 311 (ID: 20090108)}
\textbf{Date: }January 08,2009
\vspace{6pt}

What is the diagnosis?
\vspace{12pt}

\textbf{Options:}
\begin{enumerate}
\item[A.] Spider angioma
\item[B.] Hereditary hemorrhagic telangiectasia
\item[C.] Pyogenic granuloma
\item[D.] Nodular melanoma
\item[E.] Cherry hemangioma
\end{enumerate}

\textbf{Image:}
\begin{center}
\includegraphics[width=0.95\textwidth,height=0.50\textheight,width=0.90\textwidth,keepaspectratio]{images/nejm_20090108.jpg}
\end{center}
\vspace{12pt}
\newpage

\section*{Question 312 (ID: 20090115)}
\textbf{Date: }January 15,2009
\vspace{6pt}

Which one of the following biochemical measures would be most likely to be elevated in this patient?
\vspace{12pt}

\textbf{Options:}
\begin{enumerate}
\item[A.] Alkaline phosphatase
\item[B.] Calcium
\item[C.] Ferritin
\item[D.] Phosphorus
\item[E.] 25-hydroxy-vitamin D
\end{enumerate}

\textbf{Image:}
\begin{center}
\includegraphics[width=0.57\textwidth,height=0.50\textheight,width=0.90\textwidth,keepaspectratio]{images/nejm_20090115.jpg}
\end{center}
\vspace{12pt}
\newpage

\section*{Question 313 (ID: 20090122)}
\textbf{Date: }January 22,2009
\vspace{6pt}

This patient presented following a high-speed motor vehicle crash. Which structure has been disrupted?
\vspace{12pt}

\textbf{Options:}
\begin{enumerate}
\item[B.] Diaphragm
\item[C.] Esophagus
\item[D.] Myocardium
\item[E.] Trachea
\end{enumerate}

\textbf{Image:}
\begin{center}
\includegraphics[width=0.91\textwidth,height=0.50\textheight,width=0.90\textwidth,keepaspectratio]{images/nejm_20090122.jpg}
\end{center}
\vspace{12pt}
\newpage

\section*{Question 314 (ID: 20090129)}
\textbf{Date: }January 29,2009
\vspace{6pt}

What is the diagnosis?
\vspace{12pt}

\textbf{Options:}
\begin{enumerate}
\item[A.] Epidural hematoma
\item[B.] Glioblastoma multiforme
\item[C.] Meningioma
\item[D.] Subarachnoid hemorrhage
\item[E.] Subdural hematoma
\end{enumerate}

\textbf{Image:}
\begin{center}
\includegraphics[width=0.73\textwidth,height=0.50\textheight,width=0.90\textwidth,keepaspectratio]{images/nejm_20090129.jpg}
\end{center}
\vspace{12pt}
\newpage

\section*{Question 315 (ID: 20090205)}
\textbf{Date: }February 05,2009
\vspace{6pt}

What is the diagnosis?
\vspace{12pt}

\textbf{Options:}
\begin{enumerate}
\item[A.] Contact dermatitis
\item[B.] Discoid lupus erythematosus
\item[C.] Melanoma
\item[D.] Nummular eczema
\item[E.] Tinea corporis
\end{enumerate}

\textbf{Image:}
\begin{center}
\includegraphics[width=0.95\textwidth,height=0.50\textheight,width=0.90\textwidth,keepaspectratio]{images/nejm_20090205.jpg}
\end{center}
\vspace{12pt}
\newpage

\section*{Question 316 (ID: 20090212)}
\textbf{Date: }February 12,2009
\vspace{6pt}

This 22-year-old man presented with a 1-month history of severe pubic itch that was worst at night. What is the most appropriate topical treatment?
\vspace{12pt}

\textbf{Options:}
\begin{enumerate}
\item[A.] Hydrocortisone
\item[B.] Hydroxyzine
\item[C.] Mupirocin
\item[D.] Permethrin
\item[E.] Selenium sulfide
\end{enumerate}

\textbf{Image:}
\begin{center}
\includegraphics[width=0.95\textwidth,height=0.50\textheight,width=0.90\textwidth,keepaspectratio]{images/nejm_20090212.jpg}
\end{center}
\vspace{12pt}
\newpage

\section*{Question 317 (ID: 20090219)}
\textbf{Date: }February 19,2009
\vspace{6pt}

What clinical presentation would be expected in this patient?
\vspace{12pt}

\textbf{Options:}
\begin{enumerate}
\item[A.] Asymmetrical mydriasis
\item[B.] Ataxic hemiparesis
\item[C.] Hypothermia
\item[D.] Quadriplegia
\item[E.] Upward gaze palsy
\end{enumerate}

\textbf{Image:}
\begin{center}
\includegraphics[width=0.76\textwidth,height=0.50\textheight,width=0.90\textwidth,keepaspectratio]{images/nejm_20090219.jpg}
\end{center}
\vspace{12pt}
\newpage

\section*{Question 318 (ID: 20090226)}
\textbf{Date: }February 26,2009
\vspace{6pt}

Serum levels of which one of the following laboratory tests would be expected to be most abnormal in this patient?
\vspace{12pt}

\textbf{Options:}
\begin{enumerate}
\item[A.] 17-hydroxyprogesterone
\item[B.] Angiotensin-converting enzyme
\item[C.] Anti-tissue transglutaminase antibody
\item[D.] Prolactin
\item[E.] Vitamin B6
\end{enumerate}

\textbf{Image:}
\begin{center}
\includegraphics[width=0.95\textwidth,height=0.50\textheight,width=0.90\textwidth,keepaspectratio]{images/nejm_20090226.jpg}
\end{center}
\vspace{12pt}
\newpage

\section*{Question 319 (ID: 20090305)}
\textbf{Date: }March 05,2009
\vspace{6pt}

Which one of the following conditions is the most likely to be responsible for this clinical picture?
\vspace{12pt}

\textbf{Options:}
\begin{enumerate}
\item[A.] Excessive fluoride supplementation
\item[B.] Hyperbilirubinemia
\item[C.] Treatment with tetracycline
\item[D.] Trichophyton rubrum infection
\item[E.] Pseudomonas aeruginosa infection
\end{enumerate}

\textbf{Image:}
\begin{center}
\includegraphics[width=0.79\textwidth,height=0.50\textheight,width=0.90\textwidth,keepaspectratio]{images/nejm_20090305.jpg}
\end{center}
\vspace{12pt}
\newpage

\section*{Question 320 (ID: 20090312)}
\textbf{Date: }March 12,2009
\vspace{6pt}

Where is the abnormality on this chest radiograph?
\vspace{12pt}

\textbf{Options:}
\begin{enumerate}
\item[A.] Left lower lobe
\item[B.] Left upper lobe
\item[C.] Right lower lobe
\item[D.] Right upper lobe
\item[E.] Superior mediastinum
\end{enumerate}

\textbf{Image:}
\begin{center}
\includegraphics[width=0.69\textwidth,height=0.50\textheight,width=0.90\textwidth,keepaspectratio]{images/nejm_20090312.jpg}
\end{center}
\vspace{12pt}
\newpage

\section*{Question 321 (ID: 20090319)}
\textbf{Date: }March 19,2009
\vspace{6pt}

What is the diagnosis?
\vspace{12pt}

\textbf{Options:}
\begin{enumerate}
\item[A.] Coarctation of the aorta
\item[B.] Lung cancer
\item[C.] Pneumothorax
\item[D.] Rib fracture
\item[E.] Substernal goiter
\end{enumerate}

\textbf{Image:}
\begin{center}
\includegraphics[width=0.95\textwidth,height=0.50\textheight,width=0.90\textwidth,keepaspectratio]{images/nejm_20090319.jpg}
\end{center}
\vspace{12pt}
\newpage

\section*{Question 322 (ID: 20090326)}
\textbf{Date: }March 26,2009
\vspace{6pt}

The appearance of this Tunisian woman's ear is most consistent with which one of the following infectious diseases?
\vspace{12pt}

\textbf{Options:}
\begin{enumerate}
\item[A.] Leprosy
\item[B.] Leishmaniasis
\item[C.] Syphilis
\item[D.] Tuberculosis
\end{enumerate}

\textbf{Image:}
\begin{center}
\includegraphics[width=0.95\textwidth,height=0.50\textheight,width=0.90\textwidth,keepaspectratio]{images/nejm_20090326.jpg}
\end{center}
\vspace{12pt}
\newpage

\section*{Question 323 (ID: 20090402)}
\textbf{Date: }April 02,2009
\vspace{6pt}

What is the diagnosis?
\vspace{12pt}

\textbf{Options:}
\begin{enumerate}
\item[A.] Carcinoid syndrome
\item[B.] Dermatomyositis
\item[C.] Endocarditis
\item[D.] Lichen planus
\item[E.] Porphyria
\end{enumerate}

\textbf{Image:}
\begin{center}
\includegraphics[width=0.85\textwidth,height=0.50\textheight,width=0.90\textwidth,keepaspectratio]{images/nejm_20090402.jpg}
\end{center}
\vspace{12pt}
\newpage

\section*{Question 324 (ID: 20090409)}
\textbf{Date: }April 09,2009
\vspace{6pt}

What term is used to describe this finding?
\vspace{12pt}

\textbf{Options:}
\begin{enumerate}
\item[A.] Arc eye
\item[B.] Asthenopia
\item[C.] Choroideremia
\item[D.] Coloboma
\item[E.] Corectopia
\end{enumerate}

\textbf{Image:}
\begin{center}
\includegraphics[width=0.95\textwidth,height=0.50\textheight,width=0.90\textwidth,keepaspectratio]{images/nejm_20090409.jpg}
\end{center}
\vspace{12pt}
\newpage

\section*{Question 325 (ID: 20090416)}
\textbf{Date: }April 16,2009
\vspace{6pt}

What is the diagnosis?
\vspace{12pt}

\textbf{Options:}
\begin{enumerate}
\item[A.] Epidermolysis bullosa
\item[B.] Hereditary hemorrhagic telangiectasia
\item[C.] Neurofibromatosis
\item[D.] Peutz-Jeghers syndrome
\item[E.] Scleroderma
\end{enumerate}

\textbf{Image:}
\begin{center}
\includegraphics[width=0.95\textwidth,height=0.50\textheight,width=0.90\textwidth,keepaspectratio]{images/nejm_20090416.jpg}
\end{center}
\vspace{12pt}
\newpage

\section*{Question 326 (ID: 20090423)}
\textbf{Date: }April 23,2009
\vspace{6pt}

What is the diagnosis?
\vspace{12pt}

\textbf{Options:}
\begin{enumerate}
\item[A.] Hypercholesterolemia
\item[B.] Leukemia
\item[C.] Lichen planus
\item[D.] Pompe's disease
\item[E.] Psoriasis
\end{enumerate}

\textbf{Image:}
\begin{center}
\includegraphics[width=0.95\textwidth,height=0.50\textheight,width=0.90\textwidth,keepaspectratio]{images/nejm_20090423.jpg}
\end{center}
\vspace{12pt}
\newpage

\section*{Question 327 (ID: 20090430)}
\textbf{Date: }April 30,2009
\vspace{6pt}

What is the diagnosis?
\vspace{12pt}

\textbf{Options:}
\begin{enumerate}
\item[A.] Coxsackievirus infection
\item[B.] Gorlin's syndrome
\item[C.] Herpes zoster
\item[D.] Orbital cellulitis
\item[E.] Superficial pyoderma
\end{enumerate}

\textbf{Image:}
\begin{center}
\includegraphics[width=0.95\textwidth,height=0.50\textheight,width=0.90\textwidth,keepaspectratio]{images/nejm_20090430.jpg}
\end{center}
\vspace{12pt}
\newpage

\section*{Question 328 (ID: 20090507)}
\textbf{Date: }May 07,2009
\vspace{6pt}

Which one of the following medications is most likely to be responsible for this finding?
\vspace{12pt}

\textbf{Options:}
\begin{enumerate}
\item[A.] Amiodarone
\item[B.] Cinacalcet
\item[C.] Lithium
\item[D.] Palifermin
\item[E.] Trastuzumab
\end{enumerate}

\textbf{Image:}
\begin{center}
\includegraphics[width=0.95\textwidth,height=0.50\textheight,width=0.90\textwidth,keepaspectratio]{images/nejm_20090507.jpg}
\end{center}
\vspace{12pt}
\newpage

\section*{Question 329 (ID: 20090514)}
\textbf{Date: }May 14,2009
\vspace{6pt}

What is the diagnosis?
\vspace{12pt}

\textbf{Options:}
\begin{enumerate}
\item[A.] Babesiosis
\item[B.] Fabry's disease
\item[C.] Glucose-6-phosphate dehydrogenase deficiency
\item[D.] Pernicious anemia
\item[E.] Thalassemia minor
\end{enumerate}

\textbf{Image:}
\begin{center}
\includegraphics[width=0.95\textwidth,height=0.50\textheight,width=0.90\textwidth,keepaspectratio]{images/nejm_20090514.jpg}
\end{center}
\vspace{12pt}
\newpage

\section*{Question 330 (ID: 20090521)}
\textbf{Date: }May 21,2009
\vspace{6pt}

What is the most likely diagnosis?
\vspace{12pt}

\textbf{Options:}
\begin{enumerate}
\item[A.] Amyloidosis
\item[B.] Craniopharyngioma
\item[C.] Leukemia
\item[D.] Neuroblastoma
\item[E.] von Willebrand's disease
\end{enumerate}

\textbf{Image:}
\begin{center}
\includegraphics[width=0.95\textwidth,height=0.50\textheight,width=0.90\textwidth,keepaspectratio]{images/nejm_20090521.jpg}
\end{center}
\vspace{12pt}
\newpage

\section*{Question 331 (ID: 20090528)}
\textbf{Date: }May 28,2009
\vspace{6pt}

What is the diagnosis?
\vspace{12pt}

\textbf{Options:}
\begin{enumerate}
\item[A.] Aspergillosis
\item[B.] Hydatid disease
\item[C.] Plombage
\item[D.] Squamous cell carcinoma
\item[E.] Wegener's granulomatosis
\end{enumerate}

\textbf{Image:}
\begin{center}
\includegraphics[width=0.75\textwidth,height=0.50\textheight,width=0.90\textwidth,keepaspectratio]{images/nejm_20090528.jpg}
\end{center}
\vspace{12pt}
\newpage

\section*{Question 332 (ID: 20090604)}
\textbf{Date: }June 04,2009
\vspace{6pt}

What is the diagnosis?
\vspace{12pt}

\textbf{Options:}
\begin{enumerate}
\item[A.] Anaplastic thyroid carcinoma
\item[B.] Graves' disease
\item[C.] Hashimoto's thyroiditis
\item[D.] Medullary thyroid carcinoma
\item[E.] Thyroid lymphoma
\end{enumerate}

\textbf{Image:}
\begin{center}
\includegraphics[width=0.91\textwidth,height=0.50\textheight,width=0.90\textwidth,keepaspectratio]{images/nejm_20090604.jpg}
\end{center}
\vspace{12pt}
\newpage

\section*{Question 333 (ID: 20090611)}
\textbf{Date: }June 11,2009
\vspace{6pt}

Which one of the following signs would you expect to find in this 76-year-old woman with anemia?
\vspace{12pt}

\textbf{Options:}
\begin{enumerate}
\item[A.] Bulbar palsy
\item[B.] Flaccid paralysis of the upper extremities
\item[C.] Lower-extremity spasticity
\item[D.] Lower-extremity thermoanesthesia
\item[E.] Romberg's sign
\end{enumerate}

\textbf{Image:}
\begin{center}
\includegraphics[width=0.69\textwidth,height=0.50\textheight,width=0.90\textwidth,keepaspectratio]{images/nejm_20090611.jpg}
\end{center}
\vspace{12pt}
\newpage

\section*{Question 334 (ID: 20090618)}
\textbf{Date: }June 18,2009
\vspace{6pt}

What is the diagnosis?
\vspace{12pt}

\textbf{Options:}
\begin{enumerate}
\item[A.] Acromegaly
\item[B.] Amyloidosis
\item[C.] Lipomatosis
\item[D.] Mucopolysaccharidosis type 1
\item[E.] Multiple endocrine neoplasia type 2b
\end{enumerate}

\textbf{Image:}
\begin{center}
\includegraphics[width=0.95\textwidth,height=0.50\textheight,width=0.90\textwidth,keepaspectratio]{images/nejm_20090618.jpg}
\end{center}
\vspace{12pt}
\newpage

\section*{Question 335 (ID: 20090625)}
\textbf{Date: }June 25,2009
\vspace{6pt}

What is the diagnosis in this 56-year-old man with abdominal pain?
\vspace{12pt}

\textbf{Options:}
\begin{enumerate}
\item[A.] Acute pancreatitis
\item[B.] Colonic volvulus
\item[C.] Colonic carcinoma
\item[D.] Mesenteric ischemia
\item[E.] Pneumoperitoneum
\end{enumerate}

\textbf{Image:}
\begin{center}
\includegraphics[width=0.69\textwidth,height=0.50\textheight,width=0.90\textwidth,keepaspectratio]{images/nejm_20090625.jpg}
\end{center}
\vspace{12pt}
\newpage

\section*{Question 336 (ID: 20090702)}
\textbf{Date: }July 02,2009
\vspace{6pt}

What is the diagnosis?
\vspace{12pt}

\textbf{Options:}
\begin{enumerate}
\item[A.] Hemorrhagic episcleritis
\item[B.] Intravitreal hemorrhage
\item[C.] Loa loa
\item[D.] Osteogenesis imperfecta
\item[E.] Traumatic bleb
\end{enumerate}

\textbf{Image:}
\begin{center}
\includegraphics[width=0.95\textwidth,height=0.50\textheight,width=0.90\textwidth,keepaspectratio]{images/nejm_20090702.jpg}
\end{center}
\vspace{12pt}
\newpage

\section*{Question 337 (ID: 20090709)}
\textbf{Date: }July 09,2009
\vspace{6pt}

This blood smear was taken from an immunodeficient man with fever, night sweats, and weight loss. What is the diagnosis?
\vspace{12pt}

\textbf{Options:}
\begin{enumerate}
\item[A.] Candida albicans
\item[B.] Coccidioides immitis
\item[C.] Cryptococcus neoformans
\item[D.] Histoplasma capsulatum
\end{enumerate}

\textbf{Image:}
\begin{center}
\includegraphics[width=0.95\textwidth,height=0.50\textheight,width=0.90\textwidth,keepaspectratio]{images/nejm_20090709.jpg}
\end{center}
\vspace{12pt}
\newpage

\section*{Question 338 (ID: 20090716)}
\textbf{Date: }July 16,2009
\vspace{6pt}

What underlying disease is most likely to have been present in this patient?
\vspace{12pt}

\textbf{Options:}
\begin{enumerate}
\item[A.] Atrial fibrillation
\item[B.] Chronic constipation
\item[C.] Chronic renal insufficiency
\item[D.] Diabetes mellitus
\item[E.] Thalassemia
\end{enumerate}

\textbf{Image:}
\begin{center}
\includegraphics[width=0.69\textwidth,height=0.50\textheight,width=0.90\textwidth,keepaspectratio]{images/nejm_20090716.jpg}
\end{center}
\vspace{12pt}
\newpage

\section*{Question 339 (ID: 20090723)}
\textbf{Date: }July 23,2009
\vspace{6pt}

What is the diagnosis?
\vspace{12pt}

\textbf{Options:}
\begin{enumerate}
\item[A.] Digital pilonidal sinus
\item[C.] Paronychia
\item[D.] Scabies
\item[E.] Xanthomata
\end{enumerate}

\textbf{Image:}
\begin{center}
\includegraphics[width=0.95\textwidth,height=0.50\textheight,width=0.90\textwidth,keepaspectratio]{images/nejm_20090723.jpg}
\end{center}
\vspace{12pt}
\newpage

\section*{Question 340 (ID: 20090730)}
\textbf{Date: }July 30,2009
\vspace{6pt}

What physical findings would be expected to be present in this patient?
\vspace{12pt}

\textbf{Options:}
\begin{enumerate}
\item[A.] Quadriplegia with bilateral gaze paresis
\item[B.] Left hemiparesis, gaze deviated to the left
\item[C.] Left hemiparesis, gaze deviated to the right
\item[D.] Right hemiparesis, gaze deviated to the left
\item[E.] Right hemiparesis, gaze deviated to the right
\end{enumerate}

\textbf{Image:}
\begin{center}
\includegraphics[width=0.57\textwidth,height=0.50\textheight,width=0.90\textwidth,keepaspectratio]{images/nejm_20090730.jpg}
\end{center}
\vspace{12pt}
\newpage

\section*{Question 341 (ID: 20090806)}
\textbf{Date: }August 06,2009
\vspace{6pt}

This 6-week old presented with scrotal pain. What is the diagnosis?
\vspace{12pt}

\textbf{Options:}
\begin{enumerate}
\item[A.] Bilateral cryptorchidism
\item[B.] Bilateral inguinal hernias
\item[C.] Isolated right hydrocele
\item[D.] Left retractile testis
\item[E.] Left testicular torsion
\end{enumerate}

\textbf{Image:}
\begin{center}
\includegraphics[width=0.95\textwidth,height=0.50\textheight,width=0.90\textwidth,keepaspectratio]{images/nejm_20090806.jpg}
\end{center}
\vspace{12pt}
\newpage

\section*{Question 342 (ID: 20090813)}
\textbf{Date: }August 13,2009
\vspace{6pt}

What is the diagnosis?
\vspace{12pt}

\textbf{Options:}
\begin{enumerate}
\item[A.] Diverticulosis
\item[B.] Familial adenomatous polyposis
\item[C.] Foreign object ingestion
\item[D.] Pneumatosis cystoides coli
\item[E.] Trichinosis
\end{enumerate}

\textbf{Image:}
\begin{center}
\includegraphics[width=0.95\textwidth,height=0.50\textheight,width=0.90\textwidth,keepaspectratio]{images/nejm_20090813.jpg}
\end{center}
\vspace{12pt}
\newpage

\section*{Question 343 (ID: 20090820)}
\textbf{Date: }August 20,2009
\vspace{6pt}

What is the diagnosis?
\vspace{12pt}

\textbf{Options:}
\begin{enumerate}
\item[A.] Cytomegalovirus retinitis
\item[B.] Glaucoma
\item[C.] Malignant hypertension
\item[D.] Panretinal photocoagulation
\item[E.] Retinal vein thrombosis
\end{enumerate}

\textbf{Image:}
\begin{center}
\includegraphics[width=0.95\textwidth,height=0.50\textheight,width=0.90\textwidth,keepaspectratio]{images/nejm_20090820.jpg}
\end{center}
\vspace{12pt}
\newpage

\section*{Question 344 (ID: 20090827)}
\textbf{Date: }August 27,2009
\vspace{6pt}

What is the diagnosis?
\vspace{12pt}

\textbf{Options:}
\begin{enumerate}
\item[A.] Diverticular abscess
\item[B.] Hirschsprung's disease
\item[C.] Sigmoid volvulus
\item[D.] Small-bowel obstruction
\item[E.] Trichobezoar
\end{enumerate}

\textbf{Image:}
\begin{center}
\includegraphics[width=0.83\textwidth,height=0.50\textheight,width=0.90\textwidth,keepaspectratio]{images/nejm_20090827.jpg}
\end{center}
\vspace{12pt}
\newpage

\section*{Question 345 (ID: 20090903)}
\textbf{Date: }September 03,2009
\vspace{6pt}

This patient with a history of rheumatoid arthritis presented with a several-month history of a painful left calf. What is the diagnosis?
\vspace{12pt}

\textbf{Options:}
\begin{enumerate}
\item[A.] Baker's cyst
\item[B.] Gastrocnemius tear
\item[C.] Meniscal cyst
\item[D.] Lipoblastoma
\item[E.] Septic arthritis
\end{enumerate}

\textbf{Image:}
\begin{center}
\includegraphics[width=0.4\textwidth,height=0.50\textheight,width=0.90\textwidth,keepaspectratio]{images/nejm_20090903.jpg}
\end{center}
\vspace{12pt}
\newpage

\section*{Question 346 (ID: 20090910)}
\textbf{Date: }September 10,2009
\vspace{6pt}

Which one of the following medications is most characteristically associated with the illustrated finding?
\vspace{12pt}

\textbf{Options:}
\begin{enumerate}
\item[A.] Alendronate
\item[B.] Carbamazepine
\item[C.] Imatinib
\item[D.] Ropinirole
\item[E.] Zidovudine
\end{enumerate}

\textbf{Image:}
\begin{center}
\includegraphics[width=0.95\textwidth,height=0.50\textheight,width=0.90\textwidth,keepaspectratio]{images/nejm_20090910.jpg}
\end{center}
\vspace{12pt}
\newpage

\section*{Question 347 (ID: 20090917)}
\textbf{Date: }September 17,2009
\vspace{6pt}

This technetium-99m sulfur colloid scan was performed after the patient presented with abdominal pain. Howell-Jolly bodies were present on a peripheral-blood smear. What is the diagnosis?
\vspace{12pt}

\textbf{Options:}
\begin{enumerate}
\item[A.] Hemochromatosis
\item[B.] Hydatidiform mole
\item[C.] Multiple myeloma
\item[D.] Pelvic spleen
\item[E.] Uterine sarcoma
\end{enumerate}

\textbf{Image:}
\begin{center}
\includegraphics[width=0.95\textwidth,height=0.50\textheight,width=0.90\textwidth,keepaspectratio]{images/nejm_20090917.jpg}
\end{center}
\vspace{12pt}
\newpage

\section*{Question 348 (ID: 20090924)}
\textbf{Date: }September 24,2009
\vspace{6pt}

This patient presented with unilateral rhinorrhea. What is the diagnosis?
\vspace{12pt}

\textbf{Options:}
\begin{enumerate}
\item[A.] Nasal foreign body
\item[B.] Osteoma
\item[C.] Pituitary tumor
\item[D.] Skull fracture
\item[E.] Sinusitis
\end{enumerate}

\textbf{Image:}
\begin{center}
\includegraphics[width=0.66\textwidth,height=0.50\textheight,width=0.90\textwidth,keepaspectratio]{images/nejm_20090924.jpg}
\end{center}
\vspace{12pt}
\newpage

\section*{Question 349 (ID: 20091001)}
\textbf{Date: }October 01,2009
\vspace{6pt}

This patient had multiple myeloma. What diagnosis is suggested from this image of the axilla?
\vspace{12pt}

\textbf{Options:}
\begin{enumerate}
\item[A.] Acanthosis nigricans
\item[B.] Crusted scabies
\item[C.] Erysipelas
\item[D.] Malacoplakia
\item[E.] Psoriasis
\end{enumerate}

\textbf{Image:}
\begin{center}
\includegraphics[width=0.67\textwidth,height=0.50\textheight,width=0.90\textwidth,keepaspectratio]{images/nejm_20091001.jpg}
\end{center}
\vspace{12pt}
\newpage

\section*{Question 350 (ID: 20091008)}
\textbf{Date: }October 08,2009
\vspace{6pt}

This appearance was noticed during routine colonoscopy. What is the diagnosis?
\vspace{12pt}

\textbf{Options:}
\begin{enumerate}
\item[A.] Addison's disease
\item[B.] Hemochromatosis
\item[C.] Laxative use
\item[D.] Metastatic melanoma
\item[E.] von Hippel-Lindau disease
\end{enumerate}

\textbf{Image:}
\begin{center}
\includegraphics[width=0.85\textwidth,height=0.50\textheight,width=0.90\textwidth,keepaspectratio]{images/nejm_20091008.jpg}
\end{center}
\vspace{12pt}
\newpage

\section*{Question 351 (ID: 20091015)}
\textbf{Date: }October 15,2009
\vspace{6pt}

What is the diagnosis?
\vspace{12pt}

\textbf{Options:}
\begin{enumerate}
\item[A.] Acanthosis nigricans
\item[B.] Lymphogranuloma venereum
\item[C.] Seborrheic keratoses
\item[D.] Secondary syphilis
\item[E.] Bowenoid papulosis
\end{enumerate}

\textbf{Image:}
\begin{center}
\includegraphics[width=0.95\textwidth,height=0.50\textheight,width=0.90\textwidth,keepaspectratio]{images/nejm_20091015.jpg}
\end{center}
\vspace{12pt}
\newpage

\section*{Question 352 (ID: 20091022)}
\textbf{Date: }October 22,2009
\vspace{6pt}

What is the diagnosis?
\vspace{12pt}

\textbf{Options:}
\begin{enumerate}
\item[A.] Brachial plexopathy
\item[B.] Clavicular fracture
\item[C.] Lipodystrophy
\item[D.] Polyostotic fibrous dysplasia
\item[E.] Rupture of the trapezius muscle
\end{enumerate}

\textbf{Image:}
\begin{center}
\includegraphics[width=0.56\textwidth,height=0.50\textheight,width=0.90\textwidth,keepaspectratio]{images/nejm_20091022.jpg}
\end{center}
\vspace{12pt}
\newpage

\section*{Question 353 (ID: 20091029)}
\textbf{Date: }October 29,2009
\vspace{6pt}

What is the diagnosis?
\vspace{12pt}

\textbf{Options:}
\begin{enumerate}
\item[A.] Lipoatrophy
\item[B.] Myositis ossificans
\item[C.] Rhabdomyolysis
\item[D.] Sarcoma
\item[E.] Syphilitic gumma
\end{enumerate}

\textbf{Image:}
\begin{center}
\includegraphics[width=0.95\textwidth,height=0.50\textheight,width=0.90\textwidth,keepaspectratio]{images/nejm_20091029.jpg}
\end{center}
\vspace{12pt}
\newpage

\section*{Question 354 (ID: 20091105)}
\textbf{Date: }November 05,2009
\vspace{6pt}

What is the diagnosis?
\vspace{12pt}

\textbf{Options:}
\begin{enumerate}
\item[A.] Bowel infarction
\item[B.] Crohn's disease
\item[C.] Cystic fibrosis
\item[D.] Echinococcosis
\item[E.] Lead poisoning
\end{enumerate}

\textbf{Image:}
\begin{center}
\includegraphics[width=0.67\textwidth,height=0.50\textheight,width=0.90\textwidth,keepaspectratio]{images/nejm_20091105.jpg}
\end{center}
\vspace{12pt}
\newpage

\section*{Question 355 (ID: 20091112)}
\textbf{Date: }November 12,2009
\vspace{6pt}

What is the diagnosis?
\vspace{12pt}

\textbf{Options:}
\begin{enumerate}
\item[A.] Cellulitis
\item[C.] Osteoarthritis
\item[D.] Rheumatoid arthritis
\item[E.] Septic arthritis
\end{enumerate}

\textbf{Image:}
\begin{center}
\includegraphics[width=0.81\textwidth,height=0.50\textheight,width=0.90\textwidth,keepaspectratio]{images/nejm_20091112.jpg}
\end{center}
\vspace{12pt}
\newpage

\section*{Question 356 (ID: 20091119)}
\textbf{Date: }November 19,2009
\vspace{6pt}

What is the diagnosis?
\vspace{12pt}

\textbf{Options:}
\begin{enumerate}
\item[A.] Amebic liver abscess
\item[B.] Diffuse bilioma
\item[C.] Echinococcosis
\item[D.] Hepatocellular carcinoma
\item[E.] Polycystic liver disease
\end{enumerate}

\textbf{Image:}
\begin{center}
\includegraphics[width=0.95\textwidth,height=0.50\textheight,width=0.90\textwidth,keepaspectratio]{images/nejm_20091119.jpg}
\end{center}
\vspace{12pt}
\newpage

\section*{Question 357 (ID: 20091126)}
\textbf{Date: }November 26,2009
\vspace{6pt}

The appearance of this foot is most consistent with which one of the following diagnoses?
\vspace{12pt}

\textbf{Options:}
\begin{enumerate}
\item[A.] Buruli ulcer
\item[B.] Chromoblastomycosis
\item[C.] Leishmaniasis
\item[D.] Madura foot
\item[E.] Swimming pool granuloma
\end{enumerate}

\textbf{Image:}
\begin{center}
\includegraphics[width=0.95\textwidth,height=0.50\textheight,width=0.90\textwidth,keepaspectratio]{images/nejm_20091126.jpg}
\end{center}
\vspace{12pt}
\newpage

\section*{Question 358 (ID: 20091203)}
\textbf{Date: }December 03,2009
\vspace{6pt}

What is the diagnosis?
\vspace{12pt}

\textbf{Options:}
\begin{enumerate}
\item[A.] Central retinal vein occlusion
\item[B.] Cholesterol embolism
\item[C.] Retinoschisis
\item[D.] Temporal arteritis
\item[E.] Toxoplasmosis
\end{enumerate}

\textbf{Image:}
\begin{center}
\includegraphics[width=0.95\textwidth,height=0.50\textheight,width=0.90\textwidth,keepaspectratio]{images/nejm_20091203.jpg}
\end{center}
\vspace{12pt}
\newpage

\section*{Question 359 (ID: 20091210)}
\textbf{Date: }December 10,2009
\vspace{6pt}

What is the diagnosis?
\vspace{12pt}

\textbf{Options:}
\begin{enumerate}
\item[A.] Atlanto-occipital dislocation
\item[B.] Atlanto-axial subluxation
\item[C.] Pillar fracture
\item[D.] Spinous process avulsion
\item[E.] Wedge fracture
\end{enumerate}

\textbf{Image:}
\begin{center}
\includegraphics[width=0.52\textwidth,height=0.50\textheight,width=0.90\textwidth,keepaspectratio]{images/nejm_20091210.jpg}
\end{center}
\vspace{12pt}
\newpage

\section*{Question 360 (ID: 20091217)}
\textbf{Date: }December 17,2009
\vspace{6pt}

This patient had a history of breast-cancer surgery. Which one of the following rib abnormalities is present on her follow-up imaging?
\vspace{12pt}

\textbf{Options:}
\begin{enumerate}
\item[A.] Absent rib
\item[B.] Inferior rib notching
\item[C.] Rib metastases
\item[D.] Sternocostal anomaly
\item[E.] Supernumerary rib
\end{enumerate}

\textbf{Image:}
\begin{center}
\includegraphics[width=0.72\textwidth,height=0.50\textheight,width=0.90\textwidth,keepaspectratio]{images/nejm_20091217.jpg}
\end{center}
\vspace{12pt}
\newpage

\section*{Question 361 (ID: 20091224)}
\textbf{Date: }December 24,2009
\vspace{6pt}

What is the diagnosis?
\vspace{12pt}

\textbf{Options:}
\begin{enumerate}
\item[A.] Eczema
\item[B.] Herpetic whitlow
\item[C.] Impetigo
\item[D.] Scabies
\item[E.] Tinea pedis
\end{enumerate}

\textbf{Image:}
\begin{center}
\includegraphics[width=0.95\textwidth,height=0.50\textheight,width=0.90\textwidth,keepaspectratio]{images/nejm_20091224.jpg}
\end{center}
\vspace{12pt}
\newpage

\section*{Question 362 (ID: 20091231)}
\textbf{Date: }December 31,2009
\vspace{6pt}

This patient had a 4-year history of diabetes. What is the diagnosis?
\vspace{12pt}

\textbf{Options:}
\begin{enumerate}
\item[A.] Bullous disease of diabetes
\item[B.] Necrobiosis lipoidica diabeticorum
\item[C.] Necrolytic migratory erythema
\item[D.] Pigmented purpuric dermatosis
\item[E.] Scleredema of diabetes
\end{enumerate}

\textbf{Image:}
\begin{center}
\includegraphics[width=0.95\textwidth,height=0.50\textheight,width=0.90\textwidth,keepaspectratio]{images/nejm_20091231.jpg}
\end{center}
\vspace{12pt}
\newpage

\section*{Question 363 (ID: 20100107)}
\textbf{Date: }January 07,2010
\vspace{6pt}

What is the diagnosis?
\vspace{12pt}

\textbf{Options:}
\begin{enumerate}
\item[A.] Lichen planus
\item[B.] Mycosis fungoides
\item[C.] Ostraceous psoriasis
\item[D.] Paraneoplastic pemphigoid
\item[E.] Staphylococcal scalded skin syndrome
\end{enumerate}

\textbf{Image:}
\begin{center}
\includegraphics[width=0.63\textwidth,height=0.50\textheight,width=0.90\textwidth,keepaspectratio]{images/nejm_20100107.jpg}
\end{center}
\vspace{12pt}
\newpage

\section*{Question 364 (ID: 20100114)}
\textbf{Date: }January 14,2010
\vspace{6pt}

What is the diagnosis?
\vspace{12pt}

\textbf{Options:}
\begin{enumerate}
\item[A.] Beta-thalassemia
\item[B.] Eisenmenger syndrome
\item[C.] Metastatic small cell lung carcinoma
\item[D.] Sarcoidosis
\item[E.] Scleroderma
\end{enumerate}

\textbf{Image:}
\begin{center}
\includegraphics[width=0.95\textwidth,height=0.50\textheight,width=0.90\textwidth,keepaspectratio]{images/nejm_20100114.jpg}
\end{center}
\vspace{12pt}
\newpage

\section*{Question 365 (ID: 20100121)}
\textbf{Date: }January 21,2010
\vspace{6pt}

What is the diagnosis?
\vspace{12pt}

\textbf{Options:}
\begin{enumerate}
\item[A.] Cutaneous larva migrans
\item[B.] Dirofilariasis
\item[C.] Gnathostomiasis
\item[D.] Paragonimiasis
\item[E.] Toxocariasis
\end{enumerate}

\textbf{Image:}
\begin{center}
\includegraphics[width=0.95\textwidth,height=0.50\textheight,width=0.90\textwidth,keepaspectratio]{images/nejm_20100121.jpg}
\end{center}
\vspace{12pt}
\newpage

\section*{Question 366 (ID: 20100128)}
\textbf{Date: }January 28,2010
\vspace{6pt}

This patient with a history of alcohol abuse developed alopecia with fine, brittle scalp hair, diarrhea, and angular cheilitis. Measurement of which one of the following metals is most likely to be diagnostic?
\vspace{12pt}

\textbf{Options:}
\begin{enumerate}
\item[A.] Chromium
\item[B.] Copper
\item[C.] Manganese
\item[D.] Selenium
\end{enumerate}

\textbf{Image:}
\begin{center}
\includegraphics[width=0.95\textwidth,height=0.50\textheight,width=0.90\textwidth,keepaspectratio]{images/nejm_20100128.jpg}
\end{center}
\vspace{12pt}
\newpage

\section*{Question 367 (ID: 20100204)}
\textbf{Date: }February 04,2010
\vspace{6pt}

What is the diagnosis?
\vspace{12pt}

\textbf{Options:}
\begin{enumerate}
\item[A.] Cutaneous leishmaniasis
\item[B.] Leprosy
\item[C.] Leukemia cutis
\item[D.] Syphilis
\end{enumerate}

\textbf{Image:}
\begin{center}
\includegraphics[width=0.95\textwidth,height=0.50\textheight,width=0.90\textwidth,keepaspectratio]{images/nejm_20100204.jpg}
\end{center}
\vspace{12pt}
\newpage

\section*{Question 368 (ID: 20100211)}
\textbf{Date: }February 11,2010
\vspace{6pt}

What is the diagnosis?
\vspace{12pt}

\textbf{Options:}
\begin{enumerate}
\item[A.] Gastric outlet obstruction
\item[B.] Hirschsprung disease
\item[C.] Ileal intussusception
\item[D.] Ulcerative colitis
\item[E.] Zollinger-Ellison syndrome
\end{enumerate}

\textbf{Image:}
\begin{center}
\includegraphics[width=0.88\textwidth,height=0.50\textheight,width=0.90\textwidth,keepaspectratio]{images/nejm_20100211.jpg}
\end{center}
\vspace{12pt}
\newpage

\section*{Question 369 (ID: 20100218)}
\textbf{Date: }February 18,2010
\vspace{6pt}

This 12-year-old presented with decreased appetite and vomiting. What is the most likely diagnosis?
\vspace{12pt}

\textbf{Options:}
\begin{enumerate}
\item[A.] Endometrioma
\item[B.] Hernia
\item[C.] Keloid
\item[D.] Intraabdominal cancer
\item[E.] Urachal duct cyst
\end{enumerate}

\textbf{Image:}
\begin{center}
\includegraphics[width=0.6\textwidth,height=0.50\textheight,width=0.90\textwidth,keepaspectratio]{images/nejm_20100218.jpg}
\end{center}
\vspace{12pt}
\newpage

\section*{Question 370 (ID: 20100225)}
\textbf{Date: }February 25,2010
\vspace{6pt}

What is the diagnosis in this baby girl who was delivered at 30 weeks' gestation?
\vspace{12pt}

\textbf{Options:}
\begin{enumerate}
\item[A.] Congenital cytomegalovirus infection
\item[B.] Ecthyma gangrenosum
\item[C.] Homozygous protein C deficiency
\item[D.] Klippel-Trenaunay syndrome
\item[E.] Langerhans cell histiocytosis
\end{enumerate}

\textbf{Image:}
\begin{center}
\includegraphics[width=0.95\textwidth,height=0.50\textheight,width=0.90\textwidth,keepaspectratio]{images/nejm_20100225.jpg}
\end{center}
\vspace{12pt}
\newpage

\section*{Question 371 (ID: 20100304)}
\textbf{Date: }March 04,2010
\vspace{6pt}

What is the diagnosis?
\vspace{12pt}

\textbf{Options:}
\begin{enumerate}
\item[A.] Dermatobia hominis
\item[B.] Pediculus humanus capitis
\item[C.] Pediculus humanus corporis
\item[D.] Phthirus pubis
\item[E.] Sarcoptes scabiei
\end{enumerate}

\textbf{Image:}
\begin{center}
\includegraphics[width=0.95\textwidth,height=0.50\textheight,width=0.90\textwidth,keepaspectratio]{images/nejm_20100304.jpg}
\end{center}
\vspace{12pt}
\newpage

\section*{Question 372 (ID: 20100311)}
\textbf{Date: }March 11,2010
\vspace{6pt}

A woman underwent colonoscopy after presenting with colicky abdominal pain and loose stool. What type of worm is causing this presentation?
\vspace{12pt}

\textbf{Options:}
\begin{enumerate}
\item[A.] Ascaris lumbricoides
\item[B.] Diphyllobothrium latum
\item[C.] Necator americanus
\item[D.] Trichinella spiralis
\item[E.] Trichuris trichiura
\end{enumerate}

\textbf{Image:}
\begin{center}
\includegraphics[width=0.95\textwidth,height=0.50\textheight,width=0.90\textwidth,keepaspectratio]{images/nejm_20100311.jpg}
\end{center}
\vspace{12pt}
\newpage

\section*{Question 373 (ID: 20100318)}
\textbf{Date: }March 18,2010
\vspace{6pt}

What is the diagnosis?
\vspace{12pt}

\textbf{Options:}
\begin{enumerate}
\item[A.] Chemical burn
\item[B.] Cicatricial pemphigoid
\item[C.] Epidermomycosis
\item[D.] Herpes zoster
\item[E.] Squamous cell carcinoma
\end{enumerate}

\textbf{Image:}
\begin{center}
\includegraphics[width=0.63\textwidth,height=0.50\textheight,width=0.90\textwidth,keepaspectratio]{images/nejm_20100318.jpg}
\end{center}
\vspace{12pt}
\newpage

\section*{Question 374 (ID: 20100325)}
\textbf{Date: }March 25,2010
\vspace{6pt}

What is the diagnosis?
\vspace{12pt}

\textbf{Options:}
\begin{enumerate}
\item[A.] Bullous pemphigoid
\item[B.] Dermatitis herpetiformis
\item[C.] Impetigo
\item[D.] Syphilis
\item[E.] Varicella
\end{enumerate}

\textbf{Image:}
\begin{center}
\includegraphics[width=0.95\textwidth,height=0.50\textheight,width=0.90\textwidth,keepaspectratio]{images/nejm_20100325.jpg}
\end{center}
\vspace{12pt}
\newpage

\section*{Question 375 (ID: 20100401)}
\textbf{Date: }April 01,2010
\vspace{6pt}

This 4-year-old boy presented with a 5-day history of mild fever and malaise and a 3-day history of rash. What is the diagnosis?
\vspace{12pt}

\textbf{Options:}
\begin{enumerate}
\item[A.] Erythema infectiosum
\item[B.] Hand, foot, and mouth disease
\item[C.] Kawasaki disease
\item[D.] Measles
\item[E.] Pityriasis rosea
\end{enumerate}

\textbf{Image:}
\begin{center}
\includegraphics[width=0.95\textwidth,height=0.50\textheight,width=0.90\textwidth,keepaspectratio]{images/nejm_20100401.jpg}
\end{center}
\vspace{12pt}
\newpage

\section*{Question 376 (ID: 20100408)}
\textbf{Date: }April 08,2010
\vspace{6pt}

This patient was trying to look right when the image was taken. What is the diagnosis?
\vspace{12pt}

\textbf{Options:}
\begin{enumerate}
\item[A.] Internuclear ophthalmoplegia
\item[B.] Left fourth cranial nerve palsy
\item[C.] Left sixth cranial nerve palsy
\item[D.] Right fourth cranial nerve palsy
\item[E.] Right sixth cranial nerve palsy
\end{enumerate}

\textbf{Image:}
\begin{center}
\includegraphics[width=0.95\textwidth,height=0.50\textheight,width=0.90\textwidth,keepaspectratio]{images/nejm_20100408.jpg}
\end{center}
\vspace{12pt}
\newpage

\section*{Question 377 (ID: 20100415)}
\textbf{Date: }April 15,2010
\vspace{6pt}

What is the diagnosis?
\vspace{12pt}

\textbf{Options:}
\begin{enumerate}
\item[A.] Duodenal perforation
\item[B.] Emphysematous pyelonephritis
\item[C.] Perinephric cyst
\item[D.] Situs inversus
\item[E.] Ureterocele
\end{enumerate}

\textbf{Image:}
\begin{center}
\includegraphics[width=0.95\textwidth,height=0.50\textheight,width=0.90\textwidth,keepaspectratio]{images/nejm_20100415.jpg}
\end{center}
\vspace{12pt}
\newpage

\section*{Question 378 (ID: 20100422)}
\textbf{Date: }April 22,2010
\vspace{6pt}

What is the most likely diagnosis?
\vspace{12pt}

\textbf{Options:}
\begin{enumerate}
\item[A.] End-stage renal disease
\item[B.] Graves' disease
\item[C.] Iron deficiency
\item[D.] Pseudohypoparathyroidism
\item[E.] Sarcoidosis
\end{enumerate}

\textbf{Image:}
\begin{center}
\includegraphics[width=0.95\textwidth,height=0.50\textheight,width=0.90\textwidth,keepaspectratio]{images/nejm_20100422.jpg}
\end{center}
\vspace{12pt}
\newpage

\section*{Question 379 (ID: 20100429)}
\textbf{Date: }April 29,2010
\vspace{6pt}

What is the diagnosis?
\vspace{12pt}

\textbf{Options:}
\begin{enumerate}
\item[A.] Cerebral aneurysm
\item[B.] Chiari I malformation
\item[C.] Neurofibromatosis
\item[D.] Paget disease
\item[E.] Subdural hematoma
\end{enumerate}

\textbf{Image:}
\begin{center}
\includegraphics[width=0.95\textwidth,height=0.50\textheight,width=0.90\textwidth,keepaspectratio]{images/nejm_20100429.jpg}
\end{center}
\vspace{12pt}
\newpage

\section*{Question 380 (ID: 20100506)}
\textbf{Date: }May 06,2010
\vspace{6pt}

What is the most likely diagnosis?
\vspace{12pt}

\textbf{Options:}
\begin{enumerate}
\item[A.] Cutaneous larva migrans
\item[B.] Glucagonoma
\item[C.] Lung cancer
\item[D.] Systemic lupus erythematosus
\item[E.] Ulcerative colitis
\end{enumerate}

\textbf{Image:}
\begin{center}
\includegraphics[width=0.63\textwidth,height=0.50\textheight,width=0.90\textwidth,keepaspectratio]{images/nejm_20100506.jpg}
\end{center}
\vspace{12pt}
\newpage

\section*{Question 381 (ID: 20100513)}
\textbf{Date: }May 13,2010
\vspace{6pt}

What is the most likely cause of this presentation?
\vspace{12pt}

\textbf{Options:}
\begin{enumerate}
\item[A.] Chemotherapy
\item[B.] Hyperthyroidism
\item[C.] Nephrotic syndrome
\item[D.] Psoriasis
\item[E.] Syphilis
\end{enumerate}

\textbf{Image:}
\begin{center}
\includegraphics[width=0.95\textwidth,height=0.50\textheight,width=0.90\textwidth,keepaspectratio]{images/nejm_20100513.jpg}
\end{center}
\vspace{12pt}
\newpage

\section*{Question 382 (ID: 20100520)}
\textbf{Date: }May 20,2010
\vspace{6pt}

What is the diagnosis?
\vspace{12pt}

\textbf{Options:}
\begin{enumerate}
\item[A.] Erythema ab igne
\item[B.] Onchocerciasis
\item[C.] Sarcoidosis
\item[D.] Syphilis
\item[E.] Tuberculosis
\end{enumerate}

\textbf{Image:}
\begin{center}
\includegraphics[width=0.56\textwidth,height=0.50\textheight,width=0.90\textwidth,keepaspectratio]{images/nejm_20100520.jpg}
\end{center}
\vspace{12pt}
\newpage

\section*{Question 383 (ID: 20100527)}
\textbf{Date: }May 27,2010
\vspace{6pt}

This woman presented with right-sided pulsatile tinnitus without any hearing loss. What is the diagnosis?
\vspace{12pt}

\textbf{Options:}
\begin{enumerate}
\item[A.] Acoustic neuroma
\item[B.] Carotid aneurysm
\item[C.] Cholesteatoma
\item[D.] Eosinophilic granuloma
\item[E.] Glomus tympanicum
\end{enumerate}

\textbf{Image:}
\begin{center}
\includegraphics[width=0.95\textwidth,height=0.50\textheight,width=0.90\textwidth,keepaspectratio]{images/nejm_20100527.jpg}
\end{center}
\vspace{12pt}
\newpage

\section*{Question 384 (ID: 20100603)}
\textbf{Date: }June 03,2010
\vspace{6pt}

This rash appeared 9 hours after a scuba dive. What is the diagnosis?
\vspace{12pt}

\textbf{Options:}
\begin{enumerate}
\item[A.] Decompression sickness
\item[B.] Jellyfish envenomation
\item[C.] Mycobacterium marinum infection
\item[D.] Phylum Porifera contact dermatitis
\item[E.] Scombroid fish poisoning
\end{enumerate}

\textbf{Image:}
\begin{center}
\includegraphics[width=0.95\textwidth,height=0.50\textheight,width=0.90\textwidth,keepaspectratio]{images/nejm_20100603.jpg}
\end{center}
\vspace{12pt}
\newpage

\section*{Question 385 (ID: 20100610)}
\textbf{Date: }June 10,2010
\vspace{6pt}

This patient reported bleeding gums. What is the diagnosis?
\vspace{12pt}

\textbf{Options:}
\begin{enumerate}
\item[A.] Acromegaly
\item[B.] Dermatomyositis
\item[C.] Monocytic leukemia
\item[D.] Periodontitis
\item[E.] Scurvy
\end{enumerate}

\textbf{Image:}
\begin{center}
\includegraphics[width=0.95\textwidth,height=0.50\textheight,width=0.90\textwidth,keepaspectratio]{images/nejm_20100610.jpg}
\end{center}
\vspace{12pt}
\newpage

\section*{Question 386 (ID: 20100617)}
\textbf{Date: }June 17,2010
\vspace{6pt}

What is the diagnosis?
\vspace{12pt}

\textbf{Options:}
\begin{enumerate}
\item[A.] Ankylosing spondylitis
\item[B.] Ascending cholangitis
\item[C.] Intestinal obstruction
\item[D.] Pancreatitis
\item[E.] Perforation of a viscus
\end{enumerate}

\textbf{Image:}
\begin{center}
\includegraphics[width=0.56\textwidth,height=0.50\textheight,width=0.90\textwidth,keepaspectratio]{images/nejm_20100617.jpg}
\end{center}
\vspace{12pt}
\newpage

\section*{Question 387 (ID: 20100624)}
\textbf{Date: }June 24,2010
\vspace{6pt}

What diagnosis is suggested by these spirography findings?
\vspace{12pt}

\textbf{Options:}
\begin{enumerate}
\item[A.] Intrathoracic localized obstruction
\item[B.] Fixed inspiratory obstruction
\item[C.] Pneumothorax
\item[D.] Restrictive lung disease
\item[E.] Variable extrathoracic obstruction
\end{enumerate}

\textbf{Image:}
\begin{center}
\includegraphics[width=0.72\textwidth,height=0.50\textheight,width=0.90\textwidth,keepaspectratio]{images/nejm_20100624.jpg}
\end{center}
\vspace{12pt}
\newpage

\section*{Question 388 (ID: 20100701)}
\textbf{Date: }July 01,2010
\vspace{6pt}

What is the most likely diagnosis in this 32-year-old man?
\vspace{12pt}

\textbf{Options:}
\begin{enumerate}
\item[A.] Craniopharyngioma
\item[B.] Cushing disease
\item[C.] Meningioma
\item[D.] Prolactinoma
\item[E.] Rathke cleft cyst
\end{enumerate}

\textbf{Image:}
\begin{center}
\includegraphics[width=0.84\textwidth,height=0.50\textheight,width=0.90\textwidth,keepaspectratio]{images/nejm_20100701.jpg}
\end{center}
\vspace{12pt}
\newpage

\section*{Question 389 (ID: 20100708)}
\textbf{Date: }July 08,2010
\vspace{6pt}

What is the diagnosis?
\vspace{12pt}

\textbf{Options:}
\begin{enumerate}
\item[A.] Angioedema
\item[B.] Cholesterol embolism
\item[C.] Compartment syndrome
\item[D.] Raynaud disease
\item[E.] Subclavian vein thrombosis
\end{enumerate}

\textbf{Image:}
\begin{center}
\includegraphics[width=0.63\textwidth,height=0.50\textheight,width=0.90\textwidth,keepaspectratio]{images/nejm_20100708.jpg}
\end{center}
\vspace{12pt}
\newpage

\section*{Question 390 (ID: 20100715)}
\textbf{Date: }July 15,2010
\vspace{6pt}

What medication is most likely to be responsible for the finding in this woman's hair?
\vspace{12pt}

\textbf{Options:}
\begin{enumerate}
\item[A.] Carbamazepine
\item[B.] Chloroquine
\item[C.] Danazol
\item[D.] Desipramine
\item[E.] Verapamil
\end{enumerate}

\textbf{Image:}
\begin{center}
\includegraphics[width=0.95\textwidth,height=0.50\textheight,width=0.90\textwidth,keepaspectratio]{images/nejm_20100715.jpg}
\end{center}
\vspace{12pt}
\newpage

\section*{Question 391 (ID: 20100722)}
\textbf{Date: }July 22,2010
\vspace{6pt}

What is the diagnosis?
\vspace{12pt}

\textbf{Options:}
\begin{enumerate}
\item[A.] Aspiration pneumonia
\item[B.] Caplan syndrome
\item[C.] Pleural perforation
\item[D.] Intestinal malrotation
\item[E.] Small-cell lung cancer
\end{enumerate}

\textbf{Image:}
\begin{center}
\includegraphics[width=0.95\textwidth,height=0.50\textheight,width=0.90\textwidth,keepaspectratio]{images/nejm_20100722.jpg}
\end{center}
\vspace{12pt}
\newpage

\section*{Question 392 (ID: 20100729)}
\textbf{Date: }July 29,2010
\vspace{6pt}

Which laboratory measure would be expected to be most abnormal in this patient?
\vspace{12pt}

\textbf{Options:}
\begin{enumerate}
\item[A.] Alkaline phosphatase
\item[B.] \{beta\}2-microglobulin
\item[C.] Cortisol
\item[D.] Insulin-like growth factor 1
\item[E.] Mean corpuscular volume
\end{enumerate}

\textbf{Image:}
\begin{center}
\includegraphics[width=0.95\textwidth,height=0.50\textheight,width=0.90\textwidth,keepaspectratio]{images/nejm_20100729.jpg}
\end{center}
\vspace{12pt}
\newpage

\section*{Question 393 (ID: 20100805)}
\textbf{Date: }August 05,2010
\vspace{6pt}

An 18-year-old woman presented with visual loss in the right eye. What diagnosis is suggested by this finding on ophthalmoscopy?
\vspace{12pt}

\textbf{Options:}
\begin{enumerate}
\item[A.] Cystic fibrosis
\item[B.] Marfan syndrome
\item[C.] Sickle cell disease
\item[D.] Tay-Sachs disease
\item[E.] Von  Hippel-Lindau disease
\end{enumerate}

\textbf{Image:}
\begin{center}
\includegraphics[width=0.83\textwidth,height=0.50\textheight,width=0.90\textwidth,keepaspectratio]{images/nejm_20100805.jpg}
\end{center}
\vspace{12pt}
\newpage

\section*{Question 394 (ID: 20100812)}
\textbf{Date: }August 12,2010
\vspace{6pt}

What is the diagnosis?
\vspace{12pt}

\textbf{Options:}
\begin{enumerate}
\item[A.] Bladder diverticula
\item[B.] Orthotopic neobladder
\item[C.] Schistosomiasis
\item[D.] Transitional-cell carcinoma
\item[E.] Vesicolithiasis
\end{enumerate}

\textbf{Image:}
\begin{center}
\includegraphics[width=0.95\textwidth,height=0.50\textheight,width=0.90\textwidth,keepaspectratio]{images/nejm_20100812.jpg}
\end{center}
\vspace{12pt}
\newpage

\section*{Question 395 (ID: 20100819)}
\textbf{Date: }August 19,2010
\vspace{6pt}

This man presented 1 day after being treated with penicillin for a toothache. What is the diagnosis?
\vspace{12pt}

\textbf{Options:}
\begin{enumerate}
\item[A.] Acute parotitis
\item[B.] Angioneurotic edema
\item[C.] Ludwig's angina
\item[D.] Penicillin allergy
\item[E.] Peritonsillar abscess
\end{enumerate}

\textbf{Image:}
\begin{center}
\includegraphics[width=0.55\textwidth,height=0.50\textheight,width=0.90\textwidth,keepaspectratio]{images/nejm_20100819.jpg}
\end{center}
\vspace{12pt}
\newpage

\section*{Question 396 (ID: 20100826)}
\textbf{Date: }August 26,2010
\vspace{6pt}

What is the diagnosis?
\vspace{12pt}

\textbf{Options:}
\begin{enumerate}
\item[A.] Closed-angle glaucoma
\item[B.] Cholesterol embolism
\item[C.] Diabetes mellitus
\item[D.] Malignant hypertension
\item[E.] Syphilis
\end{enumerate}

\textbf{Image:}
\begin{center}
\includegraphics[width=0.95\textwidth,height=0.50\textheight,width=0.90\textwidth,keepaspectratio]{images/nejm_20100826.jpg}
\end{center}
\vspace{12pt}
\newpage

\section*{Question 397 (ID: 20100902)}
\textbf{Date: }September 02,2010
\vspace{6pt}

This 20-year-old man was evaluated for fever. What is the diagnosis?
\vspace{12pt}

\textbf{Options:}
\begin{enumerate}
\item[A.] Allergic bronchopulmonary aspergillosis
\item[B.] Cystic fibrosis
\item[C.] Primary hyperparathyroidism
\item[D.] Sarcoidosis
\item[E.] Tuberculosis
\end{enumerate}

\textbf{Image:}
\begin{center}
\includegraphics[width=0.95\textwidth,height=0.50\textheight,width=0.90\textwidth,keepaspectratio]{images/nejm_20100902.jpg}
\end{center}
\vspace{12pt}
\newpage

\section*{Question 398 (ID: 20100909)}
\textbf{Date: }September 09,2010
\vspace{6pt}

What laboratory evaluation is most likely to be abnormal in this 10-year-old patient who also had discrete facial erythema?
\vspace{12pt}

\textbf{Options:}
\begin{enumerate}
\item[A.] CD4 count
\item[B.] Creatine kinase
\item[C.] Rheumatoid factor
\item[D.] Thyrotropin
\end{enumerate}

\textbf{Image:}
\begin{center}
\includegraphics[width=0.95\textwidth,height=0.50\textheight,width=0.90\textwidth,keepaspectratio]{images/nejm_20100909.jpg}
\end{center}
\vspace{12pt}
\newpage

\section*{Question 399 (ID: 20100916)}
\textbf{Date: }September 16,2010
\vspace{6pt}

What is the diagnosis?
\vspace{12pt}

\textbf{Options:}
\begin{enumerate}
\item[A.] Acute myelogenous leukemia
\item[B.] Ehrlichiosis
\item[C.] Lead poisoning
\item[D.] Malaria
\item[E.] Pompe disease
\end{enumerate}

\textbf{Image:}
\begin{center}
\includegraphics[width=0.95\textwidth,height=0.50\textheight,width=0.90\textwidth,keepaspectratio]{images/nejm_20100916.jpg}
\end{center}
\vspace{12pt}
\newpage

\section*{Question 400 (ID: 20100923)}
\textbf{Date: }September 23,2010
\vspace{6pt}

This patient had subclinical hypothyroidism. What is the diagnosis?
\vspace{12pt}

\textbf{Options:}
\begin{enumerate}
\item[A.] Ectopic thyroid
\item[B.] Hashimoto thyroiditis
\item[C.] Iodine deficiency
\item[D.] Pendred syndrome
\item[E.] Thyroglossal duct cyst
\end{enumerate}

\textbf{Image:}
\begin{center}
\includegraphics[width=0.95\textwidth,height=0.50\textheight,width=0.90\textwidth,keepaspectratio]{images/nejm_20100923.jpg}
\end{center}
\vspace{12pt}
\newpage

\section*{Question 401 (ID: 20100930)}
\textbf{Date: }September 30,2010
\vspace{6pt}

Movement, bathing, and exercise triggered pain in this patient's right hand. What is the most likely diagnosis?
\vspace{12pt}

\textbf{Options:}
\begin{enumerate}
\item[A.] Brachial plexus injury
\item[B.] Erythromelalgia
\item[C.] Frostbite
\item[D.] Raynaud phenomenon
\item[E.] Subclavian artery stenosis
\end{enumerate}

\textbf{Image:}
\begin{center}
\includegraphics[width=0.95\textwidth,height=0.50\textheight,width=0.90\textwidth,keepaspectratio]{images/nejm_20100930.jpg}
\end{center}
\vspace{12pt}
\newpage

\section*{Question 402 (ID: 20101007)}
\textbf{Date: }October 07,2010
\vspace{6pt}

What is the diagnosis?
\vspace{12pt}

\textbf{Options:}
\begin{enumerate}
\item[A.] Achondroplasia
\item[B.] Ankylosing spondylitis
\item[C.] Osteogenesis imperfecta
\item[D.] Pectus excavatum
\item[E.] Pneumothorax
\end{enumerate}

\textbf{Image:}
\begin{center}
\includegraphics[width=0.68\textwidth,height=0.50\textheight,width=0.90\textwidth,keepaspectratio]{images/nejm_20101007.jpg}
\end{center}
\vspace{12pt}
\newpage

\section*{Question 403 (ID: 20101014)}
\textbf{Date: }October 14,2010
\vspace{6pt}

This patient was recovering from diabetic ketoacidosis (DKA). What is the diagnosis?
\vspace{12pt}

\textbf{Options:}
\begin{enumerate}
\item[A.] Basilar artery thrombosis
\item[B.] Cerebral edema
\item[C.] Cobalamin deficiency
\item[D.] Osmotic demyelination syndrome
\item[E.] Vermian atrophy
\end{enumerate}

\textbf{Image:}
\begin{center}
\includegraphics[width=0.62\textwidth,height=0.50\textheight,width=0.90\textwidth,keepaspectratio]{images/nejm_20101014.jpg}
\end{center}
\vspace{12pt}
\newpage

\section*{Question 404 (ID: 20101021)}
\textbf{Date: }October 21,2010
\vspace{6pt}

This patient had hypertension. What is the diagnosis?
\vspace{12pt}

\textbf{Options:}
\begin{enumerate}
\item[A.] Coarctation of the aorta
\item[B.] Pheochromocytoma
\item[C.] Primary hyperaldosteronism
\item[D.] Renal artery stenosis
\item[E.] Systemic sclerosis
\end{enumerate}

\textbf{Image:}
\begin{center}
\includegraphics[width=0.95\textwidth,height=0.50\textheight,width=0.90\textwidth,keepaspectratio]{images/nejm_20101021.jpg}
\end{center}
\vspace{12pt}
\newpage

\section*{Question 405 (ID: 20101028)}
\textbf{Date: }October 28,2010
\vspace{6pt}

This patient was found to have new-onset hyperglycemia. What is the diagnosis?
\vspace{12pt}

\textbf{Options:}
\begin{enumerate}
\item[A.] Hemochromatosis
\item[B.] Necrolytic migratory erythema
\item[C.] Psoriasis
\item[D.] Scleredema of diabetes
\item[E.] Tinea versicolor
\end{enumerate}

\textbf{Image:}
\begin{center}
\includegraphics[width=0.95\textwidth,height=0.50\textheight,width=0.90\textwidth,keepaspectratio]{images/nejm_20101028.jpg}
\end{center}
\vspace{12pt}
\newpage

\section*{Question 406 (ID: 20101104)}
\textbf{Date: }November 04,2010
\vspace{6pt}

What is the diagnosis?
\vspace{12pt}

\textbf{Options:}
\begin{enumerate}
\item[A.] Atrophic candidiasis
\item[B.] Erythroplakia
\item[C.] Lingual thyroid
\item[D.] Median rhomboid glossitis
\item[E.] Tertiary syphilis
\end{enumerate}

\textbf{Image:}
\begin{center}
\includegraphics[width=0.75\textwidth,height=0.50\textheight,width=0.90\textwidth,keepaspectratio]{images/nejm_20101104.jpg}
\end{center}
\vspace{12pt}
\newpage

\section*{Question 407 (ID: 20101111)}
\textbf{Date: }November 11,2010
\vspace{6pt}

This 10-year-old presented with abdominal pain. What is the diagnosis?
\vspace{12pt}

\textbf{Options:}
\begin{enumerate}
\item[A.] Hirschsprung's disease
\item[B.] Meckel's diverticulum
\item[C.] Peptic ulceration
\item[D.] Pyloric outlet obstruction
\item[E.] Situs inversus
\end{enumerate}

\textbf{Image:}
\begin{center}
\includegraphics[width=0.59\textwidth,height=0.50\textheight,width=0.90\textwidth,keepaspectratio]{images/nejm_20101111.jpg}
\end{center}
\vspace{12pt}
\newpage

\section*{Question 408 (ID: 20101118)}
\textbf{Date: }November 18,2010
\vspace{6pt}

What is the diagnosis?
\vspace{12pt}

\textbf{Options:}
\begin{enumerate}
\item[A.] Meningioma
\item[B.] Osteitis fibrosa cystica
\item[C.] Paget's disease
\item[D.] Plasmacytoma
\item[E.] Thalassemia
\end{enumerate}

\textbf{Image:}
\begin{center}
\includegraphics[width=0.85\textwidth,height=0.50\textheight,width=0.90\textwidth,keepaspectratio]{images/nejm_20101118.jpg}
\end{center}
\vspace{12pt}
\newpage

\section*{Question 409 (ID: 20101125)}
\textbf{Date: }November 25,2010
\vspace{6pt}

A 65-year-old man presented with skin lesions six weeks after returning from a vacation in Belize at the beach and in the rain forest. What is the diagnosis?
\vspace{12pt}

\textbf{Options:}
\begin{enumerate}
\item[A.] Bot fly
\item[B.] Cutaneous leishmaniasis
\item[C.] Onchocerciasis
\item[D.] Plague
\item[E.] Tungiasis
\end{enumerate}

\textbf{Image:}
\begin{center}
\includegraphics[width=0.95\textwidth,height=0.50\textheight,width=0.90\textwidth,keepaspectratio]{images/nejm_20101125.jpg}
\end{center}
\vspace{12pt}
\newpage

\section*{Question 410 (ID: 20101202)}
\textbf{Date: }December 02,2010
\vspace{6pt}

What is the diagnosis?
\vspace{12pt}

\textbf{Options:}
\begin{enumerate}
\item[A.] Arsenic poisoning
\item[B.] Cushing's syndrome
\item[C.] Pellagra
\item[D.] Self-flagellation
\item[E.] Treatment with bleomycin
\end{enumerate}

\textbf{Image:}
\begin{center}
\includegraphics[width=0.77\textwidth,height=0.50\textheight,width=0.90\textwidth,keepaspectratio]{images/nejm_20101202.jpg}
\end{center}
\vspace{12pt}
\newpage

\section*{Question 411 (ID: 20101209)}
\textbf{Date: }December 09,2010
\vspace{6pt}

What is the diagnosis?
\vspace{12pt}

\textbf{Options:}
\begin{enumerate}
\item[A.] Ascaris lumbricoides
\item[B.] Dirofilaria repens
\item[C.] Paragonimus westermani
\item[D.] Toxocara canis
\item[E.] Wuchereria bancrofti
\end{enumerate}

\textbf{Image:}
\begin{center}
\includegraphics[width=0.95\textwidth,height=0.50\textheight,width=0.90\textwidth,keepaspectratio]{images/nejm_20101209.jpg}
\end{center}
\vspace{12pt}
\newpage

\section*{Question 412 (ID: 20101216)}
\textbf{Date: }December 16,2010
\vspace{6pt}

What is the diagnosis?
\vspace{12pt}

\textbf{Options:}
\begin{enumerate}
\item[A.] Anthrax
\item[B.] Cellulitis
\item[C.] Lichen planus
\item[D.] Psoriasis
\end{enumerate}

\textbf{Image:}
\begin{center}
\includegraphics[width=0.95\textwidth,height=0.50\textheight,width=0.90\textwidth,keepaspectratio]{images/nejm_20101216.jpg}
\end{center}
\vspace{12pt}
\newpage

\section*{Question 413 (ID: 20101223)}
\textbf{Date: }December 23,2010
\vspace{6pt}

This patient had diabetes. What is the diagnosis?
\vspace{12pt}

\textbf{Options:}
\begin{enumerate}
\item[A.] Ichthyosis
\item[B.] Ischemic gangrene
\item[C.] Necrobiosis lipoidica diabeticorum
\item[D.] Phlegmasia cerulea dolens
\item[E.] Pyoderma gangrenosum
\end{enumerate}

\textbf{Image:}
\begin{center}
\includegraphics[width=0.76\textwidth,height=0.50\textheight,width=0.90\textwidth,keepaspectratio]{images/nejm_20101223.jpg}
\end{center}
\vspace{12pt}
\newpage

\section*{Question 414 (ID: 20101230)}
\textbf{Date: }December 30,2010
\vspace{6pt}

This rash had been present for 2 weeks. What is the diagnosis?
\vspace{12pt}

\textbf{Options:}
\begin{enumerate}
\item[A.] Coxsackie A
\item[B.] Gonococcus
\item[C.] Measles
\item[D.] Parvovirus B19
\item[E.] Treponema pallidum
\end{enumerate}

\textbf{Image:}
\begin{center}
\includegraphics[width=0.95\textwidth,height=0.50\textheight,width=0.90\textwidth,keepaspectratio]{images/nejm_20101230.jpg}
\end{center}
\vspace{12pt}
\newpage

\section*{Question 415 (ID: 20110106)}
\textbf{Date: }January 06,2011
\vspace{6pt}

What is the diagnosis?
\vspace{12pt}

\textbf{Options:}
\begin{enumerate}
\item[A.] Acromegaly
\item[B.] Adrenal insufficiency
\item[C.] Histiocytosis
\item[D.] Osteopetrosis
\item[E.] Paget's disease
\end{enumerate}

\textbf{Image:}
\begin{center}
\includegraphics[width=0.73\textwidth,height=0.50\textheight,width=0.90\textwidth,keepaspectratio]{images/nejm_20110106.jpg}
\end{center}
\vspace{12pt}
\newpage

\section*{Question 416 (ID: 20110113)}
\textbf{Date: }January 13,2011
\vspace{6pt}

This patient presented with renal failure. What would be the expected finding on renal biopsy?
\vspace{12pt}

\textbf{Options:}
\begin{enumerate}
\item[A.] Cholesterol crystals
\item[B.] Crescentic glomerulonephritis
\item[C.] Heme pigment
\item[D.] Renal cortical necrosis
\item[E.] Tubulointerstitial nephritis
\end{enumerate}

\textbf{Image:}
\begin{center}
\includegraphics[width=0.95\textwidth,height=0.50\textheight,width=0.90\textwidth,keepaspectratio]{images/nejm_20110113.jpg}
\end{center}
\vspace{12pt}
\newpage

\section*{Question 417 (ID: 20110120)}
\textbf{Date: }January 20,2011
\vspace{6pt}

Deficiency of which one of the following dietary components is most likely to have caused this rash?
\vspace{12pt}

\textbf{Options:}
\begin{enumerate}
\item[A.] Biotin
\item[B.] Folate
\item[C.] Niacin
\item[D.] Riboflavin
\item[E.] Vitamin C
\end{enumerate}

\textbf{Image:}
\begin{center}
\includegraphics[width=0.95\textwidth,height=0.50\textheight,width=0.90\textwidth,keepaspectratio]{images/nejm_20110120.jpg}
\end{center}
\vspace{12pt}
\newpage

\section*{Question 418 (ID: 20110127)}
\textbf{Date: }January 27,2011
\vspace{6pt}

This patient with diffuse, nonencapsulated fatty deposits is most likely to have a history of which one of the following?
\vspace{12pt}

\textbf{Options:}
\begin{enumerate}
\item[A.] Alcohol dependency
\item[B.] Hypercalcemia
\item[C.] Renal insufficiency
\item[D.] Rheumatoid arthritis
\item[E.] Tuberculosis
\end{enumerate}

\textbf{Image:}
\begin{center}
\includegraphics[width=0.83\textwidth,height=0.50\textheight,width=0.90\textwidth,keepaspectratio]{images/nejm_20110127.jpg}
\end{center}
\vspace{12pt}
\newpage

\section*{Question 419 (ID: 20110203)}
\textbf{Date: }February 03,2011
\vspace{6pt}

What is the diagnosis?
\vspace{12pt}

\textbf{Options:}
\begin{enumerate}
\item[A.] Bancroftian filariasis
\item[B.] Deep-vein thrombosis
\item[C.] Erythema ab igne
\item[D.] Lipodermatosclerosis
\item[E.] Superficial thrombophlebitis
\end{enumerate}

\textbf{Image:}
\begin{center}
\includegraphics[width=0.95\textwidth,height=0.50\textheight,width=0.90\textwidth,keepaspectratio]{images/nejm_20110203.jpg}
\end{center}
\vspace{12pt}
\newpage

\section*{Question 420 (ID: 20110210)}
\textbf{Date: }February 10,2011
\vspace{6pt}

What is the most likely diagnosis?
\vspace{12pt}

\textbf{Options:}
\begin{enumerate}
\item[A.] Acromegaly
\item[B.] Cystic fibrosis
\item[C.] Eisenmenger's syndrome
\item[D.] Squamous-cell lung cancer
\item[E.] Ulcerative colitis
\end{enumerate}

\textbf{Image:}
\begin{center}
\includegraphics[width=0.89\textwidth,height=0.50\textheight,width=0.90\textwidth,keepaspectratio]{images/nejm_20110210.jpg}
\end{center}
\vspace{12pt}
\newpage

\section*{Question 421 (ID: 20110217)}
\textbf{Date: }February 17,2011
\vspace{6pt}

What is the diagnosis?
\vspace{12pt}

\textbf{Options:}
\begin{enumerate}
\item[A.] Anterior uveitis
\item[B.] Carotid cavernous sinus fistula
\item[C.] Graves' disease
\item[D.] Orbital lymphoma
\item[E.] Scleral rupture
\end{enumerate}

\textbf{Image:}
\begin{center}
\includegraphics[width=0.8\textwidth,height=0.50\textheight,width=0.90\textwidth,keepaspectratio]{images/nejm_20110217.jpg}
\end{center}
\vspace{12pt}
\newpage

\section*{Question 422 (ID: 20110224)}
\textbf{Date: }February 24,2011
\vspace{6pt}

What laboratory test is most appropriate for this patient?
\vspace{12pt}

\textbf{Options:}
\begin{enumerate}
\item[A.] Alpha-fetoprotein
\item[B.] Beta2-microglobulin
\item[C.] Calcitonin
\item[D.] Insulin-like growth factor-1
\item[E.] Red-cell transketolase
\end{enumerate}

\textbf{Image:}
\begin{center}
\includegraphics[width=0.95\textwidth,height=0.50\textheight,width=0.90\textwidth,keepaspectratio]{images/nejm_20110224.jpg}
\end{center}
\vspace{12pt}
\newpage

\section*{Question 423 (ID: 20110303)}
\textbf{Date: }March 03,2011
\vspace{6pt}

What is the diagnosis?
\vspace{12pt}

\textbf{Options:}
\begin{enumerate}
\item[A.] Adrenal adenoma
\item[B.] Gastric bezoar
\item[C.] Pancreatic pseudocyst
\item[D.] Pulmonary echinococcosis
\item[E.] Splenic cyst
\end{enumerate}

\textbf{Image:}
\begin{center}
\includegraphics[width=0.72\textwidth,height=0.50\textheight,width=0.90\textwidth,keepaspectratio]{images/nejm_20110303.jpg}
\end{center}
\vspace{12pt}
\newpage

\section*{Question 424 (ID: 20110310)}
\textbf{Date: }March 10,2011
\vspace{6pt}

What is the diagnosis?
\vspace{12pt}

\textbf{Options:}
\begin{enumerate}
\item[A.] Neurofibromatosis
\item[B.] Rhinophyma
\item[C.] Rosacea
\item[D.] Systemic lupus erythematosus
\item[E.] Tuberous sclerosis
\end{enumerate}

\textbf{Image:}
\begin{center}
\includegraphics[width=0.95\textwidth,height=0.50\textheight,width=0.90\textwidth,keepaspectratio]{images/nejm_20110310.jpg}
\end{center}
\vspace{12pt}
\newpage

\section*{Question 425 (ID: 20110317)}
\textbf{Date: }March 17,2011
\vspace{6pt}

Which vector was responsible for infecting this 14-year-old immigrant from Cameroon?
\vspace{12pt}

\textbf{Options:}
\begin{enumerate}
\item[A.] Chrysops fly
\item[B.] Mosquito
\item[C.] Reduviid bug
\item[D.] Sand fly
\item[E.] Tsetse fly
\end{enumerate}

\textbf{Image:}
\begin{center}
\includegraphics[width=0.95\textwidth,height=0.50\textheight,width=0.90\textwidth,keepaspectratio]{images/nejm_20110317.jpg}
\end{center}
\vspace{12pt}
\newpage

\section*{Question 426 (ID: 20110324)}
\textbf{Date: }March 24,2011
\vspace{6pt}

Ultraviolet light was shone on this patient's rash. What is the diagnosis?
\vspace{12pt}

\textbf{Options:}
\begin{enumerate}
\item[A.] Erythrasma
\item[B.] Intertrigo
\item[C.] Pityriasis rosea
\item[D.] Psoriasis
\item[E.] Tinea versicolor
\end{enumerate}

\textbf{Image:}
\begin{center}
\includegraphics[width=0.88\textwidth,height=0.50\textheight,width=0.90\textwidth,keepaspectratio]{images/nejm_20110324.jpg}
\end{center}
\vspace{12pt}
\newpage

\section*{Question 427 (ID: 20110331)}
\textbf{Date: }March 31,2011
\vspace{6pt}

What is the diagnosis?
\vspace{12pt}

\textbf{Options:}
\begin{enumerate}
\item[A.] Intestinal malrotation
\item[B.] Intestinal perforation
\item[C.] Lynch syndrome
\item[D.] Mesenteric ischemia
\item[E.] Portal vein thrombosis
\end{enumerate}

\textbf{Image:}
\begin{center}
\includegraphics[width=0.88\textwidth,height=0.50\textheight,width=0.90\textwidth,keepaspectratio]{images/nejm_20110331.jpg}
\end{center}
\vspace{12pt}
\newpage

\section*{Question 428 (ID: 20110407)}
\textbf{Date: }April 07,2011
\vspace{6pt}

What is the diagnosis?
\vspace{12pt}

\textbf{Options:}
\begin{enumerate}
\item[A.] Dermatitis herpetiformis
\item[B.] Impetigo
\item[C.] Measles
\item[D.] Secondary syphilis
\item[E.] Varicella
\end{enumerate}

\textbf{Image:}
\begin{center}
\includegraphics[width=0.95\textwidth,height=0.50\textheight,width=0.90\textwidth,keepaspectratio]{images/nejm_20110407.jpg}
\end{center}
\vspace{12pt}
\newpage

\section*{Question 429 (ID: 20110414)}
\textbf{Date: }April 14,2011
\vspace{6pt}

What is the most likely diagnosis?
\vspace{12pt}

\textbf{Options:}
\begin{enumerate}
\item[A.] Calciphylaxis
\item[B.] Factor V Leiden
\item[C.] Protein C deficiency
\item[D.] Scleroderma
\item[E.] Waldenstrom's macroglobulinemia
\end{enumerate}

\textbf{Image:}
\begin{center}
\includegraphics[width=0.95\textwidth,height=0.50\textheight,width=0.90\textwidth,keepaspectratio]{images/nejm_20110414.jpg}
\end{center}
\vspace{12pt}
\newpage

\section*{Question 430 (ID: 20110421)}
\textbf{Date: }April 21,2011
\vspace{6pt}

What is the diagnosis?
\vspace{12pt}

\textbf{Options:}
\begin{enumerate}
\item[A.] Erythema marginatum
\item[B.] Erythema multiforme
\item[C.] Pityriasis rosea
\item[D.] Tinea versicolor
\item[E.] Tuberculoid leprosy
\end{enumerate}

\textbf{Image:}
\begin{center}
\includegraphics[width=0.95\textwidth,height=0.50\textheight,width=0.90\textwidth,keepaspectratio]{images/nejm_20110421.jpg}
\end{center}
\vspace{12pt}
\newpage

\section*{Question 431 (ID: 20110428)}
\textbf{Date: }April 28,2011
\vspace{6pt}

What is the diagnosis?
\vspace{12pt}

\textbf{Options:}
\begin{enumerate}
\item[A.] Addison's disease
\item[B.] Alkaptonuria
\item[C.] Hematoma
\item[D.] Melanoma
\item[E.] Zidovudine exposure
\end{enumerate}

\textbf{Image:}
\begin{center}
\includegraphics[width=0.95\textwidth,height=0.50\textheight,width=0.90\textwidth,keepaspectratio]{images/nejm_20110428.jpg}
\end{center}
\vspace{12pt}
\newpage

\section*{Question 432 (ID: 20110505)}
\textbf{Date: }May 05,2011
\vspace{6pt}

A prolapse such as this may be effectively reduced by applying what common household substance to the tissue?
\vspace{12pt}

\textbf{Options:}
\begin{enumerate}
\item[A.] Baking powder
\end{enumerate}

\textbf{Image:}
\begin{center}
\includegraphics[width=0.95\textwidth,height=0.50\textheight,width=0.90\textwidth,keepaspectratio]{images/nejm_20110505.jpg}
\end{center}
\vspace{12pt}
\newpage

\section*{Question 433 (ID: 20110512)}
\textbf{Date: }May 12,2011
\vspace{6pt}

What is the diagnosis?
\vspace{12pt}

\textbf{Options:}
\begin{enumerate}
\item[A.] Graves' disease
\item[B.] Paraganglioma
\item[C.] Superior vena cava obstruction
\item[D.] Tertiary hyperparathyroidism
\item[E.] Zenker's diverticulum
\end{enumerate}

\textbf{Image:}
\begin{center}
\includegraphics[width=0.5\textwidth,height=0.50\textheight,width=0.90\textwidth,keepaspectratio]{images/nejm_20110512.jpg}
\end{center}
\vspace{12pt}
\newpage

\section*{Question 434 (ID: 20110519)}
\textbf{Date: }May 19,2011
\vspace{6pt}

This woman's umbilical nodule bled intermittently. What is the most likely diagnosis?
\vspace{12pt}

\textbf{Options:}
\begin{enumerate}
\item[A.] Endometrioma
\item[B.] Metastatic adenocarcinoma
\item[C.] Omphalith
\item[D.] Umbilical hernia
\item[E.] Urachal cyst
\end{enumerate}

\textbf{Image:}
\begin{center}
\includegraphics[width=0.95\textwidth,height=0.50\textheight,width=0.90\textwidth,keepaspectratio]{images/nejm_20110519.jpg}
\end{center}
\vspace{12pt}
\newpage

\section*{Question 435 (ID: 20110526)}
\textbf{Date: }May 26,2011
\vspace{6pt}

What is the diagnosis?
\vspace{12pt}

\textbf{Options:}
\begin{enumerate}
\item[A.] Erysipeloid
\item[B.] Psoriasis
\item[C.] Rosacea
\item[D.] Seborrheic dermatitis
\item[E.] Systemic lupus erythematosus
\end{enumerate}

\textbf{Image:}
\begin{center}
\includegraphics[width=0.95\textwidth,height=0.50\textheight,width=0.90\textwidth,keepaspectratio]{images/nejm_20110526.jpg}
\end{center}
\vspace{12pt}
\newpage

\section*{Question 436 (ID: 20110602)}
\textbf{Date: }June 02,2011
\vspace{6pt}

What is the diagnosis?
\vspace{12pt}

\textbf{Options:}
\begin{enumerate}
\item[A.] Carpal tunnel syndrome
\item[C.] Hypoparathyroidism
\item[D.] Rheumatoid arthritis
\item[E.] Systemic sclerosis
\end{enumerate}

\textbf{Image:}
\begin{center}
\includegraphics[width=0.84\textwidth,height=0.50\textheight,width=0.90\textwidth,keepaspectratio]{images/nejm_20110602.jpg}
\end{center}
\vspace{12pt}
\newpage

\section*{Question 437 (ID: 20110609)}
\textbf{Date: }June 09,2011
\vspace{6pt}

This patient had hypercalcemia with a suppressed parathyroid hormone level. What is the diagnosis?
\vspace{12pt}

\textbf{Options:}
\begin{enumerate}
\item[A.] Hodgkin's lymphoma
\item[B.] Hyperthyroidism
\item[C.] Parathyroid carcinoma
\item[D.] Small-cell lung cancer
\item[E.] Squamous-cell lung cancer
\end{enumerate}

\textbf{Image:}
\begin{center}
\includegraphics[width=0.75\textwidth,height=0.50\textheight,width=0.90\textwidth,keepaspectratio]{images/nejm_20110609.jpg}
\end{center}
\vspace{12pt}
\newpage

\section*{Question 438 (ID: 20110616)}
\textbf{Date: }June 16,2011
\vspace{6pt}

These lesions are often associated with which one of the following disorders?
\vspace{12pt}

\textbf{Options:}
\begin{enumerate}
\item[A.] Bulimia nervosa
\item[B.] Carcinoid tumor
\item[C.] Crohn's disease
\item[D.] Marfan's syndrome
\item[E.] Squamous-cell lung carcinoma
\end{enumerate}

\textbf{Image:}
\begin{center}
\includegraphics[width=0.95\textwidth,height=0.50\textheight,width=0.90\textwidth,keepaspectratio]{images/nejm_20110616.jpg}
\end{center}
\vspace{12pt}
\newpage

\section*{Question 439 (ID: 20110623)}
\textbf{Date: }June 23,2011
\vspace{6pt}

What is the diagnosis?
\vspace{12pt}

\textbf{Options:}
\begin{enumerate}
\item[A.] Acute liver failure
\item[B.] Cushing's syndrome
\item[C.] Familial partial lipodystrophy
\item[D.] Pancreatic adenocarcinoma
\item[E.] Thrombosis of the inferior vena cava
\end{enumerate}

\textbf{Image:}
\begin{center}
\includegraphics[width=0.67\textwidth,height=0.50\textheight,width=0.90\textwidth,keepaspectratio]{images/nejm_20110623.jpg}
\end{center}
\vspace{12pt}
\newpage

\section*{Question 440 (ID: 20110630)}
\textbf{Date: }June 30,2011
\vspace{6pt}

What is the diagnosis?
\vspace{12pt}

\textbf{Options:}
\begin{enumerate}
\item[A.] Leukocytoclastic vasculitis
\item[B.] Multiple myeloma
\item[C.] Polycythemia vera
\item[D.] Primary hypothyroidism
\item[E.] Tolosa-Hunt syndrome
\end{enumerate}

\textbf{Image:}
\begin{center}
\includegraphics[width=0.95\textwidth,height=0.50\textheight,width=0.90\textwidth,keepaspectratio]{images/nejm_20110630.jpg}
\end{center}
\vspace{12pt}
\newpage

\section*{Question 441 (ID: 20110707)}
\textbf{Date: }July 07,2011
\vspace{6pt}

What is the diagnosis?
\vspace{12pt}

\textbf{Options:}
\begin{enumerate}
\item[A.] Esophageal rupture
\item[B.] Flail chest
\item[C.] Hydropneumothorax
\item[D.] Lymphangioleiomyomatosis
\item[E.] Phrenic nerve palsy
\end{enumerate}

\textbf{Image:}
\begin{center}
\includegraphics[width=0.69\textwidth,height=0.50\textheight,width=0.90\textwidth,keepaspectratio]{images/nejm_20110707.jpg}
\end{center}
\vspace{12pt}
\newpage

\section*{Question 442 (ID: 20110714)}
\textbf{Date: }July 14,2011
\vspace{6pt}

This patient was anicteric. What is the diagnosis?
\vspace{12pt}

\textbf{Options:}
\begin{enumerate}
\item[A.] Acromegaly
\item[B.] Addison's disease
\item[C.] Diabetes mellitus
\item[D.] Graves' disease
\item[E.] Somatostatinoma
\end{enumerate}

\textbf{Image:}
\begin{center}
\includegraphics[width=0.45\textwidth,height=0.50\textheight,width=0.90\textwidth,keepaspectratio]{images/nejm_20110714.jpg}
\end{center}
\vspace{12pt}
\newpage

\section*{Question 443 (ID: 20110721)}
\textbf{Date: }July 21,2011
\vspace{6pt}

What is the diagnosis?
\vspace{12pt}

\textbf{Options:}
\begin{enumerate}
\item[A.] Dermoid cyst
\item[B.] Hemangioma
\item[C.] Mucocele
\item[D.] Mucosal neuroma
\item[E.] Pyogenic granuloma
\end{enumerate}

\textbf{Image:}
\begin{center}
\includegraphics[width=0.95\textwidth,height=0.50\textheight,width=0.90\textwidth,keepaspectratio]{images/nejm_20110721.jpg}
\end{center}
\vspace{12pt}
\newpage

\section*{Question 444 (ID: 20110728)}
\textbf{Date: }July 28,2011
\vspace{6pt}

What is the diagnosis?
\vspace{12pt}

\textbf{Options:}
\begin{enumerate}
\item[A.] Anterior subluxation of C4
\item[B.] Atlantoaxial dislocation
\item[C.] Epiglottitis
\item[D.] Esophageal diverticulum
\item[E.] Medullary thyroid carcinoma
\end{enumerate}

\textbf{Image:}
\begin{center}
\includegraphics[width=0.92\textwidth,height=0.50\textheight,width=0.90\textwidth,keepaspectratio]{images/nejm_20110728.jpg}
\end{center}
\vspace{12pt}
\newpage

\section*{Question 445 (ID: 20110804)}
\textbf{Date: }August 04,2011
\vspace{6pt}

This lesion developed on the chest of a 35-year-old man during treatment for a respiratory tract infection. What is the diagnosis?
\vspace{12pt}

\textbf{Options:}
\begin{enumerate}
\item[A.] Acute urticaria
\item[B.] Drug-induced pemphigus
\item[C.] Erythema annulare
\item[D.] Fixed drug eruption
\item[E.] Stevens-Johnson syndrome
\end{enumerate}

\textbf{Image:}
\begin{center}
\includegraphics[width=0.95\textwidth,height=0.50\textheight,width=0.90\textwidth,keepaspectratio]{images/nejm_20110804.jpg}
\end{center}
\vspace{12pt}
\newpage

\section*{Question 446 (ID: 20110811)}
\textbf{Date: }August 11,2011
\vspace{6pt}

What is the diagnosis?
\vspace{12pt}

\textbf{Options:}
\begin{enumerate}
\item[A.] Calciphylaxis
\item[B.] Epidermoid cysts
\item[C.] Hidradenitis suppurativa
\item[D.] Onchocerciasis
\item[E.] Scrotal calcinosis
\end{enumerate}

\textbf{Image:}
\begin{center}
\includegraphics[width=0.95\textwidth,height=0.50\textheight,width=0.90\textwidth,keepaspectratio]{images/nejm_20110811.jpg}
\end{center}
\vspace{12pt}
\newpage

\section*{Question 447 (ID: 20110818)}
\textbf{Date: }August 18,2011
\vspace{6pt}

Treatment with a medication from which one of the following drug groups has been associated with the problem identified in the abdominal radiograph?
\vspace{12pt}

\textbf{Options:}
\begin{enumerate}
\item[A.] Alpha-glucosidase inhibitor
\item[B.] Biguanide
\item[C.] Dipeptidyl-peptidase 4 inhibitor
\item[D.] Sulfonylurea
\item[E.] Thiazolidinedione
\end{enumerate}

\textbf{Image:}
\begin{center}
\includegraphics[width=0.65\textwidth,height=0.50\textheight,width=0.90\textwidth,keepaspectratio]{images/nejm_20110818.jpg}
\end{center}
\vspace{12pt}
\newpage

\section*{Question 448 (ID: 20110825)}
\textbf{Date: }August 25,2011
\vspace{6pt}

What is the diagnosis?
\vspace{12pt}

\textbf{Options:}
\begin{enumerate}
\item[A.] Bullous pemphigoid
\item[B.] Hypoparathyroidism
\item[C.] Osteosarcoma
\item[D.] Scleroderma
\item[E.] Wilson's disease
\end{enumerate}

\textbf{Image:}
\begin{center}
\includegraphics[width=0.95\textwidth,height=0.50\textheight,width=0.90\textwidth,keepaspectratio]{images/nejm_20110825.jpg}
\end{center}
\vspace{12pt}
\newpage

\section*{Question 449 (ID: 20110901)}
\textbf{Date: }September 01,2011
\vspace{6pt}

Which one of the following medications is most likely to be responsible for this appearance?
\vspace{12pt}

\textbf{Options:}
\begin{enumerate}
\item[A.] Clarithromycin
\item[B.] Dexamethasone
\item[C.] Doxorubicin
\item[D.] Efavirenz
\item[E.] Ferrous sulfate
\end{enumerate}

\textbf{Image:}
\begin{center}
\includegraphics[width=0.95\textwidth,height=0.50\textheight,width=0.90\textwidth,keepaspectratio]{images/nejm_20110901.jpg}
\end{center}
\vspace{12pt}
\newpage

\section*{Question 450 (ID: 20110908)}
\textbf{Date: }September 08,2011
\vspace{6pt}

Polarized microscopical examination of a specimen from a patient's elbow revealed what type of crystals?
\vspace{12pt}

\textbf{Options:}
\begin{enumerate}
\item[A.] Calcium apatite
\item[B.] Calcium oxalate
\item[C.] Calcium pyrophosphate dihydrate
\item[D.] Iron sulfide
\item[E.] Monosodium urate
\end{enumerate}

\textbf{Image:}
\begin{center}
\includegraphics[width=0.87\textwidth,height=0.50\textheight,width=0.90\textwidth,keepaspectratio]{images/nejm_20110908.jpg}
\end{center}
\vspace{12pt}
\newpage

\section*{Question 451 (ID: 20110915)}
\textbf{Date: }September 15,2011
\vspace{6pt}

This patient had a history of chronic hepatitis C virus (HCV) infection and facial hypertrichosis. What is the diagnosis?
\vspace{12pt}

\textbf{Options:}
\begin{enumerate}
\item[A.] Acute leukemia
\item[B.] Arsenic keratosis
\item[C.] Dermatomyositis
\item[D.] Porphyria cutanea tarda
\item[E.] Secondary syphilis
\end{enumerate}

\textbf{Image:}
\begin{center}
\includegraphics[width=0.95\textwidth,height=0.50\textheight,width=0.90\textwidth,keepaspectratio]{images/nejm_20110915.jpg}
\end{center}
\vspace{12pt}
\newpage

\section*{Question 452 (ID: 20110922)}
\textbf{Date: }September 22,2011
\vspace{6pt}

What is the diagnosis?
\vspace{12pt}

\textbf{Options:}
\begin{enumerate}
\item[A.] Melanoma
\item[B.] Onychomycosis
\item[C.] Psoriasis
\item[D.] Yellow-nail syndrome
\item[E.] Zinc deficiency
\end{enumerate}

\textbf{Image:}
\begin{center}
\includegraphics[width=0.95\textwidth,height=0.50\textheight,width=0.90\textwidth,keepaspectratio]{images/nejm_20110922.jpg}
\end{center}
\vspace{12pt}
\newpage

\section*{Question 453 (ID: 20110929)}
\textbf{Date: }September 29,2011
\vspace{6pt}

What is the diagnosis?
\vspace{12pt}

\textbf{Options:}
\begin{enumerate}
\item[A.] Diaphragmatic rupture
\item[B.] Lung abscess
\item[C.] Marfan syndrome
\item[D.] Plombage
\item[E.] Tuberculosis
\end{enumerate}

\textbf{Image:}
\begin{center}
\includegraphics[width=0.7\textwidth,height=0.50\textheight,width=0.90\textwidth,keepaspectratio]{images/nejm_20110929.jpg}
\end{center}
\vspace{12pt}
\newpage

\section*{Question 454 (ID: 20111006)}
\textbf{Date: }October 06,2011
\vspace{6pt}

This patient was being treated by an endocrinologist for which one of the following conditions?
\vspace{12pt}

\textbf{Options:}
\begin{enumerate}
\item[A.] Acromegaly
\item[B.] Cushing's disease
\item[C.] Graves' disease
\item[D.] Hashimoto's thyroiditis
\item[E.] Type 1 diabetes
\end{enumerate}

\textbf{Image:}
\begin{center}
\includegraphics[width=0.66\textwidth,height=0.50\textheight,width=0.90\textwidth,keepaspectratio]{images/nejm_20111006.jpg}
\end{center}
\vspace{12pt}
\newpage

\section*{Question 455 (ID: 20111013)}
\textbf{Date: }October 13,2011
\vspace{6pt}

These skin changes developed during treatment for end-stage renal failure. What is the diagnosis?
\vspace{12pt}

\textbf{Options:}
\begin{enumerate}
\item[A.] Amyloidosis
\item[B.] Nephrogenic fibrosing dermopathy
\item[C.] Syphilis
\item[D.] Scleroderma
\item[E.] Uremic arteriopathy
\end{enumerate}

\textbf{Image:}
\begin{center}
\includegraphics[width=0.78\textwidth,height=0.50\textheight,width=0.90\textwidth,keepaspectratio]{images/nejm_20111013.jpg}
\end{center}
\vspace{12pt}
\newpage

\section*{Question 456 (ID: 20111020)}
\textbf{Date: }October 20,2011
\vspace{6pt}

What is the diagnosis?
\vspace{12pt}

\textbf{Options:}
\begin{enumerate}
\item[A.] Bowenoid papulosis
\item[B.] Condyloma acuminatum
\item[C.] Molluscum contagiosum
\item[D.] Seborrheic keratosis
\item[E.] Verrucous carcinoma
\end{enumerate}

\textbf{Image:}
\begin{center}
\includegraphics[width=0.55\textwidth,height=0.50\textheight,width=0.90\textwidth,keepaspectratio]{images/nejm_20111020.jpg}
\end{center}
\vspace{12pt}
\newpage

\section*{Question 457 (ID: 20111027)}
\textbf{Date: }October 27,2011
\vspace{6pt}

What is the diagnosis?
\vspace{12pt}

\textbf{Options:}
\begin{enumerate}
\item[A.] Aphthous stomatitis
\item[B.] Foliate papillae
\item[C.] Fordyce granules
\item[D.] Herpes zoster
\item[E.] Syphilis
\end{enumerate}

\textbf{Image:}
\begin{center}
\includegraphics[width=0.83\textwidth,height=0.50\textheight,width=0.90\textwidth,keepaspectratio]{images/nejm_20111027.jpg}
\end{center}
\vspace{12pt}
\newpage

\section*{Question 458 (ID: 20111103)}
\textbf{Date: }November 03,2011
\vspace{6pt}

What is the diagnosis in this infant who was born at term?
\vspace{12pt}

\textbf{Options:}
\begin{enumerate}
\item[A.] Bilateral cryptorchidism
\item[B.] Congenital asplenia
\item[C.] Congenital toxoplasmosis
\item[D.] Neonatal epididymitis
\item[E.] Retroperitoneal hemorrhage
\end{enumerate}

\textbf{Image:}
\begin{center}
\includegraphics[width=0.56\textwidth,height=0.50\textheight,width=0.90\textwidth,keepaspectratio]{images/nejm_20111103.jpg}
\end{center}
\vspace{12pt}
\newpage

\section*{Question 459 (ID: 20111110)}
\textbf{Date: }November 10,2011
\vspace{6pt}

What is the diagnosis?
\vspace{12pt}

\textbf{Options:}
\begin{enumerate}
\item[A.] Cholesteatoma
\item[B.] Globus tympanicus
\item[C.] Herpes zoster
\item[D.] Otitis media
\item[E.] Squamous-cell carcinoma
\end{enumerate}

\textbf{Image:}
\begin{center}
\includegraphics[width=0.95\textwidth,height=0.50\textheight,width=0.90\textwidth,keepaspectratio]{images/nejm_20111110.jpg}
\end{center}
\vspace{12pt}
\newpage

\section*{Question 460 (ID: 20111117)}
\textbf{Date: }November 17,2011
\vspace{6pt}

What is the diagnosis?
\vspace{12pt}

\textbf{Options:}
\begin{enumerate}
\item[A.] Klippel-Trénaunay-Weber syndrome
\item[B.] McCune-Albright syndrome
\item[C.] Multiple lipomatosis
\item[D.] Neurofibromatosis
\item[E.] Proteus syndrome
\end{enumerate}

\textbf{Image:}
\begin{center}
\includegraphics[width=0.95\textwidth,height=0.50\textheight,width=0.90\textwidth,keepaspectratio]{images/nejm_20111117.jpg}
\end{center}
\vspace{12pt}
\newpage

\section*{Question 461 (ID: 20111124)}
\textbf{Date: }November 24,2011
\vspace{6pt}

This patient presented with a 1-day history of fever, acute painful symmetric polyarthritis, abdominal pain, and hematemesis. What is the diagnosis?
\vspace{12pt}

\textbf{Options:}
\begin{enumerate}
\item[A.] Bacterial endocarditis
\item[B.] Henoch-Schönlein purpura
\item[C.] Purpura fulminans
\item[D.] Rheumatic fever
\item[E.] Rocky Mountain spotted fever
\end{enumerate}

\textbf{Image:}
\begin{center}
\includegraphics[width=0.95\textwidth,height=0.50\textheight,width=0.90\textwidth,keepaspectratio]{images/nejm_20111124.jpg}
\end{center}
\vspace{12pt}
\newpage

\section*{Question 462 (ID: 20111201)}
\textbf{Date: }December 01,2011
\vspace{6pt}

What cranial nerve palsy is most clearly illustrated in this image?
\vspace{12pt}

\textbf{Options:}
\begin{enumerate}
\item[A.] Left facial nerve
\item[B.] Left glossopharyngeal nerve
\item[C.] Left hypoglossal nerve
\item[D.] Right glossopharyngeal nerve
\item[E.] Right hypoglossal nerve
\end{enumerate}

\textbf{Image:}
\begin{center}
\includegraphics[width=0.83\textwidth,height=0.50\textheight,width=0.90\textwidth,keepaspectratio]{images/nejm_20111201.jpg}
\end{center}
\vspace{12pt}
\newpage

\section*{Question 463 (ID: 20111208)}
\textbf{Date: }December 08,2011
\vspace{6pt}

What is the diagnosis?
\vspace{12pt}

\textbf{Options:}
\begin{enumerate}
\item[A.] Dermatitis herpetiformis
\item[B.] Erythema multiforme
\item[C.] Guttate psoriasis
\item[D.] Pemphigus vulgaris
\item[E.] Varicella-zoster infection
\end{enumerate}

\textbf{Image:}
\begin{center}
\includegraphics[width=0.94\textwidth,height=0.50\textheight,width=0.90\textwidth,keepaspectratio]{images/nejm_20111208.jpg}
\end{center}
\vspace{12pt}
\newpage

\section*{Question 464 (ID: 20111215)}
\textbf{Date: }December 15,2011
\vspace{6pt}

What is the diagnosis?
\vspace{12pt}

\textbf{Options:}
\begin{enumerate}
\item[A.] Achalasia
\item[B.] Aortic dissection
\item[C.] Esophageal perforation
\item[D.] Pneumopericardium
\item[E.] Ventricular rupture
\end{enumerate}

\textbf{Image:}
\begin{center}
\includegraphics[width=0.89\textwidth,height=0.50\textheight,width=0.90\textwidth,keepaspectratio]{images/nejm_20111215.jpg}
\end{center}
\vspace{12pt}
\newpage

\section*{Question 465 (ID: 20111222)}
\textbf{Date: }December 22,2011
\vspace{6pt}

What is the diagnosis?
\vspace{12pt}

\textbf{Options:}
\begin{enumerate}
\item[A.] Cushing's syndrome
\item[B.] Decompression sickness
\item[C.] Electrocution
\item[D.] Filiariasis
\item[E.] Lymphangitis
\end{enumerate}

\textbf{Image:}
\begin{center}
\includegraphics[width=0.9\textwidth,height=0.50\textheight,width=0.90\textwidth,keepaspectratio]{images/nejm_20111222.jpg}
\end{center}
\vspace{12pt}
\newpage

\section*{Question 466 (ID: 20111229)}
\textbf{Date: }December 29,2011
\vspace{6pt}

What structure has ruptured in this image?
\vspace{12pt}

\textbf{Options:}
\begin{enumerate}
\item[A.] Diaphragm
\item[B.] Esophagus
\item[C.] Intercostal muscle
\item[D.] Interventricular septum
\item[E.] Pericardium
\end{enumerate}

\textbf{Image:}
\begin{center}
\includegraphics[width=0.82\textwidth,height=0.50\textheight,width=0.90\textwidth,keepaspectratio]{images/nejm_20111229.jpg}
\end{center}
\vspace{12pt}
\newpage

\section*{Question 467 (ID: 20120105)}
\textbf{Date: }January 05,2012
\vspace{6pt}

This finding appeared on microscopical examination of the bronchoalveolar-lavage fluid of a patient with pulmonary nodules and cavitations. What is the diagnosis?
\vspace{12pt}

\textbf{Options:}
\begin{enumerate}
\item[A.] Asbestosis
\item[B.] Ascariasis
\item[C.] Aspergillosis
\item[D.] Coal-worker's pneumoconiosis
\item[E.] Paragonimiasis
\end{enumerate}

\textbf{Image:}
\begin{center}
\includegraphics[width=0.85\textwidth,height=0.50\textheight,width=0.90\textwidth,keepaspectratio]{images/nejm_20120105.jpg}
\end{center}
\vspace{12pt}
\newpage

\section*{Question 468 (ID: 20120112)}
\textbf{Date: }January 12,2012
\vspace{6pt}

What is the diagnosis?
\vspace{12pt}

\textbf{Options:}
\begin{enumerate}
\item[A.] Gonococcal arthritis
\item[B.] Heberden's node
\item[C.] Rheumatoid arthritis
\item[D.] Sarcoidosis
\item[E.] Tophaceous gout
\end{enumerate}

\textbf{Image:}
\begin{center}
\includegraphics[width=0.79\textwidth,height=0.50\textheight,width=0.90\textwidth,keepaspectratio]{images/nejm_20120112.jpg}
\end{center}
\vspace{12pt}
\newpage

\section*{Question 469 (ID: 20120119)}
\textbf{Date: }January 19,2012
\vspace{6pt}

This patient presented with cough. What diagnosis accounts for the combination of findings on the bone scan?
\vspace{12pt}

\textbf{Options:}
\begin{enumerate}
\item[A.] Adverse effect of chronic glucocorticoids
\item[B.] Mastocytosis
\item[C.] Metastatic lung cancer
\item[D.] Osteomalacia with fracture
\item[E.] Paget's disease
\end{enumerate}

\textbf{Image:}
\begin{center}
\includegraphics[width=0.72\textwidth,height=0.50\textheight,width=0.90\textwidth,keepaspectratio]{images/nejm_20120119.jpg}
\end{center}
\vspace{12pt}
\newpage

\section*{Question 470 (ID: 20120126)}
\textbf{Date: }January 26,2012
\vspace{6pt}

What is the diagnosis?
\vspace{12pt}

\textbf{Options:}
\begin{enumerate}
\item[A.] Cushing's syndrome
\item[B.] Divarication of the rectus abdominis
\item[C.] Familial partial lipodystrophy of the Dunnigan type
\item[D.] Insulin lipohypertrophy
\item[E.] Morgagni hernia
\end{enumerate}

\textbf{Image:}
\begin{center}
\includegraphics[width=0.77\textwidth,height=0.50\textheight,width=0.90\textwidth,keepaspectratio]{images/nejm_20120126.jpg}
\end{center}
\vspace{12pt}
\newpage

\section*{Question 471 (ID: 20120202)}
\textbf{Date: }February 02,2012
\vspace{6pt}

This 23-year-old man was involved in a motor vehicle accident. What is the diagnosis?
\vspace{12pt}

\textbf{Options:}
\begin{enumerate}
\item[A.] Aortic dissection
\item[B.] Cardiac rupture
\item[C.] Diaphragmatic rupture
\item[D.] Pneumothorax
\item[E.] Vertebral fractures
\end{enumerate}

\textbf{Image:}
\begin{center}
\includegraphics[width=0.86\textwidth,height=0.50\textheight,width=0.90\textwidth,keepaspectratio]{images/nejm_20120202.jpg}
\end{center}
\vspace{12pt}
\newpage

\section*{Question 472 (ID: 20120209)}
\textbf{Date: }February 09,2012
\vspace{6pt}

What is the diagnosis?
\vspace{12pt}

\textbf{Options:}
\begin{enumerate}
\item[A.] Arcus juvenilus
\item[B.] Calcific band keratopathy
\item[C.] Herpetic keratitis
\item[D.] Kayser-Fleischer ring
\item[E.] Vernal conjunctivitis
\end{enumerate}

\textbf{Image:}
\begin{center}
\includegraphics[width=0.95\textwidth,height=0.50\textheight,width=0.90\textwidth,keepaspectratio]{images/nejm_20120209.jpg}
\end{center}
\vspace{12pt}
\newpage

\section*{Question 473 (ID: 20120216)}
\textbf{Date: }February 16,2012
\vspace{6pt}

What is the most likely underlying diagnosis?
\vspace{12pt}

\textbf{Options:}
\begin{enumerate}
\item[A.] Cirrhosis
\item[B.] Chronic renal failure
\item[C.] Hypothyroidism
\item[D.] Myeloma
\item[E.] Sickle cell disease
\end{enumerate}

\textbf{Image:}
\begin{center}
\includegraphics[width=0.83\textwidth,height=0.50\textheight,width=0.90\textwidth,keepaspectratio]{images/nejm_20120216.jpg}
\end{center}
\vspace{12pt}
\newpage

\section*{Question 474 (ID: 20120223)}
\textbf{Date: }February 23,2012
\vspace{6pt}

This patient had diabetes. What is the diagnosis?
\vspace{12pt}

\textbf{Options:}
\begin{enumerate}
\item[A.] Diabetic bullae
\item[B.] Eruptive xanthomas
\item[C.] Lichen amyloidosis
\item[D.] Necrobiosis lipoidica
\item[E.] Pustular psoriasis
\end{enumerate}

\textbf{Image:}
\begin{center}
\includegraphics[width=0.95\textwidth,height=0.50\textheight,width=0.90\textwidth,keepaspectratio]{images/nejm_20120223.jpg}
\end{center}
\vspace{12pt}
\newpage

\section*{Question 475 (ID: 20120301)}
\textbf{Date: }March 01,2012
\vspace{6pt}

This patient was hypothyroid. What is the diagnosis?
\vspace{12pt}

\textbf{Options:}
\begin{enumerate}
\item[A.] Branchial cleft cyst
\item[B.] Ectopic thyroid
\item[C.] Laryngocele
\item[D.] Lipoma
\item[E.] Papillary thyroid cancer
\end{enumerate}

\textbf{Image:}
\begin{center}
\includegraphics[width=0.64\textwidth,height=0.50\textheight,width=0.90\textwidth,keepaspectratio]{images/nejm_20120301.jpg}
\end{center}
\vspace{12pt}
\newpage

\section*{Question 476 (ID: 20120308)}
\textbf{Date: }March 08,2012
\vspace{6pt}

What is the diagnosis?
\vspace{12pt}

\textbf{Options:}
\begin{enumerate}
\item[A.] Coarctation of the aorta
\item[B.] Left atrial enlargement
\item[C.] Phrenic nerve palsy
\item[D.] Pulmonary embolism
\item[E.] Sarcoidosis
\end{enumerate}

\textbf{Image:}
\begin{center}
\includegraphics[width=0.89\textwidth,height=0.50\textheight,width=0.90\textwidth,keepaspectratio]{images/nejm_20120308.jpg}
\end{center}
\vspace{12pt}
\newpage

\section*{Question 477 (ID: 20120315)}
\textbf{Date: }March 15,2012
\vspace{6pt}

What is the diagnosis?
\vspace{12pt}

\textbf{Options:}
\begin{enumerate}
\item[A.] Arcus senilis
\item[B.] Galactosialidosis
\item[C.] Hemochromatosis
\item[D.] Keratoconus
\item[E.] Wilson's disease
\end{enumerate}

\textbf{Image:}
\begin{center}
\includegraphics[width=0.95\textwidth,height=0.50\textheight,width=0.90\textwidth,keepaspectratio]{images/nejm_20120315.jpg}
\end{center}
\vspace{12pt}
\newpage

\section*{Question 478 (ID: 20120322)}
\textbf{Date: }March 22,2012
\vspace{6pt}

What is the diagnosis?
\vspace{12pt}

\textbf{Options:}
\begin{enumerate}
\item[A.] Beta-galactosidase deficiency
\item[B.] Fordyce's angiokeratomas
\item[C.] Radiation dermatitis
\item[D.] Scabies
\item[E.] Varicocele
\end{enumerate}

\textbf{Image:}
\begin{center}
\includegraphics[width=0.63\textwidth,height=0.50\textheight,width=0.90\textwidth,keepaspectratio]{images/nejm_20120322.jpg}
\end{center}
\vspace{12pt}
\newpage

\section*{Question 479 (ID: 20120329)}
\textbf{Date: }March 29,2012
\vspace{6pt}

Which coronary artery is occluded?
\vspace{12pt}

\textbf{Options:}
\begin{enumerate}
\item[A.] Left anterior descending
\item[B.] Left circumflex
\item[C.] Left diagonal branch
\item[D.] Left main stem
\end{enumerate}

\textbf{Image:}
\begin{center}
\includegraphics[width=0.95\textwidth,height=0.50\textheight,width=0.90\textwidth,keepaspectratio]{images/nejm_20120329.jpg}
\end{center}
\vspace{12pt}
\newpage

\section*{Question 480 (ID: 20120405)}
\textbf{Date: }April 05,2012
\vspace{6pt}

What is the diagnosis?
\vspace{12pt}

\textbf{Options:}
\begin{enumerate}
\item[A.] Granulomatosis with polyangiitis (Wegener's)
\item[B.] Lepromatous leprosy
\item[C.] Neurofibromatosis type 1
\item[D.] Sarcoidosis
\item[E.] Tertiary syphilis
\end{enumerate}

\textbf{Image:}
\begin{center}
\includegraphics[width=0.74\textwidth,height=0.50\textheight,width=0.90\textwidth,keepaspectratio]{images/nejm_20120405.jpg}
\end{center}
\vspace{12pt}
\newpage

\section*{Question 481 (ID: 20120412)}
\textbf{Date: }April 12,2012
\vspace{6pt}

What is the diagnosis?
\vspace{12pt}

\textbf{Options:}
\begin{enumerate}
\item[A.] Cryptorchidism
\item[B.] Femoral-artery pseudoaneurysm
\item[C.] Iliopectineal bursitis
\item[D.] Inguinal hernia
\item[E.] Metastatic inguinal lymphadenopathy
\end{enumerate}

\textbf{Image:}
\begin{center}
\includegraphics[width=0.95\textwidth,height=0.50\textheight,width=0.90\textwidth,keepaspectratio]{images/nejm_20120412.jpg}
\end{center}
\vspace{12pt}
\newpage

\section*{Question 482 (ID: 20120419)}
\textbf{Date: }April 19,2012
\vspace{6pt}

Which of these medications, used to treat this patient's glioblastoma multiforme, is most likely to have contributed to this complication?
\vspace{12pt}

\textbf{Options:}
\begin{enumerate}
\item[A.] Acetazolamide
\item[B.] Dexamethasone
\item[C.] Levetiracetam
\item[D.] Temozolomide
\item[E.] Topotecan
\end{enumerate}

\textbf{Image:}
\begin{center}
\includegraphics[width=0.8\textwidth,height=0.50\textheight,width=0.90\textwidth,keepaspectratio]{images/nejm_20120419.jpg}
\end{center}
\vspace{12pt}
\newpage

\section*{Question 483 (ID: 20120426)}
\textbf{Date: }April 26,2012
\vspace{6pt}

What is the most likely infecting organism in this patient with prostatitis?
\vspace{12pt}

\textbf{Options:}
\begin{enumerate}
\item[A.] Blastomyces dermatitidis
\item[B.] Escherichia coli
\item[C.] Histoplasma capsulatum
\item[D.] Pseudomonas aeruginosa
\item[E.] Staphylococcus aureus
\end{enumerate}

\textbf{Image:}
\begin{center}
\includegraphics[width=0.59\textwidth,height=0.50\textheight,width=0.90\textwidth,keepaspectratio]{images/nejm_20120426.jpg}
\end{center}
\vspace{12pt}
\newpage

\section*{Question 484 (ID: 20120503)}
\textbf{Date: }May 03,2012
\vspace{6pt}

What is the diagnosis?
\vspace{12pt}

\textbf{Options:}
\begin{enumerate}
\item[A.] Digoxin poisoning
\item[B.] Hyperkalemia
\item[C.] Intra-aortic balloon pump
\item[D.] Pericardial effusion
\item[E.] Right fascicular block
\end{enumerate}

\textbf{Image:}
\begin{center}
\includegraphics[width=0.95\textwidth,height=0.50\textheight,width=0.90\textwidth,keepaspectratio]{images/nejm_20120503.jpg}
\end{center}
\vspace{12pt}
\newpage

\section*{Question 485 (ID: 20120510)}
\textbf{Date: }May 10,2012
\vspace{6pt}

What is the most likely diagnosis?
\vspace{12pt}

\textbf{Options:}
\begin{enumerate}
\item[A.] Cutaneous T-cell lymphoma
\item[B.] Eosinophilic folliculitis
\item[C.] Impetigo
\item[D.] Pustular rosacea
\item[E.] Subcorneal pustular dermatosis
\end{enumerate}

\textbf{Image:}
\begin{center}
\includegraphics[width=0.87\textwidth,height=0.50\textheight,width=0.90\textwidth,keepaspectratio]{images/nejm_20120510.jpg}
\end{center}
\vspace{12pt}
\newpage

\section*{Question 486 (ID: 20120517)}
\textbf{Date: }May 17,2012
\vspace{6pt}

What is the diagnosis?
\vspace{12pt}

\textbf{Options:}
\begin{enumerate}
\item[A.] Cystic fibrosis
\item[B.] Non-small-cell lung cancer
\item[C.] Sarcoidosis
\item[D.] Thymoma
\item[E.] Tuberculosis
\end{enumerate}

\textbf{Image:}
\begin{center}
\includegraphics[width=0.95\textwidth,height=0.50\textheight,width=0.90\textwidth,keepaspectratio]{images/nejm_20120517.jpg}
\end{center}
\vspace{12pt}
\newpage

\section*{Question 487 (ID: 20120524)}
\textbf{Date: }May 24,2012
\vspace{6pt}

What is the most likely diagnosis?
\vspace{12pt}

\textbf{Options:}
\begin{enumerate}
\item[A.] Anklylosing spondylitis
\item[B.] Ascariasis
\item[C.] Diabetes mellitus
\item[D.] Paget's disease
\item[E.] Prostate cancer
\end{enumerate}

\textbf{Image:}
\begin{center}
\includegraphics[width=0.95\textwidth,height=0.50\textheight,width=0.90\textwidth,keepaspectratio]{images/nejm_20120524.jpg}
\end{center}
\vspace{12pt}
\newpage

\section*{Question 488 (ID: 20120531)}
\textbf{Date: }May 31,2012
\vspace{6pt}

What is the most likely diagnosis?
\vspace{12pt}

\textbf{Options:}
\begin{enumerate}
\item[A.] Cat scratch disease
\item[C.] Sarcoidosis
\item[D.] Sjogren's syndrome
\item[E.] Tuberculosis
\end{enumerate}

\textbf{Image:}
\begin{center}
\includegraphics[width=0.95\textwidth,height=0.50\textheight,width=0.90\textwidth,keepaspectratio]{images/nejm_20120531.jpg}
\end{center}
\vspace{12pt}
\newpage

\section*{Question 489 (ID: 20120607)}
\textbf{Date: }June 07,2012
\vspace{6pt}

What is the diagnosis?
\vspace{12pt}

\textbf{Options:}
\begin{enumerate}
\item[A.] Bulimia nervosa
\item[B.] Dental abscess
\item[D.] Obstructive parotitis
\item[E.] Peritonsillar abscess
\end{enumerate}

\textbf{Image:}
\begin{center}
\includegraphics[width=0.95\textwidth,height=0.50\textheight,width=0.90\textwidth,keepaspectratio]{images/nejm_20120607.jpg}
\end{center}
\vspace{12pt}
\newpage

\section*{Question 490 (ID: 20120614)}
\textbf{Date: }June 14,2012
\vspace{6pt}

What is the diagnosis?
\vspace{12pt}

\textbf{Options:}
\begin{enumerate}
\item[A.] Abscess
\item[B.] Baker's cyst
\item[C.] Bursitis
\item[E.] Septic arthritis
\end{enumerate}

\textbf{Image:}
\begin{center}
\includegraphics[width=0.71\textwidth,height=0.50\textheight,width=0.90\textwidth,keepaspectratio]{images/nejm_20120614.jpg}
\end{center}
\vspace{12pt}
\newpage

\section*{Question 491 (ID: 20120621)}
\textbf{Date: }June 21,2012
\vspace{6pt}

What is the diagnosis?
\vspace{12pt}

\textbf{Options:}
\begin{enumerate}
\item[A.] Erythema induratum
\item[B.] Erythema nodosum
\item[C.] Granuloma annulare
\item[D.] Necrobiosis lipoidica
\item[E.] Pretibial myxedema
\end{enumerate}

\textbf{Image:}
\begin{center}
\includegraphics[width=0.55\textwidth,height=0.50\textheight,width=0.90\textwidth,keepaspectratio]{images/nejm_20120621.jpg}
\end{center}
\vspace{12pt}
\newpage

\section*{Question 492 (ID: 20120628)}
\textbf{Date: }June 28,2012
\vspace{6pt}

What is the diagnosis?
\vspace{12pt}

\textbf{Options:}
\begin{enumerate}
\item[B.] Esophageal obstruction
\item[C.] Laryngomalacia
\item[D.] Spina bifida
\item[E.] Zenker diverticulum
\end{enumerate}

\textbf{Image:}
\begin{center}
\includegraphics[width=0.69\textwidth,height=0.50\textheight,width=0.90\textwidth,keepaspectratio]{images/nejm_20120628.jpg}
\end{center}
\vspace{12pt}
\newpage

\section*{Question 493 (ID: 20120705)}
\textbf{Date: }July 05,2012
\vspace{6pt}

What is the diagnosis?
\vspace{12pt}

\textbf{Options:}
\begin{enumerate}
\item[A.] Beckwith-Wiedemann syndrome
\item[B.] Hypoglossal nerve injury
\item[C.] Lingual carcinoma
\item[D.] Sarcoidosis
\item[E.] Vitamin B12 deficiency
\end{enumerate}

\textbf{Image:}
\begin{center}
\includegraphics[width=0.95\textwidth,height=0.50\textheight,width=0.90\textwidth,keepaspectratio]{images/nejm_20120705.jpg}
\end{center}
\vspace{12pt}
\newpage

\section*{Question 494 (ID: 20120712)}
\textbf{Date: }July 12,2012
\vspace{6pt}

What is the diagnosis?
\vspace{12pt}

\textbf{Options:}
\begin{enumerate}
\item[A.] Convergent strabismus
\item[B.] Immature cataract
\item[C.] Intraocular tumor
\item[D.] Retinitis pigmentosa
\item[E.] Tay-Sachs disease
\end{enumerate}

\textbf{Image:}
\begin{center}
\includegraphics[width=0.95\textwidth,height=0.50\textheight,width=0.90\textwidth,keepaspectratio]{images/nejm_20120712.jpg}
\end{center}
\vspace{12pt}
\newpage

\section*{Question 495 (ID: 20120719)}
\textbf{Date: }July 19,2012
\vspace{6pt}

What is the diagnosis?
\vspace{12pt}

\textbf{Options:}
\begin{enumerate}
\item[A.] Congenital nevus
\item[B.] Psoriasis
\item[C.] Tuberous sclerosis
\item[D.] Vitiligo
\item[E.] Waardenburg syndrome type II
\end{enumerate}

\textbf{Image:}
\begin{center}
\includegraphics[width=0.59\textwidth,height=0.50\textheight,width=0.90\textwidth,keepaspectratio]{images/nejm_20120719.jpg}
\end{center}
\vspace{12pt}
\newpage

\section*{Question 496 (ID: 20120726)}
\textbf{Date: }July 26,2012
\vspace{6pt}

What is the diagnosis?
\vspace{12pt}

\textbf{Options:}
\begin{enumerate}
\item[A.] Acute mesenteric ischemia
\item[B.] Gastric carcinoma
\item[C.] Intraabdominal hernia
\item[D.] Retroperitoneal fibrosis
\item[E.] Trichobezoar
\end{enumerate}

\textbf{Image:}
\begin{center}
\includegraphics[width=0.86\textwidth,height=0.50\textheight,width=0.90\textwidth,keepaspectratio]{images/nejm_20120726.jpg}
\end{center}
\vspace{12pt}
\newpage

\section*{Question 497 (ID: 20120802)}
\textbf{Date: }August 02,2012
\vspace{6pt}

What is the diagnosis?
\vspace{12pt}

\textbf{Options:}
\begin{enumerate}
\item[A.] Congenital hypothyroidism
\item[B.] Congenital syphilis
\item[C.] Dentinogenesis imperfecta
\item[D.] Neonatal hyperbilirubinemia
\item[E.] Williams syndrome
\end{enumerate}

\textbf{Image:}
\begin{center}
\includegraphics[width=0.95\textwidth,height=0.50\textheight,width=0.90\textwidth,keepaspectratio]{images/nejm_20120802.jpg}
\end{center}
\vspace{12pt}
\newpage

\section*{Question 498 (ID: 20120809)}
\textbf{Date: }August 09,2012
\vspace{6pt}

What diagnosis is suggested by the results of this coronary angiogram?
\vspace{12pt}

\textbf{Options:}
\begin{enumerate}
\item[A.] Diabetes mellitus
\item[B.] Fibromuscular dysplasia
\item[C.] Marfan syndrome
\item[D.] Polyarteritis nodosa
\item[E.] Takayasu arteritis
\end{enumerate}

\textbf{Image:}
\begin{center}
\includegraphics[width=0.83\textwidth,height=0.50\textheight,width=0.90\textwidth,keepaspectratio]{images/nejm_20120809.jpg}
\end{center}
\vspace{12pt}
\newpage

\section*{Question 499 (ID: 20120816)}
\textbf{Date: }August 16,2012
\vspace{6pt}

What is the most likely diagnosis in this 12-year-old boy with recurrent infections?
\vspace{12pt}

\textbf{Options:}
\begin{enumerate}
\item[A.] Brucellosis
\item[B.] Burkitt's lymphoma
\item[C.] Chronic granulomatous disease
\item[D.] Neurofibromatosis
\item[E.] Tuberculous lymphadenitis
\end{enumerate}

\textbf{Image:}
\begin{center}
\includegraphics[width=0.95\textwidth,height=0.50\textheight,width=0.90\textwidth,keepaspectratio]{images/nejm_20120816.jpg}
\end{center}
\vspace{12pt}
\newpage

\section*{Question 500 (ID: 20120823)}
\textbf{Date: }August 23,2012
\vspace{6pt}

This finding was recognized after cardiopulmonary resuscitation. What is the diagnosis?
\vspace{12pt}

\textbf{Options:}
\begin{enumerate}
\item[A.] Air embolism
\item[B.] Central pontine myelinolysis
\item[C.] Intraventricular hemorrhage
\item[D.] Middle cerebral artery dissection
\item[E.] Skull fracture
\end{enumerate}

\textbf{Image:}
\begin{center}
\includegraphics[width=0.9\textwidth,height=0.50\textheight,width=0.90\textwidth,keepaspectratio]{images/nejm_20120823.jpg}
\end{center}
\vspace{12pt}
\newpage

\section*{Question 501 (ID: 20120830)}
\textbf{Date: }August 30,2012
\vspace{6pt}

What is the diagnosis in this patient with abdominal pain?
\vspace{12pt}

\textbf{Options:}
\begin{enumerate}
\item[A.] Amyloidosis
\item[B.] Familial Mediterranean fever
\item[C.] Lead poisoning
\item[D.] Leukemia
\item[E.] Scurvy
\end{enumerate}

\textbf{Image:}
\begin{center}
\includegraphics[width=0.95\textwidth,height=0.50\textheight,width=0.90\textwidth,keepaspectratio]{images/nejm_20120830.jpg}
\end{center}
\vspace{12pt}
\newpage

\section*{Question 502 (ID: 20120906)}
\textbf{Date: }September 06,2012
\vspace{6pt}

What is the diagnosis?
\vspace{12pt}

\textbf{Options:}
\begin{enumerate}
\item[A.] Bedbug bites
\item[B.] Dermatitis herpetiformis
\item[C.] Ecthyma
\item[D.] Guttate psoriasis
\item[E.] Lichen planus
\end{enumerate}

\textbf{Image:}
\begin{center}
\includegraphics[width=0.83\textwidth,height=0.50\textheight,width=0.90\textwidth,keepaspectratio]{images/nejm_20120906.jpg}
\end{center}
\vspace{12pt}
\newpage

\section*{Question 503 (ID: 20120913)}
\textbf{Date: }September 13,2012
\vspace{6pt}

This patient presented with hearing loss and left ear fullness. What is the diagnosis?
\vspace{12pt}

\textbf{Options:}
\begin{enumerate}
\item[A.] Cholesteatoma
\item[B.] Fracture of the temporal bone
\item[C.] Ochronosis
\item[D.] Otomycosis
\item[E.] Relapsing polychondritis
\end{enumerate}

\textbf{Image:}
\begin{center}
\includegraphics[width=0.92\textwidth,height=0.50\textheight,width=0.90\textwidth,keepaspectratio]{images/nejm_20120913.jpg}
\end{center}
\vspace{12pt}
\newpage

\section*{Question 504 (ID: 20120920)}
\textbf{Date: }September 20,2012
\vspace{6pt}

This girl had developmental delay. What is the diagnosis?
\vspace{12pt}

\textbf{Options:}
\begin{enumerate}
\item[A.] Cushing's syndrome
\item[B.] Neurofibromatosis
\item[C.] Pigmentary mosaicism
\item[D.] Sarcoidosis
\item[E.] von Hippel-Lindau disease
\end{enumerate}

\textbf{Image:}
\begin{center}
\includegraphics[width=0.56\textwidth,height=0.50\textheight,width=0.90\textwidth,keepaspectratio]{images/nejm_20120920.jpg}
\end{center}
\vspace{12pt}
\newpage

\section*{Question 505 (ID: 20120927)}
\textbf{Date: }September 27,2012
\vspace{6pt}

This 4-year-old child had intense pruritus. What is the diagnosis?
\vspace{12pt}

\textbf{Options:}
\begin{enumerate}
\item[A.] Bacterial dermatitis
\item[B.] Pilonidal abscess
\item[C.] Psoriasis
\item[D.] Scarlet fever
\item[E.] Sexual abuse
\end{enumerate}

\textbf{Image:}
\begin{center}
\includegraphics[width=0.66\textwidth,height=0.50\textheight,width=0.90\textwidth,keepaspectratio]{images/nejm_20120927.jpg}
\end{center}
\vspace{12pt}
\newpage

\section*{Question 506 (ID: 20121004)}
\textbf{Date: }October 04,2012
\vspace{6pt}

Which one of the following medications is most likely to have led to this appearance?
\vspace{12pt}

\textbf{Options:}
\begin{enumerate}
\item[A.] Bicalutamide
\item[B.] Clomiphene
\item[C.] Glipizide
\item[D.] Mesalamine
\item[E.] Rifampin
\end{enumerate}

\textbf{Image:}
\begin{center}
\includegraphics[width=0.95\textwidth,height=0.50\textheight,width=0.90\textwidth,keepaspectratio]{images/nejm_20121004.jpg}
\end{center}
\vspace{12pt}
\newpage

\section*{Question 507 (ID: 20121011)}
\textbf{Date: }October 11,2012
\vspace{6pt}

What is the most likely diagnosis?
\vspace{12pt}

\textbf{Options:}
\begin{enumerate}
\item[A.] Angioedema
\item[B.] Facial palsy
\item[C.] Frey's syndrome
\item[D.] Parotid adenoma
\item[E.] Parotitis
\end{enumerate}

\textbf{Image:}
\begin{center}
\includegraphics[width=0.83\textwidth,height=0.50\textheight,width=0.90\textwidth,keepaspectratio]{images/nejm_20121011.jpg}
\end{center}
\vspace{12pt}
\newpage

\section*{Question 508 (ID: 20121018)}
\textbf{Date: }October 18,2012
\vspace{6pt}

What presentation is most likely to accompany this angiographic finding?
\vspace{12pt}

\textbf{Options:}
\begin{enumerate}
\item[A.] Binocular visual loss
\item[B.] Hemineglect
\item[C.] Paratonic rigidity
\item[D.] Urinary incontinence
\item[E.] Verbal agnosia
\end{enumerate}

\textbf{Image:}
\begin{center}
\includegraphics[width=0.85\textwidth,height=0.50\textheight,width=0.90\textwidth,keepaspectratio]{images/nejm_20121018.jpg}
\end{center}
\vspace{12pt}
\newpage

\section*{Question 509 (ID: 20121025)}
\textbf{Date: }October 25,2012
\vspace{6pt}

What is the diagnosis?
\vspace{12pt}

\textbf{Options:}
\begin{enumerate}
\item[A.] Addison's disease
\item[B.] Arsenic poisoning
\item[C.] Carotenemia
\item[D.] Glaucoma
\item[E.] Ochronosis
\end{enumerate}

\textbf{Image:}
\begin{center}
\includegraphics[width=0.93\textwidth,height=0.50\textheight,width=0.90\textwidth,keepaspectratio]{images/nejm_20121025.jpg}
\end{center}
\vspace{12pt}
\newpage

\section*{Question 510 (ID: 20121101)}
\textbf{Date: }November 01,2012
\vspace{6pt}

What is the diagnosis?
\vspace{12pt}

\textbf{Options:}
\begin{enumerate}
\item[A.] Basilar artery aneurysm
\item[B.] Central pontine myelinolysis
\item[C.] Dandy-Walker malformation
\item[D.] Glioma
\item[E.] Pontine stroke
\end{enumerate}

\textbf{Image:}
\begin{center}
\includegraphics[width=0.95\textwidth,height=0.50\textheight,width=0.90\textwidth,keepaspectratio]{images/nejm_20121101.jpg}
\end{center}
\vspace{12pt}
\newpage

\section*{Question 511 (ID: 20121108)}
\textbf{Date: }November 08,2012
\vspace{6pt}

What procedure was most recently performed on this patient?
\vspace{12pt}

\textbf{Options:}
\begin{enumerate}
\item[A.] Mitral-valve repair
\item[B.] Pancreatectomy
\item[C.] Roux-en-Y gastric bypass
\item[D.] Thyroidectomy
\item[E.] Total hip replacement
\end{enumerate}

\textbf{Image:}
\begin{center}
\includegraphics[width=0.93\textwidth,height=0.50\textheight,width=0.90\textwidth,keepaspectratio]{images/nejm_20121108.jpg}
\end{center}
\vspace{12pt}
\newpage

\section*{Question 512 (ID: 20121115)}
\textbf{Date: }November 15,2012
\vspace{6pt}

What is the diagnosis?
\vspace{12pt}

\textbf{Options:}
\begin{enumerate}
\item[A.] Chlamydial cervicitis
\item[B.] Endometriosis
\item[C.] Mesonephric remnant
\item[D.] Syphilitic chancre
\item[E.] Squamous-cell carcinoma
\end{enumerate}

\textbf{Image:}
\begin{center}
\includegraphics[width=0.57\textwidth,height=0.50\textheight,width=0.90\textwidth,keepaspectratio]{images/nejm_20121115.jpg}
\end{center}
\vspace{12pt}
\newpage

\section*{Question 513 (ID: 20121122)}
\textbf{Date: }November 22,2012
\vspace{6pt}

What is the most likely diagnosis?
\vspace{12pt}

\textbf{Options:}
\begin{enumerate}
\item[A.] Hairy-cell leukemia
\item[B.] Hodgkin's lymphoma
\item[C.] Infectious mononucleosis
\item[D.] Multiple myeloma
\item[E.] Pernicious anemia
\end{enumerate}

\textbf{Image:}
\begin{center}
\includegraphics[width=0.95\textwidth,height=0.50\textheight,width=0.90\textwidth,keepaspectratio]{images/nejm_20121122.jpg}
\end{center}
\vspace{12pt}
\newpage

\section*{Question 514 (ID: 20121129)}
\textbf{Date: }November 29,2012
\vspace{6pt}

What is the most likely diagnosis?
\vspace{12pt}

\textbf{Options:}
\begin{enumerate}
\item[A.] Pilonidal sinus
\item[B.] Melanoma
\item[C.] Molluscum contagiosum
\item[D.] Keratoacanthoma
\item[E.] Prurigo nodularis
\end{enumerate}

\textbf{Image:}
\begin{center}
\includegraphics[width=0.95\textwidth,height=0.50\textheight,width=0.90\textwidth,keepaspectratio]{images/nejm_20121129.jpg}
\end{center}
\vspace{12pt}
\newpage

\section*{Question 515 (ID: 20121206)}
\textbf{Date: }December 06,2012
\vspace{6pt}

What is the diagnosis?
\vspace{12pt}

\textbf{Options:}
\begin{enumerate}
\item[A.] Nephrogenic systemic sclerosis
\item[B.] Keratoderma
\item[C.] Poikiloderma
\item[D.] Punctate psoriasis
\item[E.] Verrucae
\end{enumerate}

\textbf{Image:}
\begin{center}
\includegraphics[width=0.95\textwidth,height=0.50\textheight,width=0.90\textwidth,keepaspectratio]{images/nejm_20121206.jpg}
\end{center}
\vspace{12pt}
\newpage

\section*{Question 516 (ID: 20121213)}
\textbf{Date: }December 13,2012
\vspace{6pt}

What is the most likely diagnosis?
\vspace{12pt}

\textbf{Options:}
\begin{enumerate}
\item[A.] Abscess
\item[B.] Cavernous hemangioma
\item[C.] Lipoma
\item[D.] Mucocele
\item[E.] Sialolith
\end{enumerate}

\textbf{Image:}
\begin{center}
\includegraphics[width=0.95\textwidth,height=0.50\textheight,width=0.90\textwidth,keepaspectratio]{images/nejm_20121213.jpg}
\end{center}
\vspace{12pt}
\newpage

\section*{Question 517 (ID: 20121220)}
\textbf{Date: }December 20,2012
\vspace{6pt}

What is the diagnosis?
\vspace{12pt}

\textbf{Options:}
\begin{enumerate}
\item[A.] Albright's hereditary osteodystrophy
\item[B.] Hypertrophic osteoarthropathy
\item[C.] Median nerve palsy
\item[D.] Osteoarthritis
\item[E.] Rheumatoid arthritis
\end{enumerate}

\textbf{Image:}
\begin{center}
\includegraphics[width=0.95\textwidth,height=0.50\textheight,width=0.90\textwidth,keepaspectratio]{images/nejm_20121220.jpg}
\end{center}
\vspace{12pt}
\newpage

\section*{Question 518 (ID: 20121227)}
\textbf{Date: }December 27,2012
\vspace{6pt}

This ocular nodule had several black hairs. What is the diagnosis?
\vspace{12pt}

\textbf{Options:}
\begin{enumerate}
\item[A.] Chalazion
\item[B.] Dermoid
\item[C.] Melanoma
\item[D.] Phlyctenule
\item[E.] Pinguecula
\end{enumerate}

\textbf{Image:}
\begin{center}
\includegraphics[width=0.66\textwidth,height=0.50\textheight,width=0.90\textwidth,keepaspectratio]{images/nejm_20121227.jpg}
\end{center}
\vspace{12pt}
\newpage

\section*{Question 519 (ID: 20130103)}
\textbf{Date: }January 03,2013
\vspace{6pt}

What is the diagnosis?
\vspace{12pt}

\textbf{Options:}
\begin{enumerate}
\item[A.] Hand-foot-mouth disease
\item[B.] Herpes simplex virus infection
\item[C.] Herpes zoster virus infection
\item[D.] Folliculitis
\item[E.] Scalded skin syndrome
\end{enumerate}

\textbf{Image:}
\begin{center}
\includegraphics[width=0.95\textwidth,height=0.50\textheight,width=0.90\textwidth,keepaspectratio]{images/nejm_20130103.jpg}
\end{center}
\vspace{12pt}
\newpage

\section*{Question 520 (ID: 20130110)}
\textbf{Date: }January 10,2013
\vspace{6pt}

This blood smear was from a child who developed a high fever after returning from a camping trip in eastern California. What is the diagnosis?
\vspace{12pt}

\textbf{Options:}
\begin{enumerate}
\item[A.] Babesiosis
\item[B.] Brucellosis
\item[C.] Leptospirosis
\item[D.] Rocky Mountain spotted fever
\item[E.] Tickborne relapsing fever
\end{enumerate}

\textbf{Image:}
\begin{center}
\includegraphics[width=0.95\textwidth,height=0.50\textheight,width=0.90\textwidth,keepaspectratio]{images/nejm_20130110.jpg}
\end{center}
\vspace{12pt}
\newpage

\section*{Question 521 (ID: 20130117)}
\textbf{Date: }January 17,2013
\vspace{6pt}

What diagnosis is implied by the results of this gallium scan?
\vspace{12pt}

\textbf{Options:}
\begin{enumerate}
\item[B.] Paget's disease
\item[C.] Sarcoidosis
\item[D.] Septic emboli
\item[E.] Tuberculosis
\end{enumerate}

\textbf{Image:}
\begin{center}
\includegraphics[width=0.46\textwidth,height=0.50\textheight,width=0.90\textwidth,keepaspectratio]{images/nejm_20130117.jpg}
\end{center}
\vspace{12pt}
\newpage

\section*{Question 522 (ID: 20130124)}
\textbf{Date: }January 24,2013
\vspace{6pt}

This patient had an elevated lipase level. What is the diagnosis?
\vspace{12pt}

\textbf{Options:}
\begin{enumerate}
\item[A.] Adenovirus infection
\item[B.] Calciphylaxis
\item[C.] Erythema nodosum
\item[D.] Panniculitis
\item[E.] Tuberculosis
\end{enumerate}

\textbf{Image:}
\begin{center}
\includegraphics[width=0.66\textwidth,height=0.50\textheight,width=0.90\textwidth,keepaspectratio]{images/nejm_20130124.jpg}
\end{center}
\vspace{12pt}
\newpage

\section*{Question 523 (ID: 20130131)}
\textbf{Date: }January 31,2013
\vspace{6pt}

This patient had presented with weight loss. What is the most likely diagnosis?
\vspace{12pt}

\textbf{Options:}
\begin{enumerate}
\item[A.] Gastric cancer
\item[B.] Lung cancer
\item[C.] Melanoma
\item[D.] Nasopharyngeal cancer
\item[E.] Thyroid cancer
\end{enumerate}

\textbf{Image:}
\begin{center}
\includegraphics[width=0.95\textwidth,height=0.50\textheight,width=0.90\textwidth,keepaspectratio]{images/nejm_20130131.jpg}
\end{center}
\vspace{12pt}
\newpage

\section*{Question 524 (ID: 20130207)}
\textbf{Date: }February 07,2013
\vspace{6pt}

These oral ulcers were painless. What is the most likely diagnosis?
\vspace{12pt}

\textbf{Options:}
\begin{enumerate}
\item[A.] Chancroid
\item[B.] Herpes simplex
\item[C.] Iron deficiency
\item[D.] Squamous cell cancer
\item[E.] Syphilis
\end{enumerate}

\textbf{Image:}
\begin{center}
\includegraphics[width=0.95\textwidth,height=0.50\textheight,width=0.90\textwidth,keepaspectratio]{images/nejm_20130207.jpg}
\end{center}
\vspace{12pt}
\newpage

\section*{Question 525 (ID: 20130214)}
\textbf{Date: }February 14,2013
\vspace{6pt}

What is the diagnosis?
\vspace{12pt}

\textbf{Options:}
\begin{enumerate}
\item[A.] Aphthous stomatitis
\item[B.] Bullous pemphigoid
\item[C.] Chickenpox
\item[D.] Herpes zoster
\item[E.] Paraneoplastic pemphigus
\end{enumerate}

\textbf{Image:}
\begin{center}
\includegraphics[width=0.95\textwidth,height=0.50\textheight,width=0.90\textwidth,keepaspectratio]{images/nejm_20130214.jpg}
\end{center}
\vspace{12pt}
\newpage

\section*{Question 526 (ID: 20130221)}
\textbf{Date: }February 21,2013
\vspace{6pt}

What is the diagnosis?
\vspace{12pt}

\textbf{Options:}
\begin{enumerate}
\item[A.] Parulis
\item[B.] Pyogenic granuloma
\item[C.] Peripheral ossifying fibroma
\item[D.] Retrocuspid papillae
\item[E.] Torus mandibularis
\end{enumerate}

\textbf{Image:}
\begin{center}
\includegraphics[width=0.95\textwidth,height=0.50\textheight,width=0.90\textwidth,keepaspectratio]{images/nejm_20130221.jpg}
\end{center}
\vspace{12pt}
\newpage

\section*{Question 527 (ID: 20130228)}
\textbf{Date: }February 28,2013
\vspace{6pt}

This specimen was resected from a child with intestinal obstruction. What is the most likely diagnosis?
\vspace{12pt}

\textbf{Options:}
\begin{enumerate}
\item[A.] Ascariasis
\item[B.] Kala-azar
\item[C.] Meckel's diverticulum
\item[D.] Strongyloidiasis
\item[E.] Trichobezoar
\end{enumerate}

\textbf{Image:}
\begin{center}
\includegraphics[width=0.95\textwidth,height=0.50\textheight,width=0.90\textwidth,keepaspectratio]{images/nejm_20130228.jpg}
\end{center}
\vspace{12pt}
\newpage

\section*{Question 528 (ID: 20130307)}
\textbf{Date: }March 07,2013
\vspace{6pt}

What is the diagnosis?
\vspace{12pt}

\textbf{Options:}
\begin{enumerate}
\item[A.] Addison's disease
\item[B.] Amalgam tattoo
\item[C.] Blue nevus
\item[D.] Hemangioma
\item[E.] Melanoma
\end{enumerate}

\textbf{Image:}
\begin{center}
\includegraphics[width=0.95\textwidth,height=0.50\textheight,width=0.90\textwidth,keepaspectratio]{images/nejm_20130307.jpg}
\end{center}
\vspace{12pt}
\newpage

\section*{Question 529 (ID: 20130314)}
\textbf{Date: }March 14,2013
\vspace{6pt}

What is the diagnosis?
\vspace{12pt}

\textbf{Options:}
\begin{enumerate}
\item[A.] Acromegaly
\item[B.] Ankylosing spondylitis
\item[C.] Fluorosis
\item[D.] Mastocytosis
\item[E.] Multiple myeloma
\end{enumerate}

\textbf{Image:}
\begin{center}
\includegraphics[width=0.48\textwidth,height=0.50\textheight,width=0.90\textwidth,keepaspectratio]{images/nejm_20130314.jpg}
\end{center}
\vspace{12pt}
\newpage

\section*{Question 530 (ID: 20130321)}
\textbf{Date: }March 21,2013
\vspace{6pt}

What is the diagnosis?
\vspace{12pt}

\textbf{Options:}
\begin{enumerate}
\item[A.] Bulimia nervosa
\item[B.] Cocaine abuse
\item[C.] Lichen planus
\item[D.] Ludwig's angina
\item[E.] Osteonecrosis
\end{enumerate}

\textbf{Image:}
\begin{center}
\includegraphics[width=0.95\textwidth,height=0.50\textheight,width=0.90\textwidth,keepaspectratio]{images/nejm_20130321.jpg}
\end{center}
\vspace{12pt}
\newpage

\section*{Question 531 (ID: 20130328)}
\textbf{Date: }March 28,2013
\vspace{6pt}

What is the diagnosis?
\vspace{12pt}

\textbf{Options:}
\begin{enumerate}
\item[A.] Carcinoid syndrome
\item[B.] Mastocytosis
\item[C.] Normal pregnancy
\item[D.] Radial-artery occlusion
\item[E.] Raynaud's phenomenon
\end{enumerate}

\textbf{Image:}
\begin{center}
\includegraphics[width=0.91\textwidth,height=0.50\textheight,width=0.90\textwidth,keepaspectratio]{images/nejm_20130328.jpg}
\end{center}
\vspace{12pt}
\newpage

\section*{Question 532 (ID: 20130404)}
\textbf{Date: }April 04,2013
\vspace{6pt}

What is the diagnosis?
\vspace{12pt}

\textbf{Options:}
\begin{enumerate}
\item[A.] Adrenal cancer
\item[B.] Echinococcal infection
\item[C.] Meckel's diverticulitis
\item[D.] Pneumatosis intestinalis
\item[E.] Trichobezoar
\end{enumerate}

\textbf{Image:}
\begin{center}
\includegraphics[width=0.64\textwidth,height=0.50\textheight,width=0.90\textwidth,keepaspectratio]{images/nejm_20130404.jpg}
\end{center}
\vspace{12pt}
\newpage

\section*{Question 533 (ID: 20130411)}
\textbf{Date: }April 11,2013
\vspace{6pt}

What is the diagnosis?
\vspace{12pt}

\textbf{Options:}
\begin{enumerate}
\item[A.] Amelanotic melanoma
\item[B.] Angioma
\item[C.] Dermal nevus
\item[D.] Pyogenic granuloma
\end{enumerate}

\textbf{Image:}
\begin{center}
\includegraphics[width=0.95\textwidth,height=0.50\textheight,width=0.90\textwidth,keepaspectratio]{images/nejm_20130411.jpg}
\end{center}
\vspace{12pt}
\newpage

\section*{Question 534 (ID: 20130418)}
\textbf{Date: }April 18,2013
\vspace{6pt}

What diagnosis explains these lesions that developed after acupuncture to the area?
\vspace{12pt}

\textbf{Options:}
\begin{enumerate}
\item[A.] Herpetic whitlow
\item[B.] Nummular eczema
\item[C.] Psoriasis
\item[D.] Ringworm
\item[E.] Scabies
\end{enumerate}

\textbf{Image:}
\begin{center}
\includegraphics[width=0.95\textwidth,height=0.50\textheight,width=0.90\textwidth,keepaspectratio]{images/nejm_20130418.jpg}
\end{center}
\vspace{12pt}
\newpage

\section*{Question 535 (ID: 20130425)}
\textbf{Date: }April 25,2013
\vspace{6pt}

In addition to neurofibromatosis, what other examination finding would you expect for this patient?
\vspace{12pt}

\textbf{Options:}
\begin{enumerate}
\item[A.] Aortic regurgitation
\item[B.] Chvostek sign
\item[C.] Lisch nodules
\item[D.] Thyroid bruit
\item[E.] Web neck
\end{enumerate}

\textbf{Image:}
\begin{center}
\includegraphics[width=0.61\textwidth,height=0.50\textheight,width=0.90\textwidth,keepaspectratio]{images/nejm_20130425.jpg}
\end{center}
\vspace{12pt}
\newpage

\section*{Question 536 (ID: 20130502)}
\textbf{Date: }May 02,2013
\vspace{6pt}

What is the diagnosis?
\vspace{12pt}

\textbf{Options:}
\begin{enumerate}
\item[A.] Achalasia
\item[B.] Esophageal cancer
\item[C.] Esophageal web
\item[D.] Gastroesophageal reflux disease
\item[E.] Nutcracker esophagus
\end{enumerate}

\textbf{Image:}
\begin{center}
\includegraphics[width=0.5\textwidth,height=0.50\textheight,width=0.90\textwidth,keepaspectratio]{images/nejm_20130502.jpg}
\end{center}
\vspace{12pt}
\newpage

\section*{Question 537 (ID: 20130509)}
\textbf{Date: }May 09,2013
\vspace{6pt}

What is the most likely diagnosis?
\vspace{12pt}

\textbf{Options:}
\begin{enumerate}
\item[A.] Herpes simplex infection
\item[B.] Inflammatory bowel disease
\item[C.] Peutz-Jeghers syndrome
\item[D.] Scurvy
\item[E.] Syphilis
\end{enumerate}

\textbf{Image:}
\begin{center}
\includegraphics[width=0.95\textwidth,height=0.50\textheight,width=0.90\textwidth,keepaspectratio]{images/nejm_20130509.jpg}
\end{center}
\vspace{12pt}
\newpage

\section*{Question 538 (ID: 20130516)}
\textbf{Date: }May 16,2013
\vspace{6pt}

What is the diagnosis?
\vspace{12pt}

\textbf{Options:}
\begin{enumerate}
\item[A.] Familial hypertriglyceridemia
\item[B.] Injection-drug abuse
\item[C.] Sarcoidosis
\item[D.] Systemic sclerosis
\item[E.] Takayasu's arteritis
\end{enumerate}

\textbf{Image:}
\begin{center}
\includegraphics[width=0.95\textwidth,height=0.50\textheight,width=0.90\textwidth,keepaspectratio]{images/nejm_20130516.jpg}
\end{center}
\vspace{12pt}
\newpage

\section*{Question 539 (ID: 20130523)}
\textbf{Date: }May 23,2013
\vspace{6pt}

What is the diagnosis?
\vspace{12pt}

\textbf{Options:}
\begin{enumerate}
\item[A.] Creutzfeldt-Jakob disease
\item[B.] Herpes simplex encephalitis
\item[C.] Neurocysticercosis
\item[D.] Polyarteritis nodosa
\item[E.] Toxoplasmosis
\end{enumerate}

\textbf{Image:}
\begin{center}
\includegraphics[width=0.95\textwidth,height=0.50\textheight,width=0.90\textwidth,keepaspectratio]{images/nejm_20130523.jpg}
\end{center}
\vspace{12pt}
\newpage

\section*{Question 540 (ID: 20130530)}
\textbf{Date: }May 30,2013
\vspace{6pt}

What is the cause of this patient's dyspnea?
\vspace{12pt}

\textbf{Options:}
\begin{enumerate}
\item[A.] Mitral stenosis
\item[B.] Pneumonia
\item[C.] Pneumothorax
\item[D.] Pulmonary embolism
\item[E.] Sarcoidosis
\end{enumerate}

\textbf{Image:}
\begin{center}
\includegraphics[width=0.95\textwidth,height=0.50\textheight,width=0.90\textwidth,keepaspectratio]{images/nejm_20130530.jpg}
\end{center}
\vspace{12pt}
\newpage

\section*{Question 541 (ID: 20130606)}
\textbf{Date: }June 06,2013
\vspace{6pt}

What is the most likely diagnosis?
\vspace{12pt}

\textbf{Options:}
\begin{enumerate}
\item[A.] Atherosclerosis
\item[B.] Fabry's disease
\item[D.] Hashimoto's disease
\item[E.] Ulcerative colitis
\end{enumerate}

\textbf{Image:}
\begin{center}
\includegraphics[width=0.71\textwidth,height=0.50\textheight,width=0.90\textwidth,keepaspectratio]{images/nejm_20130606.jpg}
\end{center}
\vspace{12pt}
\newpage

\section*{Question 542 (ID: 20130613)}
\textbf{Date: }June 13,2013
\vspace{6pt}

What is the diagnosis for this patient with end-stage renal disease who developed skin changes after an imaging procedure?
\vspace{12pt}

\textbf{Options:}
\begin{enumerate}
\item[A.] Actinic elastosis
\item[B.] Calciphylaxis
\item[C.] Nephrogenic systemic fibrosis
\item[D.] Porphyria cutanea tarda
\item[E.] Uremic frost
\end{enumerate}

\textbf{Image:}
\begin{center}
\includegraphics[width=0.95\textwidth,height=0.50\textheight,width=0.90\textwidth,keepaspectratio]{images/nejm_20130613.jpg}
\end{center}
\vspace{12pt}
\newpage

\section*{Question 543 (ID: 20130620)}
\textbf{Date: }June 20,2013
\vspace{6pt}

This patient presented with xerostomia and xerophthalmia. What is the diagnosis?
\vspace{12pt}

\textbf{Options:}
\begin{enumerate}
\item[A.] Angioedema
\item[B.] Contact dermatitis
\item[C.] Follicular lymphoma
\item[D.] Hypothyroidism
\item[E.] Sjögren's syndrome
\end{enumerate}

\textbf{Image:}
\begin{center}
\includegraphics[width=0.75\textwidth,height=0.50\textheight,width=0.90\textwidth,keepaspectratio]{images/nejm_20130620.jpg}
\end{center}
\vspace{12pt}
\newpage

\section*{Question 544 (ID: 20130627)}
\textbf{Date: }June 27,2013
\vspace{6pt}

What is the most likely cause of these plaques that developed after the patient's home heating system failed?
\vspace{12pt}

\textbf{Options:}
\begin{enumerate}
\item[A.] Cold agglutinins
\item[B.] Methemoglobinemia
\item[C.] Multiple myeloma
\item[D.] Scleroderma
\item[E.] Von Willebrand disease
\end{enumerate}

\textbf{Image:}
\begin{center}
\includegraphics[width=0.66\textwidth,height=0.50\textheight,width=0.90\textwidth,keepaspectratio]{images/nejm_20130627.jpg}
\end{center}
\vspace{12pt}
\newpage

\section*{Question 545 (ID: 20130704)}
\textbf{Date: }July 04,2013
\vspace{6pt}

What is the diagnosis?
\vspace{12pt}

\textbf{Options:}
\begin{enumerate}
\item[A.] Alopecia areata
\item[B.] Androgenic alopecia
\item[C.] Discoid lupus
\item[D.] Fibrosing alopecia
\item[E.] Hypothyroidism
\end{enumerate}

\textbf{Image:}
\begin{center}
\includegraphics[width=0.95\textwidth,height=0.50\textheight,width=0.90\textwidth,keepaspectratio]{images/nejm_20130704.jpg}
\end{center}
\vspace{12pt}
\newpage

\section*{Question 546 (ID: 20130711)}
\textbf{Date: }July 11,2013
\vspace{6pt}

What is the diagnosis?
\vspace{12pt}

\textbf{Options:}
\begin{enumerate}
\item[A.] Congenital  melanocytic nevus
\item[B.] Epidermal nevus syndrome
\item[C.] Neurofibromatosis
\item[D.] Nevus of Ota
\item[E.] Spitz nevus
\end{enumerate}

\textbf{Image:}
\begin{center}
\includegraphics[width=0.95\textwidth,height=0.50\textheight,width=0.90\textwidth,keepaspectratio]{images/nejm_20130711.jpg}
\end{center}
\vspace{12pt}
\newpage

\section*{Question 547 (ID: 20130718)}
\textbf{Date: }July 18,2013
\vspace{6pt}

What is the diagnosis?
\vspace{12pt}

\textbf{Options:}
\begin{enumerate}
\item[A.] Bullous ichthyosiform erythroderma
\item[B.] Collodion baby
\item[C.] Cutis marmorata
\item[D.] Harlequin color change
\item[E.] Staphylococcal pyoderma
\end{enumerate}

\textbf{Image:}
\begin{center}
\includegraphics[width=0.84\textwidth,height=0.50\textheight,width=0.90\textwidth,keepaspectratio]{images/nejm_20130718.jpg}
\end{center}
\vspace{12pt}
\newpage

\section*{Question 548 (ID: 20130725)}
\textbf{Date: }July 25,2013
\vspace{6pt}

What is the diagnosis in this patient with dry eyes, dry mouth, and hilar adenopathy?
\vspace{12pt}

\textbf{Options:}
\begin{enumerate}
\item[A.] Bulemia nervosa
\item[C.] Sjogren's syndrome
\item[D.] Serum sickness
\item[E.] Uveoparotid fever
\end{enumerate}

\textbf{Image:}
\begin{center}
\includegraphics[width=0.64\textwidth,height=0.50\textheight,width=0.90\textwidth,keepaspectratio]{images/nejm_20130725.jpg}
\end{center}
\vspace{12pt}
\newpage

\section*{Question 549 (ID: 20130801)}
\textbf{Date: }August 01,2013
\vspace{6pt}

What is the most likely diagnosis?
\vspace{12pt}

\textbf{Options:}
\begin{enumerate}
\item[A.] Discoid lupus
\item[B.] Erythema migrans
\item[C.] Psoriasis
\item[D.] Tinea corporis
\item[E.] Urticaria
\end{enumerate}

\textbf{Image:}
\begin{center}
\includegraphics[width=0.95\textwidth,height=0.50\textheight,width=0.90\textwidth,keepaspectratio]{images/nejm_20130801.jpg}
\end{center}
\vspace{12pt}
\newpage

\section*{Question 550 (ID: 20130808)}
\textbf{Date: }August 08,2013
\vspace{6pt}

What is the diagnosis in this 42-year-old man with bone pain?
\vspace{12pt}

\textbf{Options:}
\begin{enumerate}
\item[A.] Hypervitaminosis A
\item[B.] Gaucher's disease
\item[C.] Multiple myeloma
\item[D.] Osteopetrosis
\item[E.] Thalassemia minor
\end{enumerate}

\textbf{Image:}
\begin{center}
\includegraphics[width=0.95\textwidth,height=0.50\textheight,width=0.90\textwidth,keepaspectratio]{images/nejm_20130808.jpg}
\end{center}
\vspace{12pt}
\newpage

\section*{Question 551 (ID: 20130815)}
\textbf{Date: }August 15,2013
\vspace{6pt}

What is the diagnosis in this patient who had been bitten by a tsetse fly?
\vspace{12pt}

\textbf{Options:}
\begin{enumerate}
\item[A.] Chagas' disease
\item[B.] Filariasis
\item[C.] Leishmaniasis
\item[D.] Trypanosomiasis
\item[E.] Tuberculous scrofula
\end{enumerate}

\textbf{Image:}
\begin{center}
\includegraphics[width=0.95\textwidth,height=0.50\textheight,width=0.90\textwidth,keepaspectratio]{images/nejm_20130815.jpg}
\end{center}
\vspace{12pt}
\newpage

\section*{Question 552 (ID: 20130822)}
\textbf{Date: }August 22,2013
\vspace{6pt}

What is the cause of this rash that developed in a patient after cesarean section in association with fever and hypotension?
\vspace{12pt}

\textbf{Options:}
\begin{enumerate}
\item[A.] Kawasaki disease
\item[B.] Pseudoporphyria
\item[C.] Scalded skin syndrome
\item[D.] Toxic epidermal necrolysis
\item[E.] Toxic shock syndrome
\end{enumerate}

\textbf{Image:}
\begin{center}
\includegraphics[width=0.58\textwidth,height=0.50\textheight,width=0.90\textwidth,keepaspectratio]{images/nejm_20130822.jpg}
\end{center}
\vspace{12pt}
\newpage

\section*{Question 553 (ID: 20130829)}
\textbf{Date: }August 29,2013
\vspace{6pt}

What is the diagnosis?
\vspace{12pt}

\textbf{Options:}
\begin{enumerate}
\item[A.] Bowenoid papulosis
\item[B.] Fibrous pseudotumor
\item[C.] Keloid
\item[D.] Onchocerciasis
\item[E.] Scrotal calcinosis
\end{enumerate}

\textbf{Image:}
\begin{center}
\includegraphics[width=0.59\textwidth,height=0.50\textheight,width=0.90\textwidth,keepaspectratio]{images/nejm_20130829.jpg}
\end{center}
\vspace{12pt}
\newpage

\section*{Question 554 (ID: 20130905)}
\textbf{Date: }September 05,2013
\vspace{6pt}

What is the diagnosis?
\vspace{12pt}

\textbf{Options:}
\begin{enumerate}
\item[A.] Cholesteatoma
\item[B.] Glomus tumor
\item[C.] Suppurative otitis media
\item[D.] Serous otitis media
\item[E.] Tympanic hyperectasis
\end{enumerate}

\textbf{Image:}
\begin{center}
\includegraphics[width=0.84\textwidth,height=0.50\textheight,width=0.90\textwidth,keepaspectratio]{images/nejm_20130905.jpg}
\end{center}
\vspace{12pt}
\newpage

\section*{Question 555 (ID: 20130912)}
\textbf{Date: }September 12,2013
\vspace{6pt}

What is the diagnosis in this 70-year-old patient who developed bilateral fluctuant masses over each elbow?
\vspace{12pt}

\textbf{Options:}
\begin{enumerate}
\item[A.] Gout nodulosis
\item[B.] Lipoma
\item[C.] Medial epicondylitis
\item[D.] Pseudogout
\item[E.] Rheumatoid nodules
\end{enumerate}

\textbf{Image:}
\begin{center}
\includegraphics[width=0.95\textwidth,height=0.50\textheight,width=0.90\textwidth,keepaspectratio]{images/nejm_20130912.jpg}
\end{center}
\vspace{12pt}
\newpage

\section*{Question 556 (ID: 20130919)}
\textbf{Date: }September 19,2013
\vspace{6pt}

What is the diagnosis in this 36-year-old male?
\vspace{12pt}

\textbf{Options:}
\begin{enumerate}
\item[A.] Amyloidosis
\item[B.] Beriberi
\item[C.] Hypertrophic cardiomyopathy
\item[D.] Sarcoidosis
\item[E.] Uremic pericarditis
\end{enumerate}

\textbf{Image:}
\begin{center}
\includegraphics[width=0.95\textwidth,height=0.50\textheight,width=0.90\textwidth,keepaspectratio]{images/nejm_20130919.jpg}
\end{center}
\vspace{12pt}
\newpage

\section*{Question 557 (ID: 20130926)}
\textbf{Date: }September 26,2013
\vspace{6pt}

What is the diagnosis in this patient who had erosions on the dorsal surfaces of her hands?
\vspace{12pt}

\textbf{Options:}
\begin{enumerate}
\item[A.] Bulimia nervosa
\item[B.] Cushing's syndrome
\item[C.] Hypothyroidism
\item[D.] Porphyria cutanea tarda
\item[E.] Theca-cell tumor of the ovary
\end{enumerate}

\textbf{Image:}
\begin{center}
\includegraphics[width=0.58\textwidth,height=0.50\textheight,width=0.90\textwidth,keepaspectratio]{images/nejm_20130926.jpg}
\end{center}
\vspace{12pt}
\newpage

\section*{Question 558 (ID: 20131003)}
\textbf{Date: }October 03,2013
\vspace{6pt}

What is the diagnosis?
\vspace{12pt}

\textbf{Options:}
\begin{enumerate}
\item[A.] Chewing tobacco
\item[B.] Implantation of amalgam
\item[C.] Lead poisoning
\item[D.] Melanoma
\item[E.] Peutz-Jegher's syndrome
\end{enumerate}

\textbf{Image:}
\begin{center}
\includegraphics[width=0.95\textwidth,height=0.50\textheight,width=0.90\textwidth,keepaspectratio]{images/nejm_20131003.jpg}
\end{center}
\vspace{12pt}
\newpage

\section*{Question 559 (ID: 20131010)}
\textbf{Date: }October 10,2013
\vspace{6pt}

What is the diagnosis in this patient who appeared sick, had a bull neck, and was tachycardic?
\vspace{12pt}

\textbf{Options:}
\begin{enumerate}
\item[A.] Diphtheria
\item[B.] Epiglottitis
\item[C.] Infectious mononucleosis
\item[D.] Rheumatic fever
\item[E.] Streptococcal pharyngitis
\end{enumerate}

\textbf{Image:}
\begin{center}
\includegraphics[width=0.95\textwidth,height=0.50\textheight,width=0.90\textwidth,keepaspectratio]{images/nejm_20131010.jpg}
\end{center}
\vspace{12pt}
\newpage

\section*{Question 560 (ID: 20131017)}
\textbf{Date: }October 17,2013
\vspace{6pt}

What serologic test is most likely to be positive in this patient?
\vspace{12pt}

\textbf{Options:}
\begin{enumerate}
\item[A.] Anticentromere antibody
\item[B.] Anti-double stranded DNA
\item[C.] Anti-ribonucleoprotein antibody
\item[D.] Anti-Ro antibody
\item[E.] Anti-Smith antibody
\end{enumerate}

\textbf{Image:}
\begin{center}
\includegraphics[width=0.95\textwidth,height=0.50\textheight,width=0.90\textwidth,keepaspectratio]{images/nejm_20131017.jpg}
\end{center}
\vspace{12pt}
\newpage

\section*{Question 561 (ID: 20131024)}
\textbf{Date: }October 24,2013
\vspace{6pt}

What is the diagnosis in this patient who had a relative afferent pupillary defect?
\vspace{12pt}

\textbf{Options:}
\begin{enumerate}
\item[A.] Multiple sclerosis
\item[B.] Retinal ischemia
\item[C.] Retinitis pigmentosa
\item[D.] Toxoplasmosis
\item[E.] Wilson's disease
\end{enumerate}

\textbf{Image:}
\begin{center}
\includegraphics[width=0.95\textwidth,height=0.50\textheight,width=0.90\textwidth,keepaspectratio]{images/nejm_20131024.jpg}
\end{center}
\vspace{12pt}
\newpage

\section*{Question 562 (ID: 20131031)}
\textbf{Date: }October 31,2013
\vspace{6pt}

What is the diagnosis in this patient who had abdominal pain?
\vspace{12pt}

\textbf{Options:}
\begin{enumerate}
\item[A.] Familial Mediterranean fever
\item[B.] Henoch-Schönlein purpura
\item[C.] Meningococcemia
\item[D.] Polyarteritis nodosa
\item[E.] Rocky Mountain spotted fever
\end{enumerate}

\textbf{Image:}
\begin{center}
\includegraphics[width=0.82\textwidth,height=0.50\textheight,width=0.90\textwidth,keepaspectratio]{images/nejm_20131031.jpg}
\end{center}
\vspace{12pt}
\newpage

\section*{Question 563 (ID: 20131107)}
\textbf{Date: }November 07,2013
\vspace{6pt}

What are these crystals that were aspirated from the bursa of an elbow of a patient with rheumatoid arthritis?
\vspace{12pt}

\textbf{Options:}
\begin{enumerate}
\item[A.] Cholesterol
\item[B.] Calcium apatite
\item[C.] Calcium oxalate
\item[D.] Calcium pyrophosphate dihydrate
\item[E.] Monosodium urate
\end{enumerate}

\textbf{Image:}
\begin{center}
\includegraphics[width=0.63\textwidth,height=0.50\textheight,width=0.90\textwidth,keepaspectratio]{images/nejm_20131107.jpg}
\end{center}
\vspace{12pt}
\newpage

\section*{Question 564 (ID: 20131114)}
\textbf{Date: }November 14,2013
\vspace{6pt}

What is the diagnosis?
\vspace{12pt}

\textbf{Options:}
\begin{enumerate}
\item[A.] Chronic aortic dissection
\item[B.] Cirrhosis
\item[C.] Hypoplastic right heart syndrome
\item[D.] Superior vena caval obstruction
\item[E.] Transposition of the great vessels
\end{enumerate}

\textbf{Image:}
\begin{center}
\includegraphics[width=0.86\textwidth,height=0.50\textheight,width=0.90\textwidth,keepaspectratio]{images/nejm_20131114.jpg}
\end{center}
\vspace{12pt}
\newpage

\section*{Question 565 (ID: 20131121)}
\textbf{Date: }November 21,2013
\vspace{6pt}

What is the diagnosis in this patient who had presented with a sore throat?
\vspace{12pt}

\textbf{Options:}
\begin{enumerate}
\item[B.] Discoid lupus
\item[C.] Impetigo
\item[D.] Mastocytosis
\item[E.] Syphilis
\end{enumerate}

\textbf{Image:}
\begin{center}
\includegraphics[width=0.72\textwidth,height=0.50\textheight,width=0.90\textwidth,keepaspectratio]{images/nejm_20131121.jpg}
\end{center}
\vspace{12pt}
\newpage

\section*{Question 566 (ID: 20131128)}
\textbf{Date: }November 28,2013
\vspace{6pt}

What is the most likely diagnosis?
\vspace{12pt}

\textbf{Options:}
\begin{enumerate}
\item[A.] Alopecia areata
\item[B.] Androgenetic alopecia
\item[C.] Telogen effluvium
\item[D.] Tinea capitis
\item[E.] Trichotillomania
\end{enumerate}

\textbf{Image:}
\begin{center}
\includegraphics[width=0.95\textwidth,height=0.50\textheight,width=0.90\textwidth,keepaspectratio]{images/nejm_20131128.jpg}
\end{center}
\vspace{12pt}
\newpage

\section*{Question 567 (ID: 20131205)}
\textbf{Date: }December 05,2013
\vspace{6pt}

What is the diagnosis in this patient who presented with high fever?
\vspace{12pt}

\textbf{Options:}
\begin{enumerate}
\item[A.] Bowel infarction
\item[B.] Caval thrombophlebitis
\item[C.] Hepatoma
\item[D.] Liver abscess
\item[E.] Perforated gastric ulcer
\end{enumerate}

\textbf{Image:}
\begin{center}
\includegraphics[width=0.95\textwidth,height=0.50\textheight,width=0.90\textwidth,keepaspectratio]{images/nejm_20131205.jpg}
\end{center}
\vspace{12pt}
\newpage

\section*{Question 568 (ID: 20131212)}
\textbf{Date: }December 12,2013
\vspace{6pt}

What is the diagnosis?
\vspace{12pt}

\textbf{Options:}
\begin{enumerate}
\item[A.] Aphthous stomatitis
\item[B.] Pyogenic granuloma
\item[C.] Squamous-cell carcinoma
\item[D.] Syphilis
\item[E.] Traumatic fibroma
\end{enumerate}

\textbf{Image:}
\begin{center}
\includegraphics[width=0.95\textwidth,height=0.50\textheight,width=0.90\textwidth,keepaspectratio]{images/nejm_20131212.jpg}
\end{center}
\vspace{12pt}
\newpage

\section*{Question 569 (ID: 20131219)}
\textbf{Date: }December 19,2013
\vspace{6pt}

What is the diagnosis?
\vspace{12pt}

\textbf{Options:}
\begin{enumerate}
\item[A.] Ehlers-Danlos syndrome
\item[B.] Familial hypertriglyceridemia
\item[C.] Hidradinitis suppurativa
\item[D.] Marfan syndrome
\item[E.] Pseudoxanthoma elasticum
\end{enumerate}

\textbf{Image:}
\begin{center}
\includegraphics[width=0.56\textwidth,height=0.50\textheight,width=0.90\textwidth,keepaspectratio]{images/nejm_20131219.jpg}
\end{center}
\vspace{12pt}
\newpage

\section*{Question 570 (ID: 20131226)}
\textbf{Date: }December 26,2013
\vspace{6pt}

What is the diagnosis?
\vspace{12pt}

\textbf{Options:}
\begin{enumerate}
\item[A.] African trypanosomiasis
\item[B.] Frostbite
\item[C.] Ketatosis follicularis
\item[D.] Leprosy
\item[E.] Papillomatosis cutis lymphostatica
\end{enumerate}

\textbf{Image:}
\begin{center}
\includegraphics[width=0.59\textwidth,height=0.50\textheight,width=0.90\textwidth,keepaspectratio]{images/nejm_20131226.jpg}
\end{center}
\vspace{12pt}
\newpage

\section*{Question 571 (ID: 20140102)}
\textbf{Date: }January 02,2014
\vspace{6pt}

What is the diagnosis in this patient who presented with a 2-day history of fever, sore throat, arthralgia, and suboccipital lymphadenopathy?
\vspace{12pt}

\textbf{Options:}
\begin{enumerate}
\item[A.] Adenovirus infection
\item[B.] Infectious mononucleosis
\item[C.] Measles
\item[D.] Rubella
\item[E.] Scarlet fever
\end{enumerate}

\textbf{Image:}
\begin{center}
\includegraphics[width=0.95\textwidth,height=0.50\textheight,width=0.90\textwidth,keepaspectratio]{images/nejm_20140102.jpg}
\end{center}
\vspace{12pt}
\newpage

\section*{Question 572 (ID: 20140109)}
\textbf{Date: }January 09,2014
\vspace{6pt}

What is the diagnosis?
\vspace{12pt}

\textbf{Options:}
\begin{enumerate}
\item[A.] Basal-cell carcinoma
\item[B.] Halo nevus
\item[C.] Lichen planus
\item[D.] Malignant melanoma
\item[E.] Spitz nevus
\end{enumerate}

\textbf{Image:}
\begin{center}
\includegraphics[width=0.95\textwidth,height=0.50\textheight,width=0.90\textwidth,keepaspectratio]{images/nejm_20140109.jpg}
\end{center}
\vspace{12pt}
\newpage

\section*{Question 573 (ID: 20140116)}
\textbf{Date: }January 16,2014
\vspace{6pt}

What is the most likely diagnosis?
\vspace{12pt}

\textbf{Options:}
\begin{enumerate}
\item[A.] Age-related macular degeneration
\item[B.] Melanoma
\item[C.] Posterior vitreous detachment
\item[D.] Retinitis pigmentosa
\item[E.] Toxoplasmosis
\end{enumerate}

\textbf{Image:}
\begin{center}
\includegraphics[width=0.95\textwidth,height=0.50\textheight,width=0.90\textwidth,keepaspectratio]{images/nejm_20140116.jpg}
\end{center}
\vspace{12pt}
\newpage

\section*{Question 574 (ID: 20140123)}
\textbf{Date: }January 23,2014
\vspace{6pt}

What diagnosis is most associated with this finding?
\vspace{12pt}

\textbf{Options:}
\begin{enumerate}
\item[A.] Congenital adrenal hyperplasia
\item[B.] Porphyria cutanea tarda
\item[C.] Ovarian teratoma
\item[D.] Situs inversus
\item[E.] Spinal dysraphism
\end{enumerate}

\textbf{Image:}
\begin{center}
\includegraphics[width=0.55\textwidth,height=0.50\textheight,width=0.90\textwidth,keepaspectratio]{images/nejm_20140123.jpg}
\end{center}
\vspace{12pt}
\newpage

\section*{Question 575 (ID: 20140130)}
\textbf{Date: }January 30,2014
\vspace{6pt}

What is the diagnosis?
\vspace{12pt}

\textbf{Options:}
\begin{enumerate}
\item[A.] Bowel obstruction
\item[B.] Metastatic melanoma
\item[C.] Osteitis fibrosa cystica
\item[D.] Osteomalacia
\item[E.] Paget's disease
\end{enumerate}

\textbf{Image:}
\begin{center}
\includegraphics[width=0.92\textwidth,height=0.50\textheight,width=0.90\textwidth,keepaspectratio]{images/nejm_20140130.jpg}
\end{center}
\vspace{12pt}
\newpage

\section*{Question 576 (ID: 20140206)}
\textbf{Date: }February 06,2014
\vspace{6pt}

What is the most likely underlying cause of this problem?
\vspace{12pt}

\textbf{Options:}
\begin{enumerate}
\item[A.] Cavernous sinus thrombosis
\item[B.] Graves' disease
\item[C.] Rheumatoid arthritis
\item[D.] Syphilis
\item[E.] Vitamin E deficiency
\end{enumerate}

\textbf{Image:}
\begin{center}
\includegraphics[width=0.95\textwidth,height=0.50\textheight,width=0.90\textwidth,keepaspectratio]{images/nejm_20140206.jpg}
\end{center}
\vspace{12pt}
\newpage

\section*{Question 577 (ID: 20140213)}
\textbf{Date: }February 13,2014
\vspace{6pt}

What diagnosis explains the loss of visual acuity in this woman who is at 36 weeks of gestation?
\vspace{12pt}

\textbf{Options:}
\begin{enumerate}
\item[A.] Central serous chorioretinitis
\item[B.] Diabetes mellitus
\item[C.] Glaucoma
\item[D.] Graves' disease
\item[E.] Preeclampsia
\end{enumerate}

\textbf{Image:}
\begin{center}
\includegraphics[width=0.95\textwidth,height=0.50\textheight,width=0.90\textwidth,keepaspectratio]{images/nejm_20140213.jpg}
\end{center}
\vspace{12pt}
\newpage

\section*{Question 578 (ID: 20140220)}
\textbf{Date: }February 20,2014
\vspace{6pt}

What is the most likely diagnosis in this patient who underwent fine-needle aspiration after reporting several weeks of submandibular pain?
\vspace{12pt}

\textbf{Options:}
\begin{enumerate}
\item[A.] Adenoid cystic carcinoma
\item[B.] Cat scratch disease
\item[C.] Infectious mononucleosis
\item[D.] Sclerosing sialadenitis
\item[E.] Systemic lupus erythematosus
\end{enumerate}

\textbf{Image:}
\begin{center}
\includegraphics[width=0.95\textwidth,height=0.50\textheight,width=0.90\textwidth,keepaspectratio]{images/nejm_20140220.jpg}
\end{center}
\vspace{12pt}
\newpage

\section*{Question 579 (ID: 20140227)}
\textbf{Date: }February 27,2014
\vspace{6pt}

What maternal diagnosis is most likely to expain the development of this rash 2 hours after delivery of this child with trisomy 21?
\vspace{12pt}

\textbf{Options:}
\begin{enumerate}
\item[A.] Measles
\item[B.] Polyarteritis nodosa
\item[C.] Rubella
\item[D.] Streptococcal A infection
\item[E.] Systemic lupus erythematosus
\end{enumerate}

\textbf{Image:}
\begin{center}
\includegraphics[width=0.89\textwidth,height=0.50\textheight,width=0.90\textwidth,keepaspectratio]{images/nejm_20140227.jpg}
\end{center}
\vspace{12pt}
\newpage

\section*{Question 580 (ID: 20140306)}
\textbf{Date: }March 06,2014
\vspace{6pt}

What is the most likely diagnosis for this finding detected during esophageal endoscopy?
\vspace{12pt}

\textbf{Options:}
\begin{enumerate}
\item[A.] Adenocarcinoma
\item[B.] Candidiasis
\item[C.] Dieulafoy's lesion
\item[D.] Schatzki ring
\item[E.] Systemic sclerosis
\end{enumerate}

\textbf{Image:}
\begin{center}
\includegraphics[width=0.95\textwidth,height=0.50\textheight,width=0.90\textwidth,keepaspectratio]{images/nejm_20140306.jpg}
\end{center}
\vspace{12pt}
\newpage

\section*{Question 581 (ID: 20140313)}
\textbf{Date: }March 13,2014
\vspace{6pt}

What is the diagnosis in this patient who had normal serum levels of calcium and phosphorous?
\vspace{12pt}

\textbf{Options:}
\begin{enumerate}
\item[A.] Albright's hereditary osteodystrophy
\item[B.] Hallux valgus
\item[C.] Hypertrophic osteoarthropathy
\item[D.] Morton's neuroma
\item[E.] Pseudohypoparathyroidism
\end{enumerate}

\textbf{Image:}
\begin{center}
\includegraphics[width=0.95\textwidth,height=0.50\textheight,width=0.90\textwidth,keepaspectratio]{images/nejm_20140313.jpg}
\end{center}
\vspace{12pt}
\newpage

\section*{Question 582 (ID: 20140320)}
\textbf{Date: }March 20,2014
\vspace{6pt}

What is the most likely diagnosis for this boy who was born at 36 weeks gestation weighing 1800 g?
\vspace{12pt}

\textbf{Options:}
\begin{enumerate}
\item[A.] Congenital parvovirus B19 infection
\item[B.] Fetal alcohol syndrome
\item[C.] Neonatal thyrotoxicosis
\item[D.] Nesidioblastosis
\item[E.] Williams syndrome
\end{enumerate}

\textbf{Image:}
\begin{center}
\includegraphics[width=0.57\textwidth,height=0.50\textheight,width=0.90\textwidth,keepaspectratio]{images/nejm_20140320.jpg}
\end{center}
\vspace{12pt}
\newpage

\section*{Question 583 (ID: 20140327)}
\textbf{Date: }March 27,2014
\vspace{6pt}

What is the diagnosis in this patient who presented with sudden painful vision impairment after vigorous exercise?
\vspace{12pt}

\textbf{Options:}
\begin{enumerate}
\item[A.] Central retinal artery occlusion
\item[B.] Corneal ulcer
\item[C.] Dislocation of the lens
\item[D.] Episcleritis
\item[E.] Retinal detachment
\end{enumerate}

\textbf{Image:}
\begin{center}
\includegraphics[width=0.95\textwidth,height=0.50\textheight,width=0.90\textwidth,keepaspectratio]{images/nejm_20140327.jpg}
\end{center}
\vspace{12pt}
\newpage

\section*{Question 584 (ID: 20140403)}
\textbf{Date: }April 03,2014
\vspace{6pt}

What is the diagnosis?
\vspace{12pt}

\textbf{Options:}
\begin{enumerate}
\item[A.] Empyema
\item[B.] Lymphangioleiomyomatosis
\item[C.] Paraesophageal hernia
\item[D.] Pericardial effusion
\item[E.] Plombage
\end{enumerate}

\textbf{Image:}
\begin{center}
\includegraphics[width=0.8\textwidth,height=0.50\textheight,width=0.90\textwidth,keepaspectratio]{images/nejm_20140403.jpg}
\end{center}
\vspace{12pt}
\newpage

\section*{Question 585 (ID: 20140410)}
\textbf{Date: }April 10,2014
\vspace{6pt}

What is the diagnosis?
\vspace{12pt}

\textbf{Options:}
\begin{enumerate}
\item[A.] Blastomycosis
\item[B.] Celiac disease
\item[C.] Down's syndrome
\item[D.] Hyperparathyroidism
\item[E.] IgA nephropathy
\end{enumerate}

\textbf{Image:}
\begin{center}
\includegraphics[width=0.65\textwidth,height=0.50\textheight,width=0.90\textwidth,keepaspectratio]{images/nejm_20140410.jpg}
\end{center}
\vspace{12pt}
\newpage

\section*{Question 586 (ID: 20140417)}
\textbf{Date: }April 17,2014
\vspace{6pt}

What is the diagnosis?
\vspace{12pt}

\textbf{Options:}
\begin{enumerate}
\item[A.] Blepharitis
\item[B.] Entropion
\item[C.] Eyelid wart
\item[D.] Keratoconjunctivitis sicca
\item[E.] Vernal keratoconjunctivitis
\end{enumerate}

\textbf{Image:}
\begin{center}
\includegraphics[width=0.95\textwidth,height=0.50\textheight,width=0.90\textwidth,keepaspectratio]{images/nejm_20140417.jpg}
\end{center}
\vspace{12pt}
\newpage

\section*{Question 587 (ID: 20140424)}
\textbf{Date: }April 24,2014
\vspace{6pt}

What is the cause of this new skin lesion in this patient whose blood CD4+ cell count was 20 per cubic millimeter (1\%) following treatment for chronic lymphocytic leukemia?
\vspace{12pt}

\textbf{Options:}
\begin{enumerate}
\item[A.] Bacillary angiomatosis
\item[B.] Bullous impetigo
\item[C.] Cryptococcosis
\item[D.] Kaposi's sarcoma
\item[E.] Verruca vulgaris
\end{enumerate}

\textbf{Image:}
\begin{center}
\includegraphics[width=0.95\textwidth,height=0.50\textheight,width=0.90\textwidth,keepaspectratio]{images/nejm_20140424.jpg}
\end{center}
\vspace{12pt}
\newpage

\section*{Question 588 (ID: 20140501)}
\textbf{Date: }May 01,2014
\vspace{6pt}

What is the diagnosis in this patient who had been involved in a motorcycle accident?
\vspace{12pt}

\textbf{Options:}
\begin{enumerate}
\item[A.] Aortic dissection
\item[B.] Cardiac tamponade
\item[C.] Diffuse pulmonary hemorrhage
\item[D.] Tension pneumothorax
\item[E.] Traumatic diaphragmatic hernia
\end{enumerate}

\textbf{Image:}
\begin{center}
\includegraphics[width=0.92\textwidth,height=0.50\textheight,width=0.90\textwidth,keepaspectratio]{images/nejm_20140501.jpg}
\end{center}
\vspace{12pt}
\newpage

\section*{Question 589 (ID: 20140508)}
\textbf{Date: }May 08,2014
\vspace{6pt}

What is the diagnosis?
\vspace{12pt}

\textbf{Options:}
\begin{enumerate}
\item[A.] Cutaneous candidiasis
\item[B.] Granuloma annulare
\item[C.] Sarcoidosis
\item[D.] Syphilis
\item[E.] Tinea faciei
\end{enumerate}

\textbf{Image:}
\begin{center}
\includegraphics[width=0.93\textwidth,height=0.50\textheight,width=0.90\textwidth,keepaspectratio]{images/nejm_20140508.jpg}
\end{center}
\vspace{12pt}
\newpage

\section*{Question 590 (ID: 20140515)}
\textbf{Date: }May 15,2014
\vspace{6pt}

What is the most likely diagnosis for this 19-month-old girl who presented with a 1-week history of an ascending erythematous eruption?
\vspace{12pt}

\textbf{Options:}
\begin{enumerate}
\item[A.] Cellulitis
\item[B.] Hand, foot, and mouth disease
\item[C.] Herpes zoster
\item[D.] Lymphadenitis
\item[E.] Superficial thrombophlebitis
\end{enumerate}

\textbf{Image:}
\begin{center}
\includegraphics[width=0.55\textwidth,height=0.50\textheight,width=0.90\textwidth,keepaspectratio]{images/nejm_20140515.jpg}
\end{center}
\vspace{12pt}
\newpage

\section*{Question 591 (ID: 20140522)}
\textbf{Date: }May 22,2014
\vspace{6pt}

What is the most likely diagnosis in this 50-year-old woman?
\vspace{12pt}

\textbf{Options:}
\begin{enumerate}
\item[A.] Metastasis
\item[B.] Onychomycosis
\item[C.] Psoriasis
\item[D.] Thromboangiitis obliterans
\item[E.] Yellow nail syndrome
\end{enumerate}

\textbf{Image:}
\begin{center}
\includegraphics[width=0.88\textwidth,height=0.50\textheight,width=0.90\textwidth,keepaspectratio]{images/nejm_20140522.jpg}
\end{center}
\vspace{12pt}
\newpage

\section*{Question 592 (ID: 20140529)}
\textbf{Date: }May 29,2014
\vspace{6pt}

What is the most likely diagnosis in this patient with cough?
\vspace{12pt}

\textbf{Options:}
\begin{enumerate}
\item[A.] Aspergillosis
\item[B.] Coccidioidomycosis
\item[C.] Granuloma annulare
\item[D.] Sarcoidosis
\item[E.] Syphilis
\end{enumerate}

\textbf{Image:}
\begin{center}
\includegraphics[width=0.95\textwidth,height=0.50\textheight,width=0.90\textwidth,keepaspectratio]{images/nejm_20140529.jpg}
\end{center}
\vspace{12pt}
\newpage

\section*{Question 593 (ID: 20140605)}
\textbf{Date: }June 05,2014
\vspace{6pt}

What is the diagnosis?
\vspace{12pt}

\textbf{Options:}
\begin{enumerate}
\item[A.] Cataract
\item[B.] Loa loa
\item[C.] Melanoma
\item[D.] Retinoblastoma
\item[E.] Sarcoidosis
\end{enumerate}

\textbf{Image:}
\begin{center}
\includegraphics[width=0.95\textwidth,height=0.50\textheight,width=0.90\textwidth,keepaspectratio]{images/nejm_20140605.jpg}
\end{center}
\vspace{12pt}
\newpage

\section*{Question 594 (ID: 20140612)}
\textbf{Date: }June 12,2014
\vspace{6pt}

What is the most likely cause of these retinal changes?
\vspace{12pt}

\textbf{Options:}
\begin{enumerate}
\item[A.] Infective endocarditis
\item[B.] Myeloma
\item[C.] Syphilis
\item[D.] Systemic lupus erythematosus
\item[E.] Tay-Sachs disease
\end{enumerate}

\textbf{Image:}
\begin{center}
\includegraphics[width=0.87\textwidth,height=0.50\textheight,width=0.90\textwidth,keepaspectratio]{images/nejm_20140612.jpg}
\end{center}
\vspace{12pt}
\newpage

\section*{Question 595 (ID: 20140619)}
\textbf{Date: }June 19,2014
\vspace{6pt}

What is the most likely cause of these skin changes?
\vspace{12pt}

\textbf{Options:}
\begin{enumerate}
\item[A.] Asteatotic eczema
\item[B.] Chronic venous stasis
\item[C.] Necrobiosis lipoidica
\item[D.] Pretibial myxedema
\item[E.] Psoriasis
\end{enumerate}

\textbf{Image:}
\begin{center}
\includegraphics[width=0.9\textwidth,height=0.50\textheight,width=0.90\textwidth,keepaspectratio]{images/nejm_20140619.jpg}
\end{center}
\vspace{12pt}
\newpage

\section*{Question 596 (ID: 20140626)}
\textbf{Date: }June 26,2014
\vspace{6pt}

Deficiency of what micronutrient is likely to account for this rash following Roux-en-Y gastric bypass?
\vspace{12pt}

\textbf{Options:}
\begin{enumerate}
\item[A.] Copper
\item[B.] Thiamine
\item[C.] Vitamin B6
\item[D.] Vitamin C
\end{enumerate}

\textbf{Image:}
\begin{center}
\includegraphics[width=0.95\textwidth,height=0.50\textheight,width=0.90\textwidth,keepaspectratio]{images/nejm_20140626.jpg}
\end{center}
\vspace{12pt}
\newpage

\section*{Question 597 (ID: 20140703)}
\textbf{Date: }July 03,2014
\vspace{6pt}

What is the most likely diagnosis in this patient who had a 3-month history of coughing reddish sputum, weight loss, and fever?
\vspace{12pt}

\textbf{Options:}
\begin{enumerate}
\item[A.] Brucellosis
\item[B.] Gonococcal infection
\item[C.] Lymphoma
\item[D.] Sarcoidosis
\item[E.] Tuberculosis
\end{enumerate}

\textbf{Image:}
\begin{center}
\includegraphics[width=0.95\textwidth,height=0.50\textheight,width=0.90\textwidth,keepaspectratio]{images/nejm_20140703.jpg}
\end{center}
\vspace{12pt}
\newpage

\section*{Question 598 (ID: 20140710)}
\textbf{Date: }July 10,2014
\vspace{6pt}

What is the most likely diagnosis in this patient who had developed tricuspid regurgitation?
\vspace{12pt}

\textbf{Options:}
\begin{enumerate}
\item[A.] Auriculotemporal nerve syndrome
\item[B.] Carcinoid syndrome
\item[C.] Mastocytosis
\item[D.] Pheochromocytoma
\item[E.] Rosacea
\end{enumerate}

\textbf{Image:}
\begin{center}
\includegraphics[width=0.55\textwidth,height=0.50\textheight,width=0.90\textwidth,keepaspectratio]{images/nejm_20140710.jpg}
\end{center}
\vspace{12pt}
\newpage

\section*{Question 599 (ID: 20140717)}
\textbf{Date: }July 17,2014
\vspace{6pt}

A disease of which one of the following systems is most likely to have resulted in this finding?
\vspace{12pt}

\textbf{Options:}
\begin{enumerate}
\item[A.] Endocrine system
\item[B.] Gastrointestinal system
\item[C.] Hematopoietic system
\item[D.] Neurologic system
\item[E.] Respiratory system
\end{enumerate}

\textbf{Image:}
\begin{center}
\includegraphics[width=0.7\textwidth,height=0.50\textheight,width=0.90\textwidth,keepaspectratio]{images/nejm_20140717.jpg}
\end{center}
\vspace{12pt}
\newpage

\section*{Question 600 (ID: 20140724)}
\textbf{Date: }July 24,2014
\vspace{6pt}

What is the most likely diagnosis in this patient who was having difficulty swallowing?
\vspace{12pt}

\textbf{Options:}
\begin{enumerate}
\item[A.] Amyloidosis
\item[B.] Amyotrophic lateral sclerosis
\item[C.] Pellagra
\item[D.] Pernicious anemia
\item[E.] Sjogrens syndrome
\end{enumerate}

\textbf{Image:}
\begin{center}
\includegraphics[width=0.95\textwidth,height=0.50\textheight,width=0.90\textwidth,keepaspectratio]{images/nejm_20140724.jpg}
\end{center}
\vspace{12pt}
\newpage

\section*{Question 601 (ID: 20140731)}
\textbf{Date: }July 31,2014
\vspace{6pt}

What is the most likely diagnosis?
\vspace{12pt}

\textbf{Options:}
\begin{enumerate}
\item[A.] Lupus pernio
\item[B.] Melasma
\item[C.] Phototoxic dermatitis
\item[D.] Prurigo simplex
\item[E.] Systemic lupus erythematosus
\end{enumerate}

\textbf{Image:}
\begin{center}
\includegraphics[width=0.58\textwidth,height=0.50\textheight,width=0.90\textwidth,keepaspectratio]{images/nejm_20140731.jpg}
\end{center}
\vspace{12pt}
\newpage

\section*{Question 602 (ID: 20140807)}
\textbf{Date: }August 07,2014
\vspace{6pt}

What is the most likely diagnosis?
\vspace{12pt}

\textbf{Options:}
\begin{enumerate}
\item[A.] Mucocele
\item[B.] Necrotizing sialometaplasia
\item[C.] Periapical abscess
\item[D.] Periodontal abscess
\item[E.] Syphilitic gumma
\end{enumerate}

\textbf{Image:}
\begin{center}
\includegraphics[width=0.95\textwidth,height=0.50\textheight,width=0.90\textwidth,keepaspectratio]{images/nejm_20140807.jpg}
\end{center}
\vspace{12pt}
\newpage

\section*{Question 603 (ID: 20140814)}
\textbf{Date: }August 14,2014
\vspace{6pt}

What is the most likely diagnosis in this asymptomatic 17-year-old male?
\vspace{12pt}

\textbf{Options:}
\begin{enumerate}
\item[A.] Antrochoanal polyp
\item[B.] Foreign body
\item[C.] Palatine tonsillitis
\item[D.] Pharyngeal lymphoma
\item[E.] Tonsillar abscess
\end{enumerate}

\textbf{Image:}
\begin{center}
\includegraphics[width=0.95\textwidth,height=0.50\textheight,width=0.90\textwidth,keepaspectratio]{images/nejm_20140814.jpg}
\end{center}
\vspace{12pt}
\newpage

\section*{Question 604 (ID: 20140821)}
\textbf{Date: }August 21,2014
\vspace{6pt}

What is the diagnosis?
\vspace{12pt}

\textbf{Options:}
\begin{enumerate}
\item[A.] Diaphragmatic hernia
\item[B.] Gastric bezoar
\item[C.] Inferior mesenteric artery thrombosis
\item[D.] Pancreatic phlegmon
\item[E.] Small-bowel volvulus
\end{enumerate}

\textbf{Image:}
\begin{center}
\includegraphics[width=0.72\textwidth,height=0.50\textheight,width=0.90\textwidth,keepaspectratio]{images/nejm_20140821.jpg}
\end{center}
\vspace{12pt}
\newpage

\section*{Question 605 (ID: 20140828)}
\textbf{Date: }August 28,2014
\vspace{6pt}

What is the diagnosis in this patient who developed burning pain, pruritus, and edema of the hands upon their exposure to water and who had this image taken under a Wood's lamp?
\vspace{12pt}

\textbf{Options:}
\begin{enumerate}
\item[A.] Acquired acral mycosis
\item[B.] Aquagenic keratoderma
\item[C.] Dyshidrotic eczema
\item[D.] Leprosy
\item[E.] Ochronosis
\end{enumerate}

\textbf{Image:}
\begin{center}
\includegraphics[width=0.95\textwidth,height=0.50\textheight,width=0.90\textwidth,keepaspectratio]{images/nejm_20140828.jpg}
\end{center}
\vspace{12pt}
\newpage

\section*{Question 606 (ID: 20140904)}
\textbf{Date: }September 04,2014
\vspace{6pt}

What is the diagnosis?
\vspace{12pt}

\textbf{Options:}
\begin{enumerate}
\item[A.] Aerophagia
\item[B.] Congenital diaphragmatic hernia
\item[C.] Duodenal atresia
\item[D.] Gastroschesis
\item[E.] Hypertrophic pyloric stenosis
\end{enumerate}

\textbf{Image:}
\begin{center}
\includegraphics[width=0.54\textwidth,height=0.50\textheight,width=0.90\textwidth,keepaspectratio]{images/nejm_20140904.jpg}
\end{center}
\vspace{12pt}
\newpage

\section*{Question 607 (ID: 20140911)}
\textbf{Date: }September 11,2014
\vspace{6pt}

What is the diagnosis?
\vspace{12pt}

\textbf{Options:}
\begin{enumerate}
\item[A.] Angioedema
\item[B.] Cardiac tamponade
\item[C.] Necrotizing fasciitis
\item[D.] Superior vena cava syndrome
\item[E.] Toxic epidermal necrolysis
\end{enumerate}

\textbf{Image:}
\begin{center}
\includegraphics[width=0.95\textwidth,height=0.50\textheight,width=0.90\textwidth,keepaspectratio]{images/nejm_20140911.jpg}
\end{center}
\vspace{12pt}
\newpage

\section*{Question 608 (ID: 20140918)}
\textbf{Date: }September 18,2014
\vspace{6pt}

What is the most likely diagnosis in this asymptomatic male?
\vspace{12pt}

\textbf{Options:}
\begin{enumerate}
\item[A.] Dercum's disease
\item[B.] Familial multiple lipomatosis
\item[C.] Madelung's disease
\item[D.] Neurofibromatosis, type 1
\item[E.] Rheumatoid arthritis
\end{enumerate}

\textbf{Image:}
\begin{center}
\includegraphics[width=0.76\textwidth,height=0.50\textheight,width=0.90\textwidth,keepaspectratio]{images/nejm_20140918.jpg}
\end{center}
\vspace{12pt}
\newpage

\section*{Question 609 (ID: 20140925)}
\textbf{Date: }September 25,2014
\vspace{6pt}

What is the most likely diagnosis in this 47-year-old patient with an acute headache?
\vspace{12pt}

\textbf{Options:}
\begin{enumerate}
\item[A.] Encephalitis
\item[B.] Meningitis
\item[C.] Paget's disease
\item[D.] Stroke
\item[E.] Subdural hemorrhage
\end{enumerate}

\textbf{Image:}
\begin{center}
\includegraphics[width=0.95\textwidth,height=0.50\textheight,width=0.90\textwidth,keepaspectratio]{images/nejm_20140925.jpg}
\end{center}
\vspace{12pt}
\newpage

\section*{Question 610 (ID: 20141002)}
\textbf{Date: }October 02,2014
\vspace{6pt}

What is the most likely diagnosis?
\vspace{12pt}

\textbf{Options:}
\begin{enumerate}
\item[A.] Ankyloglossia
\item[B.] Atrophic glossitis
\item[C.] Creutzfeldt-Jakob disease
\item[D.] Hypoglossal nerve palsy
\item[E.] Motor neuron disease
\end{enumerate}

\textbf{Image:}
\begin{center}
\includegraphics[width=0.95\textwidth,height=0.50\textheight,width=0.90\textwidth,keepaspectratio]{images/nejm_20141002.jpg}
\end{center}
\vspace{12pt}
\newpage

\section*{Question 611 (ID: 20141009)}
\textbf{Date: }October 09,2014
\vspace{6pt}

What is the diagnosis?
\vspace{12pt}

\textbf{Options:}
\begin{enumerate}
\item[A.] Erythema chronicum migrans
\item[B.] Granuloma annulare
\item[C.] Pityriasis rosea
\item[D.] Sarcoidosis
\item[E.] Tinea circinata
\end{enumerate}

\textbf{Image:}
\begin{center}
\includegraphics[width=0.95\textwidth,height=0.50\textheight,width=0.90\textwidth,keepaspectratio]{images/nejm_20141009.jpg}
\end{center}
\vspace{12pt}
\newpage

\section*{Question 612 (ID: 20141016)}
\textbf{Date: }October 16,2014
\vspace{6pt}

What is the diagnosis in this patient with axillary lymphadenopathy?
\vspace{12pt}

\textbf{Options:}
\begin{enumerate}
\item[A.] Bacterial felon
\item[B.] Erysipelas
\item[C.] Herpetic whitlow
\item[D.] Paronychia
\item[E.] Scabies
\end{enumerate}

\textbf{Image:}
\begin{center}
\includegraphics[width=0.86\textwidth,height=0.50\textheight,width=0.90\textwidth,keepaspectratio]{images/nejm_20141016.jpg}
\end{center}
\vspace{12pt}
\newpage

\section*{Question 613 (ID: 20141023)}
\textbf{Date: }October 23,2014
\vspace{6pt}

What is the diagnosis in this asymptomatic male from North Africa?
\vspace{12pt}

\textbf{Options:}
\begin{enumerate}
\item[A.] Cutaneous leishmaniasis
\item[B.] Myiasis
\item[C.] Paronychia
\item[D.] Syphilis
\end{enumerate}

\textbf{Image:}
\begin{center}
\includegraphics[width=0.62\textwidth,height=0.50\textheight,width=0.90\textwidth,keepaspectratio]{images/nejm_20141023.jpg}
\end{center}
\vspace{12pt}
\newpage

\section*{Question 614 (ID: 20141030)}
\textbf{Date: }October 30,2014
\vspace{6pt}

What is the most likely diagnosis in this patient who had suffered a blunt head trauma 2 months earlier?
\vspace{12pt}

\textbf{Options:}
\begin{enumerate}
\item[A.] Acute angle-closure glaucoma
\item[B.] Carotid cavernous fistula
\item[C.] Ocular lymphoma
\item[D.] Periorbital cellulitis
\item[E.] Thyroid ophthalmopathy
\end{enumerate}

\textbf{Image:}
\begin{center}
\includegraphics[width=0.89\textwidth,height=0.50\textheight,width=0.90\textwidth,keepaspectratio]{images/nejm_20141030.jpg}
\end{center}
\vspace{12pt}
\newpage

\section*{Question 615 (ID: 20141113)}
\textbf{Date: }November 13,2014
\vspace{6pt}

What is the most likely diagnosis?
\vspace{12pt}

\textbf{Options:}
\begin{enumerate}
\item[A.] Achalasia
\item[B.] Diffuse esophageal spasm
\item[C.] Esophageal stricture
\item[D.] Hiatal hernia
\item[E.] Systemic sclerosis
\end{enumerate}

\textbf{Image:}
\begin{center}
\includegraphics[width=0.38\textwidth,height=0.50\textheight,width=0.90\textwidth,keepaspectratio]{images/nejm_20141113.jpg}
\end{center}
\vspace{12pt}
\newpage

\section*{Question 616 (ID: 20141120)}
\textbf{Date: }November 20,2014
\vspace{6pt}

What is the diagnosis?
\vspace{12pt}

\textbf{Options:}
\begin{enumerate}
\item[A.] Adrenal carcinoma
\item[B.] Autosplenectomy
\item[C.] Diverticular abscess
\item[D.] Emphysematous pyelonephritis
\item[E.] Renal vein thrombosis
\end{enumerate}

\textbf{Image:}
\begin{center}
\includegraphics[width=0.58\textwidth,height=0.50\textheight,width=0.90\textwidth,keepaspectratio]{images/nejm_20141120.jpg}
\end{center}
\vspace{12pt}
\newpage

\section*{Question 617 (ID: 20141127)}
\textbf{Date: }November 27,2014
\vspace{6pt}

What is the diagnosis in this 52-year-old man with epigastric pain?
\vspace{12pt}

\textbf{Options:}
\begin{enumerate}
\item[A.] Bezoar
\item[B.] Diverticulitis
\item[C.] Intussusception
\item[D.] Mesenteric infarction
\item[E.] Pseudomembranous colitis
\end{enumerate}

\textbf{Image:}
\begin{center}
\includegraphics[width=0.95\textwidth,height=0.50\textheight,width=0.90\textwidth,keepaspectratio]{images/nejm_20141127.jpg}
\end{center}
\vspace{12pt}
\newpage

\section*{Question 618 (ID: 20141204)}
\textbf{Date: }December 04,2014
\vspace{6pt}

This patient would be predicted to have a higher-than-average risk for which one of the following diseases?
\vspace{12pt}

\textbf{Options:}
\begin{enumerate}
\item[A.] Cirrhosis
\item[B.] Coronary artery disease
\item[D.] Hemorrhagic stroke
\item[E.] Hypothyroidism
\end{enumerate}

\textbf{Image:}
\begin{center}
\includegraphics[width=0.41\textwidth,height=0.50\textheight,width=0.90\textwidth,keepaspectratio]{images/nejm_20141204.jpg}
\end{center}
\vspace{12pt}
\newpage

\section*{Question 619 (ID: 20141211)}
\textbf{Date: }December 11,2014
\vspace{6pt}

What is the most likely diagnosis?
\vspace{12pt}

\textbf{Options:}
\begin{enumerate}
\item[A.] Hypophosphatasia
\item[B.] Osteosarcoma
\item[C.] Paget's disease
\item[D.] Rickets
\item[E.] Tuberculosis
\end{enumerate}

\textbf{Image:}
\begin{center}
\includegraphics[width=0.55\textwidth,height=0.50\textheight,width=0.90\textwidth,keepaspectratio]{images/nejm_20141211.jpg}
\end{center}
\vspace{12pt}
\newpage

\section*{Question 620 (ID: 20141218)}
\textbf{Date: }December 18,2014
\vspace{6pt}

What is the diagnosis?
\vspace{12pt}

\textbf{Options:}
\begin{enumerate}
\item[A.] Aortic aneurysm
\item[B.] Aortic coarctation
\item[C.] Diaphragmatic hernia
\item[D.] Esophageal diverticulum
\item[E.] Pericardial effusion
\end{enumerate}

\textbf{Image:}
\begin{center}
\includegraphics[width=0.95\textwidth,height=0.50\textheight,width=0.90\textwidth,keepaspectratio]{images/nejm_20141218.jpg}
\end{center}
\vspace{12pt}
\newpage

\section*{Question 621 (ID: 20141225)}
\textbf{Date: }December 25,2014
\vspace{6pt}

What is the diagnosis in this patient with chest pain?
\vspace{12pt}

\textbf{Options:}
\begin{enumerate}
\item[A.] Anxiety
\item[B.] Myocarditis
\item[C.] Pericardial effusion
\item[D.] Pulmonary embolism
\item[E.] Stenosis of the left anterior descending artery
\end{enumerate}

\textbf{Image:}
\begin{center}
\includegraphics[width=0.95\textwidth,height=0.50\textheight,width=0.90\textwidth,keepaspectratio]{images/nejm_20141225.jpg}
\end{center}
\vspace{12pt}
\newpage

\section*{Question 622 (ID: 20150101)}
\textbf{Date: }January 01,2015
\vspace{6pt}

What is the most likely diagnosis in this woman who developed eosinophilia and a diffuse erythematous eruption while being treated with hydroxychloroquine for Sjogren's syndrome?
\vspace{12pt}

\textbf{Options:}
\begin{enumerate}
\item[A.] Bullous impetigo
\item[B.] Exanthematous pustulosis
\item[C.] Exfoliative dermatitis
\item[D.] Stevens-Johnson syndrome
\item[E.] Sweet’s syndrome
\end{enumerate}

\textbf{Image:}
\begin{center}
\includegraphics[width=0.95\textwidth,height=0.50\textheight,width=0.90\textwidth,keepaspectratio]{images/nejm_20150101.jpg}
\end{center}
\vspace{12pt}
\newpage

\section*{Question 623 (ID: 20150108)}
\textbf{Date: }January 08,2015
\vspace{6pt}

What is the most likely diagnosis in this patient from Mauritania whose foot had developed these abnormalities over several years?
\vspace{12pt}

\textbf{Options:}
\begin{enumerate}
\item[A.] Actinomycetoma
\item[B.] Kaposi’s sarcoma
\item[C.] Leishmaniasis
\item[D.] Malignant melanoma
\end{enumerate}

\textbf{Image:}
\begin{center}
\includegraphics[width=0.49\textwidth,height=0.50\textheight,width=0.90\textwidth,keepaspectratio]{images/nejm_20150108.jpg}
\end{center}
\vspace{12pt}
\newpage

\section*{Question 624 (ID: 20150115)}
\textbf{Date: }January 15,2015
\vspace{6pt}

What is the diagnosis?
\vspace{12pt}

\textbf{Options:}
\begin{enumerate}
\item[A.] Gastric volvulus
\item[B.] Ileal intussusception
\item[C.] Mesenteric ischemia
\item[D.] Ogilvie’s syndrome
\item[E.] Perforated viscus
\end{enumerate}

\textbf{Image:}
\begin{center}
\includegraphics[width=0.95\textwidth,height=0.50\textheight,width=0.90\textwidth,keepaspectratio]{images/nejm_20150115.jpg}
\end{center}
\vspace{12pt}
\newpage

\section*{Question 625 (ID: 20150122)}
\textbf{Date: }January 22,2015
\vspace{6pt}

What is the diagnosis in this patient who developed chest pain after coughing?
\vspace{12pt}

\textbf{Options:}
\begin{enumerate}
\item[A.] Lung cancer
\item[B.] Lung herniation
\item[C.] Pneumothorax
\item[D.] Pulmonary embolism
\item[E.] Tuberculosis
\end{enumerate}

\textbf{Image:}
\begin{center}
\includegraphics[width=0.86\textwidth,height=0.50\textheight,width=0.90\textwidth,keepaspectratio]{images/nejm_20150122.jpg}
\end{center}
\vspace{12pt}
\newpage

\section*{Question 626 (ID: 20150129)}
\textbf{Date: }January 29,2015
\vspace{6pt}

What is the diagnosis in this patient with abdominal pain?
\vspace{12pt}

\textbf{Options:}
\begin{enumerate}
\item[A.] Gastrointestinal stromal tumor
\item[B.] Obstipation
\item[C.] Pancreatic phlegmon
\item[D.] Subphrenic abscess
\item[E.] Trichobezoar
\end{enumerate}

\textbf{Image:}
\begin{center}
\includegraphics[width=0.95\textwidth,height=0.50\textheight,width=0.90\textwidth,keepaspectratio]{images/nejm_20150129.jpg}
\end{center}
\vspace{12pt}
\newpage

\section*{Question 627 (ID: 20150205)}
\textbf{Date: }February 05,2015
\vspace{6pt}

What is the diagnosis?
\vspace{12pt}

\textbf{Options:}
\begin{enumerate}
\item[A.] Acute lymphangitis
\item[B.] Cellulitis
\item[C.] Deep vein thrombosis
\item[D.] Myositis
\item[E.] Superficial thrombophlebitis
\end{enumerate}

\textbf{Image:}
\begin{center}
\includegraphics[width=0.62\textwidth,height=0.50\textheight,width=0.90\textwidth,keepaspectratio]{images/nejm_20150205.jpg}
\end{center}
\vspace{12pt}
\newpage

\section*{Question 628 (ID: 20150212)}
\textbf{Date: }February 12,2015
\vspace{6pt}

What is the most likely diagnosis in this patient with hypothermia and bradycardia?
\vspace{12pt}

\textbf{Options:}
\begin{enumerate}
\item[A.] Cushing's syndrome
\item[B.] Generalized lipodystrophy
\item[C.] Hypothyroidism
\item[D.] Primary systemic amyloidosis
\item[E.] Superior vena cava syndrome
\end{enumerate}

\textbf{Image:}
\begin{center}
\includegraphics[width=0.6\textwidth,height=0.50\textheight,width=0.90\textwidth,keepaspectratio]{images/nejm_20150212.jpg}
\end{center}
\vspace{12pt}
\newpage

\section*{Question 629 (ID: 20141106)}
\textbf{Date: }November 06,2014
\vspace{6pt}

What is the most likely underlying diagnosis in this 82-year-old patient with diabetes mellitus who had undergone a total hip replacement 10 years previously?
\vspace{12pt}

\textbf{Options:}
\begin{enumerate}
\item[A.] Colon cancer
\item[B.] Hypogammaglobulinemia
\item[C.] Hypophosphatasia
\item[D.] Osteosarcoma
\item[E.] Tuberculosis
\end{enumerate}

\textbf{Image:}
\begin{center}
\includegraphics[width=0.5\textwidth,height=0.50\textheight,width=0.90\textwidth,keepaspectratio]{images/nejm_20141106.jpg}
\end{center}
\vspace{12pt}
\newpage

\section*{Question 630 (ID: 20150219)}
\textbf{Date: }February 19,2015
\vspace{6pt}

What is the diagnosis in this 31-year-old woman with gradual bilateral vision loss?
\vspace{12pt}

\textbf{Options:}
\begin{enumerate}
\item[A.] Glaucoma
\item[B.] Graves’ ophthalmopathy
\item[C.] Marfan’s syndrome
\item[D.] Retinitis pigmentosa
\item[E.] Temporal arteritis
\end{enumerate}

\textbf{Image:}
\begin{center}
\includegraphics[width=0.91\textwidth,height=0.50\textheight,width=0.90\textwidth,keepaspectratio]{images/nejm_20150219.jpg}
\end{center}
\vspace{12pt}
\newpage

\section*{Question 631 (ID: 20150226)}
\textbf{Date: }February 26,2015
\vspace{6pt}

What is the most likely cause of this long-standing dermatologic finding?
\vspace{12pt}

\textbf{Options:}
\begin{enumerate}
\item[A.] Acromegaly
\item[B.] McCune-Albright syndrome
\item[C.] Neurofibromatosis
\item[D.] Shingles
\item[E.] Speckled lentiginous naevi
\end{enumerate}

\textbf{Image:}
\begin{center}
\includegraphics[width=0.74\textwidth,height=0.50\textheight,width=0.90\textwidth,keepaspectratio]{images/nejm_20150226.jpg}
\end{center}
\vspace{12pt}
\newpage

\section*{Question 632 (ID: 20150305)}
\textbf{Date: }March 05,2015
\vspace{6pt}

What is the diagnosis in this patient who developed acute penile pain and detumescence during intercourse?
\vspace{12pt}

\textbf{Options:}
\begin{enumerate}
\item[A.] Penile contusion
\item[B.] Penile fracture
\item[C.] Traumatic epididymitis
\item[D.] Urethral rupture
\item[E.] Zipper injury
\end{enumerate}

\textbf{Image:}
\begin{center}
\includegraphics[width=0.62\textwidth,height=0.50\textheight,width=0.90\textwidth,keepaspectratio]{images/nejm_20150305.jpg}
\end{center}
\vspace{12pt}
\newpage

\section*{Question 633 (ID: 20150312)}
\textbf{Date: }March 12,2015
\vspace{6pt}

What is the most likely diagnosis?
\vspace{12pt}

\textbf{Options:}
\begin{enumerate}
\item[A.] Candidiasis
\item[B.] Geographic tongue
\item[C.] Herpes zoster
\item[D.] Lichen planus
\item[E.] Pemphigus
\end{enumerate}

\textbf{Image:}
\begin{center}
\includegraphics[width=0.57\textwidth,height=0.50\textheight,width=0.90\textwidth,keepaspectratio]{images/nejm_20150312.jpg}
\end{center}
\vspace{12pt}
\newpage

\section*{Question 634 (ID: 20150319)}
\textbf{Date: }March 19,2015
\vspace{6pt}

What is the most likely diagnosis?
\vspace{12pt}

\textbf{Options:}
\begin{enumerate}
\item[A.] Becker’s nevus
\item[B.] Hypomelanosis of Ito
\item[C.] Mongolian spots
\item[D.] Neurofibromatosis
\item[E.] Speckled lentiginous nevus
\end{enumerate}

\textbf{Image:}
\begin{center}
\includegraphics[width=0.55\textwidth,height=0.50\textheight,width=0.90\textwidth,keepaspectratio]{images/nejm_20150319.jpg}
\end{center}
\vspace{12pt}
\newpage

\section*{Question 635 (ID: 20150326)}
\textbf{Date: }March 26,2015
\vspace{6pt}

Infection with which one of the following organisms is the most likely cause of this rash?
\vspace{12pt}

\textbf{Options:}
\begin{enumerate}
\item[A.] Coxsackie virus type A16
\item[B.] Echovirus type 16
\item[C.] Group A streptococcus
\item[D.] Herpes simplex virus type 1
\item[E.] Norwalk virus
\end{enumerate}

\textbf{Image:}
\begin{center}
\includegraphics[width=0.95\textwidth,height=0.50\textheight,width=0.90\textwidth,keepaspectratio]{images/nejm_20150326.jpg}
\end{center}
\vspace{12pt}
\newpage

\section*{Question 636 (ID: 20150402)}
\textbf{Date: }April 02,2015
\vspace{6pt}

What is the most likely diagnosis in this 65-year-old woman with a 7-day history of throat pain and difficulty swallowing?
\vspace{12pt}

\textbf{Options:}
\begin{enumerate}
\item[A.] Epiglottitis
\item[B.] Foreign body aspiration
\item[C.] Peritonsillar abscess
\item[D.] Retropharyngeal abscess
\item[E.] Thyroid nodule
\end{enumerate}

\textbf{Image:}
\begin{center}
\includegraphics[width=0.93\textwidth,height=0.50\textheight,width=0.90\textwidth,keepaspectratio]{images/nejm_20150402.jpg}
\end{center}
\vspace{12pt}
\newpage

\section*{Question 637 (ID: 20150409)}
\textbf{Date: }April 09,2015
\vspace{6pt}

What is the most likely cause of this physical exam finding in a 65-year-old woman undergoing treatment for breast cancer?
\vspace{12pt}

\textbf{Options:}
\begin{enumerate}
\item[A.] Vitamin E deficiency
\item[B.] Chemotherapy
\item[C.] Scleroderma
\item[D.] Chronic inflammatory demyelinating polyneuropathy
\item[E.] Mercury toxicity
\end{enumerate}

\textbf{Image:}
\begin{center}
\includegraphics[width=0.62\textwidth,height=0.50\textheight,width=0.90\textwidth,keepaspectratio]{images/nejm_20150409.jpg}
\end{center}
\vspace{12pt}
\newpage

\section*{Question 638 (ID: 20150416)}
\textbf{Date: }April 16,2015
\vspace{6pt}

What is the most likely diagnosis in this 50-year-old woman with atrial fibrillation who sees halos around lights?
\vspace{12pt}

\textbf{Options:}
\begin{enumerate}
\item[A.] Early cataract
\item[B.] Herpes simplex virus keratitis
\item[C.] Digoxin-associated visual disturbance
\item[D.] Amaurosis fugax
\item[E.] Amiodarone-induced vortex keratopathy
\end{enumerate}

\textbf{Image:}
\begin{center}
\includegraphics[width=0.95\textwidth,height=0.50\textheight,width=0.90\textwidth,keepaspectratio]{images/nejm_20150416.jpg}
\end{center}
\vspace{12pt}
\newpage

\section*{Question 639 (ID: 20150423)}
\textbf{Date: }April 23,2015
\vspace{6pt}

The following physical exam finding is most commonly associated with which one of the following?
\vspace{12pt}

\textbf{Options:}
\begin{enumerate}
\item[A.] Chronic kidney disease
\item[B.] Thyroid dysfunction
\item[C.] End-stage liver disease
\item[D.] Pulmonary disease
\item[E.] Systemic infection
\end{enumerate}

\textbf{Image:}
\begin{center}
\includegraphics[width=0.95\textwidth,height=0.50\textheight,width=0.90\textwidth,keepaspectratio]{images/nejm_20150423.jpg}
\end{center}
\vspace{12pt}
\newpage

\section*{Question 640 (ID: 20150430)}
\textbf{Date: }April 30,2015
\vspace{6pt}

What is the most likely diagnosis?
\vspace{12pt}

\textbf{Options:}
\begin{enumerate}
\item[A.] Staphyloma
\item[B.] Pigmentary glaucoma
\item[C.] Conjunctival hemorrhage
\item[D.] Melanoma
\item[E.] Squamous-cell carcinoma
\end{enumerate}

\textbf{Image:}
\begin{center}
\includegraphics[width=0.95\textwidth,height=0.50\textheight,width=0.90\textwidth,keepaspectratio]{images/nejm_20150430.jpg}
\end{center}
\vspace{12pt}
\newpage

\section*{Question 641 (ID: 20150507)}
\textbf{Date: }May 07,2015
\vspace{6pt}

What is the most likely cause of this finding in a woman undergoing treatment for breast cancer with spinal metastases?
\vspace{12pt}

\textbf{Options:}
\begin{enumerate}
\item[A.] Radiation recall
\item[B.] Tinea corporis
\item[C.] Lichen simplex chronicus
\item[D.] Squamous-cell carcinoma
\item[E.] Acute febrile neutrophilic dermatosis
\end{enumerate}

\textbf{Image:}
\begin{center}
\includegraphics[width=0.95\textwidth,height=0.50\textheight,width=0.90\textwidth,keepaspectratio]{images/nejm_20150507.jpg}
\end{center}
\vspace{12pt}
\newpage

\section*{Question 642 (ID: 20150514)}
\textbf{Date: }May 14,2015
\vspace{6pt}

What is the most likely diagnosis?
\vspace{12pt}

\textbf{Options:}
\begin{enumerate}
\item[A.] Oral lichen planus
\item[B.] Severe migratory glossitis
\item[C.] Invasive mucormycosis
\item[D.] Oral leukoplakia
\item[E.] Verrucous carcinoma
\end{enumerate}

\textbf{Image:}
\begin{center}
\includegraphics[width=0.73\textwidth,height=0.50\textheight,width=0.90\textwidth,keepaspectratio]{images/nejm_20150514.jpg}
\end{center}
\vspace{12pt}
\newpage

\section*{Question 643 (ID: 20150521)}
\textbf{Date: }May 21,2015
\vspace{6pt}

What is the most likely cause of this finding?
\vspace{12pt}

\textbf{Options:}
\begin{enumerate}
\item[A.] Periorbital cellulitis
\item[B.] Lyme disease
\item[C.] Dermatomyositis
\item[D.] Periocular atopic dermatitis
\item[E.] Polymorphous light eruption
\end{enumerate}

\textbf{Image:}
\begin{center}
\includegraphics[width=0.78\textwidth,height=0.50\textheight,width=0.90\textwidth,keepaspectratio]{images/nejm_20150521.jpg}
\end{center}
\vspace{12pt}
\newpage

\section*{Question 644 (ID: 20150528)}
\textbf{Date: }May 28,2015
\vspace{6pt}

What is the most likely diagnosis in a woman who presents with fever, cough, rhinorrhea, and this exam finding?
\vspace{12pt}

\textbf{Options:}
\begin{enumerate}
\item[A.] Polymorphous light eruption
\item[B.] Dengue fever
\item[C.] Measles
\item[D.] Stevens-Johnson syndrome
\item[E.] Acute cutaneous lupus erythematosus
\end{enumerate}

\textbf{Image:}
\begin{center}
\includegraphics[width=0.95\textwidth,height=0.50\textheight,width=0.90\textwidth,keepaspectratio]{images/nejm_20150528.jpg}
\end{center}
\vspace{12pt}
\newpage

\section*{Question 645 (ID: 20150604)}
\textbf{Date: }June 04,2015
\vspace{6pt}

What is the most likely diagnosis in a full-term neonate with electrolyte abnormalities and this exam finding?
\vspace{12pt}

\textbf{Options:}
\begin{enumerate}
\item[A.] Congenital adrenal hyperplasia
\item[B.] Hypospadias with cryptorchidism
\item[C.] Preterm ovarian hyperstimulation syndrome
\item[D.] Turner syndrome
\item[E.] Mixed gonadal dysgenesis
\end{enumerate}

\textbf{Image:}
\begin{center}
\includegraphics[width=0.95\textwidth,height=0.50\textheight,width=0.90\textwidth,keepaspectratio]{images/nejm_20150604.jpg}
\end{center}
\vspace{12pt}
\newpage

\section*{Question 646 (ID: 20150611)}
\textbf{Date: }June 11,2015
\vspace{6pt}

What is the most likely cause of this finding in a man from Mexico who had a biopsy showing gram-negative bacilli?
\vspace{12pt}

\textbf{Options:}
\begin{enumerate}
\item[A.] Klebsiella rhinoscleromatis
\item[B.] Mycobacterium leprae
\item[C.] Acinetobacter baumannii
\item[D.] Treponema pallidum
\item[E.] Mycobacterium tuberculosis
\end{enumerate}

\textbf{Image:}
\begin{center}
\includegraphics[width=0.73\textwidth,height=0.50\textheight,width=0.90\textwidth,keepaspectratio]{images/nejm_20150611.jpg}
\end{center}
\vspace{12pt}
\newpage

\section*{Question 647 (ID: 20150618)}
\textbf{Date: }June 18,2015
\vspace{6pt}

What is the most likely diagnosis for this lesion that has been present since birth?
\vspace{12pt}

\textbf{Options:}
\begin{enumerate}
\item[A.] Tinea versicolor
\item[B.] Acne vulgaris
\item[C.] Keratosis pilaris
\item[D.] Perforating folliculitis
\item[E.] Nevus comedonicus
\end{enumerate}

\textbf{Image:}
\begin{center}
\includegraphics[width=0.95\textwidth,height=0.50\textheight,width=0.90\textwidth,keepaspectratio]{images/nejm_20150618.jpg}
\end{center}
\vspace{12pt}
\newpage

\section*{Question 648 (ID: 20150625)}
\textbf{Date: }June 25,2015
\vspace{6pt}

What is the name for this finding that is reducible on physical exam?
\vspace{12pt}

\textbf{Options:}
\begin{enumerate}
\item[A.] Heberden’s nodes
\item[B.] Rheumatoid nodules
\item[C.] Jaccoud’s arthropathy
\item[D.] Pseudogout tophi
\item[E.] Erosive arthropathy of rheumatoid arthritis
\end{enumerate}

\textbf{Image:}
\begin{center}
\includegraphics[width=0.95\textwidth,height=0.50\textheight,width=0.90\textwidth,keepaspectratio]{images/nejm_20150625.jpg}
\end{center}
\vspace{12pt}
\newpage

\section*{Question 649 (ID: 20150702)}
\textbf{Date: }July 02,2015
\vspace{6pt}

What is the most likely diagnosis in this woman with dermatomyositis?
\vspace{12pt}

\textbf{Options:}
\begin{enumerate}
\item[A.] Soft-tissue calcinosis
\item[B.] Deep-vein thrombosis
\item[C.] Leukemia cutis
\item[D.] Chondrosarcoma
\item[E.] Calciphylaxis
\end{enumerate}

\textbf{Image:}
\begin{center}
\includegraphics[width=0.95\textwidth,height=0.50\textheight,width=0.90\textwidth,keepaspectratio]{images/nejm_20150702.jpg}
\end{center}
\vspace{12pt}
\newpage

\section*{Question 650 (ID: 20150709)}
\textbf{Date: }July 09,2015
\vspace{6pt}

What is the most likely diagnosis of this scalp lesion present since birth?
\vspace{12pt}

\textbf{Options:}
\begin{enumerate}
\item[A.] Cerebriform nevus sebaceous
\item[B.] Verruca vulgaris
\item[C.] Basal-cell carcinoma of the scalp
\item[D.] Dermal melanosis
\item[E.] Seborrheic keratosis
\end{enumerate}

\textbf{Image:}
\begin{center}
\includegraphics[width=0.64\textwidth,height=0.50\textheight,width=0.90\textwidth,keepaspectratio]{images/nejm_20150709.jpg}
\end{center}
\vspace{12pt}
\newpage

\section*{Question 651 (ID: 20150716)}
\textbf{Date: }July 16,2015
\vspace{6pt}

In this traumatic injury with foreign-object penetration to the neck, what would be the most appropriate management by first responders?
\vspace{12pt}

\textbf{Options:}
\begin{enumerate}
\item[A.] Remove the foreign object completely
\item[B.] Leave the foreign object in place
\item[C.] Apply cervical collar to stabilize the neck
\item[D.] Irrigate the wound with low-pressure saline flush
\item[E.] Intubate the patient to ensure airway patency
\end{enumerate}

\textbf{Image:}
\begin{center}
\includegraphics[width=0.47\textwidth,height=0.50\textheight,width=0.90\textwidth,keepaspectratio]{images/nejm_20150716.jpg}
\end{center}
\vspace{12pt}
\newpage

\section*{Question 652 (ID: 20150723)}
\textbf{Date: }July 23,2015
\vspace{6pt}

What is the most appropriate therapy for this 8-year-old child who presents with 7 days of fever, rash, and this physical examination finding?
\vspace{12pt}

\textbf{Options:}
\begin{enumerate}
\item[A.] Intravenous or oral doxycycline
\item[B.] NSAIDs and systemic glucocorticoids
\item[C.] Penicillin or azithromycin
\item[D.] Intravenous valacyclovir
\item[E.] Aspirin and intravenous immune globulin
\end{enumerate}

\textbf{Image:}
\begin{center}
\includegraphics[width=0.53\textwidth,height=0.50\textheight,width=0.90\textwidth,keepaspectratio]{images/nejm_20150723.jpg}
\end{center}
\vspace{12pt}
\newpage

\section*{Question 653 (ID: 20150730)}
\textbf{Date: }July 30,2015
\vspace{6pt}

Which of the following is the least likely differential diagnosis associated with this image finding?
\vspace{12pt}

\textbf{Options:}
\begin{enumerate}
\item[A.] Lead exposure
\item[B.] Polyvinyl chloride exposure
\item[C.] Hyperparathyroidism
\item[D.] Systemic sclerosis
\item[E.] Diabetes
\end{enumerate}

\textbf{Image:}
\begin{center}
\includegraphics[width=0.95\textwidth,height=0.50\textheight,width=0.90\textwidth,keepaspectratio]{images/nejm_20150730.jpg}
\end{center}
\vspace{12pt}
\newpage

\section*{Question 654 (ID: 20150806)}
\textbf{Date: }August 06,2015
\vspace{6pt}

Which of the following is the least likely differential diagnosis associated with this image finding?
\vspace{12pt}

\textbf{Options:}
\begin{enumerate}
\item[A.] Pulmonary hypertension
\item[B.] Neoplasm
\item[C.] Vasculitis
\item[D.] Septic emboli
\item[E.] Infection
\end{enumerate}

\textbf{Image:}
\begin{center}
\includegraphics[width=0.95\textwidth,height=0.50\textheight,width=0.90\textwidth,keepaspectratio]{images/nejm_20150806.jpg}
\end{center}
\vspace{12pt}
\newpage

\section*{Question 655 (ID: 20150813)}
\textbf{Date: }August 13,2015
\vspace{6pt}

What is the most appropriate management of this patient who presents with fatigue and dyspnea?
\vspace{12pt}

\textbf{Options:}
\begin{enumerate}
\item[A.] Prompt coronary revascularization
\item[B.] Pericardiocentesis
\item[C.] Thoracentesis
\item[D.] Diuresis
\item[E.] Pacemaker placement
\end{enumerate}

\textbf{Image:}
\begin{center}
\includegraphics[width=0.95\textwidth,height=0.50\textheight,width=0.90\textwidth,keepaspectratio]{images/nejm_20150813.jpg}
\end{center}
\vspace{12pt}
\newpage

\section*{Question 656 (ID: 20150820)}
\textbf{Date: }August 20,2015
\vspace{6pt}

What is the most appropriate treatment for a child with recurrent hematuria and the following finding?
\vspace{12pt}

\textbf{Options:}
\begin{enumerate}
\item[A.] Fluconazole
\item[B.] Metronidazole
\item[C.] Praziquantel
\item[D.] Steroids
\item[E.] Azathioprine
\end{enumerate}

\textbf{Image:}
\begin{center}
\includegraphics[width=0.95\textwidth,height=0.50\textheight,width=0.90\textwidth,keepaspectratio]{images/nejm_20150820.jpg}
\end{center}
\vspace{12pt}
\newpage

\section*{Question 657 (ID: 20150827)}
\textbf{Date: }August 27,2015
\vspace{6pt}

What is the most likely diagnosis in an infant with poor feeding and vomiting?
\vspace{12pt}

\textbf{Options:}
\begin{enumerate}
\item[A.] Hirschsprung's disease
\item[B.] Incarcerated inguinal hernia
\item[C.] Meckel's diverticulum
\item[D.] Gastric pneumatosis
\item[E.] Intussusception
\end{enumerate}

\textbf{Image:}
\begin{center}
\includegraphics[width=0.66\textwidth,height=0.50\textheight,width=0.90\textwidth,keepaspectratio]{images/nejm_20150827.jpg}
\end{center}
\vspace{12pt}
\newpage

\section*{Question 658 (ID: 20150903)}
\textbf{Date: }September 03,2015
\vspace{6pt}

Which of the following is the most likely diagnosis for this incidental finding on an MRI?
\vspace{12pt}

\textbf{Options:}
\begin{enumerate}
\item[A.] Type 1 Chiari malformation
\item[B.] Cavernous angioma
\item[C.] Asymptomatic cortical infarct
\item[D.] Aneurysm
\item[E.] Arachnoid cyst
\end{enumerate}

\textbf{Image:}
\begin{center}
\includegraphics[width=0.8\textwidth,height=0.50\textheight,width=0.90\textwidth,keepaspectratio]{images/nejm_20150903.jpg}
\end{center}
\vspace{12pt}
\newpage

\section*{Question 659 (ID: 20150910)}
\textbf{Date: }September 10,2015
\vspace{6pt}

Which of the following conditions is NOT commonly associated with this finding?
\vspace{12pt}

\textbf{Options:}
\begin{enumerate}
\item[A.] Chronic liver disease
\item[B.] Cutaneous collagenous vasculopathy
\item[C.] Hemophilia A
\item[D.] CREST syndrome
\item[E.] Rosacea
\end{enumerate}

\textbf{Image:}
\begin{center}
\includegraphics[width=0.63\textwidth,height=0.50\textheight,width=0.90\textwidth,keepaspectratio]{images/nejm_20150910.jpg}
\end{center}
\vspace{12pt}
\newpage

\section*{Question 660 (ID: 20150917)}
\textbf{Date: }September 17,2015
\vspace{6pt}

Which of the following is NOT a common risk factor associated with this image finding?
\vspace{12pt}

\textbf{Options:}
\begin{enumerate}
\item[A.] Leukemia
\item[B.] Crohn's disease
\item[C.] Phenytoin use
\item[D.] Pregnancy
\end{enumerate}

\textbf{Image:}
\begin{center}
\includegraphics[width=0.95\textwidth,height=0.50\textheight,width=0.90\textwidth,keepaspectratio]{images/nejm_20150917.jpg}
\end{center}
\vspace{12pt}
\newpage

\section*{Question 661 (ID: 20150924)}
\textbf{Date: }September 24,2015
\vspace{6pt}

Which of the following is the most likely cause of a painless swelling seen in this image?
\vspace{12pt}

\textbf{Options:}
\begin{enumerate}
\item[A.] Coronary angiography
\item[B.] Median nerve compression
\item[C.] Gout flare
\item[D.] Rheumatoid arthritis
\item[E.] Bullous pemphigoid
\end{enumerate}

\textbf{Image:}
\begin{center}
\includegraphics[width=0.95\textwidth,height=0.50\textheight,width=0.90\textwidth,keepaspectratio]{images/nejm_20150924.jpg}
\end{center}
\vspace{12pt}
\newpage

\section*{Question 662 (ID: 20151001)}
\textbf{Date: }October 01,2015
\vspace{6pt}

Which one of the following is a typical step in the life cycle of ascaris as seen in this image?
\vspace{12pt}

\textbf{Options:}
\begin{enumerate}
\item[A.] Eggs hatch in the large intestine
\item[B.] Larvae penetrate the lungs
\item[C.] Larvae develop into adult worms in the lungs
\item[D.] Eggs hatch in the small intestine
\item[E.] Larvae penetrate the perianal mucosal surface
\end{enumerate}

\textbf{Image:}
\begin{center}
\includegraphics[width=0.95\textwidth,height=0.50\textheight,width=0.90\textwidth,keepaspectratio]{images/nejm_20151001.jpg}
\end{center}
\vspace{12pt}
\newpage

\section*{Question 663 (ID: 20151008)}
\textbf{Date: }October 08,2015
\vspace{6pt}

Which one of the following is the least likely cause of this patient’s symptoms of retrobulbar swelling and diplopia?
\vspace{12pt}

\textbf{Options:}
\begin{enumerate}
\item[A.] Dermatomyositis
\item[B.] Cancer
\item[C.] Oral contraceptive use
\item[D.] Facial trauma
\item[E.] Sinusitis
\end{enumerate}

\textbf{Image:}
\begin{center}
\includegraphics[width=0.95\textwidth,height=0.50\textheight,width=0.90\textwidth,keepaspectratio]{images/nejm_20151008.jpg}
\end{center}
\vspace{12pt}
\newpage

\section*{Question 664 (ID: 20151015)}
\textbf{Date: }October 15,2015
\vspace{6pt}

Which one of the following is the best test to diagnose the condition seen in this image?
\vspace{12pt}

\textbf{Options:}
\begin{enumerate}
\item[A.] Skin wound culture
\item[B.] Autoantibodies to desmosomes in the epidermis
\item[C.] Epicutaneous patch testing
\item[D.] HSV serology
\item[E.] IgG and complement deposition at the dermis-epidermis junction
\end{enumerate}

\textbf{Image:}
\begin{center}
\includegraphics[width=0.95\textwidth,height=0.50\textheight,width=0.90\textwidth,keepaspectratio]{images/nejm_20151015.jpg}
\end{center}
\vspace{12pt}
\newpage

\section*{Question 665 (ID: 20151022)}
\textbf{Date: }October 22,2015
\vspace{6pt}

Which of the following is an incorrect statement related to painful vesicular toe rash of 3 months' duration?
\vspace{12pt}

\textbf{Options:}
\begin{enumerate}
\item[A.] Infection is associated with immunosuppressant use.
\item[B.] Infection is typically localized to the site of inoculation.
\item[C.] Transmission occurs with contaminated water contact.
\item[D.] Infection may travel along the lymphatic system leading to lymphangitis.
\item[E.] Transmission is associated with aerosolized contaminated soil exposure.
\end{enumerate}

\textbf{Image:}
\begin{center}
\includegraphics[width=0.95\textwidth,height=0.50\textheight,width=0.90\textwidth,keepaspectratio]{images/nejm_20151022.jpg}
\end{center}
\vspace{12pt}
\newpage

\section*{Question 666 (ID: 20151029)}
\textbf{Date: }October 29,2015
\vspace{6pt}

Which one of the following choices would be the least useful diagnostic test for this heel ulcer of 2 years' duration?
\vspace{12pt}

\textbf{Options:}
\begin{enumerate}
\item[A.] Hemoglobin A1c
\item[B.] Skin biopsy
\item[C.] Duplex ultrasound of the lower extremity
\item[D.] Wound culture
\item[E.] Ankle-brachial index
\end{enumerate}

\textbf{Image:}
\begin{center}
\includegraphics[width=0.62\textwidth,height=0.50\textheight,width=0.90\textwidth,keepaspectratio]{images/nejm_20151029.jpg}
\end{center}
\vspace{12pt}
\newpage

\section*{Question 667 (ID: 20151105)}
\textbf{Date: }November 05,2015
\vspace{6pt}

Which of the following choices is the least likely diagnosis for the sudden painless vision loss in this patient?
\vspace{12pt}

\textbf{Options:}
\begin{enumerate}
\item[A.] Vitreous hemorrhage
\item[B.] Retinal artery occlusion
\item[C.] Retinal vein occlusion
\item[D.] Occipital stroke
\item[E.] Narrow angle glaucoma
\end{enumerate}

\textbf{Image:}
\begin{center}
\includegraphics[width=0.95\textwidth,height=0.50\textheight,width=0.90\textwidth,keepaspectratio]{images/nejm_20151105.jpg}
\end{center}
\vspace{12pt}
\newpage

\section*{Question 668 (ID: 20151112)}
\textbf{Date: }November 12,2015
\vspace{6pt}

Which one of the following treatments is the treatment of choice for the ulcerated breast lesions seen in this image?
\vspace{12pt}

\textbf{Options:}
\begin{enumerate}
\item[A.] Radiation therapy
\item[B.] Doxorubicin
\item[C.] Penicillin
\item[D.] Mastectomy
\item[E.] Hydrocortisone cream
\end{enumerate}

\textbf{Image:}
\begin{center}
\includegraphics[width=0.95\textwidth,height=0.50\textheight,width=0.90\textwidth,keepaspectratio]{images/nejm_20151112.jpg}
\end{center}
\vspace{12pt}
\newpage

\section*{Question 669 (ID: 20151119)}
\textbf{Date: }November 19,2015
\vspace{6pt}

What is the most likely diagnosis associated with symptoms of pruritic skin changes seen in these images?
\vspace{12pt}

\textbf{Options:}
\begin{enumerate}
\item[A.] Malignancy
\item[B.] Psoriasis
\item[C.] Inflammatory bowel disease
\item[D.] Acute hepatitis C infection
\item[E.] Cutaneous sarcoidosis
\end{enumerate}

\textbf{Image:}
\begin{center}
\includegraphics[width=0.95\textwidth,height=0.50\textheight,width=0.90\textwidth,keepaspectratio]{images/nejm_20151119.jpg}
\end{center}
\vspace{12pt}
\newpage

\section*{Question 670 (ID: 20151126)}
\textbf{Date: }November 26,2015
\vspace{6pt}

What is the most likely diagnosis associated with fever and jaundice and these image findings?
\vspace{12pt}

\textbf{Options:}
\begin{enumerate}
\item[A.] Pancreatitis
\item[B.] Pyelonephritis
\item[C.] Pericarditis
\item[D.] Pancreatic malignancy
\item[E.] Pylephlebitis
\end{enumerate}

\textbf{Image:}
\begin{center}
\includegraphics[width=0.6\textwidth,height=0.50\textheight,width=0.90\textwidth,keepaspectratio]{images/nejm_20151126.jpg}
\end{center}
\vspace{12pt}
\newpage

\section*{Question 671 (ID: 20151203)}
\textbf{Date: }December 03,2015
\vspace{6pt}

What is the most likely diagnosis associated with this finding?
\vspace{12pt}

\textbf{Options:}
\begin{enumerate}
\item[A.] Squamous-cell carcinoma
\item[B.] Paronychia
\item[C.] Verruca vulgaris
\item[D.] Onychomycosis
\item[E.] Pyogenic granuloma
\end{enumerate}

\textbf{Image:}
\begin{center}
\includegraphics[width=0.95\textwidth,height=0.50\textheight,width=0.90\textwidth,keepaspectratio]{images/nejm_20151203.jpg}
\end{center}
\vspace{12pt}
\newpage

\section*{Question 672 (ID: 20151210)}
\textbf{Date: }December 10,2015
\vspace{6pt}

Which of the following is the least likely cause of eschars as seen here?
\vspace{12pt}

\textbf{Options:}
\begin{enumerate}
\item[A.] Scrub typhus
\item[B.] Cutaneous anthrax
\item[C.] Rickettsia akari
\item[D.] Chickenpox
\item[E.] Brown recluse spider bite
\end{enumerate}

\textbf{Image:}
\begin{center}
\includegraphics[width=0.95\textwidth,height=0.50\textheight,width=0.90\textwidth,keepaspectratio]{images/nejm_20151210.jpg}
\end{center}
\vspace{12pt}
\newpage

\section*{Question 673 (ID: 20151217)}
\textbf{Date: }December 17,2015
\vspace{6pt}

What is the most likely diagnosis associated with these findings and symptoms of eye and ear pain?
\vspace{12pt}

\textbf{Options:}
\begin{enumerate}
\item[A.] Polychondritis
\item[B.] Sjogren's syndrome
\item[C.] Granulomatosis with polyangiitis
\item[D.] Takayasu arteritis
\item[E.] Sarcoidosis
\end{enumerate}

\textbf{Image:}
\begin{center}
\includegraphics[width=0.95\textwidth,height=0.50\textheight,width=0.90\textwidth,keepaspectratio]{images/nejm_20151217.jpg}
\end{center}
\vspace{12pt}
\newpage

\section*{Question 674 (ID: 20151224)}
\textbf{Date: }December 24,2015
\vspace{6pt}

What radiographic finding is observed here?
\vspace{12pt}

\textbf{Options:}
\begin{enumerate}
\item[A.] Air in the subcutaneous space
\item[B.] Diffuse osteopenia
\item[C.] Ground-glass opacity
\item[D.] Kerley’s B lines
\item[E.] Air bronchogram
\end{enumerate}

\textbf{Image:}
\begin{center}
\includegraphics[width=0.95\textwidth,height=0.50\textheight,width=0.90\textwidth,keepaspectratio]{images/nejm_20151224.jpg}
\end{center}
\vspace{12pt}
\newpage

\section*{Question 675 (ID: 20151231)}
\textbf{Date: }December 31,2015
\vspace{6pt}

What is the most likely diagnosis in a 60-year-old man with blurry vision for 2 months and the findings seen here?
\vspace{12pt}

\textbf{Options:}
\begin{enumerate}
\item[B.] Syphilis
\item[C.] Closed-angle glaucoma
\item[D.] Hypertensive crisis
\item[E.] Waldenström’s macroglobulinemia
\end{enumerate}

\textbf{Image:}
\begin{center}
\includegraphics[width=0.86\textwidth,height=0.50\textheight,width=0.90\textwidth,keepaspectratio]{images/nejm_20151231.jpg}
\end{center}
\vspace{12pt}
\newpage

\section*{Question 676 (ID: 20160107)}
\textbf{Date: }January 07,2016
\vspace{6pt}

What is the most likely diagnosis of this penile lesion found in a 54-year-old man with lymphadenopathy and weight loss?
\vspace{12pt}

\textbf{Options:}
\begin{enumerate}
\item[A.] Squamous-cell carcinoma
\item[B.] Syphilis
\item[C.] Extramammary Paget disease
\item[D.] Condyloma acuminatum
\item[E.] Kaposi’s sarcoma
\end{enumerate}

\textbf{Image:}
\begin{center}
\includegraphics[width=0.95\textwidth,height=0.50\textheight,width=0.90\textwidth,keepaspectratio]{images/nejm_20160107.jpg}
\end{center}
\vspace{12pt}
\newpage

\section*{Question 677 (ID: 20160114)}
\textbf{Date: }January 14,2016
\vspace{6pt}

What is the cause of this patient's abdominal discomfort and chronic anemia and these findings?
\vspace{12pt}

\textbf{Options:}
\begin{enumerate}
\item[A.] Bites from infected sand flies
\item[B.] Swimming in infected waterways in the Nile delta
\item[C.] Consumption of undercooked, infected fish
\item[D.] Walking barefoot on infected soil
\item[E.] Consumption of undercooked, infected beef
\end{enumerate}

\textbf{Image:}
\begin{center}
\includegraphics[width=0.95\textwidth,height=0.50\textheight,width=0.90\textwidth,keepaspectratio]{images/nejm_20160114.jpg}
\end{center}
\vspace{12pt}
\newpage

\section*{Question 678 (ID: 20160121)}
\textbf{Date: }January 21,2016
\vspace{6pt}

Which is the most likely cause for this patient’s erosive, painless upper-lip sore and these findings on the feet?
\vspace{12pt}

\textbf{Options:}
\begin{enumerate}
\item[A.] Herpes simplex virus infection
\item[B.] Treponema pallidum infection
\item[C.] Lymphogranuloma venereum
\item[D.] Behcet syndrome
\item[E.] Haemophilus ducreyi infection
\end{enumerate}

\textbf{Image:}
\begin{center}
\includegraphics[width=0.66\textwidth,height=0.50\textheight,width=0.90\textwidth,keepaspectratio]{images/nejm_20160121.jpg}
\end{center}
\vspace{12pt}
\newpage

\section*{Question 679 (ID: 20160128)}
\textbf{Date: }January 28,2016
\vspace{6pt}

Which one of the following is a likely finding in a patient with fever, cough, malaise, recent travel to Africa, and this skin rash?
\vspace{12pt}

\textbf{Options:}
\begin{enumerate}
\item[A.] Thrombocytosis
\item[B.] Neutrophilia
\item[C.] Eosinophilia
\item[D.] Lymphocytosis
\item[E.] Monocytosis
\end{enumerate}

\textbf{Image:}
\begin{center}
\includegraphics[width=0.8\textwidth,height=0.50\textheight,width=0.90\textwidth,keepaspectratio]{images/nejm_20160128.jpg}
\end{center}
\vspace{12pt}
\newpage

\section*{Question 680 (ID: 20160204)}
\textbf{Date: }February 04,2016
\vspace{6pt}

Which of the following choices is the most likely diagnosis in a patient presenting with 3 months of headache, jaw claudication, and these scalp findings?
\vspace{12pt}

\textbf{Options:}
\begin{enumerate}
\item[A.] Behcet's syndrome
\item[B.] Polyarteritis nodosa
\item[C.] Granulomatosis with polyangiitis
\item[D.] Takayasu arteritis
\item[E.] Giant-cell arteritis
\end{enumerate}

\textbf{Image:}
\begin{center}
\includegraphics[width=0.95\textwidth,height=0.50\textheight,width=0.90\textwidth,keepaspectratio]{images/nejm_20160204.jpg}
\end{center}
\vspace{12pt}
\newpage

\section*{Question 681 (ID: 20160211)}
\textbf{Date: }February 11,2016
\vspace{6pt}

What is the most likely diagnosis in a man with a 3-month history of migratory lesions on the tongue?
\vspace{12pt}

\textbf{Options:}
\begin{enumerate}
\item[A.] Geographic tongue
\item[B.] Oral candidiasis
\item[C.] Lichen planus
\item[D.] Oral hairy leukoplakia
\item[E.] Pemphigus vulgaris
\end{enumerate}

\textbf{Image:}
\begin{center}
\includegraphics[width=0.95\textwidth,height=0.50\textheight,width=0.90\textwidth,keepaspectratio]{images/nejm_20160211.jpg}
\end{center}
\vspace{12pt}
\newpage

\section*{Question 682 (ID: 20160218)}
\textbf{Date: }February 18,2016
\vspace{6pt}

What is the most likely diagnosis in a man with recurrent facial flushing, foreign-body sensation, and blurry vision?
\vspace{12pt}

\textbf{Options:}
\begin{enumerate}
\item[A.] Rosacea
\item[B.] Blepharitis
\item[C.] Sjogren’s syndrome
\item[D.] Anterior uveitis
\item[E.] Conjunctival hemorrhage
\end{enumerate}

\textbf{Image:}
\begin{center}
\includegraphics[width=0.95\textwidth,height=0.50\textheight,width=0.90\textwidth,keepaspectratio]{images/nejm_20160218.jpg}
\end{center}
\vspace{12pt}
\newpage

\section*{Question 683 (ID: 20160225)}
\textbf{Date: }February 25,2016
\vspace{6pt}

What is the most likely diagnosis in a man with chest pain, hypotension, dizziness, and this ECG finding?
\vspace{12pt}

\textbf{Options:}
\begin{enumerate}
\item[A.] Cardiac tamponade
\item[B.] Left ventricle rupture
\item[C.] Right ventricle aneurysm
\item[D.] Right coronary artery occlusion
\item[E.] Left anterior descending artery occlusion
\end{enumerate}

\textbf{Image:}
\begin{center}
\includegraphics[width=0.95\textwidth,height=0.50\textheight,width=0.90\textwidth,keepaspectratio]{images/nejm_20160225.jpg}
\end{center}
\vspace{12pt}
\newpage

\section*{Question 684 (ID: 20160303)}
\textbf{Date: }March 03,2016
\vspace{6pt}

What is the most likely diagnosis in a woman with a chronic depigmenting rash with periods of remission and these findings?
\vspace{12pt}

\textbf{Options:}
\begin{enumerate}
\item[A.] Tinea versicolor
\item[B.] Psoriasis
\item[C.] Seborrheic dermatitis
\item[D.] Vitiligo
\item[E.] Pityriasis rosea
\end{enumerate}

\textbf{Image:}
\begin{center}
\includegraphics[width=0.95\textwidth,height=0.50\textheight,width=0.90\textwidth,keepaspectratio]{images/nejm_20160303.jpg}
\end{center}
\vspace{12pt}
\newpage

\section*{Question 685 (ID: 20160310)}
\textbf{Date: }March 10,2016
\vspace{6pt}

What is the most likely cerebrospinal fluid (CSF) finding from the lumbar puncture seen here in a patient with lower-extremity weakness?
\vspace{12pt}

\textbf{Options:}
\begin{enumerate}
\item[A.] Bilirubin elevation
\item[B.] Pseudomonas aeruginosa
\item[C.] Lymphocytosis
\item[D.] Bleeding diathesis
\item[E.] Elevated protein
\end{enumerate}

\textbf{Image:}
\begin{center}
\includegraphics[width=0.95\textwidth,height=0.50\textheight,width=0.90\textwidth,keepaspectratio]{images/nejm_20160310.jpg}
\end{center}
\vspace{12pt}
\newpage

\section*{Question 686 (ID: 20160317)}
\textbf{Date: }March 17,2016
\vspace{6pt}

What is the most likely diagnosis in a patient with generalized weakness and this finding?
\vspace{12pt}

\textbf{Options:}
\begin{enumerate}
\item[A.] Lichen planus pigmentosus
\item[B.] Smoker’s melanosis
\item[C.] Cushing’s disease
\item[D.] Diffuse melanosis cutis
\item[E.] Hemochromatosis
\end{enumerate}

\textbf{Image:}
\begin{center}
\includegraphics[width=0.95\textwidth,height=0.50\textheight,width=0.90\textwidth,keepaspectratio]{images/nejm_20160317.jpg}
\end{center}
\vspace{12pt}
\newpage

\section*{Question 687 (ID: 20160324)}
\textbf{Date: }March 24,2016
\vspace{6pt}

What is the best treatment for this patient with right thigh swelling and these findings?
\vspace{12pt}

\textbf{Options:}
\begin{enumerate}
\item[A.] Glucocorticoids
\item[B.] Androgen deprivation therapy
\item[C.] Anticoagulation
\item[D.] Chemotherapy
\item[E.] Bisphosphonates
\end{enumerate}

\textbf{Image:}
\begin{center}
\includegraphics[width=0.8\textwidth,height=0.50\textheight,width=0.90\textwidth,keepaspectratio]{images/nejm_20160324.jpg}
\end{center}
\vspace{12pt}
\newpage

\section*{Question 688 (ID: 20160331)}
\textbf{Date: }March 31,2016
\vspace{6pt}

What is the most likely diagnosis in a patient with this migratory, pruritic skin eruption?
\vspace{12pt}

\textbf{Options:}
\begin{enumerate}
\item[A.] Cutaneous larva migrans
\item[B.] Contact dermatitis
\item[C.] Paragonimiasis
\item[D.] Myiasis
\item[E.] Loiasis
\end{enumerate}

\textbf{Image:}
\begin{center}
\includegraphics[width=0.42\textwidth,height=0.50\textheight,width=0.90\textwidth,keepaspectratio]{images/nejm_20160331.jpg}
\end{center}
\vspace{12pt}
\newpage

\section*{Question 689 (ID: 20160407)}
\textbf{Date: }April 07,2016
\vspace{6pt}

What is the most likely diagnosis in a patient with dysarthria and gait instability and this finding?
\vspace{12pt}

\textbf{Options:}
\begin{enumerate}
\item[A.] Huntington’s disease
\item[B.] Wilson’s disease
\item[C.] Middle cerebral artery stroke
\item[D.] NMDA receptor encephalitis
\item[E.] Metronidazole-associated encephalopathy
\end{enumerate}

\textbf{Image:}
\begin{center}
\includegraphics[width=0.81\textwidth,height=0.50\textheight,width=0.90\textwidth,keepaspectratio]{images/nejm_20160407.jpg}
\end{center}
\vspace{12pt}
\newpage

\section*{Question 690 (ID: 20160414)}
\textbf{Date: }April 14,2016
\vspace{6pt}

What is the most likely cause of this physical finding?
\vspace{12pt}

\textbf{Options:}
\begin{enumerate}
\item[A.] Ascariasis
\item[B.] Esophagogastric bypass
\item[C.] Hirschsprung's disease
\item[D.] Acute intestinal pseudo-obstruction
\item[E.] Cutaneous leishmaniasis
\end{enumerate}

\textbf{Image:}
\begin{center}
\includegraphics[width=0.75\textwidth,height=0.50\textheight,width=0.90\textwidth,keepaspectratio]{images/nejm_20160414.jpg}
\end{center}
\vspace{12pt}
\newpage

\section*{Question 691 (ID: 20160421)}
\textbf{Date: }April 21,2016
\vspace{6pt}

What is the most likely diagnosis in an elderly woman with dyspnea and these findings?
\vspace{12pt}

\textbf{Options:}
\begin{enumerate}
\item[A.] Esophageal tumor
\item[B.] Vein-graft aneurysms
\item[C.] Thymoma
\item[D.] Liposarcoma
\item[E.] Hemangioma
\end{enumerate}

\textbf{Image:}
\begin{center}
\includegraphics[width=0.95\textwidth,height=0.50\textheight,width=0.90\textwidth,keepaspectratio]{images/nejm_20160421.jpg}
\end{center}
\vspace{12pt}
\newpage

\section*{Question 692 (ID: 20160428)}
\textbf{Date: }April 28,2016
\vspace{6pt}

What is the most likely diagnosis in this man with left knee swelling and these findings for the past 2 years?
\vspace{12pt}

\textbf{Options:}
\begin{enumerate}
\item[A.] Nonpulmonary tuberculosis
\item[B.] Tularemia
\item[C.] Cutaneous anthrax
\item[D.] Sporotrichosis
\item[E.] Actinomycosis
\end{enumerate}

\textbf{Image:}
\begin{center}
\includegraphics[width=0.95\textwidth,height=0.50\textheight,width=0.90\textwidth,keepaspectratio]{images/nejm_20160428.jpg}
\end{center}
\vspace{12pt}
\newpage

\section*{Question 693 (ID: 20160505)}
\textbf{Date: }May 05,2016
\vspace{6pt}

What is the most likely cause of these skull radiographic findings?
\vspace{12pt}

\textbf{Options:}
\begin{enumerate}
\item[A.] Calvarial osteomyelitis
\item[B.] Scalocephaly
\item[C.] Langerhans' cell histiocytosis
\item[D.] Carpenter's syndrome
\item[E.] Sickle cell disease
\end{enumerate}

\textbf{Image:}
\begin{center}
\includegraphics[width=0.95\textwidth,height=0.50\textheight,width=0.90\textwidth,keepaspectratio]{images/nejm_20160505.jpg}
\end{center}
\vspace{12pt}
\newpage

\section*{Question 694 (ID: 20160512)}
\textbf{Date: }May 12,2016
\vspace{6pt}

What is the most likely diagnosis in a man with oral pain, urinary frequency, thirst, and these radiographic findings?
\vspace{12pt}

\textbf{Options:}
\begin{enumerate}
\item[A.] Vitamin D excess
\item[B.] Scurvy
\item[C.] Langerhans'-cell histiocytosis
\item[D.] Paget's disease of bone
\item[E.] Metastatic prostate cancer
\end{enumerate}

\textbf{Image:}
\begin{center}
\includegraphics[width=0.95\textwidth,height=0.50\textheight,width=0.90\textwidth,keepaspectratio]{images/nejm_20160512.jpg}
\end{center}
\vspace{12pt}
\newpage

\section*{Question 695 (ID: 20160519)}
\textbf{Date: }May 19,2016
\vspace{6pt}

Which of the following is the most likely diagnosis?
\vspace{12pt}

\textbf{Options:}
\begin{enumerate}
\item[A.] Varicocele
\item[B.] Indirect hiatal hernia
\item[C.] Testicular torsion
\item[D.] Hydrocele
\item[E.] Epididymitis
\end{enumerate}

\textbf{Image:}
\begin{center}
\includegraphics[width=0.95\textwidth,height=0.50\textheight,width=0.90\textwidth,keepaspectratio]{images/nejm_20160519.jpg}
\end{center}
\vspace{12pt}
\newpage

\section*{Question 696 (ID: 20160526)}
\textbf{Date: }May 26,2016
\vspace{6pt}

Which one of the following is likely to be positive in a man with acute renal failure and these findings?
\vspace{12pt}

\textbf{Options:}
\begin{enumerate}
\item[A.] IgA anti-tissue transglutaminase
\item[B.] IgG antibodies against Lyme disease
\item[C.] Helicobacter pylori stool antigen
\item[D.] Antimitochondrial antibodies
\item[E.] Antineutrophil cytoplasmic autoantibody
\end{enumerate}

\textbf{Image:}
\begin{center}
\includegraphics[width=0.58\textwidth,height=0.50\textheight,width=0.90\textwidth,keepaspectratio]{images/nejm_20160526.jpg}
\end{center}
\vspace{12pt}
\newpage

\section*{Question 697 (ID: 20160602)}
\textbf{Date: }June 02,2016
\vspace{6pt}

Which three conditions are likely and consistent with this ECG following parathyroidectomy?
\vspace{12pt}

\textbf{Options:}
\begin{enumerate}
\item[A.] Hypocalcemia, hypophosphatemia, and hyperkalemia
\item[B.] Hypocalcemia,  hypophosphatemia, and hypokalemia
\item[C.] Hypercalcemia, hyperphosphatemia, and hyperkalemia
\item[D.] Hypercalcemia,  hypophosphatemia, and hyperkalemia
\item[E.] Hypocalcemia, hyperphosphatemia, and hypokalemia
\end{enumerate}

\textbf{Image:}
\begin{center}
\includegraphics[width=0.95\textwidth,height=0.50\textheight,width=0.90\textwidth,keepaspectratio]{images/nejm_20160602.jpg}
\end{center}
\vspace{12pt}
\newpage

\section*{Question 698 (ID: 20160609)}
\textbf{Date: }June 09,2016
\vspace{6pt}

Which is the most likely diagnosis in a man with constitutional symptoms?
\vspace{12pt}

\textbf{Options:}
\begin{enumerate}
\item[A.] Cytomegalovirus infection
\item[B.] Acute lymphoblastic leukemia
\item[C.] Brucellosis
\item[D.] Hepatitis C virus infection
\item[E.] Perinephritic abscess
\end{enumerate}

\textbf{Image:}
\begin{center}
\includegraphics[width=0.86\textwidth,height=0.50\textheight,width=0.90\textwidth,keepaspectratio]{images/nejm_20160609.jpg}
\end{center}
\vspace{12pt}
\newpage

\section*{Question 699 (ID: 20160616)}
\textbf{Date: }June 16,2016
\vspace{6pt}

What is the most likely diagnosis in a man with acute skin eruptions?
\vspace{12pt}

\textbf{Options:}
\begin{enumerate}
\item[A.] Reactive epidermal hyperplasia
\item[B.] Pyogenic granuloma
\item[C.] Eccrine poroma
\item[D.] Chromomycosis
\item[E.] Iododerma
\end{enumerate}

\textbf{Image:}
\begin{center}
\includegraphics[width=0.95\textwidth,height=0.50\textheight,width=0.90\textwidth,keepaspectratio]{images/nejm_20160616.jpg}
\end{center}
\vspace{12pt}
\newpage

\section*{Question 700 (ID: 20160623)}
\textbf{Date: }June 23,2016
\vspace{6pt}

What is the likely cause of these painless oral lesions?
\vspace{12pt}

\textbf{Options:}
\begin{enumerate}
\item[A.] Human papillomavirus
\item[B.] Varicella-zoster virus
\item[C.] Guttate psoriasis
\item[D.] Verrucous carcinoma
\item[E.] Molluscum contagiosum
\end{enumerate}

\textbf{Image:}
\begin{center}
\includegraphics[width=0.95\textwidth,height=0.50\textheight,width=0.90\textwidth,keepaspectratio]{images/nejm_20160623.jpg}
\end{center}
\vspace{12pt}
\newpage

\section*{Question 701 (ID: 20160630)}
\textbf{Date: }June 30,2016
\vspace{6pt}

What is typically true of the Dubin-Johnson syndrome, as seen here?
\vspace{12pt}

\textbf{Options:}
\begin{enumerate}
\item[A.] Elevated alanine and aspartate aminotransferases
\item[B.] Defect in glucuronidation leading to hyperbilirubinemia
\item[C.] Increased prothrombin time
\item[D.] Autosomal dominant inheritance
\item[E.] About 50\% of the serum bilirubin is conjugated
\end{enumerate}

\textbf{Image:}
\begin{center}
\includegraphics[width=0.95\textwidth,height=0.50\textheight,width=0.90\textwidth,keepaspectratio]{images/nejm_20160630.jpg}
\end{center}
\vspace{12pt}
\newpage

\section*{Question 702 (ID: 20160707)}
\textbf{Date: }July 07,2016
\vspace{6pt}

In a patient with a history of renal transplantation, now on cyclosporine, this finding is most likely associated with which underlying diagnosis?
\vspace{12pt}

\textbf{Options:}
\begin{enumerate}
\item[A.] Rheumatoid arthritis
\item[B.] Mycobacterial Infection
\item[D.] Calciphylaxis
\item[E.] Calcium pyrophosphate deposition disease
\end{enumerate}

\textbf{Image:}
\begin{center}
\includegraphics[width=0.95\textwidth,height=0.50\textheight,width=0.90\textwidth,keepaspectratio]{images/nejm_20160707.jpg}
\end{center}
\vspace{12pt}
\newpage

\section*{Question 703 (ID: 20160714)}
\textbf{Date: }July 14,2016
\vspace{6pt}

An 18-year-old woman with recurrent epigastric discomfort and diarrhea is diagnosed with Giardia lamblia infection, and subsequently undergoes gastroduodenoscopy with the findings in this image. What is the diagnosis?
\vspace{12pt}

\textbf{Options:}
\begin{enumerate}
\item[A.] Glycogenic acanthosis
\item[B.] Basal-cell hyperplasia
\item[C.] Celiac disease
\item[D.] Nodular lymphoid hyperplasia
\item[E.] Familial adenomatous polyposis
\end{enumerate}

\textbf{Image:}
\begin{center}
\includegraphics[width=0.95\textwidth,height=0.50\textheight,width=0.90\textwidth,keepaspectratio]{images/nejm_20160714.jpg}
\end{center}
\vspace{12pt}
\newpage

\section*{Question 704 (ID: 20160721)}
\textbf{Date: }July 21,2016
\vspace{6pt}

What is the most likely cause of this patient’s presentation with rash and pain on the left side of his face, and plaque over the anterior two-thirds of his tongue?
\vspace{12pt}

\textbf{Options:}
\begin{enumerate}
\item[A.] Herpes simplex virus
\item[B.] Enterovirus
\item[C.] Contact dermatitis
\item[D.] Bullous pemphigoid
\item[E.] Varicella-zoster virus
\end{enumerate}

\textbf{Image:}
\begin{center}
\includegraphics[width=0.95\textwidth,height=0.50\textheight,width=0.90\textwidth,keepaspectratio]{images/nejm_20160721.jpg}
\end{center}
\vspace{12pt}
\newpage

\section*{Question 705 (ID: 20160728)}
\textbf{Date: }July 28,2016
\vspace{6pt}

What is the diagnosis in this 3-year-old girl with recent upper respiratory infection and new pruritic blanching rash?
\vspace{12pt}

\textbf{Options:}
\begin{enumerate}
\item[A.] Erythema multiforme
\item[B.] Mastocytosis
\item[C.] Urticaria multiforme
\item[D.] Henoch-Schonlein purpura
\item[E.] Erythema migrans
\end{enumerate}

\textbf{Image:}
\begin{center}
\includegraphics[width=0.62\textwidth,height=0.50\textheight,width=0.90\textwidth,keepaspectratio]{images/nejm_20160728.jpg}
\end{center}
\vspace{12pt}
\newpage

\section*{Question 706 (ID: 20160804)}
\textbf{Date: }August 04,2016
\vspace{6pt}

A man presented with new peripheral edema. What diagnosis does his chest X-ray suggest?
\vspace{12pt}

\textbf{Options:}
\begin{enumerate}
\item[A.] Superior vena cava syndrome
\item[B.] Dilated thoracic aorta
\item[C.] Right middle lobe pneumonia
\item[D.] Cardiac tamponade
\item[E.] Giant right atrium
\end{enumerate}

\textbf{Image:}
\begin{center}
\includegraphics[width=0.94\textwidth,height=0.50\textheight,width=0.90\textwidth,keepaspectratio]{images/nejm_20160804.jpg}
\end{center}
\vspace{12pt}
\newpage

\section*{Question 707 (ID: 20160811)}
\textbf{Date: }August 11,2016
\vspace{6pt}

A 36-year-old woman presented with uncontrolled hypertension after an emergency cesarean section and was found to have bilateral renal artery stenosis and this imaging finding. What is the most likely diagnosis?
\vspace{12pt}

\textbf{Options:}
\begin{enumerate}
\item[A.] Takayasu’s arteritis
\item[B.] Aortic dissection
\item[C.] Classical polyarteritis nodosa
\item[D.] Abdominal aortic aneurysm
\item[E.] Microscopic polyangiitis
\end{enumerate}

\textbf{Image:}
\begin{center}
\includegraphics[width=0.87\textwidth,height=0.50\textheight,width=0.90\textwidth,keepaspectratio]{images/nejm_20160811.jpg}
\end{center}
\vspace{12pt}
\newpage

\section*{Question 708 (ID: 20160818)}
\textbf{Date: }August 18,2016
\vspace{6pt}

A 79-year-old woman presented with a 1-year history of gradual neurological decline involving confusion, memory loss, and imbalance. What is the most likely diagnosis?
\vspace{12pt}

\textbf{Options:}
\begin{enumerate}
\item[A.] Colloid cyst
\item[B.] Glioblastoma multiforme
\item[C.] Thrombosed intracranial aneurysm
\item[D.] Choroid plexus metastases
\item[E.] Intraventricular meningioma
\end{enumerate}

\textbf{Image:}
\begin{center}
\includegraphics[width=0.68\textwidth,height=0.50\textheight,width=0.90\textwidth,keepaspectratio]{images/nejm_20160818.jpg}
\end{center}
\vspace{12pt}
\newpage

\section*{Question 709 (ID: 20160825)}
\textbf{Date: }August 25,2016
\vspace{6pt}

A 42-year-old man with diabetes presented with a history urinary incontinence and severe dysuria over several days. What diagnosis does his computed tomographic (CT) scan suggest?
\vspace{12pt}

\textbf{Options:}
\begin{enumerate}
\item[A.] Benign prostatic hypertrophy
\item[B.] Bladder perforation
\item[C.] Urethritis
\item[D.] Emphysematous prostatitis
\item[E.] Colovesical fistula
\end{enumerate}

\textbf{Image:}
\begin{center}
\includegraphics[width=0.7\textwidth,height=0.50\textheight,width=0.90\textwidth,keepaspectratio]{images/nejm_20160825.jpg}
\end{center}
\vspace{12pt}
\newpage

\section*{Question 710 (ID: 20160901)}
\textbf{Date: }September 01,2016
\vspace{6pt}

A 64-year-old man presented with a two-year history of this asymptomatic skin condition, which was symmetrically distributed over his trunk, arms and legs. Laboratory tests demonstrated a raised fasting blood glucose level. What is the most likely diagnosis?
\vspace{12pt}

\textbf{Options:}
\begin{enumerate}
\item[A.] Chronic urticaria
\item[B.] Pityriasis rosea
\item[C.] Cutaneous sarcoidosis
\item[D.] Systemic lupus erythematosis
\item[E.] Generalized granuloma annulare
\end{enumerate}

\textbf{Image:}
\begin{center}
\includegraphics[width=0.59\textwidth,height=0.50\textheight,width=0.90\textwidth,keepaspectratio]{images/nejm_20160901.jpg}
\end{center}
\vspace{12pt}
\newpage

\section*{Question 711 (ID: 20160908)}
\textbf{Date: }September 08,2016
\vspace{6pt}

An 88-year-old man with a history of a quadruple coronary-artery bypass grafting procedure and transapical transcatheter aortic-valve replacement presented with this skin condition. What is the cause of this appearance?
\vspace{12pt}

\textbf{Options:}
\begin{enumerate}
\item[A.] Warfarin necrosis
\item[B.] Subacute bacterial endocarditis
\item[C.] Cryoglobulinaemia
\item[D.] Cholesterol embolization
\item[E.] Heparin-induced thrombocytopenia
\end{enumerate}

\textbf{Image:}
\begin{center}
\includegraphics[width=0.93\textwidth,height=0.50\textheight,width=0.90\textwidth,keepaspectratio]{images/nejm_20160908.jpg}
\end{center}
\vspace{12pt}
\newpage

\section*{Question 712 (ID: 20160915)}
\textbf{Date: }September 15,2016
\vspace{6pt}

An Eritrean man presented with 3 years of intermittent fevers; workup revealed anemia, elevated aminotransferase levels, and this finding on the blood smear. What organism caused these symptoms?
\vspace{12pt}

\textbf{Options:}
\begin{enumerate}
\item[A.] Wuchereria bancrofti
\item[B.] Onchocerca volvulus
\item[C.] Borrelia recurrentis
\item[D.] Trypanosoma brucei
\item[E.] Plamodium vivax
\end{enumerate}

\textbf{Image:}
\begin{center}
\includegraphics[width=0.87\textwidth,height=0.50\textheight,width=0.90\textwidth,keepaspectratio]{images/nejm_20160915.jpg}
\end{center}
\vspace{12pt}
\newpage

\section*{Question 713 (ID: 20160922)}
\textbf{Date: }September 22,2016
\vspace{6pt}

A woman presented with this tender lesion on her abdomen; she was previously healthy, did not recall a recent insect bite, and otherwise had no symptoms. She improved with antibiotics. What is the diagnosis?
\vspace{12pt}

\textbf{Options:}
\begin{enumerate}
\item[A.] Sporotrichoid lymphocutaneous infection
\item[B.] Streptococcal cellulitis with lymphangitis
\item[C.] Cutaneous lymphangitis carcinomatosa
\item[D.] Early localized lyme disease
\item[E.] Cutaneous lupus erythematosus
\end{enumerate}

\textbf{Image:}
\begin{center}
\includegraphics[width=0.95\textwidth,height=0.50\textheight,width=0.90\textwidth,keepaspectratio]{images/nejm_20160922.jpg}
\end{center}
\vspace{12pt}
\newpage

\section*{Question 714 (ID: 20160929)}
\textbf{Date: }September 29,2016
\vspace{6pt}

A 38-year-old woman with a history of three cesarean sections was admitted to the hospital in labor at 36 weeks of gestation. Following an emergency cesarean section, histopathological examination identified invasion of chorionic villi into but not through the myometrium. What is the diagnosis?
\vspace{12pt}

\textbf{Options:}
\begin{enumerate}
\item[A.] Placenta accreta
\item[B.] Placenta increta
\item[C.] Placenta percreta
\item[D.] Decidua
\item[E.] Choriocarcinoma
\end{enumerate}

\textbf{Image:}
\begin{center}
\includegraphics[width=0.46\textwidth,height=0.50\textheight,width=0.90\textwidth,keepaspectratio]{images/nejm_20160929.jpg}
\end{center}
\vspace{12pt}
\newpage

\section*{Question 715 (ID: 20161006)}
\textbf{Date: }October 06,2016
\vspace{6pt}

A 60-year-old man presented with abdominal pain, hypotension, and deranged renal function. What is the diagnosis?
\vspace{12pt}

\textbf{Options:}
\begin{enumerate}
\item[A.] Pseudomembranous colitis
\item[B.] Acute mesenteric ischemia
\item[C.] Diverticulitis
\item[D.] Ulcerative colitis
\item[E.] Pseudomelanosis coli
\end{enumerate}

\textbf{Image:}
\begin{center}
\includegraphics[width=0.89\textwidth,height=0.50\textheight,width=0.90\textwidth,keepaspectratio]{images/nejm_20161006.jpg}
\end{center}
\vspace{12pt}
\newpage

\section*{Question 716 (ID: 20161013)}
\textbf{Date: }October 13,2016
\vspace{6pt}

A 28-year-old woman had a painful vesicular lesion that erupted into this erythematous rash, which did not improve with antibiotics. What is the diagnosis?
\vspace{12pt}

\textbf{Options:}
\begin{enumerate}
\item[A.] Cutaneous lupus
\item[B.] Erysipelas
\item[C.] Dermatomyositis
\item[D.] Rosacea
\item[E.] Fixed drug eruption
\end{enumerate}

\textbf{Image:}
\begin{center}
\includegraphics[width=0.64\textwidth,height=0.50\textheight,width=0.90\textwidth,keepaspectratio]{images/nejm_20161013.jpg}
\end{center}
\vspace{12pt}
\newpage

\section*{Question 717 (ID: 20161020)}
\textbf{Date: }October 20,2016
\vspace{6pt}

A 51-year-old male with diabetes presented with 2 weeks of general malaise and fever. What is the diagnosis?
\vspace{12pt}

\textbf{Options:}
\begin{enumerate}
\item[A.] Small bowel obstruction
\item[B.] Emphysematous pyelonephritis
\item[C.] Paralytic ileus
\item[D.] Renocolic fistula
\item[E.] Splenic abscess
\end{enumerate}

\textbf{Image:}
\begin{center}
\includegraphics[width=0.9\textwidth,height=0.50\textheight,width=0.90\textwidth,keepaspectratio]{images/nejm_20161020.jpg}
\end{center}
\vspace{12pt}
\newpage

\section*{Question 718 (ID: 20161027)}
\textbf{Date: }October 27,2016
\vspace{6pt}

A 53-year-old woman presented with intensely itchy skin lesions and burning in her mouth. The following appearances are noted on examination of her wrist. What is the diagnosis?
\vspace{12pt}

\textbf{Options:}
\begin{enumerate}
\item[A.] Lichen sclerosus
\item[B.] Lichen planus
\item[C.] Psoriasis
\item[D.] Eczema
\item[E.] Tinea cruris
\end{enumerate}

\textbf{Image:}
\begin{center}
\includegraphics[width=0.95\textwidth,height=0.50\textheight,width=0.90\textwidth,keepaspectratio]{images/nejm_20161027.jpg}
\end{center}
\vspace{12pt}
\newpage

\section*{Question 719 (ID: 20161103)}
\textbf{Date: }November 03,2016
\vspace{6pt}

A 68-year-old man presented with unilateral ptosis with no other symptoms. Application of an ice pack to the left eye improved his symptoms. What is the diagnosis?
\vspace{12pt}

\textbf{Options:}
\begin{enumerate}
\item[A.] Bell’s palsy
\item[B.] Myasthenia gravis
\item[C.] Benign essential blepharospasm
\item[D.] Multiple sclerosis
\item[E.] Myotonic dystrophy
\end{enumerate}

\textbf{Image:}
\begin{center}
\includegraphics[width=0.85\textwidth,height=0.50\textheight,width=0.90\textwidth,keepaspectratio]{images/nejm_20161103.jpg}
\end{center}
\vspace{12pt}
\newpage

\section*{Question 720 (ID: 20161110)}
\textbf{Date: }November 10,2016
\vspace{6pt}

A man presented with difficulty walking and urinary incontinence. On examination, his pupils were nonreactive to bright light but constricted when focusing on a near object. What is the diagnosis?
\vspace{12pt}

\textbf{Options:}
\begin{enumerate}
\item[A.] Multiple sclerosis
\item[B.] Tabes dorsalis
\item[C.] Parinaud's syndrome
\item[D.] Sarcoidosis
\item[E.] Horner’s syndrome
\end{enumerate}

\textbf{Image:}
\begin{center}
\includegraphics[width=0.95\textwidth,height=0.50\textheight,width=0.90\textwidth,keepaspectratio]{images/nejm_20161110.jpg}
\end{center}
\vspace{12pt}
\newpage

\section*{Question 721 (ID: 20161117)}
\textbf{Date: }November 17,2016
\vspace{6pt}

An 81-year-old woman presented with a 1-week history of vomiting. What is the diagnosis?
\vspace{12pt}

\textbf{Options:}
\begin{enumerate}
\item[A.] Lung abscess
\item[B.] Perforated duodenal ulcer
\item[C.] Epiphrenic diverticulum
\item[D.] Paraesophageal hernia
\item[E.] Esophageal duplication cyst
\end{enumerate}

\textbf{Image:}
\begin{center}
\includegraphics[width=0.93\textwidth,height=0.50\textheight,width=0.90\textwidth,keepaspectratio]{images/nejm_20161117.jpg}
\end{center}
\vspace{12pt}
\newpage

\section*{Question 722 (ID: 20161124)}
\textbf{Date: }November 24,2016
\vspace{6pt}

A 44-year-old man with a history of quadriplegia presented to the emergency department with symptoms of a urinary tract infection. What is the diagnosis?
\vspace{12pt}

\textbf{Options:}
\begin{enumerate}
\item[A.] Bladder perforation
\item[B.] Neurogenic megacolon
\item[C.] Chronic constipation
\item[D.] Gastric distention
\item[E.] Ulcerative colitis
\end{enumerate}

\textbf{Image:}
\begin{center}
\includegraphics[width=0.89\textwidth,height=0.50\textheight,width=0.90\textwidth,keepaspectratio]{images/nejm_20161124.jpg}
\end{center}
\vspace{12pt}
\newpage

\section*{Question 723 (ID: 20161201)}
\textbf{Date: }December 01,2016
\vspace{6pt}

An 18-year-old Fijian woman presented with a 3-year history of a concentric scaly rash on her trunk and upper and lower limbs. Her brother had similar skin findings. What is the diagnosis?
\vspace{12pt}

\textbf{Options:}
\begin{enumerate}
\item[A.] Erythema gyratum repens
\item[B.] Tinea corporis
\item[C.] Discoid eczema
\item[D.] Tinea imbricata
\item[E.] Erythrasma
\end{enumerate}

\textbf{Image:}
\begin{center}
\includegraphics[width=0.95\textwidth,height=0.50\textheight,width=0.90\textwidth,keepaspectratio]{images/nejm_20161201.jpg}
\end{center}
\vspace{12pt}
\newpage

\section*{Question 724 (ID: 20161208)}
\textbf{Date: }December 08,2016
\vspace{6pt}

A 60-year-old woman presented with fevers, chills, myalgias, headache, and two pruritic areas on her back after returning from Botswana and Zimbabwe. She was found to have thrombocytopenia, and this finding on her blood smear. What organism is responsible for her presentation?
\vspace{12pt}

\textbf{Options:}
\begin{enumerate}
\item[A.] Wuchereria bancrofti
\item[B.] Toxoplasma gondii
\item[C.] Plasmodium falciparum
\item[D.] Trypanosoma brucei
\item[E.] Leishmania donovani
\end{enumerate}

\textbf{Image:}
\begin{center}
\includegraphics[width=0.95\textwidth,height=0.50\textheight,width=0.90\textwidth,keepaspectratio]{images/nejm_20161208.jpg}
\end{center}
\vspace{12pt}
\newpage

\section*{Question 725 (ID: 20161215)}
\textbf{Date: }December 15,2016
\vspace{6pt}

A 36-year-old man presented with 1 month of fever and pain in his shoulders and knees after having a sore throat. He subsequently developed this evanescent nonpruritic macular rash (marked with a skin marking pen). What is the diagnosis?
\vspace{12pt}

\textbf{Options:}
\begin{enumerate}
\item[A.] Erythema infectiosum
\item[B.] Erythema multiforme
\item[C.] Erythema migrans
\item[D.] Erythema nodosum
\item[E.] Erythema marginatum
\end{enumerate}

\textbf{Image:}
\begin{center}
\includegraphics[width=0.95\textwidth,height=0.50\textheight,width=0.90\textwidth,keepaspectratio]{images/nejm_20161215.jpg}
\end{center}
\vspace{12pt}
\newpage

\section*{Question 726 (ID: 20161222)}
\textbf{Date: }December 22,2016
\vspace{6pt}

A 23-year-old woman presented with a 1-month history of headache, syncope, weight gain. Physical examination revealed hypertrophic appearance of the thigh and calf muscles, and imaging of her thighs is shown. What is the diagnosis?
\vspace{12pt}

\textbf{Options:}
\begin{enumerate}
\item[A.] Septic emboli
\item[B.] Echinococcus granulosus
\item[C.] Myxoid sarcoma
\item[D.] Cysticercosis
\item[E.] Toxoplasma gondii
\end{enumerate}

\textbf{Image:}
\begin{center}
\includegraphics[width=0.8\textwidth,height=0.50\textheight,width=0.90\textwidth,keepaspectratio]{images/nejm_20161222.jpg}
\end{center}
\vspace{12pt}
\newpage

\section*{Question 727 (ID: 20161229)}
\textbf{Date: }December 29,2016
\vspace{6pt}

A 90-year-old man presented with a 1-month history of progressive edema in both of his legs, malaise, and dizziness associated with hypoglycemia. His chest radiograph demonstrated abnormal findings. What is the diagnosis?
\vspace{12pt}

\textbf{Options:}
\begin{enumerate}
\item[A.] Gastric bezoar
\item[B.] Gastric polyp
\item[C.] Pyloric stenosis
\item[D.] Gastric cancer
\item[E.] Peptic ulcer disease
\end{enumerate}

\textbf{Image:}
\begin{center}
\includegraphics[width=0.79\textwidth,height=0.50\textheight,width=0.90\textwidth,keepaspectratio]{images/nejm_20161229.jpg}
\end{center}
\vspace{12pt}
\newpage

\section*{Question 728 (ID: 20170105)}
\textbf{Date: }January 05,2017
\vspace{6pt}

A 68-year-old woman presented with unstable angina and underwent cardiac catheterization by radial access. There was difficulty advancing the guidewire; a brachial angiogram revealed the following. What is the diagnosis?
\vspace{12pt}

\textbf{Options:}
\begin{enumerate}
\item[A.] Takayasu’s arteritis
\item[B.] Arteriovenous malformation
\item[C.] Brachial artery stenosis
\item[D.] Fibromuscular dysplasia
\item[E.] Atherosclerotic peripheral artery disease
\end{enumerate}

\textbf{Image:}
\begin{center}
\includegraphics[width=0.95\textwidth,height=0.50\textheight,width=0.90\textwidth,keepaspectratio]{images/nejm_20170105.jpg}
\end{center}
\vspace{12pt}
\newpage

\section*{Question 729 (ID: 20170112)}
\textbf{Date: }January 12,2017
\vspace{6pt}

A 56-year-old man with a history of Roux-en-Y gastric bypass presented with abdominal pain, nausea, and bilious emesis. What is the diagnosis?
\vspace{12pt}

\textbf{Options:}
\begin{enumerate}
\item[A.] Anastamotic stricture
\item[B.] Incisional hernia
\item[C.] Intussusception
\item[D.] Dumping syndrome
\item[E.] Internal hernia
\end{enumerate}

\textbf{Image:}
\begin{center}
\includegraphics[width=0.82\textwidth,height=0.50\textheight,width=0.90\textwidth,keepaspectratio]{images/nejm_20170112.jpg}
\end{center}
\vspace{12pt}
\newpage

\section*{Question 730 (ID: 20170119)}
\textbf{Date: }January 19,2017
\vspace{6pt}

A 72-year-old woman presented with skin changes over her left breast. She had a history of breast cancer 5 years earlier and was treated with a lumpectomy and radiation therapy. What is the diagnosis?
\vspace{12pt}

\textbf{Options:}
\begin{enumerate}
\item[A.] Inflammatory breast cancer
\item[B.] Mastitis
\item[C.] Radiation-induced angiosarcoma
\item[D.] Fat necrosis
\item[E.] Mondor’s disease
\end{enumerate}

\textbf{Image:}
\begin{center}
\includegraphics[width=0.95\textwidth,height=0.50\textheight,width=0.90\textwidth,keepaspectratio]{images/nejm_20170119.jpg}
\end{center}
\vspace{12pt}
\newpage

\section*{Question 731 (ID: 20170126)}
\textbf{Date: }January 26,2017
\vspace{6pt}

A 50-year-old woman presented with an 18-month history of asymptomatic skin lesions on her forehead and the dorsum of her hands. What is the diagnosis?
\vspace{12pt}

\textbf{Options:}
\begin{enumerate}
\item[A.] Necrobiosis lipoidica
\item[B.] Actinic keratosis
\item[C.] Tinea capitis
\item[D.] Actinic granuloma
\item[E.] Cutaneous sarcoidosis
\end{enumerate}

\textbf{Image:}
\begin{center}
\includegraphics[width=0.95\textwidth,height=0.50\textheight,width=0.90\textwidth,keepaspectratio]{images/nejm_20170126.jpg}
\end{center}
\vspace{12pt}
\newpage

\section*{Question 732 (ID: 20170202)}
\textbf{Date: }February 02,2017
\vspace{6pt}

A 15-year-old boy presented with decreased vision in both eyes, and a slit lamp examination revealed this finding. What is the diagnosis?
\vspace{12pt}

\textbf{Options:}
\begin{enumerate}
\item[A.] Juvenile xanthogranuloma
\item[B.] Persistent hyperplastic primary vitreous
\item[C.] Protruding iris collarette
\item[D.] Iris duplication
\item[E.] Persistent pupillary membranes
\end{enumerate}

\textbf{Image:}
\begin{center}
\includegraphics[width=0.95\textwidth,height=0.50\textheight,width=0.90\textwidth,keepaspectratio]{images/nejm_20170202.jpg}
\end{center}
\vspace{12pt}
\newpage

\section*{Question 733 (ID: 20170209)}
\textbf{Date: }February 09,2017
\vspace{6pt}

A 60-year-old man with atrial fibrillation presented with chest pain and numbness in both arms, then collapsed. This image from his CT scan shows air in which anatomical space?
\vspace{12pt}

\textbf{Options:}
\begin{enumerate}
\item[A.] Left ventricle
\item[B.] Right ventricle
\item[C.] Right atrium
\item[D.] Left atrium
\item[E.] Pericardial space
\end{enumerate}

\textbf{Image:}
\begin{center}
\includegraphics[width=0.92\textwidth,height=0.50\textheight,width=0.90\textwidth,keepaspectratio]{images/nejm_20170209.jpg}
\end{center}
\vspace{12pt}
\newpage

\section*{Question 734 (ID: 20170216)}
\textbf{Date: }February 16,2017
\vspace{6pt}

A previously healthy 5-year-old boy presented with a 4-day history of nausea, vomiting, and intermittent abdominal pain. On examination he had mild periorbital edema. A computed tomography scan with contrast demonstrated these findings. What is the diagnosis?
\vspace{12pt}

\textbf{Options:}
\begin{enumerate}
\item[A.] Hyperplastic polyps
\item[B.] Reflux gastropathy
\item[C.] Zollinger-Ellison syndrome
\item[D.] Ménétrier’s disease
\item[E.] Cronkite-Canada syndrome
\end{enumerate}

\textbf{Image:}
\begin{center}
\includegraphics[width=0.95\textwidth,height=0.50\textheight,width=0.90\textwidth,keepaspectratio]{images/nejm_20170216.jpg}
\end{center}
\vspace{12pt}
\newpage

\section*{Question 735 (ID: 20170223)}
\textbf{Date: }February 23,2017
\vspace{6pt}

A 67-year-old man with a substantial cardiac history and renal disease presented to the emergency department with neck swelling, dysphagia, and hoarseness. Computed tomographic scans of his neck showed these findings. What is the diagnosis?
\vspace{12pt}

\textbf{Options:}
\begin{enumerate}
\item[A.] Acute bacterial sialadenitis
\item[C.] Iodide-associated sialadenitis
\item[D.] Sialolithiasis
\item[E.] Mikulicz syndrome
\end{enumerate}

\textbf{Image:}
\begin{center}
\includegraphics[width=0.74\textwidth,height=0.50\textheight,width=0.90\textwidth,keepaspectratio]{images/nejm_20170223.jpg}
\end{center}
\vspace{12pt}
\newpage

\section*{Question 736 (ID: 20170302)}
\textbf{Date: }March 02,2017
\vspace{6pt}

A 65-year-old man presented with a recent history of weight loss, early satiety, mild flank pain, and this rash on his legs. CT scan revealed a mass in the head of the pancreas. Which neuroendocrine tumor is associated with this characteristic rash?
\vspace{12pt}

\textbf{Options:}
\begin{enumerate}
\item[A.] Insulinoma
\item[B.] Gastrinoma
\item[C.] Glucagonoma
\item[D.] Vasoactive intestinal peptide-secreting tumor (vipoma)
\item[E.] Somatostatinoma
\end{enumerate}

\textbf{Image:}
\begin{center}
\includegraphics[width=0.62\textwidth,height=0.50\textheight,width=0.90\textwidth,keepaspectratio]{images/nejm_20170302.jpg}
\end{center}
\vspace{12pt}
\newpage

\section*{Question 737 (ID: 20170309)}
\textbf{Date: }March 09,2017
\vspace{6pt}

A 37-year-old woman presented with symptoms consistent with allergic conjunctivitis, but was found to have this incidental finding. What is the diagnosis?
\vspace{12pt}

\textbf{Options:}
\begin{enumerate}
\item[A.] Aniridia
\item[B.] Prominent iris collarette
\item[C.] Coloboma of the iris
\item[D.] Granulomatous iritis
\item[E.] Persistent pupillary membranes
\end{enumerate}

\textbf{Image:}
\begin{center}
\includegraphics[width=0.95\textwidth,height=0.50\textheight,width=0.90\textwidth,keepaspectratio]{images/nejm_20170309.jpg}
\end{center}
\vspace{12pt}
\newpage

\section*{Question 738 (ID: 20170316)}
\textbf{Date: }March 16,2017
\vspace{6pt}

An 86-year-old woman with hypertension presents with epigastric pain and does not report any respiratory symptoms. Her chest radiograph demonstrates these findings. What is the diagnosis?
\vspace{12pt}

\textbf{Options:}
\begin{enumerate}
\item[A.] Oleothorax
\item[B.] Hemothorax
\item[C.] Calcified empyema
\item[D.] Non-small-cell lung cancer
\item[E.] Asbestosis
\end{enumerate}

\textbf{Image:}
\begin{center}
\includegraphics[width=0.72\textwidth,height=0.50\textheight,width=0.90\textwidth,keepaspectratio]{images/nejm_20170316.jpg}
\end{center}
\vspace{12pt}
\newpage

\section*{Question 739 (ID: 20170323)}
\textbf{Date: }March 23,2017
\vspace{6pt}

A 46-year-old woman who had recently returned from a trip to the Ivory Coast presented to the emergency department with a 10-day history of pain and swelling in her right arm. What is the diagnosis?
\vspace{12pt}

\textbf{Options:}
\begin{enumerate}
\item[A.] Spotted flesh fly
\item[B.] Tumbu fly
\item[C.] Lund fly
\item[D.] Human botfly
\item[E.] Lesser house fly
\end{enumerate}

\textbf{Image:}
\begin{center}
\includegraphics[width=0.47\textwidth,height=0.50\textheight,width=0.90\textwidth,keepaspectratio]{images/nejm_20170323.jpg}
\end{center}
\vspace{12pt}
\newpage

\section*{Question 740 (ID: 20170330)}
\textbf{Date: }March 30,2017
\vspace{6pt}

A 29-year-old woman presented 4 months after an uncomplicated vaginal delivery with a non-tender mass protruding from the vaginal canal. What is the diagnosis?
\vspace{12pt}

\textbf{Options:}
\begin{enumerate}
\item[A.] Skene duct cyst
\item[B.] Urethral diverticulum
\item[C.] Bartholin gland cyst
\item[D.] Nabothian cyst
\item[E.] Gartner duct cyst
\end{enumerate}

\textbf{Image:}
\begin{center}
\includegraphics[width=0.95\textwidth,height=0.50\textheight,width=0.90\textwidth,keepaspectratio]{images/nejm_20170330.jpg}
\end{center}
\vspace{12pt}
\newpage

\section*{Question 741 (ID: 20170406)}
\textbf{Date: }April 06,2017
\vspace{6pt}

A 3-year-old boy presented with a 6-month history of gross hematuria, and was found to have kidney, bladder, and ureteral stones. Which of the following factors predisposed this boy to genitourinary stone formation?
\vspace{12pt}

\textbf{Options:}
\begin{enumerate}
\item[A.] Phenylketonuria
\item[B.] Ornithine transcarbamylase deficiency
\item[C.] Primary hyperoxaluria
\item[D.] Acid maltase deficiency (Pompe disease)
\item[E.] Pyruvate kinase deficiency
\end{enumerate}

\textbf{Image:}
\begin{center}
\includegraphics[width=0.55\textwidth,height=0.50\textheight,width=0.90\textwidth,keepaspectratio]{images/nejm_20170406.jpg}
\end{center}
\vspace{12pt}
\newpage

\section*{Question 742 (ID: 20170413)}
\textbf{Date: }April 13,2017
\vspace{6pt}

A 6-year-old girl presented with short stature, impaired hearing, and vision loss at 2 years of age. A radiograph of her wrists show the following features. What is the diagnosis?
\vspace{12pt}

\textbf{Options:}
\begin{enumerate}
\item[A.] Lead poisoning
\item[B.] Osteopetrosis
\item[C.] Fibrous dysplasia
\item[D.] Hypoparathyroidism
\item[E.] Hypervitaminosis D
\end{enumerate}

\textbf{Image:}
\begin{center}
\includegraphics[width=0.67\textwidth,height=0.50\textheight,width=0.90\textwidth,keepaspectratio]{images/nejm_20170413.jpg}
\end{center}
\vspace{12pt}
\newpage

\section*{Question 743 (ID: 20170420)}
\textbf{Date: }April 20,2017
\vspace{6pt}

A 30-year-old man presented with this lesion on his right lower eyelid that had grown over 3 days. Two weeks before his presentation, a cyst had ruptured on the same eyelid. What is the diagnosis?
\vspace{12pt}

\textbf{Options:}
\begin{enumerate}
\item[A.] Squamous papilloma
\item[B.] Amelanotic melanoma
\item[C.] Squamous-cell carcinoma
\item[D.] Pyogenic granuloma
\item[E.] Suture granuloma
\end{enumerate}

\textbf{Image:}
\begin{center}
\includegraphics[width=0.63\textwidth,height=0.50\textheight,width=0.90\textwidth,keepaspectratio]{images/nejm_20170420.jpg}
\end{center}
\vspace{12pt}
\newpage

\section*{Question 744 (ID: 20170427)}
\textbf{Date: }April 27,2017
\vspace{6pt}

A 25-year-old man presented to the emergency department with severe neck pain after a high-speed motor vehicle collision resulted in neck hyperextension. Axial CT scan of the C2 vertebrae is shown here. What is the diagnosis?
\vspace{12pt}

\textbf{Options:}
\begin{enumerate}
\item[A.] Traumatic spondylolisthesis of the axis (hangman’s fracture)
\item[B.] Odontoid fracture (dens fracture)
\item[C.] Cervical spinous process fracture (Clay-shoveler fracture)
\item[D.] Flexion tear drop fracture
\item[E.] Cervical burst fracture (Jefferson fracture)
\end{enumerate}

\textbf{Image:}
\begin{center}
\includegraphics[width=0.85\textwidth,height=0.50\textheight,width=0.90\textwidth,keepaspectratio]{images/nejm_20170427.jpg}
\end{center}
\vspace{12pt}
\newpage

\section*{Question 745 (ID: 20170504)}
\textbf{Date: }May 04,2017
\vspace{6pt}

An 80-year-old man presented with this lesion on the bottom of his left foot, which had been growing for the past 2 years. It had become ulcerated and bled intermittently. What is the diagnosis?
\vspace{12pt}

\textbf{Options:}
\begin{enumerate}
\item[A.] Microvenular hemangioma
\item[B.] Benign lymphangioendothelioma
\item[C.] Amelanotic melanoma
\item[D.] Pyogenic granuloma
\item[E.] Bacillary angiomatosis
\end{enumerate}

\textbf{Image:}
\begin{center}
\includegraphics[width=0.79\textwidth,height=0.50\textheight,width=0.90\textwidth,keepaspectratio]{images/nejm_20170504.jpg}
\end{center}
\vspace{12pt}
\newpage

\section*{Question 746 (ID: 20170511)}
\textbf{Date: }May 11,2017
\vspace{6pt}

A 71-year-old man presented to the nephrology department for the evaluation of chronic kidney disease. He had erythematous papules on his face and these lesions on his toenails. What is the diagnosis?
\vspace{12pt}

\textbf{Options:}
\begin{enumerate}
\item[A.] Squamous-cell carcinoma
\item[B.] Verrucae vulgaris
\item[C.] Cutaneous horn
\item[D.] Pyogenic granuloma
\item[E.] Periungual fibroma
\end{enumerate}

\textbf{Image:}
\begin{center}
\includegraphics[width=0.78\textwidth,height=0.50\textheight,width=0.90\textwidth,keepaspectratio]{images/nejm_20170511.jpg}
\end{center}
\vspace{12pt}
\newpage

\section*{Question 747 (ID: 20170518)}
\textbf{Date: }May 18,2017
\vspace{6pt}

A 69-year-old man presented with progressive neutropenia and anemia and this result of his bone marrow aspiration. What is the diagnosis?
\vspace{12pt}

\textbf{Options:}
\begin{enumerate}
\item[A.] Acute promyelocytic leukemia (APL)
\item[B.] Chronic lymphocytic leukemia (CLL)
\item[C.] Multiple myeloma (MM)
\item[D.] Acute lymphocytic leukemia (ALL)
\item[E.] Chronic myeloid leukemia (CML)
\end{enumerate}

\textbf{Image:}
\begin{center}
\includegraphics[width=0.95\textwidth,height=0.50\textheight,width=0.90\textwidth,keepaspectratio]{images/nejm_20170518.jpg}
\end{center}
\vspace{12pt}
\newpage

\section*{Question 748 (ID: 20170525)}
\textbf{Date: }May 25,2017
\vspace{6pt}

A 3-year-old boy presented with fever, vomiting, and these skin lesions. He was found to have bacteremia with Neisseria meningitidis. What is the diagnosis?
\vspace{12pt}

\textbf{Options:}
\begin{enumerate}
\item[A.] Vitamin C deficiency
\item[B.] Drug-induced vasculitis
\item[C.] Purpura fulminans
\item[D.] Henloch-Schonlein purpura
\item[E.] Idiopathic thrombocytopenic purpura
\end{enumerate}

\textbf{Image:}
\begin{center}
\includegraphics[width=0.56\textwidth,height=0.50\textheight,width=0.90\textwidth,keepaspectratio]{images/nejm_20170525.jpg}
\end{center}
\vspace{12pt}
\newpage

\section*{Question 749 (ID: 20170601)}
\textbf{Date: }June 01,2017
\vspace{6pt}

A 53-year-old man presented with dull, aching pain in his right nipple. What is the diagnosis?
\vspace{12pt}

\textbf{Options:}
\begin{enumerate}
\item[A.] Trousseau’s syndrome
\item[B.] Mondor’s disease
\item[C.] Bancroft’s Sign
\item[D.] May-Thurner syndrome
\item[E.] Paget-Schroetter disease
\end{enumerate}

\textbf{Image:}
\begin{center}
\includegraphics[width=0.73\textwidth,height=0.50\textheight,width=0.90\textwidth,keepaspectratio]{images/nejm_20170601.jpg}
\end{center}
\vspace{12pt}
\newpage

\section*{Question 750 (ID: 20170608)}
\textbf{Date: }June 08,2017
\vspace{6pt}

A 47-year-old Laotian man with a history of dermatomyositis presented with a 4-day history of abdominal pain, fever, melena, and hemoptysis and a 2-week history of rash. A punch biopsy of the rash revealed the findings seen here. What is the diagnosis?
\vspace{12pt}

\textbf{Options:}
\begin{enumerate}
\item[A.] Taenia solium
\item[B.] Entamoeba histolytica
\item[C.] Capillaria
\item[D.] Trichuris trichuria
\item[E.] Strongyloides stercoralis
\end{enumerate}

\textbf{Image:}
\begin{center}
\includegraphics[width=0.84\textwidth,height=0.50\textheight,width=0.90\textwidth,keepaspectratio]{images/nejm_20170608.jpg}
\end{center}
\vspace{12pt}
\newpage

\section*{Question 751 (ID: 20170615)}
\textbf{Date: }June 15,2017
\vspace{6pt}

A 69-year-old woman with chronic kidney disease and myelofibrosis presented with a 2-day history of dyspnea and general weakness. Magnetic resonance imaging of the abdomen is shown here. What is the diagnosis?
\vspace{12pt}

\textbf{Options:}
\begin{enumerate}
\item[A.] Renal-cell carcinoma
\item[B.] Amyloidosis
\item[C.] Angiomylipoma
\item[D.] Perirenal extramedullary hematopoiesis
\item[E.] Liposarcoma
\end{enumerate}

\textbf{Image:}
\begin{center}
\includegraphics[width=0.77\textwidth,height=0.50\textheight,width=0.90\textwidth,keepaspectratio]{images/nejm_20170615.jpg}
\end{center}
\vspace{12pt}
\newpage

\section*{Question 752 (ID: 20170622)}
\textbf{Date: }June 22,2017
\vspace{6pt}

A healthy 66-year-old nonsmoking woman presented with recurrent pain and bruising in her right middle finger. She denied trauma or exposure to cold, and had normal complete blood count and coagulation studies. What is the diagnosis?
\vspace{12pt}

\textbf{Options:}
\begin{enumerate}
\item[A.] Ehler’s-Danlos syndrome
\item[B.] Achenbach’s syndrome
\item[C.] Buerger disease
\item[D.] Raynaud syndrome
\item[E.] Von Willebrand disease
\end{enumerate}

\textbf{Image:}
\begin{center}
\includegraphics[width=0.72\textwidth,height=0.50\textheight,width=0.90\textwidth,keepaspectratio]{images/nejm_20170622.jpg}
\end{center}
\vspace{12pt}
\newpage

\section*{Question 753 (ID: 20170629)}
\textbf{Date: }June 29,2017
\vspace{6pt}

A 59-year-old previously healthy man presented with progressively worsening headaches and bluish nodular skin lesions. Fast-field echo MRI image of the brain showed this finding. What is the diagnosis?
\vspace{12pt}

\textbf{Options:}
\begin{enumerate}
\item[A.] Cerebral cavernous malformations
\item[B.] Infectious cerebral emboli
\item[C.] Hemorrhagic cerebral metastasis
\item[D.] Cerebral vasculitis
\item[E.] Cerebral air embolisms
\end{enumerate}

\textbf{Image:}
\begin{center}
\includegraphics[width=0.67\textwidth,height=0.50\textheight,width=0.90\textwidth,keepaspectratio]{images/nejm_20170629.jpg}
\end{center}
\vspace{12pt}
\newpage

\section*{Question 754 (ID: 20170706)}
\textbf{Date: }July 06,2017
\vspace{6pt}

A 31-year-old man with an 8-month history of shortness of breath and skin changes presented to the emergency department. What is the diagnosis?
\vspace{12pt}

\textbf{Options:}
\begin{enumerate}
\item[A.] Systemic lupus erythematosis
\item[B.] Vitiligo
\item[C.] Porphyria cutanea tarda
\item[D.] Scleroderma
\item[E.] Radiation exposure
\end{enumerate}

\textbf{Image:}
\begin{center}
\includegraphics[width=0.42\textwidth,height=0.50\textheight,width=0.90\textwidth,keepaspectratio]{images/nejm_20170706.jpg}
\end{center}
\vspace{12pt}
\newpage

\section*{Question 755 (ID: 20170713)}
\textbf{Date: }July 13,2017
\vspace{6pt}

A 25-year-old woman who had undergone an orthotopic heart transplantation presented with these changes on both feet. What is the diagnosis?
\vspace{12pt}

\textbf{Options:}
\begin{enumerate}
\item[A.] Helomata
\item[B.] Calluses
\item[C.] Lichen planus
\item[D.] Plantar warts
\item[E.] Seborrhoeic keratosis
\end{enumerate}

\textbf{Image:}
\begin{center}
\includegraphics[width=0.87\textwidth,height=0.50\textheight,width=0.90\textwidth,keepaspectratio]{images/nejm_20170713.jpg}
\end{center}
\vspace{12pt}
\newpage

\section*{Question 756 (ID: 20170720)}
\textbf{Date: }July 20,2017
\vspace{6pt}

A 28-year-old woman with vertigo, confusion, and falls 2 weeks after a surgical abortion at 11 weeks of gestation presents to the emergency department. Examination revealed spontaneous upbeat nystagmus, gaze-evoked nystagmus, and gait ataxia. What is the diagnosis?
\vspace{12pt}

\textbf{Options:}
\begin{enumerate}
\item[A.] Neuro-Behçet disease
\item[B.] Multiple sclerosis
\item[C.] Wernicke’s encephalopathy
\item[D.] Deep cerebral venous thrombosis
\item[E.] Osmotic myelinolysis
\end{enumerate}

\textbf{Image:}
\begin{center}
\includegraphics[width=0.66\textwidth,height=0.50\textheight,width=0.90\textwidth,keepaspectratio]{images/nejm_20170720.jpg}
\end{center}
\vspace{12pt}
\newpage

\section*{Question 757 (ID: 20170727)}
\textbf{Date: }July 27,2017
\vspace{6pt}

A 71-year-old man with cirrhosis presented to the ED with severe diffuse pruritus 3 months after being treated with oral and topical glucocorticoids for an erythematous, scaly rash. Physical exam revealed this rash on the patient’s neck, hands, and genital area. What is the most likely diagnosis?
\vspace{12pt}

\textbf{Options:}
\begin{enumerate}
\item[A.] Crusted scabies
\item[B.] Psoriasis
\item[C.] Seborrheic dermatitis
\item[D.] Tinea corporis
\item[E.] Candidiasis
\end{enumerate}

\textbf{Image:}
\begin{center}
\includegraphics[width=0.72\textwidth,height=0.50\textheight,width=0.90\textwidth,keepaspectratio]{images/nejm_20170727.jpg}
\end{center}
\vspace{12pt}
\newpage

\section*{Question 758 (ID: 20170803)}
\textbf{Date: }August 03,2017
\vspace{6pt}

What is the most likely diagnosis for these two painless periumbilical masses in a 76-year-old man with type 2 diabetes mellitus?
\vspace{12pt}

\textbf{Options:}
\begin{enumerate}
\item[A.] Abscess
\item[B.] Dermatofibrosarcoma protuberans
\item[C.] Hematoma
\item[D.] Insulin-induced lipohypertrophy
\item[E.] Nodular fasciitis
\end{enumerate}

\textbf{Image:}
\begin{center}
\includegraphics[width=0.95\textwidth,height=0.50\textheight,width=0.90\textwidth,keepaspectratio]{images/nejm_20170803.jpg}
\end{center}
\vspace{12pt}
\newpage

\section*{Question 759 (ID: 20170810)}
\textbf{Date: }August 10,2017
\vspace{6pt}

An 8-year-old girl with atopic dermatitis presented with a 3-day history of diffuse eruption of pruritic, umbilicated, erythematous vesicles with erosion and crusting on the flexor surfaces of the legs and arms. What is the diagnosis?
\vspace{12pt}

\textbf{Options:}
\begin{enumerate}
\item[A.] Impetigo
\item[B.] Eczema herpeticum
\item[C.] Contact dermatitis
\item[D.] Drug eruption
\item[E.] Dermatitis herpetiformis
\end{enumerate}

\textbf{Image:}
\begin{center}
\includegraphics[width=0.95\textwidth,height=0.50\textheight,width=0.90\textwidth,keepaspectratio]{images/nejm_20170810.jpg}
\end{center}
\vspace{12pt}
\newpage

\section*{Question 760 (ID: 20170817)}
\textbf{Date: }August 17,2017
\vspace{6pt}

A 52-year-old woman with rheumatoid arthritis, treated with sulfasalazine and azathioprine, presented with multiple painful, erythematous plaques on her palms and fingers. What is the diagnosis?
\vspace{12pt}

\textbf{Options:}
\begin{enumerate}
\item[A.] Sweet’s syndrome
\item[B.] Erythema multiforme
\item[C.] Drug eruption
\item[D.] Secondary syphilis
\item[E.] Granulomatosis with polyangitis
\end{enumerate}

\textbf{Image:}
\begin{center}
\includegraphics[width=0.65\textwidth,height=0.50\textheight,width=0.90\textwidth,keepaspectratio]{images/nejm_20170817.jpg}
\end{center}
\vspace{12pt}
\newpage

\section*{Question 761 (ID: 20170824)}
\textbf{Date: }August 24,2017
\vspace{6pt}

A 43-year-old woman had an 8-month history of non-productive cough, unresponsive to antibiotic treatment. Physical exam showed bilateral wheezing, and pulmonary function tests showed obstructive disease, unresponsive to bronchodilators. Bronchoscopy revealed the following. What is the most likely diagnosis?
\vspace{12pt}

\textbf{Options:}
\begin{enumerate}
\item[A.] Bronchial polyposis
\item[B.] Bronchial tuberculosis
\item[C.] Pulmonary sarcoidosis
\item[D.] Tracheal inhalation injury
\item[E.] Eosinophilic bronchitis
\end{enumerate}

\textbf{Image:}
\begin{center}
\includegraphics[width=0.79\textwidth,height=0.50\textheight,width=0.90\textwidth,keepaspectratio]{images/nejm_20170824.jpg}
\end{center}
\vspace{12pt}
\newpage

\section*{Question 762 (ID: 20170831)}
\textbf{Date: }August 31,2017
\vspace{6pt}

A 64-year-old man with a history of bladder cancer treated with radical cystectomy and orthotopic neobladder construction with intestinal segments presented to the emergency department with left flank pain and urinary retention. CT of the abdomen and pelvis showed a very large stone in the neobladder. What is the predominant composition of this patient’s stone?
\vspace{12pt}

\textbf{Options:}
\begin{enumerate}
\item[A.] Struvite
\item[B.] Uric acid
\item[C.] Calcium oxalate
\item[D.] Calcium phosphate
\item[E.] Cystine
\end{enumerate}

\textbf{Image:}
\begin{center}
\includegraphics[width=0.71\textwidth,height=0.50\textheight,width=0.90\textwidth,keepaspectratio]{images/nejm_20170831.jpg}
\end{center}
\vspace{12pt}
\newpage

\section*{Question 763 (ID: 20170907)}
\textbf{Date: }September 07,2017
\vspace{6pt}

A 41-year-old man presented with progressive loss of vision in both eyes. Physical examination was notable for pectus excavatum and elongated digits. His uncorrected visual acuity was 6/200 in the right eye and 20/100 in the left eye. What is the diagnosis?
\vspace{12pt}

\textbf{Options:}
\begin{enumerate}
\item[A.] Pseudoexfoliation glaucoma
\item[B.] Retinal detachment
\item[C.] Ectopia lentis
\item[D.] Traumatic cataract
\item[E.] Iridocyclitis
\end{enumerate}

\textbf{Image:}
\begin{center}
\includegraphics[width=0.95\textwidth,height=0.50\textheight,width=0.90\textwidth,keepaspectratio]{images/nejm_20170907.jpg}
\end{center}
\vspace{12pt}
\newpage

\section*{Question 764 (ID: 20170914)}
\textbf{Date: }September 14,2017
\vspace{6pt}

A 17-year-old boy from rural Mexico presented to the hospital with a 3-week history of decreased visual acuity and pain in the right eye. What is the diagnosis?
\vspace{12pt}

\textbf{Options:}
\begin{enumerate}
\item[A.] Intraocular parasite infection
\item[B.] Bacterial uveitis
\item[C.] Viral keratitis
\item[D.] Angle closure glaucoma
\item[E.] Optic neuritis
\end{enumerate}

\textbf{Image:}
\begin{center}
\includegraphics[width=0.95\textwidth,height=0.50\textheight,width=0.90\textwidth,keepaspectratio]{images/nejm_20170914.jpg}
\end{center}
\vspace{12pt}
\newpage

\section*{Question 765 (ID: 20170921)}
\textbf{Date: }September 21,2017
\vspace{6pt}

A 35-year-old man presented with sharp neck pain and a sensation of having a foreign body in his throat over the past year. What is the most likely diagnosis?
\vspace{12pt}

\textbf{Options:}
\begin{enumerate}
\item[A.] Cervical spondyloarthropathy
\item[B.] Oropharyngeal fish bone foreign body
\item[C.] Temporomandibular joint dysfunction
\item[D.] Ernest syndrome
\item[E.] Eagle’s syndrome
\end{enumerate}

\textbf{Image:}
\begin{center}
\includegraphics[width=0.79\textwidth,height=0.50\textheight,width=0.90\textwidth,keepaspectratio]{images/nejm_20170921.jpg}
\end{center}
\vspace{12pt}
\newpage

\section*{Question 766 (ID: 20170928)}
\textbf{Date: }September 28,2017
\vspace{6pt}

A 63-year-old woman with type 2 diabetes presented with a 2-day history of hematemesis and black stool. Esophagogastroduodenoscopy revealed longitudinal, black lesions in the lower esophagus. What is the diagnosis?
\vspace{12pt}

\textbf{Options:}
\begin{enumerate}
\item[A.] Malignant melanoma
\item[B.] Esophageal acanthosis nigricans
\item[C.] Acute esophageal necrosis
\item[D.] Esophageal melanocytosis
\item[E.] Coal dust deposition
\end{enumerate}

\textbf{Image:}
\begin{center}
\includegraphics[width=0.95\textwidth,height=0.50\textheight,width=0.90\textwidth,keepaspectratio]{images/nejm_20170928.jpg}
\end{center}
\vspace{12pt}
\newpage

\section*{Question 767 (ID: 20171005)}
\textbf{Date: }October 05,2017
\vspace{6pt}

A 59-year-old man with a remote history of melanoma presented to the dermatology clinic. He presented with lingual papillomas for many years, as well as multiple trichilemmomas on the face and chest, acral keratosis on the hands and feet, and macrocephaly. He also had a history of gastrointestinal polyps and a multinodular goiter. What is the diagnosis?
\vspace{12pt}

\textbf{Options:}
\begin{enumerate}
\item[A.] Birt-Hogg Dube syndrome
\item[B.] Cowden syndrome
\item[C.] Tuberous sclerosis
\item[D.] Peutz-Jeghers syndrome
\item[E.] Hereditary mixed polyposis syndrome
\end{enumerate}

\textbf{Image:}
\begin{center}
\includegraphics[width=0.62\textwidth,height=0.50\textheight,width=0.90\textwidth,keepaspectratio]{images/nejm_20171005.jpg}
\end{center}
\vspace{12pt}
\newpage

\section*{Question 768 (ID: 20171012)}
\textbf{Date: }October 12,2017
\vspace{6pt}

A 61-year-old woman with hypertension and hypothyroidism presented with acute onset of severe chest pain after the death of her dog. Electrocardiography showed ST-segment elevation in anterolateral leads, and left ventriculography showed severe apical hypokinesis. What is the likely diagnosis?
\vspace{12pt}

\textbf{Options:}
\begin{enumerate}
\item[A.] Acute coronary syndrome
\item[B.] Myocardial infarction
\item[C.] Ventricular aneurysm
\item[D.] Thyrotoxic cardiomyopathy
\item[E.] Takotsubo cardiomyopathy
\end{enumerate}

\textbf{Image:}
\begin{center}
\includegraphics[width=0.85\textwidth,height=0.50\textheight,width=0.90\textwidth,keepaspectratio]{images/nejm_20171012.jpg}
\end{center}
\vspace{12pt}
\newpage

\section*{Question 769 (ID: 20171019)}
\textbf{Date: }October 19,2017
\vspace{6pt}

An 11-year-old boy was brought to the emergency department with epistaxis and severe nasal pain. What is the likely diagnosis?
\vspace{12pt}

\textbf{Options:}
\begin{enumerate}
\item[A.] Nasal polyps
\item[B.] Nasal septum fracture
\item[C.] Pills in the nasal cavity
\item[D.] Coins in the nasal cavity
\item[E.] Button magnets in the nasal cavity
\end{enumerate}

\textbf{Image:}
\begin{center}
\includegraphics[width=0.9\textwidth,height=0.50\textheight,width=0.90\textwidth,keepaspectratio]{images/nejm_20171019.jpg}
\end{center}
\vspace{12pt}
\newpage

\section*{Question 770 (ID: 20171026)}
\textbf{Date: }October 26,2017
\vspace{6pt}

A 62-year-old woman presented to the breast-surgery clinic with stiffness of the left nipple. Palpation of the left breast, mammography, and ultrasonography revealed no abnormalities. A pigmented macule that measured 5 mm was noted on the left areola. What is the diagnosis?
\vspace{12pt}

\textbf{Options:}
\begin{enumerate}
\item[A.] Inflammatory mastitis
\item[B.] Traumatic fat necrosis
\item[C.] Cutaneous malignant melanoma
\item[D.] Invasive ductal carcinoma
\item[E.] Blue nevus
\end{enumerate}

\textbf{Image:}
\begin{center}
\includegraphics[width=0.62\textwidth,height=0.50\textheight,width=0.90\textwidth,keepaspectratio]{images/nejm_20171026.jpg}
\end{center}
\vspace{12pt}
\newpage

\section*{Question 771 (ID: 20171102)}
\textbf{Date: }November 02,2017
\vspace{6pt}

A 58-year-old woman presented to the rheumatology clinic with long-standing joint pain and deformities of both hands and feet. What material has been used in acupuncture threading techniques for joint pain?
\vspace{12pt}

\textbf{Options:}
\begin{enumerate}
\item[A.] Copper
\item[C.] Titanium
\item[D.] Cobalt
\item[E.] Silver nitrate
\end{enumerate}

\textbf{Image:}
\begin{center}
\includegraphics[width=0.95\textwidth,height=0.50\textheight,width=0.90\textwidth,keepaspectratio]{images/nejm_20171102.jpg}
\end{center}
\vspace{12pt}
\newpage

\section*{Question 772 (ID: 20171109)}
\textbf{Date: }November 09,2017
\vspace{6pt}

An 87-year-old woman with a history of peptic ulcer disease presented with severe abdominal pain. An examination showed hypotension and diffuse abdominal tenderness. Computed tomography of the abdomen revealed a classic sign of pneumoperitoneum outlining which ligament?
\vspace{12pt}

\textbf{Options:}
\begin{enumerate}
\item[A.] Round ligament of the liver
\item[B.] Gastrohepatic ligament
\item[C.] Ligament of Trietz
\item[D.] Falciform ligament
\item[E.] Hepatoduodenal ligament
\end{enumerate}

\textbf{Image:}
\begin{center}
\includegraphics[width=0.95\textwidth,height=0.50\textheight,width=0.90\textwidth,keepaspectratio]{images/nejm_20171109.jpg}
\end{center}
\vspace{12pt}
\newpage

\section*{Question 773 (ID: 20171116)}
\textbf{Date: }November 16,2017
\vspace{6pt}

A 7-year-old boy with a history of atopic dermatitis presented with a 3-month history of widespread, mildly pruritic papular skin rash. What is the likely diagnosis?
\vspace{12pt}

\textbf{Options:}
\begin{enumerate}
\item[A.] Molluscum contagiosum
\item[B.] Pityriasis rosea
\item[C.] Papular acrodermatitis
\item[D.] Varicella zoster
\item[E.] Neurofibromatosis
\end{enumerate}

\textbf{Image:}
\begin{center}
\includegraphics[width=0.73\textwidth,height=0.50\textheight,width=0.90\textwidth,keepaspectratio]{images/nejm_20171116.jpg}
\end{center}
\vspace{12pt}
\newpage

\section*{Question 774 (ID: 20171123)}
\textbf{Date: }November 23,2017
\vspace{6pt}

A 67-year-old man presented with progressive dysphagia, weight loss, regurgitation, and halitosis. A barium swallow examination showed stasis of barium in the upper esophagus with an outpouching lesion anterior to the C5 and C6 vertebrae. Upper gastrointestinal endoscopy revealed a Zenker’s diverticulum. These diverticula pass through which muscular defect?
\vspace{12pt}

\textbf{Options:}
\begin{enumerate}
\item[A.] Thyrohyoid membrane
\item[B.] Killian’s triangle
\item[C.] Laimer’s triangle
\item[D.] Killian-Jamieson’s triangle
\item[E.] Cricothyroid membrane
\end{enumerate}

\textbf{Image:}
\begin{center}
\includegraphics[width=0.58\textwidth,height=0.50\textheight,width=0.90\textwidth,keepaspectratio]{images/nejm_20171123.jpg}
\end{center}
\vspace{12pt}
\newpage

\section*{Question 775 (ID: 20171130)}
\textbf{Date: }November 30,2017
\vspace{6pt}

A 21-year-old woman with acute myeloid leukemia who was undergoing induction chemotherapy presented with a lesion on her left knee that had developed 5 days after initiation of therapy. Physical examination revealed a hemorrhagic patch with an erythematous border that progressed rapidly into a violaceous pustule. The patient developed a fever, and laboratory findings were notable for pancytopenia. Blood and wound cultures grew gram-negative rod-shaped bacterium. What is the diagnosis?
\vspace{12pt}

\textbf{Options:}
\begin{enumerate}
\item[A.] Cyroglobulinemia
\item[B.] Necrotizing fasciitis
\item[C.] Polyarteritis nodosa
\item[D.] Ecthyma gangrenosum
\item[E.] Pyoderma gangrenosum
\end{enumerate}

\textbf{Image:}
\begin{center}
\includegraphics[width=0.86\textwidth,height=0.50\textheight,width=0.90\textwidth,keepaspectratio]{images/nejm_20171130.jpg}
\end{center}
\vspace{12pt}
\newpage

\section*{Question 776 (ID: 20171207)}
\textbf{Date: }December 07,2017
\vspace{6pt}

A 23-day-old male infant was brought to the emergency department after 5 days of increasing vomiting. Physical exam revealed mild dehydration and an olive-sized abdominal mass. Upper gastrointestinal radiography showed a distended, air-filled stomach. What is the likely diagnosis?
\vspace{12pt}

\textbf{Options:}
\begin{enumerate}
\item[A.] Gastric duplication
\item[B.] Pyloric stenosis
\item[C.] Duodenal atresia
\item[D.] Intestinal malrotation
\item[E.] Mid-gut volvulus
\end{enumerate}

\textbf{Image:}
\begin{center}
\includegraphics[width=0.84\textwidth,height=0.50\textheight,width=0.90\textwidth,keepaspectratio]{images/nejm_20171207.jpg}
\end{center}
\vspace{12pt}
\newpage

\section*{Question 777 (ID: 20171214)}
\textbf{Date: }December 14,2017
\vspace{6pt}

A 27-year-old man presented with a 2-year history of acne and the recurrence of multiple painful lesions in the genital and perineal areas. Examination revealed inflamed and noninflamed nodules and scar tissue in the genitofemoral area, axilla, intergluteal area, and behind the ears. What is the diagnosis?
\vspace{12pt}

\textbf{Options:}
\begin{enumerate}
\item[A.] Blastomycosis
\item[B.] Cat scratch disease
\item[C.] Hidradenitis suppurativa
\item[D.] Lymphogranuloma venereum
\item[E.] Epidermal inclusion cyst
\end{enumerate}

\textbf{Image:}
\begin{center}
\includegraphics[width=0.49\textwidth,height=0.50\textheight,width=0.90\textwidth,keepaspectratio]{images/nejm_20171214.jpg}
\end{center}
\vspace{12pt}
\newpage

\section*{Question 778 (ID: 20171221)}
\textbf{Date: }December 21,2017
\vspace{6pt}

A 70-year-old woman presented to the neurology clinic with a 2-year history of gait disturbance, cognitive impairment, and urinary incontinence. What is the diagnosis?
\vspace{12pt}

\textbf{Options:}
\begin{enumerate}
\item[A.] Normal-pressure hydrocephalus
\item[B.] Alzheimer's disease
\item[C.] Parkinson's disease
\item[D.] Benign intracranial hypertension
\item[E.] Multiple system atrophy
\end{enumerate}

\textbf{Image:}
\begin{center}
\includegraphics[width=0.85\textwidth,height=0.50\textheight,width=0.90\textwidth,keepaspectratio]{images/nejm_20171221.jpg}
\end{center}
\vspace{12pt}
\newpage

\section*{Question 779 (ID: 20171228)}
\textbf{Date: }December 28,2017
\vspace{6pt}

An 86-year-old man presented with an absent right pectoralis major muscle and brachydactyly of the right hand since birth. What is the diagnosis?
\vspace{12pt}

\textbf{Options:}
\begin{enumerate}
\item[A.] Down syndrome
\item[B.] Poland syndrome
\item[C.] Prenatal thalidomide exposure
\item[D.] Fanconi anemia
\item[E.] Ehlers-Danlos syndrome
\end{enumerate}

\textbf{Image:}
\begin{center}
\includegraphics[width=0.62\textwidth,height=0.50\textheight,width=0.90\textwidth,keepaspectratio]{images/nejm_20171228.jpg}
\end{center}
\vspace{12pt}
\newpage

\section*{Question 780 (ID: 20180104)}
\textbf{Date: }January 04,2018
\vspace{6pt}

An 11-year-old girl presented to the hospital with 3-week history of an an ulcerated lesion on her cheek, unresponsive to a course of antibiotics. PCR testing of a swab specimen from the lesion was positive for orthopoxvirus DNA and cowpox virus-specific oligonucleotides. Exposure to what animal is the likely route of transmission?
\vspace{12pt}

\textbf{Options:}
\begin{enumerate}
\item[C.] Chickens
\item[D.] Rabbits
\end{enumerate}

\textbf{Image:}
\begin{center}
\includegraphics[width=0.95\textwidth,height=0.50\textheight,width=0.90\textwidth,keepaspectratio]{images/nejm_20180104.jpg}
\end{center}
\vspace{12pt}
\newpage

\section*{Question 781 (ID: 20180111)}
\textbf{Date: }January 11,2018
\vspace{6pt}

A 66-year-old woman presented with a 2-week history of dry cough and severe pain in the right flank. Five days before presentation, she had received a diagnosis of viral upper respiratory tract infection, but her symptoms had not abated with supportive treatment. The patient reported no trauma; she had no known sick contacts and had received the tetanus-diphtheria-acellular pertussis vaccine 8 years earlier. Physical examination revealed tenderness on palpation over the chest wall on the right side. Computed tomography of the chest and abdomen revealed a displaced fracture of the lateral aspect of the ninth rib on the right side. What is the likely diagnosis?
\vspace{12pt}

\textbf{Options:}
\begin{enumerate}
\item[A.] Bordetella pertussis
\item[B.] Chlamydial pneumonia
\item[C.] Mycoplasmal pneumonia
\item[D.] Bronchiolitis
\item[E.] Respiratory syncytial virus infection
\end{enumerate}

\textbf{Image:}
\begin{center}
\includegraphics[width=0.95\textwidth,height=0.50\textheight,width=0.90\textwidth,keepaspectratio]{images/nejm_20180111.jpg}
\end{center}
\vspace{12pt}
\newpage

\section*{Question 782 (ID: 20180118)}
\textbf{Date: }January 18,2018
\vspace{6pt}

A 54-year-old man presented with a 3-week history of cognitive deterioration. Neurologic examination revealed disorientation, horizontal gaze-evoked nystagmus, hyperreflexia, startle myoclonus, and ataxia. Brain MRI with diffusion-weighted imaging revealed hyperintensity of the cortical gyri and caudate heads. What is the diagnosis?
\vspace{12pt}

\textbf{Options:}
\begin{enumerate}
\item[A.] Dementia with Lewy bodies
\item[B.] Fronto-temporal dementia
\item[C.] Creutzfeld-Jakob disease
\item[D.] Lyme neuroborreliosis
\item[E.] Hashimoto’s encephalitis
\end{enumerate}

\textbf{Image:}
\begin{center}
\includegraphics[width=0.67\textwidth,height=0.50\textheight,width=0.90\textwidth,keepaspectratio]{images/nejm_20180118.jpg}
\end{center}
\vspace{12pt}
\newpage

\section*{Question 783 (ID: 20180125)}
\textbf{Date: }January 25,2018
\vspace{6pt}

A 76-year-old woman with a history of dementia and coronary heart disease was brought to the emergency department after she had been found lying outdoors for an undetermined period. What is this electrocardiogram finding?
\vspace{12pt}

\textbf{Options:}
\begin{enumerate}
\item[A.] Delta wave
\item[B.] Epsilon wave
\item[C.] Osborn wave
\item[D.] Inverted Q wave
\item[E.] U wave
\end{enumerate}

\textbf{Image:}
\begin{center}
\includegraphics[width=0.95\textwidth,height=0.50\textheight,width=0.90\textwidth,keepaspectratio]{images/nejm_20180125.jpg}
\end{center}
\vspace{12pt}
\newpage

\section*{Question 784 (ID: 20180201)}
\textbf{Date: }February 01,2018
\vspace{6pt}

A previously healthy 1-year-old girl was admitted to the hospital with a 4-day history of high fever accompanied by erythema and swelling of the left third finger, gingival inflammation, and tongue ulcerations. What is the likely diagnosis?
\vspace{12pt}

\textbf{Options:}
\begin{enumerate}
\item[A.] Bacterial cellulitis
\item[B.] Hand, foot, and mouth disease
\item[C.] Paronychia
\item[D.] Herpetic whitlow
\item[E.] Sporotrichosis
\end{enumerate}

\textbf{Image:}
\begin{center}
\includegraphics[width=0.95\textwidth,height=0.50\textheight,width=0.90\textwidth,keepaspectratio]{images/nejm_20180201.jpg}
\end{center}
\vspace{12pt}
\newpage

\section*{Question 785 (ID: 20180208)}
\textbf{Date: }February 08,2018
\vspace{6pt}

An 81-year-old man with hypertension presented with pain and swelling of the left thigh and lower leg that had developed during the previous several hours. He had no history of recent surgery or trauma and no known personal or family history of clotting disorders. The left lower leg was tender, cold, and swollen, and the dorsalis pedis pulse was not palpable. What is the diagnosis?
\vspace{12pt}

\textbf{Options:}
\begin{enumerate}
\item[A.] Polycythemia vera
\item[B.] Generalized essential telangiectasia
\item[C.] Phlegmasia cerulea dolens
\item[D.] Rhabdomyolysis
\item[E.] Cellulitis
\end{enumerate}

\textbf{Image:}
\begin{center}
\includegraphics[width=0.72\textwidth,height=0.50\textheight,width=0.90\textwidth,keepaspectratio]{images/nejm_20180208.jpg}
\end{center}
\vspace{12pt}
\newpage

\section*{Question 786 (ID: 20180215)}
\textbf{Date: }February 15,2018
\vspace{6pt}

A 19-year-old woman presented to the emergency department with a 7-year history of periodic lower abdominal and pelvic pain without fever, chills, or dysuria. She had not had menses nor vaginal intercourse. Physical exam revealed lower abdominal distention and a firm and tender suprapubic mass extending from the pelvis to the umbilicus. Perineal exam is seen in the image. What is the likely diagnosis?
\vspace{12pt}

\textbf{Options:}
\begin{enumerate}
\item[A.] Molar pregnancy
\item[B.] Choriocarcinoma
\item[C.] Endometrial cancer
\item[D.] Uterine fibroids
\item[E.] Hematometrocolpos
\end{enumerate}

\textbf{Image:}
\begin{center}
\includegraphics[width=0.8\textwidth,height=0.50\textheight,width=0.90\textwidth,keepaspectratio]{images/nejm_20180215.jpg}
\end{center}
\vspace{12pt}
\newpage

\section*{Question 787 (ID: 20180222)}
\textbf{Date: }February 22,2018
\vspace{6pt}

A 24-year-old woman with a 10-year history of intermittent episodes of redness and photophobia in both eyes presented to the ophthalmology clinic. On examination, the visual acuity was 20/30 in the right eye and 20/25 in the left eye. Slit-lamp examination revealed conjunctival hyperemia and peripheral corneal opacification, with inflammation and crystalline deposits on the corneal stroma consistent with interstitial keratitis. Six months later, the patient reported having vertigo, tinnitus, and hearing loss. What is the diagnosis?
\vspace{12pt}

\textbf{Options:}
\begin{enumerate}
\item[A.] Syphilis
\item[B.] Systemic lupus erythematosus
\item[C.] High lead exposure
\item[D.] Cogan’s syndrome
\item[E.] Lymphoma
\end{enumerate}

\textbf{Image:}
\begin{center}
\includegraphics[width=0.95\textwidth,height=0.50\textheight,width=0.90\textwidth,keepaspectratio]{images/nejm_20180222.jpg}
\end{center}
\vspace{12pt}
\newpage

\section*{Question 788 (ID: 20180301)}
\textbf{Date: }March 01,2018
\vspace{6pt}

A 52-year-old woman presented to the emergency department with a 24-hour history of severe pain and blurred vision in the left eye. A left eye exam revealed a mid-dilated pupil unresponsive to light, red conjunctiva, diminished visual acuity, and increased intraocular pressure. What is the likely diagnosis?
\vspace{12pt}

\textbf{Options:}
\begin{enumerate}
\item[A.] Acute conjunctivitis
\item[B.] Acute angle-closure glaucoma
\item[C.] Acute orbital compartment syndrome
\item[D.] Ulcerative keratitis
\item[E.] Vitreous hemorrhage
\end{enumerate}

\textbf{Image:}
\begin{center}
\includegraphics[width=0.95\textwidth,height=0.50\textheight,width=0.90\textwidth,keepaspectratio]{images/nejm_20180301.jpg}
\end{center}
\vspace{12pt}
\newpage

\section*{Question 789 (ID: 20180308)}
\textbf{Date: }March 08,2018
\vspace{6pt}

A 13-year-old boy presented to the orthopedic clinic with a 1-week history of bilateral knee pain. He was active in sports and participated in long jump. Examination of the right knee showed mild soft-tissue swelling and tenderness over the tibial tubercle, and a taut quadriceps muscle; the exam of the left knee was normal. Plain radiographs of both knees were done. What is the likely diagnosis?
\vspace{12pt}

\textbf{Options:}
\begin{enumerate}
\item[A.] Hoffa’s syndrome
\item[B.] Patellar tendon avulsion
\item[C.] Pes anserine bursitis
\item[D.] Osteomyelitis of the proximal tibia
\item[E.] Osgood-Schlatter disease
\end{enumerate}

\textbf{Image:}
\begin{center}
\includegraphics[width=0.95\textwidth,height=0.50\textheight,width=0.90\textwidth,keepaspectratio]{images/nejm_20180308.jpg}
\end{center}
\vspace{12pt}
\newpage

\section*{Question 790 (ID: 20180315)}
\textbf{Date: }March 15,2018
\vspace{6pt}

A 27-year-old man presented to the emergency department with abdominal pain in the left upper quadrant and a pulsatile, painful lesion on the right hand. He reported a 6-week history of fevers, decreased appetite, and night sweats and a weight loss of 12 kg. Physical examination was notable for a temperature of 38.5 degrees Celsius and a grade 3/6 diastolic murmur throughout the precordium. Laboratory studies revealed a white-cell count of 18,000 per cubic millimeter. CT angiography of the right arm revealed an aneurysm of the ulnar artery.  What is the diagnosis?
\vspace{12pt}

\textbf{Options:}
\begin{enumerate}
\item[A.] Systemic lupus erythematosus
\item[B.] Lyme disease
\item[C.] Infective endocarditis
\item[D.] Antiphospholipid syndrome
\item[E.] Kawasaki disease
\end{enumerate}

\textbf{Image:}
\begin{center}
\includegraphics[width=0.95\textwidth,height=0.50\textheight,width=0.90\textwidth,keepaspectratio]{images/nejm_20180315.jpg}
\end{center}
\vspace{12pt}
\newpage

\section*{Question 791 (ID: 20180322)}
\textbf{Date: }March 22,2018
\vspace{6pt}

A 5-year-old, fully vaccinated girl was brought to the emergency department with pruritic lesions on both legs. She had recently returned from a trip to Sierra Leone. The lesions had appeared 3 weeks into her stay there and had increased in size and ulcerated. On presentation, she was afebrile, and an area over the medial aspect of the left lower leg was ulcerated and bleeding. The C-reactive protein level was mildly elevated, and the white-cell count was normal. What is the diagnosis?
\vspace{12pt}

\textbf{Options:}
\begin{enumerate}
\item[A.] Ehlers-Danlos syndrome
\item[B.] Sickle cell disease
\item[C.] Martorell ulcer
\item[D.] Cutaneous diphtheria
\item[E.] Pyoderma gangrenosum
\end{enumerate}

\textbf{Image:}
\begin{center}
\includegraphics[width=0.95\textwidth,height=0.50\textheight,width=0.90\textwidth,keepaspectratio]{images/nejm_20180322.jpg}
\end{center}
\vspace{12pt}
\newpage

\section*{Question 792 (ID: 20180329)}
\textbf{Date: }March 29,2018
\vspace{6pt}

A 36-year-old man presented to the dermatology clinic with a 5-day history of fever, sore throat, tongue ulcerations, and blisters on his palms and soles. His 2- and 4-year-old children had similar symptoms the previous week. What is the likely diagnosis?
\vspace{12pt}

\textbf{Options:}
\begin{enumerate}
\item[A.] Erythema multiforme
\item[B.] Toxic epidermal necrolysis
\item[C.] Behçet's disease
\item[D.] Kawasaki disease
\item[E.] Hand, foot, and mouth disease
\end{enumerate}

\textbf{Image:}
\begin{center}
\includegraphics[width=0.95\textwidth,height=0.50\textheight,width=0.90\textwidth,keepaspectratio]{images/nejm_20180329.jpg}
\end{center}
\vspace{12pt}
\newpage

\section*{Question 793 (ID: 20180405)}
\textbf{Date: }April 05,2018
\vspace{6pt}

A 21-year-old woman presented to the emergency department with an 8-month history of swelling and pain in the right thigh. MRI of the right leg revealed a large periosteal femoral mass and thrombus in the right femoral vein, for which she received anticoagulation therapy. The patient subsequently experienced dyspnea and pleuritic chest pain, and a chest x-ray was done. What is the likely diagnosis?
\vspace{12pt}

\textbf{Options:}
\begin{enumerate}
\item[A.] Pulmonary embolism
\item[B.] Pulmonary sarcoidosis
\item[C.] Eosinophilic granulomatosis with polyangiitis
\item[D.] Chondroblastic osteosarcoma
\item[E.] Miliary tuberculosis
\end{enumerate}

\textbf{Image:}
\begin{center}
\includegraphics[width=0.95\textwidth,height=0.50\textheight,width=0.90\textwidth,keepaspectratio]{images/nejm_20180405.jpg}
\end{center}
\vspace{12pt}
\newpage

\section*{Question 794 (ID: 20180412)}
\textbf{Date: }April 12,2018
\vspace{6pt}

A 70-year-old man presented to the ophthalmology clinic with a 1-year history of progressive bluish discoloration of the sclerae of both eyes. He reported no ocular discomfort or blurry vision. He had previously received a diagnosis of an inflammatory arthritis for which he had been taking minocycline for more than 15 years. Examination was notable for bluish discoloration of the sclera of both eyes and pinnae of both ears. Ophthalmologic examination was otherwise normal. What is the diagnosis?
\vspace{12pt}

\textbf{Options:}
\begin{enumerate}
\item[A.] Osteogenesis imperfecta
\item[B.] Minocycline-induced pigmentation
\item[C.] Ehlers-Danlos syndrome
\item[D.] Primary acquired melanosis
\item[E.] Conjunctival melanoma
\end{enumerate}

\textbf{Image:}
\begin{center}
\includegraphics[width=0.95\textwidth,height=0.50\textheight,width=0.90\textwidth,keepaspectratio]{images/nejm_20180412.jpg}
\end{center}
\vspace{12pt}
\newpage

\section*{Question 795 (ID: 20180419)}
\textbf{Date: }April 19,2018
\vspace{6pt}

A 74-year-old woman presented to the emergency department with a 3-day history of lower abdominal pain, nausea, and vomiting. A plain-film radiograph of the abdomen showed marked dilatation of bowel loops and abrupt termination of gas within the descending colon, referred to as a colon cutoff sign. The colon cutoff sign is classically described in association with which disease process?
\vspace{12pt}

\textbf{Options:}
\begin{enumerate}
\item[A.] Diaphragmatic hernia
\item[B.] Acute pancreatitis
\item[C.] Toxic shock syndrome
\item[D.] Lead poisoning
\item[E.] Crohn’s disease
\end{enumerate}

\textbf{Image:}
\begin{center}
\includegraphics[width=0.92\textwidth,height=0.50\textheight,width=0.90\textwidth,keepaspectratio]{images/nejm_20180419.jpg}
\end{center}
\vspace{12pt}
\newpage

\section*{Question 796 (ID: 20180426)}
\textbf{Date: }April 26,2018
\vspace{6pt}

A 31-year-old man presented to the emergency department with a 2-week history of left ear swelling. He had similar episodes over the past 2 years, and a 6-month history of weight loss, fatigue, and generalized aches. Physical exam revealed a swollen and tender ear, costochondral joints, and left knee. Lab tests showed an elevated erythrocyte sedimentation rate. What is the likely diagnosis?
\vspace{12pt}

\textbf{Options:}
\begin{enumerate}
\item[A.] Infectious perichondritis
\item[B.] Rheumatoid arthritis
\item[C.] Granulomatosis with polyangitis
\item[D.] Polyarteritis nodosa
\item[E.] Relapsing polychondritis
\end{enumerate}

\textbf{Image:}
\begin{center}
\includegraphics[width=0.88\textwidth,height=0.50\textheight,width=0.90\textwidth,keepaspectratio]{images/nejm_20180426.jpg}
\end{center}
\vspace{12pt}
\newpage

\section*{Question 797 (ID: 20180503)}
\textbf{Date: }May 03,2018
\vspace{6pt}

A 31-year-old man from El Salvador, recently diagnosed with HIV/AIDS, presented to the emergency department with headache, confusion, gait instability, and fever. Brain MRI showed a large mass in the right parietal and occipital lobes. Brain aspirate identified Trypanosoma cruzi. What vector is likely responsible for transmission of this protozoa to this patient?
\vspace{12pt}

\textbf{Options:}
\begin{enumerate}
\item[A.] Anopheles mosquito
\item[B.] Triatomine bug
\item[C.] Ixodes tick
\item[D.] Phlebotomine sand fly
\item[E.] Tsetse fly
\end{enumerate}

\textbf{Image:}
\begin{center}
\includegraphics[width=0.76\textwidth,height=0.50\textheight,width=0.90\textwidth,keepaspectratio]{images/nejm_20180503.jpg}
\end{center}
\vspace{12pt}
\newpage

\section*{Question 798 (ID: 20180510)}
\textbf{Date: }May 10,2018
\vspace{6pt}

A 46-year-old man presented to the emergency department with a 1-month history of fatigue, shortness of breath, and low back pain and report of a weight loss of 30 kg over the previous 10 months. On physical examination, his conjunctiva and palms were pale. Laboratory evaluation revealed impaired kidney function and hypercalcemia. Radiographs demonstrated a “raindrop skull.” What is the diagnosis?
\vspace{12pt}

\textbf{Options:}
\begin{enumerate}
\item[A.] Amyloidosis
\item[B.] Systemic lupus erythematosus
\item[C.] Hodgkin’s lymphoma
\item[D.] Paget’s disease
\item[E.] Multiple myeloma
\end{enumerate}

\textbf{Image:}
\begin{center}
\includegraphics[width=0.95\textwidth,height=0.50\textheight,width=0.90\textwidth,keepaspectratio]{images/nejm_20180510.jpg}
\end{center}
\vspace{12pt}
\newpage

\section*{Question 799 (ID: 20180517)}
\textbf{Date: }May 17,2018
\vspace{6pt}

A 3-week-old male newborn was brought to the pediatrician with three large scalp lesions, which had appeared in the first week of life. The lesions were annular with raised papular and pustular borders and flat, hyperkeratotic central areas. The baby’s mother, who was originally from Somalia, had similar skin lesions on her upper trunk. What is the diagnosis?
\vspace{12pt}

\textbf{Options:}
\begin{enumerate}
\item[A.] Neonatal herpes simplex virus
\item[B.] Nevus sebaceous
\item[C.] Tinea capitis
\item[D.] Congenital syphilis
\item[E.] Dissecting folliculitis
\end{enumerate}

\textbf{Image:}
\begin{center}
\includegraphics[width=0.95\textwidth,height=0.50\textheight,width=0.90\textwidth,keepaspectratio]{images/nejm_20180517.jpg}
\end{center}
\vspace{12pt}
\newpage

\section*{Question 800 (ID: 20180524)}
\textbf{Date: }May 24,2018
\vspace{6pt}

A 77-year-old man was referred to the endocrinology clinic for evaluation of subclinical hyperthyroidism. Physical examination revealed an enlarged thyroid with no palpable nodules or cervical lymphadenopathy. When the patient raised both arms, he developed reversible facial congestion. What is the name of this sign?
\vspace{12pt}

\textbf{Options:}
\begin{enumerate}
\item[A.] Grave’s sign
\item[B.] Hashimoto’s sign
\item[C.] Pemberton’s sign
\item[D.] Flemingo’s flush sign
\item[E.] Lighthouse sign
\end{enumerate}

\textbf{Image:}
\begin{center}
\includegraphics[width=0.94\textwidth,height=0.50\textheight,width=0.90\textwidth,keepaspectratio]{images/nejm_20180524.jpg}
\end{center}
\vspace{12pt}
\newpage

\section*{Question 801 (ID: 20180531)}
\textbf{Date: }May 31,2018
\vspace{6pt}

A 44-year-old woman presented to the emergency department with acute chest pain after several months of progressive dyspnea. Her oxygen saturation was 92\%, and she had diminished breath sounds on the right side. Chest CT revealed a large right-sided pneumothorax and diffuse, intraparenchymal pulmonary cysts. What is the likely diagnosis?
\vspace{12pt}

\textbf{Options:}
\begin{enumerate}
\item[A.] Cystic fibrosis
\item[B.] Pneumocystis pneumonia
\item[C.] Emphysema
\item[D.] Lymphangioleiomyomatosis
\item[E.] Sjögren’s syndrome
\end{enumerate}

\textbf{Image:}
\begin{center}
\includegraphics[width=0.95\textwidth,height=0.50\textheight,width=0.90\textwidth,keepaspectratio]{images/nejm_20180531.jpg}
\end{center}
\vspace{12pt}
\newpage

\section*{Question 802 (ID: 20180607)}
\textbf{Date: }June 07,2018
\vspace{6pt}

A 63-year-old man presented to the emergency department with a 3-day history of abdominal pain that had started in the periumbilical area and subsequently shifted to the left lower quadrant. Initial laboratory tests showed a white-cell count of 12,000 per cubic millimeter (reference range, 4000 to 10,000) and a lactate level of 1.8 mmol per liter (normal value, <1.9). Contrast-enhanced computed tomography of the abdomen revealed edema of the sigmoid colon with thumbprinting. What is the diagnosis?
\vspace{12pt}

\textbf{Options:}
\begin{enumerate}
\item[A.] Secondary syphilis
\item[B.] Ischemic colitis
\item[C.] Lead poisoning
\item[D.] Chronic constipation
\item[E.] Lymphoma
\end{enumerate}

\textbf{Image:}
\begin{center}
\includegraphics[width=0.95\textwidth,height=0.50\textheight,width=0.90\textwidth,keepaspectratio]{images/nejm_20180607.jpg}
\end{center}
\vspace{12pt}
\newpage

\section*{Question 803 (ID: 20180614)}
\textbf{Date: }June 14,2018
\vspace{6pt}

A 32-year-old woman presented to an ophthalmologist with a 2-week history of nodules that moved around her face. She had first noted a nodule below her left eye. Five days later, it had moved to above her left eye, and 10 days after that to the upper lip. The nodules occasionally caused a localized itching and burning sensation. She had recently traveled to a rural area outside Moscow and recalled being frequently bitten by mosquitoes. What is the diagnosis?
\vspace{12pt}

\textbf{Options:}
\begin{enumerate}
\item[A.] Human papillomavirus warts
\item[B.] Pyogenic granuloma
\item[C.] Relapsing malaria
\item[D.] Dirofilaria repens infection
\item[E.] Nodulocystic acne
\end{enumerate}

\textbf{Image:}
\begin{center}
\includegraphics[width=0.95\textwidth,height=0.50\textheight,width=0.90\textwidth,keepaspectratio]{images/nejm_20180614.jpg}
\end{center}
\vspace{12pt}
\newpage

\section*{Question 804 (ID: 20180621)}
\textbf{Date: }June 21,2018
\vspace{6pt}

An 86-year-old woman presented to the emergency department with tongue pain 8 days after the diagnosis of giant-cell arteritis by temporal artery biopsy and treatment with glucocorticoids. Examination revealed necrotic ulceration on the right side of the tongue. Cervicofacial CT showed complete thrombosis of which one of the following arteries, on the right side?
\vspace{12pt}

\textbf{Options:}
\begin{enumerate}
\item[A.] Superior thyroid artery
\item[B.] Ascending pharyngeal artery
\item[C.] Posterior auricular artery
\item[D.] Facial artery
\item[E.] Lingual artery
\end{enumerate}

\textbf{Image:}
\begin{center}
\includegraphics[width=0.79\textwidth,height=0.50\textheight,width=0.90\textwidth,keepaspectratio]{images/nejm_20180621.jpg}
\end{center}
\vspace{12pt}
\newpage

\section*{Question 805 (ID: 20180628)}
\textbf{Date: }June 28,2018
\vspace{6pt}

A 44-year-old male construction worker presented to the dermatology clinic with nonpruritic skin lesions over his face, chest, back, arms, and legs. Six months earlier, he had fever, cough, and vomiting for 1 week. Skin biopsy showed pseudoepitheliomatous hyperplasia, intraepidermal neutrophilic abscesses, and round yeast forms with broad-based budding. What is the likely diagnosis?
\vspace{12pt}

\textbf{Options:}
\begin{enumerate}
\item[A.] Basal cell carcinoma
\item[B.] Giant keratoacanthoma
\item[C.] Pyoderma gangrenosum
\item[D.] Disseminated cutaneous blastomycosis
\item[E.] Cutaneous leishmaniasis
\end{enumerate}

\textbf{Image:}
\begin{center}
\includegraphics[width=0.69\textwidth,height=0.50\textheight,width=0.90\textwidth,keepaspectratio]{images/nejm_20180628.jpg}
\end{center}
\vspace{12pt}
\newpage

\section*{Question 806 (ID: 20180705)}
\textbf{Date: }July 05,2018
\vspace{6pt}

A 30-year-old man presented with a 15-month history of intermittent discomfort in the right upper quadrant of the abdomen. He lived in a rural area of Morocco and had occasional contact with dogs. The physical examination revealed hepatomegaly with a palpable hepatic mass. Laboratory tests showed a normal white-cell count and a normal absolute eosinophil count. Ultrasonography and computed tomography of the abdomen revealed a large cyst in the right lobe of the liver. What is the diagnosis?
\vspace{12pt}

\textbf{Options:}
\begin{enumerate}
\item[A.] Primary hepatic carcinoma
\item[B.] Autoimmune hepatitis
\item[C.] Primary biliary cholangitis
\item[D.] Amebiasis
\item[E.] Cystic echinococcosis
\end{enumerate}

\textbf{Image:}
\begin{center}
\includegraphics[width=0.95\textwidth,height=0.50\textheight,width=0.90\textwidth,keepaspectratio]{images/nejm_20180705.jpg}
\end{center}
\vspace{12pt}
\newpage

\section*{Question 807 (ID: 20180712)}
\textbf{Date: }July 12,2018
\vspace{6pt}

A 36-year-old man presented to the emergency department with a 2-week history of fever, headache, drowsiness, and photophobia. He was previously healthy and was sexually active with men. The physical examination was notable for a temperature of 38.3°C and neck stiffness. Computed tomography of the head was normal. A lumbar puncture was performed. Based on this preparation of the cerebrospinal fluid, what is the diagnosis?
\vspace{12pt}

\textbf{Options:}
\begin{enumerate}
\item[A.] Neurosyphilis
\item[B.] Cryptococcal meningitis
\item[C.] Lymphoma
\item[D.] Tuberculosis
\item[E.] Hemorrhagic cysts
\end{enumerate}

\textbf{Image:}
\begin{center}
\includegraphics[width=0.62\textwidth,height=0.50\textheight,width=0.90\textwidth,keepaspectratio]{images/nejm_20180712.jpg}
\end{center}
\vspace{12pt}
\newpage

\section*{Question 808 (ID: 20180719)}
\textbf{Date: }July 19,2018
\vspace{6pt}

A 71-year-old man presented with fever and excruciating left-hand pain that developed 12 hours after eating raw seafood. His past medical history was significant for type 2 diabetes mellitus, hypertension, and end-stage renal disease. At time of presentation, a hemorrhagic bullae measuring 3.5 by 4.5 cm had developed in the palm of his left hand. Surgical intervention was performed and a causative organism was isolated. What is the most likely organism?
\vspace{12pt}

\textbf{Options:}
\begin{enumerate}
\item[A.] Staphylococcus aureus
\item[B.] Streptococcus pyogenes
\item[C.] Haemophilus influenza
\item[D.] Vibrio vulnificus
\item[E.] Pseudomonas aeruginosa
\end{enumerate}

\textbf{Image:}
\begin{center}
\includegraphics[width=0.95\textwidth,height=0.50\textheight,width=0.90\textwidth,keepaspectratio]{images/nejm_20180719.jpg}
\end{center}
\vspace{12pt}
\newpage

\section*{Question 809 (ID: 20180726)}
\textbf{Date: }July 26,2018
\vspace{6pt}

A 4-week-old boy was brought to the emergency department after having drainage from both eyes for 2 days and redness and swelling under his left eye for 1 day. The perinatal history was uncomplicated, and he was breast-feeding well. Physical examination revealed a temperature of 38.2°C, purulent drainage from both eyes, and a 1-cm erythematous, fluctuant mass inferior to the medial canthus of the left eye. What is the most likely diagnosis?
\vspace{12pt}

\textbf{Options:}
\begin{enumerate}
\item[A.] Trichiasis
\item[B.] Gonococcal conjunctivitis
\item[C.] Chlamydial conjunctivitis
\item[D.] Simple nasolacrimal duct obstruction
\item[E.] Acute dacrocystitis
\end{enumerate}

\textbf{Image:}
\begin{center}
\includegraphics[width=0.95\textwidth,height=0.50\textheight,width=0.90\textwidth,keepaspectratio]{images/nejm_20180726.jpg}
\end{center}
\vspace{12pt}
\newpage

\section*{Question 810 (ID: 20180802)}
\textbf{Date: }August 02,2018
\vspace{6pt}

A 69-year-old man presented to the dermatology clinic with a 2-month history of a pruritic rash. The rash first appeared on his right wrist and within 2 weeks had spread to his arms, legs, and trunk. He had completed treatment for chronic hepatitis C virus (HCV) infection with a 12-week course of elbasvir and grazoprevir 3 months before the onset of the purple papular rash. Some lesions showed a fine reticulate pattern of dots and lines (Wickham’s striae). What is the diagnosis?
\vspace{12pt}

\textbf{Options:}
\begin{enumerate}
\item[A.] Drug eruption
\item[B.] Reactivation of endogenous latent varicella-zoster virus
\item[C.] Discoid lupus erythematosus
\item[D.] Lichen planus
\item[E.] Pemphigus
\end{enumerate}

\textbf{Image:}
\begin{center}
\includegraphics[width=0.81\textwidth,height=0.50\textheight,width=0.90\textwidth,keepaspectratio]{images/nejm_20180802.jpg}
\end{center}
\vspace{12pt}
\newpage

\section*{Question 811 (ID: 20180809)}
\textbf{Date: }August 09,2018
\vspace{6pt}

A 39-year-old man presented to the emergency department with anorexia, nausea, vomiting, oliguria, and diffuse deposits on his skin. The physical exam findings are associated with what underlying disease?
\vspace{12pt}

\textbf{Options:}
\begin{enumerate}
\item[A.] End-stage renal disease
\item[B.] Acute myeloid leukemia
\item[C.] Sarcoidosis
\item[D.] End-stage liver disease
\item[E.] Tuberculosis
\end{enumerate}

\textbf{Image:}
\begin{center}
\includegraphics[width=0.95\textwidth,height=0.50\textheight,width=0.90\textwidth,keepaspectratio]{images/nejm_20180809.jpg}
\end{center}
\vspace{12pt}
\newpage

\section*{Question 812 (ID: 20180816)}
\textbf{Date: }August 16,2018
\vspace{6pt}

A 31-year-old man presented to the emergency department with confusion. He was found to be anemic with a hemoglobin level of 6.7 g per deciliter. Additional laboratory tests revealed an elevated indirect bilirubin level, an elevated lactate dehydrogenase level, and undetectable haptoglobin. A peripheral-blood smear was performed. With what underlying disease process is this form of anemia most closely associated?
\vspace{12pt}

\textbf{Options:}
\begin{enumerate}
\item[A.] Decompensated cirrhosis
\item[B.] Myelodysplastic syndrome
\item[C.] Iron deficiency
\item[D.] End-stage renal disease
\item[E.] Vitamin B12 deficiency
\end{enumerate}

\textbf{Image:}
\begin{center}
\includegraphics[width=0.95\textwidth,height=0.50\textheight,width=0.90\textwidth,keepaspectratio]{images/nejm_20180816.jpg}
\end{center}
\vspace{12pt}
\newpage

\section*{Question 813 (ID: 20180823)}
\textbf{Date: }August 23,2018
\vspace{6pt}

A 4-year-old boy was brought by his father to the ophthalmology clinic with a 1-year history of enlarging white deposits in both eyes and decreasing night vision. On examination, the conjunctivae of both the right eye and the left eye appeared dry and wrinkled, with foamy, cream-colored deposits near the outer corners. The corneas were clear, the fundi were normal, and the visual acuity was 20/30 in both eyes. The child appeared pale, with hypopigmented hair, a weight of 10.5 kg (z score of less than -3), and a height of 92 cm (z score of -2.8). These physical exam findings are associated with what underlying condition?
\vspace{12pt}

\textbf{Options:}
\begin{enumerate}
\item[A.] Sjögren's syndrome
\item[B.] Pyridoxine deficiency
\item[C.] Vitamin A deficiency
\item[D.] Loaisis
\item[E.] Type 1 diabetes
\end{enumerate}

\textbf{Image:}
\begin{center}
\includegraphics[width=0.95\textwidth,height=0.50\textheight,width=0.90\textwidth,keepaspectratio]{images/nejm_20180823.jpg}
\end{center}
\vspace{12pt}
\newpage

\section*{Question 814 (ID: 20180830)}
\textbf{Date: }August 30,2018
\vspace{6pt}

A 55-year-old woman was admitted to hospital after sustaining a severe crush injury to both legs in a motor vehicle accident. A polymicrobial wound infection developed and she received antibiotic treatment. Black discoloration of her tongue was observed within 1 week of starting treatment, and the patient reported nausea and a bad taste in her mouth. Which class of antibiotic can lead to this presentation?
\vspace{12pt}

\textbf{Options:}
\begin{enumerate}
\item[A.] Tetracyclines
\item[B.] Penicillins
\item[C.] Macrolides
\item[D.] Fluoroquinolones
\item[E.] Sulfonamides
\end{enumerate}

\textbf{Image:}
\begin{center}
\includegraphics[width=0.95\textwidth,height=0.50\textheight,width=0.90\textwidth,keepaspectratio]{images/nejm_20180830.jpg}
\end{center}
\vspace{12pt}
\newpage

\section*{Question 815 (ID: 20180906)}
\textbf{Date: }September 06,2018
\vspace{6pt}

A 58-year-old man presented to the outpatient pulmonary clinic with a productive cough and a history of weight loss of 10 kg during the preceding 3 months. He was an active smoker and for the preceding 4 years had worked in a factory processing cotton. Computed tomography of the chest and abdomen revealed multiple pulmonary nodules (Panel A) associated with pericardial effusion, pleural effusions in both lungs, and multiple liver lesions. On biopsy of the lung, multiple lesions containing sulfur granules were observed along the bronchial vascular bundle (Panels B and C, with Panel C providing a closeup view of the circled area in Panel B; staining with hematoxylin and eosin). What is the most likely diagnosis?
\vspace{12pt}

\textbf{Options:}
\begin{enumerate}
\item[A.] Nocardiosis
\item[B.] Actinomycosis
\item[C.] Aspergillosis
\item[D.] Histoplasmosis
\item[E.] Blastomycosis
\end{enumerate}

\textbf{Image:}
\begin{center}
\includegraphics[width=0.95\textwidth,height=0.50\textheight,width=0.90\textwidth,keepaspectratio]{images/nejm_20180906.jpg}
\end{center}
\vspace{12pt}
\newpage

\section*{Question 816 (ID: 20180913)}
\textbf{Date: }September 13,2018
\vspace{6pt}

A 42-year-old woman presented with a 1-week history of swelling and pain in the fifth finger of her left hand. She reported no related trauma. She had systemic lupus erythematosus, treated with mycophenolate mofetil and prednisone. Physical examination of the affected finger revealed soft-tissue swelling, with erythema and warmth, that was most prominent between the proximal and distal interphalangeal joints, sparing the fingertip. The patient’s husband had recently traveled to China and developed a cough soon after his return home. What is the cause of the patient’s finger swelling?
\vspace{12pt}

\textbf{Options:}
\begin{enumerate}
\item[A.] Acute gout
\item[B.] Flare of systemic lupus erythematosus
\item[C.] Septic arthritis
\item[D.] Extrapulmonary manifestation of tuberculosis
\item[E.] Osteoarthritis
\end{enumerate}

\textbf{Image:}
\begin{center}
\includegraphics[width=0.95\textwidth,height=0.50\textheight,width=0.90\textwidth,keepaspectratio]{images/nejm_20180913.jpg}
\end{center}
\vspace{12pt}
\newpage

\section*{Question 817 (ID: 20180920)}
\textbf{Date: }September 20,2018
\vspace{6pt}

A 26-year-old previously healthy man presented to the emergency department with a 3-day history of fever, dry cough, and nonpruritic rash. A physical examination was notable for crackles on the left side of the chest and a macular, targetoid rash on his hands and feet, including the palms and soles. Over the next 3 days, severe mucositis developed that involved the lips, buccal mucosa, conjunctivae, and urethral meatus. What is the diagnosis?
\vspace{12pt}

\textbf{Options:}
\begin{enumerate}
\item[A.] Herpes simplex virus-1 gingivostomatitis
\item[B.] Stevens-Johnson syndrome
\item[C.] Behçet’s disease
\item[D.] Mucocutaneous Epstein-Barr virus
\item[E.] Mycoplasma pneumoniae-associated mucositis
\end{enumerate}

\textbf{Image:}
\begin{center}
\includegraphics[width=0.95\textwidth,height=0.50\textheight,width=0.90\textwidth,keepaspectratio]{images/nejm_20180920.jpg}
\end{center}
\vspace{12pt}
\newpage

\section*{Question 818 (ID: 20180927)}
\textbf{Date: }September 27,2018
\vspace{6pt}

A 41-year-old woman presented with worsening hair loss that had progressed over 2 years. What is the most likely diagnosis?
\vspace{12pt}

\textbf{Options:}
\begin{enumerate}
\item[A.] Androgenic alopecia
\item[B.] Discoid lupus erythematosus
\item[C.] Hypothyroidism
\item[D.] Secondary syphilis
\item[E.] Trichotillomania
\end{enumerate}

\textbf{Image:}
\begin{center}
\includegraphics[width=0.92\textwidth,height=0.50\textheight,width=0.90\textwidth,keepaspectratio]{images/nejm_20180927.jpg}
\end{center}
\vspace{12pt}
\newpage

\section*{Question 819 (ID: 20181004)}
\textbf{Date: }October 04,2018
\vspace{6pt}

A 32-year-old man presented to the neurosurgery clinic with mild balance difficulties and hearing loss in the left ear. Evaluation of the patient’s hearing revealed profound sensorineural loss on the left side and normal hearing on the right side. Magnetic resonance imaging of the brain performed after the administration of contrast material revealed tumors in both internal acoustic canals (33 by 26 by 31 mm on the left side and 32 by 28 by 30 mm on the right side), with extension in the cerebellopontine angles and brainstem compression, representing bilateral vestibular schwannomas. What hereditary condition is associated with this presentation?
\vspace{12pt}

\textbf{Options:}
\begin{enumerate}
\item[A.] Multiple endocrine neoplasia type 1
\item[B.] von Hippel-Lindau syndrome
\item[C.] Neurofibromatosis type 1
\item[D.] Li-Fraumeni syndrome
\item[E.] Neurofibromatosis type 2
\end{enumerate}

\textbf{Image:}
\begin{center}
\includegraphics[width=0.68\textwidth,height=0.50\textheight,width=0.90\textwidth,keepaspectratio]{images/nejm_20181004.jpg}
\end{center}
\vspace{12pt}
\newpage

\section*{Question 820 (ID: 20181011)}
\textbf{Date: }October 11,2018
\vspace{6pt}

A 42-year-old man who was undergoing treatment for non-Hodgkin’s lymphoma presented to the oncology clinic with changes in his fingernails. Five months earlier, he had presented with gastric-outlet obstruction and had received a diagnosis of high-grade B-cell non-Hodgkin’s lymphoma. What is the most likely underlying cause of this presentation?
\vspace{12pt}

\textbf{Options:}
\begin{enumerate}
\item[A.] Chemotherapy
\item[B.] Paraneoplastic syndrome
\item[C.] Metastatic disease of the nail bed
\item[D.] Hepatic dysfunction
\item[E.] Tetracycline
\end{enumerate}

\textbf{Image:}
\begin{center}
\includegraphics[width=0.95\textwidth,height=0.50\textheight,width=0.90\textwidth,keepaspectratio]{images/nejm_20181011.jpg}
\end{center}
\vspace{12pt}
\newpage

\section*{Question 821 (ID: 20181018)}
\textbf{Date: }October 18,2018
\vspace{6pt}

A 35-year-old male presented to the emergency psychiatry service with paranoid delusions and was found to have patchy, irregular alopecia of the scalp. What is the most likely diagnosis?
\vspace{12pt}

\textbf{Options:}
\begin{enumerate}
\item[A.] Androgenic alopecia
\item[B.] Central centrifugal cicatricial alopecia
\item[C.] Syphilitic alopecia
\item[D.] Trichotillomania
\item[E.] Hypothyroidism
\end{enumerate}

\textbf{Image:}
\begin{center}
\includegraphics[width=0.95\textwidth,height=0.50\textheight,width=0.90\textwidth,keepaspectratio]{images/nejm_20181018.jpg}
\end{center}
\vspace{12pt}
\newpage

\section*{Question 822 (ID: 20181025)}
\textbf{Date: }October 25,2018
\vspace{6pt}

A 52-year-old man presented to the emergency department with painless, bloody tears from both eyes. The bleeding had begun spontaneously approximately 2 hours earlier, had lasted a few minutes, and had recurred just before presentation. He was taking captopril for mild hypertension, and his blood pressure was normal. The clinical examination revealed slight conjunctival hyperemia without periorbital or palpebral edema. The patient had normal vision and extraocular movements. Which one of the following is NOT a cause of hemolacria?
\vspace{12pt}

\textbf{Options:}
\begin{enumerate}
\item[A.] Hemangioma
\item[B.] Infection
\item[C.] Adverse reaction to systemic antibiotics
\item[D.] Trauma to eye or surrounding structures
\item[E.] Retrograde epistaxis
\end{enumerate}

\textbf{Image:}
\begin{center}
\includegraphics[width=0.73\textwidth,height=0.50\textheight,width=0.90\textwidth,keepaspectratio]{images/nejm_20181025.jpg}
\end{center}
\vspace{12pt}
\newpage

\section*{Question 823 (ID: 20181101)}
\textbf{Date: }November 01,2018
\vspace{6pt}

A 39-year-old man presented to the emergency department with a 4-week history of abdominal pain and constipation. Physical examination of the abdomen was normal, but he was noted to have gray lines along the margins of his lower gum. What is the most likely diagnosis?
\vspace{12pt}

\textbf{Options:}
\begin{enumerate}
\item[A.] Scurvy
\item[B.] Bismuth poisoning
\item[C.] Adults Still’s disease
\item[D.] Behçet’s syndrome
\item[E.] Lead poisoning
\end{enumerate}

\textbf{Image:}
\begin{center}
\includegraphics[width=0.95\textwidth,height=0.50\textheight,width=0.90\textwidth,keepaspectratio]{images/nejm_20181101.jpg}
\end{center}
\vspace{12pt}
\newpage

\section*{Question 824 (ID: 20181108)}
\textbf{Date: }November 08,2018
\vspace{6pt}

A 56-year-old woman with no significant past medical history presented to the emergency department with worsening shortness of breath over 8 months. She had no prior history of tobacco use. Physical examination was notable for bronchial breath sounds in both lungs with oxygen saturation of 93\% on 3 liters of supplemental oxygen. A computed tomographic angiogram noted innumerable pulmonary nodules in a diffuse pattern. What is the most likely diagnosis for this presentation?
\vspace{12pt}

\textbf{Options:}
\begin{enumerate}
\item[A.] Non-small-cell lung cancer
\item[B.] Sarcoidosis
\item[C.] Langerhans cell histiocytosis
\item[D.] Histoplasmosis
\item[E.] Septic embolization
\end{enumerate}

\textbf{Image:}
\begin{center}
\includegraphics[width=0.95\textwidth,height=0.50\textheight,width=0.90\textwidth,keepaspectratio]{images/nejm_20181108.jpg}
\end{center}
\vspace{12pt}
\newpage

\section*{Question 825 (ID: 20181115)}
\textbf{Date: }November 15,2018
\vspace{6pt}

A 72-year-old man presented to the emergency department with an 11-hour history of periumbilical abdominal pain and inability to pass flatus. The pulse was 155 beats per minute, and the blood pressure 83/52 mm Hg. On physical examination, his abdomen was diffusely tender, with the most severe pain in the right upper quadrant. Computed tomography of the abdomen revealed extensive portal venous gas. What is the most common underlying cause?
\vspace{12pt}

\textbf{Options:}
\begin{enumerate}
\item[A.] Inflammatory bowel disease
\item[B.] Colonoscopy
\item[C.] Bowel ischemia
\item[D.] Intra-abdominal abscess
\item[E.] Peptic ulcer disease
\end{enumerate}

\textbf{Image:}
\begin{center}
\includegraphics[width=0.95\textwidth,height=0.50\textheight,width=0.90\textwidth,keepaspectratio]{images/nejm_20181115.jpg}
\end{center}
\vspace{12pt}
\newpage

\section*{Question 826 (ID: 20181122)}
\textbf{Date: }November 22,2018
\vspace{6pt}

The following is a cast of what anatomical structure?
\vspace{12pt}

\textbf{Options:}
\begin{enumerate}
\item[A.] Portal vein
\item[B.] Left bronchial tree
\item[C.] Femoral vein
\item[D.] Hepatic artery
\item[E.] Right bronchial tree
\end{enumerate}

\textbf{Image:}
\begin{center}
\includegraphics[width=0.73\textwidth,height=0.50\textheight,width=0.90\textwidth,keepaspectratio]{images/nejm_20181122.jpg}
\end{center}
\vspace{12pt}
\newpage

\section*{Question 827 (ID: 20181129)}
\textbf{Date: }November 29,2018
\vspace{6pt}

A 58-year-old man presented to an outpatient clinic with a 2-year history of progressive hoarseness and swelling on the left side of his neck. He worked as a farmer with no history of tobacco use. Physical examination was remarkable for nontender, compressible swelling in the left cervical region. What is the most likely diagnosis?
\vspace{12pt}

\textbf{Options:}
\begin{enumerate}
\item[A.] Saccular cyst
\item[B.] Laryngocele
\item[C.] Thyroglossal cyst
\item[D.] Scrofula
\item[E.] Lymphadenopathy
\end{enumerate}

\textbf{Image:}
\begin{center}
\includegraphics[width=0.95\textwidth,height=0.50\textheight,width=0.90\textwidth,keepaspectratio]{images/nejm_20181129.jpg}
\end{center}
\vspace{12pt}
\newpage

\section*{Question 828 (ID: 20181206)}
\textbf{Date: }December 06,2018
\vspace{6pt}

A 67-year-old woman presented to the emergency department with a 6-week history of progressive exertional dyspnea. Her medical history was notable for lung transplantation that had been performed 8 years earlier. Laboratory studies showed normocytic anemia, with a hemoglobin level of 6.9 g per deciliter (reference range, 11.9 to 17.2). White-cell and platelet counts were normal. The reticulocyte index was 0\%. Bone marrow aspiration was performed and showed giant proerythroblasts with basophilic and vacuolated cytoplasm, uncondensed chromatin, and large, intranuclear, purple-colored inclusions. What is the diagnosis?
\vspace{12pt}

\textbf{Options:}
\begin{enumerate}
\item[A.] Chronic lung allograft dysfunction
\item[B.] Acute promyelocytic leukemia
\item[C.] Adverse reaction to an immunosuppressant
\item[D.] Parvovirus B19 infection
\item[E.] Acute hepatitis B infection
\end{enumerate}

\textbf{Image:}
\begin{center}
\includegraphics[width=0.95\textwidth,height=0.50\textheight,width=0.90\textwidth,keepaspectratio]{images/nejm_20181206.jpg}
\end{center}
\vspace{12pt}
\newpage

\section*{Question 829 (ID: 20181213)}
\textbf{Date: }December 13,2018
\vspace{6pt}

An 18-year-old man presented to the emergency department with chest pain. What is the most likely childhood diagnosis associated with the findings of this coronary angiogram?
\vspace{12pt}

\textbf{Options:}
\begin{enumerate}
\item[A.] Kawasaki’s disease
\item[B.] Rheumatic heart disease
\item[C.] Marfan’s syndrome
\item[D.] Familial hypercholesterolemia
\item[E.] Hodgkin’s lymphoma
\end{enumerate}

\textbf{Image:}
\begin{center}
\includegraphics[width=0.95\textwidth,height=0.50\textheight,width=0.90\textwidth,keepaspectratio]{images/nejm_20181213.jpg}
\end{center}
\vspace{12pt}
\newpage

\section*{Question 830 (ID: 20181220)}
\textbf{Date: }December 20,2018
\vspace{6pt}

An 84-year-old man presented with fever, malaise, and discoloration and pain in his fingers and toes ongoing for 2 weeks. He had no history of smoking. On physical examination, he was afebrile with blue-black discoloration on several fingers of his left hand and purpuric lesions on both hands and feet. He had palpable radial and dorsalis pedis pulses on both sides. What is the most likely diagnosis?
\vspace{12pt}

\textbf{Options:}
\begin{enumerate}
\item[A.] Infection with Vibrio vulnificus
\item[B.] Polyarteritis nodosa
\item[C.] Necrotizing fasciitis
\item[D.] Herpetic whitlow
\item[E.] Arterial embolism
\end{enumerate}

\textbf{Image:}
\begin{center}
\includegraphics[width=0.95\textwidth,height=0.50\textheight,width=0.90\textwidth,keepaspectratio]{images/nejm_20181220.jpg}
\end{center}
\vspace{12pt}
\newpage

\section*{Question 831 (ID: 20181227)}
\textbf{Date: }December 27,2018
\vspace{6pt}

A 29-year-old man presented to the dermatology clinic with a pruritic, erythematous, and scaly rash that had first appeared 2 years earlier. He had sought no medical treatment until this presentation. His medical history included eczema during childhood and seasonal allergies. Physical examination showed erythematous, violaceous plaques that involved more than 90\% of the patient’s body-surface area, with some areas that were spared and reflect the baseline appearance of the patient’s skin. What is the diagnosis?
\vspace{12pt}

\textbf{Options:}
\begin{enumerate}
\item[A.] Seborrheic dermatitis
\item[B.] Pityriasis rubra pilaris
\item[C.] Erytherma multiforme
\item[D.] Erythrodermic psoriasis
\item[E.] Toxic epidermal necrolysis
\end{enumerate}

\textbf{Image:}
\begin{center}
\includegraphics[width=0.95\textwidth,height=0.50\textheight,width=0.90\textwidth,keepaspectratio]{images/nejm_20181227.jpg}
\end{center}
\vspace{12pt}
\newpage

\section*{Question 832 (ID: 20190103)}
\textbf{Date: }January 03,2019
\vspace{6pt}

A 32-year-old man presented to the emergency department with difficulty swallowing oral secretions and the feeling that food was stuck in his throat after he ate a pizza roll. Upper endoscopy revealed the following image after food impaction was removed. This finding is associated with what underlying diagnosis?
\vspace{12pt}

\textbf{Options:}
\begin{enumerate}
\item[A.] Gastroesophageal reflux disease
\item[B.] Crohn's disease
\item[C.] Barrett's esophagus
\item[D.] Eosinophilic esophagitis
\item[E.] Plummer Vinson syndrome
\end{enumerate}

\textbf{Image:}
\begin{center}
\includegraphics[width=0.95\textwidth,height=0.50\textheight,width=0.90\textwidth,keepaspectratio]{images/nejm_20190103.jpg}
\end{center}
\vspace{12pt}
\newpage

\section*{Question 833 (ID: 20190110)}
\textbf{Date: }January 10,2019
\vspace{6pt}

A 73-year-old woman presented to the dermatology clinic with an 11-month history of an evolving pruritic, erythematous rash on her thighs and buttocks. On physical examination she was noted to have polycyclic erythematous plaques. What is the most likely diagnosis?
\vspace{12pt}

\textbf{Options:}
\begin{enumerate}
\item[A.] Dermatomyositis
\item[B.] Subacute lupus erythematosus
\item[C.] Erythema gyratum repens
\item[D.] Tinea versicolor
\item[E.] Necrolytic migratory erythema
\end{enumerate}

\textbf{Image:}
\begin{center}
\includegraphics[width=0.59\textwidth,height=0.50\textheight,width=0.90\textwidth,keepaspectratio]{images/nejm_20190110.jpg}
\end{center}
\vspace{12pt}
\newpage

\section*{Question 834 (ID: 20190117)}
\textbf{Date: }January 17,2019
\vspace{6pt}

A 41-year-old woman presented to the ophthalmology clinic with vision that had been deteriorating during the preceding 20 years. Her subjective refraction showed that a hyperopic shift had occurred since her current corrective lenses had been prescribed. Her best corrected visual acuity was 20/25 in both eyes. Slit-lamp examination was performed, with the following corneal appearance. What is the diagnosis?
\vspace{12pt}

\textbf{Options:}
\begin{enumerate}
\item[A.] Herpes simplex keratitis
\item[B.] Trauma related corneal injury
\item[C.] Laser-assisted in situ keratomileusis
\item[D.] Radial keratotomy
\item[E.] Vitamin A deficiency
\end{enumerate}

\textbf{Image:}
\begin{center}
\includegraphics[width=0.95\textwidth,height=0.50\textheight,width=0.90\textwidth,keepaspectratio]{images/nejm_20190117.jpg}
\end{center}
\vspace{12pt}
\newpage

\section*{Question 835 (ID: 20190124)}
\textbf{Date: }January 24,2019
\vspace{6pt}

A 45-year-old man presented to the emergency department with sudden-onset abdominal pain and vomiting. On physical examination, his abdomen was diffusely tender, and he had hyperpigmented macules on the lips, oral mucosa, and nose. What is the most likely underlying diagnosis?
\vspace{12pt}

\textbf{Options:}
\begin{enumerate}
\item[A.] Peutz-Jeghers syndrome
\item[B.] Henoch-Schönlein purpura
\item[C.] Syphilis
\item[D.] Melanoacanthoma
\item[E.] Crohn’s disease
\end{enumerate}

\textbf{Image:}
\begin{center}
\includegraphics[width=0.95\textwidth,height=0.50\textheight,width=0.90\textwidth,keepaspectratio]{images/nejm_20190124.jpg}
\end{center}
\vspace{12pt}
\newpage

\section*{Question 836 (ID: 20190131)}
\textbf{Date: }January 31,2019
\vspace{6pt}

A 53-year-old man presented to the infectious disease clinic with a 4-month history of progressively enlarging, painless nodules on his scalp and perianal region. He had undergone liver and kidney transplantation 2 years ago and was receiving tacrolimus, mycophenolate mofetil, and prednisolone for immunosuppression. What is the most likely diagnosis?
\vspace{12pt}

\textbf{Options:}
\begin{enumerate}
\item[A.] Drug-induced reaction
\item[B.] Malakoplakia
\item[C.] Pyoderma gangrenosum
\item[D.] Sarcoidosis
\item[E.] Pyogenic granuloma
\end{enumerate}

\textbf{Image:}
\begin{center}
\includegraphics[width=0.74\textwidth,height=0.50\textheight,width=0.90\textwidth,keepaspectratio]{images/nejm_20190131.jpg}
\end{center}
\vspace{12pt}
\newpage

\section*{Question 837 (ID: 20190207)}
\textbf{Date: }February 07,2019
\vspace{6pt}

A 3-year-old girl presented to the emergency department after she had ingested a metal pendant. She had not vomited and had no pain in her chest. A physical examination was unremarkable. A radiograph of the chest confirmed a heart-shaped foreign body in the proximal thoracic esophagus. Ingestions of foreign bodies are most commonly reported in children 1 to 3 years of age. Which of the following ingested objects warrants immediate endoscopic removal?
\vspace{12pt}

\textbf{Options:}
\begin{enumerate}
\item[A.] Foreign bodies that have been present for longer than 24 hours
\item[B.] Small coins (e.g., a penny)
\item[C.] Chewing gum
\item[D.] Smooth plastic button
\item[E.] Pencil-top eraser
\end{enumerate}

\textbf{Image:}
\begin{center}
\includegraphics[width=0.95\textwidth,height=0.50\textheight,width=0.90\textwidth,keepaspectratio]{images/nejm_20190207.jpg}
\end{center}
\vspace{12pt}
\newpage

\section*{Question 838 (ID: 20190214)}
\textbf{Date: }February 14,2019
\vspace{6pt}

A 65-year-old man who was otherwise healthy presented to the dermatology clinic with a 5-month history of a progressively enlarging annular rash over his right hand and forearm. He worked as a farmer in southern China and noted that the erythema had developed after trauma to his hand from agricultural work. What is the most likely diagnosis?
\vspace{12pt}

\textbf{Options:}
\begin{enumerate}
\item[A.] Cutaneous leishmaniasis
\item[B.] Sarcoidosis
\item[C.] Sporotrichosis
\item[D.] Tinea versicolor
\item[E.] Scrofuladerma
\end{enumerate}

\textbf{Image:}
\begin{center}
\includegraphics[width=0.77\textwidth,height=0.50\textheight,width=0.90\textwidth,keepaspectratio]{images/nejm_20190214.jpg}
\end{center}
\vspace{12pt}
\newpage

\section*{Question 839 (ID: 20190221)}
\textbf{Date: }February 21,2019
\vspace{6pt}

A 54-year-old man presented to the oral and maxillofacial clinic with a 2-month history of difficulty chewing his food. He reported a painless brown lesion had grown on his tongue in the center of a white patch that had been present for years. Examination of the oral cavity revealed a well-circumscribed hard mass, measuring 8 mm by 7 mm and surrounded by a white patch on the right side of the tongue. What is the most likely diagnosis?
\vspace{12pt}

\textbf{Options:}
\begin{enumerate}
\item[A.] Dermoid cyst
\item[B.] Oral candidiasis
\item[C.] Lipoma
\item[D.] Spindle-cell sarcoma
\item[E.] Tongue mucocele
\end{enumerate}

\textbf{Image:}
\begin{center}
\includegraphics[width=0.95\textwidth,height=0.50\textheight,width=0.90\textwidth,keepaspectratio]{images/nejm_20190221.jpg}
\end{center}
\vspace{12pt}
\newpage

\section*{Question 840 (ID: 20190228)}
\textbf{Date: }February 28,2019
\vspace{6pt}

An otherwise healthy 9-year-old girl presented to the primary care clinic with a sore throat and fever. Her temperature was 38.5°C. Physical examination revealed swollen and tender cervical lymph nodes, an inflamed uvula, enlarged tonsils, and “doughnut” lesions on both the hard and soft palates. What is the underlying cause?
\vspace{12pt}

\textbf{Options:}
\begin{enumerate}
\item[A.] Group A streptococcus
\item[B.] Epstein-Barr virus
\item[C.] Adenovirus
\item[D.] Measles virus
\item[E.] Neisseria gonorrhea
\end{enumerate}

\textbf{Image:}
\begin{center}
\includegraphics[width=0.95\textwidth,height=0.50\textheight,width=0.90\textwidth,keepaspectratio]{images/nejm_20190228.jpg}
\end{center}
\vspace{12pt}
\newpage

\section*{Question 841 (ID: 20190307)}
\textbf{Date: }March 07,2019
\vspace{6pt}

A 73-year-old woman presented to the emergency department with a painful umbilical nodule that had been enlarging over the past 4 months. Physical examination revealed a painful and firm erythematous umbilical nodule measuring 2 cm in its largest diameter. What is the likely diagnosis?
\vspace{12pt}

\textbf{Options:}
\begin{enumerate}
\item[A.] Keloid
\item[B.] Umbilical hernia
\item[C.] Pyoderma gangrenosum
\item[D.] Pyogenic granuloma
\item[E.] Sister Mary Joseph's nodule
\end{enumerate}

\textbf{Image:}
\begin{center}
\includegraphics[width=0.95\textwidth,height=0.50\textheight,width=0.90\textwidth,keepaspectratio]{images/nejm_20190307.jpg}
\end{center}
\vspace{12pt}
\newpage

\section*{Question 842 (ID: 20190314)}
\textbf{Date: }March 14,2019
\vspace{6pt}

An 18-year-old man presented to the emergency department with a 1-week history of sore throat, fever, and malaise and a 3-day history of pleuritic chest pain and productive cough. He was found to have multiple pulmonary cavitary lesions and a right internal jugular vein thrombus. What is the most common pathogen isolated from patients with this condition?
\vspace{12pt}

\textbf{Options:}
\begin{enumerate}
\item[A.] Staphylococcus aureus
\item[B.] Aspergillus fumigatus
\item[C.] Pseudomonas aeruginosa
\item[D.] Fusobacterium necrophorum
\item[E.] Nocardia asteroides
\end{enumerate}

\textbf{Image:}
\begin{center}
\includegraphics[width=0.95\textwidth,height=0.50\textheight,width=0.90\textwidth,keepaspectratio]{images/nejm_20190314.jpg}
\end{center}
\vspace{12pt}
\newpage

\section*{Question 843 (ID: 20190321)}
\textbf{Date: }March 21,2019
\vspace{6pt}

An 18-year-old man presented to the emergency department with generalized tonic-clonic seizures. On physical examination, the patient was confused. He had swelling over the right eye and tenderness in the right testis. Magnetic resonance imaging of the head showed numerous well-defined cystic lesions throughout the cerebral cortex. What is the diagnosis?
\vspace{12pt}

\textbf{Options:}
\begin{enumerate}
\item[A.] Cerebral metastases
\item[B.] Progressive multifocal leukoencephalopathy
\item[C.] Cerebral vasculitis
\item[D.] Neurosyphilis
\item[E.] Neurocysticercosis
\end{enumerate}

\textbf{Image:}
\begin{center}
\includegraphics[width=0.72\textwidth,height=0.50\textheight,width=0.90\textwidth,keepaspectratio]{images/nejm_20190321.jpg}
\end{center}
\vspace{12pt}
\newpage

\section*{Question 844 (ID: 20190328)}
\textbf{Date: }March 28,2019
\vspace{6pt}

An otherwise healthy 10-year-old girl presented to the primary care clinic with a 10-day history of multiple itchy papules on the soles of her feet and on her toes. The lesions had black dots in the center and were painful. Two weeks earlier, the family had traveled to rural Brazil. During that time, the patient had played in a pigsty without wearing shoes. Sand fleas were removed from multiple lesions. What is the most likely diagnosis?
\vspace{12pt}

\textbf{Options:}
\begin{enumerate}
\item[A.] Coxsackievirus infection
\item[B.] Furuncular myiasis
\item[C.] Foreign body granulomas
\item[D.] Tungiasis
\item[E.] Scabies infestation
\end{enumerate}

\textbf{Image:}
\begin{center}
\includegraphics[width=0.7\textwidth,height=0.50\textheight,width=0.90\textwidth,keepaspectratio]{images/nejm_20190328.jpg}
\end{center}
\vspace{12pt}
\newpage

\section*{Question 845 (ID: 20190404)}
\textbf{Date: }April 04,2019
\vspace{6pt}

A 48-year-old man presented to the ophthalmology clinic with a 1-week history of pain, double vision, and blurred vision in his left eye. What is the abnormality depicted here?
\vspace{12pt}

\textbf{Options:}
\begin{enumerate}
\item[A.] Separation of the iris
\item[B.] Separation of the pupil
\item[C.] Separation of the lens
\item[D.] Normal eye finding
\item[E.] Separation of the sclerae
\end{enumerate}

\textbf{Image:}
\begin{center}
\includegraphics[width=0.95\textwidth,height=0.50\textheight,width=0.90\textwidth,keepaspectratio]{images/nejm_20190404.jpg}
\end{center}
\vspace{12pt}
\newpage

\section*{Question 846 (ID: 20190411)}
\textbf{Date: }April 11,2019
\vspace{6pt}

A 35-year-old man presented to the emergency department with a 2-day history of abdominal pain, headache, and brown urine. He was alert and oriented to time, place, and self. He had no purpura or petechiae. Laboratory evaluation revealed a hemoglobin level of 8.6 g per deciliter (normal range, 13.7 to 17.5) and a platelet count of 6000 per cubic millimeter (normal range, 140,000 to 370,000). Levels of indirect bilirubin and lactate dehydrogenase were elevated, haptoglobin was undetectable, and the serum creatinine level was 1 mg per deciliter (88 μmol per liter; normal range, 0.6 to 1.3 mg per deciliter [57 to 115 μmol per liter]). Test results for infection with the human immunodeficiency virus were negative. A peripheral-blood smear showed numerous schistocytes. What is the diagnosis?
\vspace{12pt}

\textbf{Options:}
\begin{enumerate}
\item[A.] Idiopathic thrombocytopenic purpura
\item[B.] Thrombotic thrombocytopenic purpura
\item[C.] Pseudothrombocytopenia
\item[D.] Hemolytic-uremic syndrome
\item[E.] Excessive splenic platelet sequestration
\end{enumerate}

\textbf{Image:}
\begin{center}
\includegraphics[width=0.95\textwidth,height=0.50\textheight,width=0.90\textwidth,keepaspectratio]{images/nejm_20190411.jpg}
\end{center}
\vspace{12pt}
\newpage

\section*{Question 847 (ID: 20190418)}
\textbf{Date: }April 18,2019
\vspace{6pt}

A 60-year-old woman presented to the ophthalmology clinic after noticing central blind spots in the visual fields of both eyes. She had a history of rheumatoid arthritis, which had been treated with hydroxychloroquine for 14 years. The retinal examination showed the pattern below. What is the most likely underlying cause of this abnormality?
\vspace{12pt}

\textbf{Options:}
\begin{enumerate}
\item[A.] Rheumatoid arthritis
\item[B.] Hydroxychloroquine toxicity
\item[C.] Age-related macular degeneration
\item[D.] Type 2 diabetes
\item[E.] Bardet-Biedl syndrome
\end{enumerate}

\textbf{Image:}
\begin{center}
\includegraphics[width=0.95\textwidth,height=0.50\textheight,width=0.90\textwidth,keepaspectratio]{images/nejm_20190418.jpg}
\end{center}
\vspace{12pt}
\newpage

\section*{Question 848 (ID: 20190425)}
\textbf{Date: }April 25,2019
\vspace{6pt}

A 37-year-old man presented with a 4-year history of excessive sweating, headaches, and joint pain. His wife had also noticed increasing skin folds on his scalp. What is the most likely diagnosis?
\vspace{12pt}

\textbf{Options:}
\begin{enumerate}
\item[A.] Acromegaly
\item[B.] Amyloidosis
\item[C.] Systemic sclerosis
\item[D.] Hypogonadism
\item[E.] Sarcoidosis
\end{enumerate}

\textbf{Image:}
\begin{center}
\includegraphics[width=0.95\textwidth,height=0.50\textheight,width=0.90\textwidth,keepaspectratio]{images/nejm_20190425.jpg}
\end{center}
\vspace{12pt}
\newpage

\section*{Question 849 (ID: 20190502)}
\textbf{Date: }May 02,2019
\vspace{6pt}

A 59-year-old woman presented to the emergency department with a 4-day history of inflammation and pain in the right eye. She had been blind in the eye for several years before presentation. Magnetic resonance imaging revealed a right orbital mass. Abdominal and thoracic imaging showed numerous hepatic masses, abdominal and thoracic lymphadenopathy, and vertebral sclerotic osseous disease. The right eye was enucleated for palliative relief and to obtain tissue for diagnosis. What is the diagnosis?
\vspace{12pt}

\textbf{Options:}
\begin{enumerate}
\item[A.] Amaurosis fugax
\item[B.] T-cell lymphoma
\item[C.] Uveal melanoma
\item[D.] Vitreous hemorrhage
\item[E.] Disseminated aspergillosis
\end{enumerate}

\textbf{Image:}
\begin{center}
\includegraphics[width=0.9\textwidth,height=0.50\textheight,width=0.90\textwidth,keepaspectratio]{images/nejm_20190502.jpg}
\end{center}
\vspace{12pt}
\newpage

\section*{Question 850 (ID: 20190509)}
\textbf{Date: }May 09,2019
\vspace{6pt}

A 51-year-old woman presented to the primary care clinic with mild worsening of chronic abdominal pain. On physical examination, the liver was enlarged. Magnetic resonance cholangiopancreatography revealed a markedly enlarged liver with numerous cystic structures (a three-dimensional maximum-intensity-projection reconstruction is shown in the figure). What is the most likely underlying diagnosis?
\vspace{12pt}

\textbf{Options:}
\begin{enumerate}
\item[A.] Alpha1-antitrypsin deficiency
\item[B.] Amebiasis
\item[C.] Autosomal dominant polycystic kidney disease
\item[D.] Hepatocellular adenomas
\item[E.] Tuberous sclerosis
\end{enumerate}

\textbf{Image:}
\begin{center}
\includegraphics[width=0.88\textwidth,height=0.50\textheight,width=0.90\textwidth,keepaspectratio]{images/nejm_20190509.jpg}
\end{center}
\vspace{12pt}
\newpage

\section*{Question 851 (ID: 20190516)}
\textbf{Date: }May 16,2019
\vspace{6pt}

A baby boy born at 39 weeks of gestation had bilious emesis, failure to pass meconium, and abdominal distention within 24 hours after birth. A water-soluble contrast enema showed a uniformly distended and shortened colon. What is the most likely diagnosis?
\vspace{12pt}

\textbf{Options:}
\begin{enumerate}
\item[A.] Hirschsprung’s disease
\item[B.] Congenital syphilis
\item[C.] Pyloric stenosis
\item[D.] Duodenal atresia
\item[E.] Necrotizing enterocolitis
\end{enumerate}

\textbf{Image:}
\begin{center}
\includegraphics[width=0.82\textwidth,height=0.50\textheight,width=0.90\textwidth,keepaspectratio]{images/nejm_20190516.jpg}
\end{center}
\vspace{12pt}
\newpage

\section*{Question 852 (ID: 20190523)}
\textbf{Date: }May 23,2019
\vspace{6pt}

A 47-year-old man presented to the cardiology clinic for evaluation of coronary-artery dilatations detected on computed tomography (CT). The patient was asymptomatic but had a history of IgG4-related disease. Laboratory tests showed a normal serum IgG4 concentration. CT angiogram of the coronary vessels revealed aneurysmal dilatation of the right coronary artery with marked, periarterial soft-tissue thickening. What is the diagnosis?
\vspace{12pt}

\textbf{Options:}
\begin{enumerate}
\item[A.] Cardiac amyloidosis
\item[B.] IgG4-related disease arteritis
\item[C.] Coronary artery disease
\item[D.] Giant cell arteritis
\item[E.] Takayasu’s arteritis
\end{enumerate}

\textbf{Image:}
\begin{center}
\includegraphics[width=0.88\textwidth,height=0.50\textheight,width=0.90\textwidth,keepaspectratio]{images/nejm_20190523.jpg}
\end{center}
\vspace{12pt}
\newpage

\section*{Question 853 (ID: 20190530)}
\textbf{Date: }May 30,2019
\vspace{6pt}

A 34-year-old man with a history of human immunodeficiency virus (HIV) infection presented to the emergency department with a 1-week history of headache, fever, and confusion. On physical examination, a large, ulcerative lesion was noted on his tongue (Panel A). The patient's CD4 count was 39 cells per cubic millimeter (reference range, 500 to 1450), and his HIV viral load was 197,000 copies per milliliter. A chest radiograph showed patchy infiltrates in both lungs. A biopsy specimen of the tongue lesion was obtained (Panel B). What is the most likely diagnosis?
\vspace{12pt}

\textbf{Options:}
\begin{enumerate}
\item[A.] Sarcoidosis
\item[B.] Lingual tuberculosis
\item[C.] Disseminated coccidioidomycosis
\item[D.] Disseminated aspergillosis
\item[E.] B-cell lymphoma
\end{enumerate}

\textbf{Image:}
\begin{center}
\includegraphics[width=0.95\textwidth,height=0.50\textheight,width=0.90\textwidth,keepaspectratio]{images/nejm_20190530.jpg}
\end{center}
\vspace{12pt}
\newpage

\section*{Question 854 (ID: 20190606)}
\textbf{Date: }June 06,2019
\vspace{6pt}

A 66-year-old man presented to the emergency department with a 3-day history of cough, congestion, and pain in his chest and on the left side of his abdomen. What is the abnormality noted incidentally on chest radiograph and computed tomography of the abdomen?
\vspace{12pt}

\textbf{Options:}
\begin{enumerate}
\item[A.] Tuberculosis
\item[B.] Sinus solitus
\item[C.] Lymphoma
\item[D.] Situs inversus totalis
\item[E.] Sarcoidosis
\end{enumerate}

\textbf{Image:}
\begin{center}
\includegraphics[width=0.95\textwidth,height=0.50\textheight,width=0.90\textwidth,keepaspectratio]{images/nejm_20190606.jpg}
\end{center}
\vspace{12pt}
\newpage

\section*{Question 855 (ID: 20190613)}
\textbf{Date: }June 13,2019
\vspace{6pt}

A previously healthy 48-year-old man presented to the emergency department with acute onset of pain in both flanks. During the preceding 6 months he had had an unintentional weight loss of approximately 5 kg. On physical examination, the following nail changes were noted. What is the finding?
\vspace{12pt}

\textbf{Options:}
\begin{enumerate}
\item[A.] Nicotine staining
\item[B.] Splinter hemorrhages
\item[C.] Ungal melanoma
\item[D.] Janeway lesion
\item[E.] Onycholysis
\end{enumerate}

\textbf{Image:}
\begin{center}
\includegraphics[width=0.95\textwidth,height=0.50\textheight,width=0.90\textwidth,keepaspectratio]{images/nejm_20190613.jpg}
\end{center}
\vspace{12pt}
\newpage

\section*{Question 856 (ID: 20190620)}
\textbf{Date: }June 20,2019
\vspace{6pt}

A 68-year-old woman presented to the otorhinolaryngology clinic with a 5-year history of swelling in the neck and the recent development of discomfort when she swallowed. What is the most likely diagnosis?
\vspace{12pt}

\textbf{Options:}
\begin{enumerate}
\item[A.] Pharyngeal abscess
\item[B.] Thyroglossal duct cyst
\item[C.] Reactive viral lymphadenopathy
\item[D.] Hodgkin's lymphoma
\item[E.] Hemangiomas
\end{enumerate}

\textbf{Image:}
\begin{center}
\includegraphics[width=0.76\textwidth,height=0.50\textheight,width=0.90\textwidth,keepaspectratio]{images/nejm_20190620.jpg}
\end{center}
\vspace{12pt}
\newpage

\section*{Question 857 (ID: 20190627)}
\textbf{Date: }June 27,2019
\vspace{6pt}

A 32-year-old woman presented to the outpatient clinic with a 2-month history of anal pruritus and bleeding. Her abdominal examination was normal, and rectal examination revealed hemorrhoids. What is the most likely diagnosis?
\vspace{12pt}

\textbf{Options:}
\begin{enumerate}
\item[A.] Hemorrhoids
\item[B.] Enterobiasis
\item[C.] Trichuriasis
\item[D.] Strongyloidiasis
\item[E.] Colorectal cancer
\end{enumerate}

\textbf{Image:}
\begin{center}
\includegraphics[width=0.95\textwidth,height=0.50\textheight,width=0.90\textwidth,keepaspectratio]{images/nejm_20190627.jpg}
\end{center}
\vspace{12pt}
\newpage

\section*{Question 858 (ID: 20190704)}
\textbf{Date: }July 04,2019
\vspace{6pt}

A 68-year-old man presented to the emergency department with a 2-day history of swelling of the neck. One week before presentation, he had a toothache, followed by fever and progressive difficulty and pain with swallowing both solids and liquids. Examination showed right submandibular and submental swelling with marked edema of the floor of the mouth, resulting in superior displacement of the tongue. What is the diagnosis?
\vspace{12pt}

\textbf{Options:}
\begin{enumerate}
\item[A.] Oropharyngeal carcinoma
\item[B.] Sarcoidosis
\item[C.] Ludwig’s angina
\item[D.] Sialolithiasis
\item[E.] IgG4-related disease
\end{enumerate}

\textbf{Image:}
\begin{center}
\includegraphics[width=0.95\textwidth,height=0.50\textheight,width=0.90\textwidth,keepaspectratio]{images/nejm_20190704.jpg}
\end{center}
\vspace{12pt}
\newpage

\section*{Question 859 (ID: 20190711)}
\textbf{Date: }July 11,2019
\vspace{6pt}

A 7-year-old boy presented to the pediatric otolaryngology clinic with a 3-year history of multiple firm, painless, slow-growing nodules on his tongue. He had a history of mild developmental delay, and a physical examination showed a high arched palate and marfanoid habitus. What is the most likely underlying diagnosis?
\vspace{12pt}

\textbf{Options:}
\begin{enumerate}
\item[A.] Gaucher’s disease
\item[B.] Multiple endocrine neoplasia type 2B
\item[C.] Tuberous sclerosis
\item[D.] Familial hypercholesterolemia
\item[E.] Marfan’s syndrome
\end{enumerate}

\textbf{Image:}
\begin{center}
\includegraphics[width=0.95\textwidth,height=0.50\textheight,width=0.90\textwidth,keepaspectratio]{images/nejm_20190711.jpg}
\end{center}
\vspace{12pt}
\newpage

\section*{Question 860 (ID: 20190718)}
\textbf{Date: }July 18,2019
\vspace{6pt}

A 60-year-old man presented to the otolaryngology clinic with a 2-month history of a firm, painless mass on the right side of his neck. He had no history of smoking or heavy alcohol use and denied any changes in voice or difficulty swallowing. Physical examination showed unilateral enlargement of the right tonsil. What is the correct diagnosis?
\vspace{12pt}

\textbf{Options:}
\begin{enumerate}
\item[A.] Peritonsillar abscess
\item[B.] Burkitt’s lymphoma
\item[C.] Tonsillar cancer
\item[D.] Mandibular osteoma
\item[E.] Benign tonsillar hypertrophy
\end{enumerate}

\textbf{Image:}
\begin{center}
\includegraphics[width=0.95\textwidth,height=0.50\textheight,width=0.90\textwidth,keepaspectratio]{images/nejm_20190718.jpg}
\end{center}
\vspace{12pt}
\newpage

\section*{Question 861 (ID: 20190725)}
\textbf{Date: }July 25,2019
\vspace{6pt}

A 48-year-old man presented to the dermatology clinic with a 6-month history of painful hand ulcerations and shortness of breath. He has no muscle weakness or arthritis. What is the diagnosis?
\vspace{12pt}

\textbf{Options:}
\begin{enumerate}
\item[A.] CREST syndrome
\item[B.] Anti-MDA5 dermatomyositis
\item[C.] Pyoderma gangrenosum
\item[D.] Cutaneous polyarteritis nodosa
\item[E.] Behcet’s disease
\end{enumerate}

\textbf{Image:}
\begin{center}
\includegraphics[width=0.95\textwidth,height=0.50\textheight,width=0.90\textwidth,keepaspectratio]{images/nejm_20190725.jpg}
\end{center}
\vspace{12pt}
\newpage

\section*{Question 862 (ID: 20190801)}
\textbf{Date: }August 01,2019
\vspace{6pt}

A 56-year-old woman presented to the emergency department with a 1-month history of abdominal pain in the left upper quadrant. Her medical history was notable for polysubstance use disorder, hepatitis C virus infection, and rheumatoid arthritis managed with naproxen. Computed tomography of the abdomen was performed. What is the most likely diagnosis?
\vspace{12pt}

\textbf{Options:}
\begin{enumerate}
\item[A.] Avascular necrosis of the spleen
\item[B.] Visceral leishmaniasis
\item[C.] Emphysematous infection of the spleen
\item[D.] Gastrosplenic fistulization
\item[E.] Bezoar
\end{enumerate}

\textbf{Image:}
\begin{center}
\includegraphics[width=0.95\textwidth,height=0.50\textheight,width=0.90\textwidth,keepaspectratio]{images/nejm_20190801.jpg}
\end{center}
\vspace{12pt}
\newpage

\section*{Question 863 (ID: 20190808)}
\textbf{Date: }August 08,2019
\vspace{6pt}

A 79-year-old man with a history of ischemic cardiomyopathy and chronic kidney disease presented 1 month after a transcatheter aortic-valve replacement (TAVR) with a serum creatinine level of 4.0 mg per deciliter (350 μmol per liter), up from a baseline of 1.6 mg per deciliter (140 μmol per liter). Renal biopsy is shown. What is the most likely diagnosis?
\vspace{12pt}

\textbf{Options:}
\begin{enumerate}
\item[A.] Calcium oxalate deposition
\item[B.] Ischemic acute tubular necrosis
\item[C.] Renal artery infarction
\item[D.] Cholesterol crystal embolization
\item[E.] Cryoglobulinemic vasculitis
\end{enumerate}

\textbf{Image:}
\begin{center}
\includegraphics[width=0.69\textwidth,height=0.50\textheight,width=0.90\textwidth,keepaspectratio]{images/nejm_20190808.jpg}
\end{center}
\vspace{12pt}
\newpage

\section*{Question 864 (ID: 20190815)}
\textbf{Date: }August 15,2019
\vspace{6pt}

An 80-year-old man presented to the emergency department with a 2-day history of abdominal pain in the right upper quadrant. On physical examination, he had abdominal tenderness in the right upper quadrant, and Murphy’s sign was positive. The alanine aminotransferase, aspartate aminotransferase, bilirubin, and lipase levels were normal. Computed tomography of the abdomen is shown. What is the diagnosis?
\vspace{12pt}

\textbf{Options:}
\begin{enumerate}
\item[A.] Ascending cholangitis
\item[B.] Duodenal diverticulitis
\item[C.] Subhepatic appendicitis
\item[D.] Emphysematous cholecystitis
\item[E.] Hepatic cyst rupture
\end{enumerate}

\textbf{Image:}
\begin{center}
\includegraphics[width=0.95\textwidth,height=0.50\textheight,width=0.90\textwidth,keepaspectratio]{images/nejm_20190815.jpg}
\end{center}
\vspace{12pt}
\newpage

\section*{Question 865 (ID: 20190822)}
\textbf{Date: }August 22,2019
\vspace{6pt}

A fully immunized 48-year-old man presented to an urgent care clinic with a 3-day history of odynophagia, fever, and progressive dyspnea. Physical examination demonstrated stridor and use of accessory muscles for respiration. A lateral radiograph of the neck was performed. What is the most likely diagnosis?
\vspace{12pt}

\textbf{Options:}
\begin{enumerate}
\item[A.] Inhaled foreign body
\item[B.] La Delfa malformation
\item[C.] Epiglottitis
\item[D.] Relapsing polychondritis
\item[E.] Tracheal perforation
\end{enumerate}

\textbf{Image:}
\begin{center}
\includegraphics[width=0.73\textwidth,height=0.50\textheight,width=0.90\textwidth,keepaspectratio]{images/nejm_20190822.jpg}
\end{center}
\vspace{12pt}
\newpage

\section*{Question 866 (ID: 20190829)}
\textbf{Date: }August 29,2019
\vspace{6pt}

A 49-year-old man with a 3-month history of progressive voice changes and pain with swallowing presents to the otorhinolaryngology clinic. He has a history of hypertension and type 2 diabetes mellitus and is a smoker. He denies any recent weight loss or other systemic symptoms. Fiberoptic laryngoscopy was performed and the following image taken. What is the most likely diagnosis?
\vspace{12pt}

\textbf{Options:}
\begin{enumerate}
\item[A.] Vocal-cord granuloma
\item[B.] Carcinoma
\item[C.] Pachydermia larynges
\item[D.] Reinke’s edema
\item[E.] Subglottic stenosis in granulomatosis with polyangiitis
\end{enumerate}

\textbf{Image:}
\begin{center}
\includegraphics[width=0.95\textwidth,height=0.50\textheight,width=0.90\textwidth,keepaspectratio]{images/nejm_20190829.jpg}
\end{center}
\vspace{12pt}
\newpage

\section*{Question 867 (ID: 20190905)}
\textbf{Date: }September 05,2019
\vspace{6pt}

A 29-year-old man with a history of aplastic anemia, who was being treated with cyclosporine, presented to the emergency department with a 4-day history of fever, cough, and difficulty breathing at rest. One day before the onset of symptoms, a pruritic rash had developed on his face, trunk, and limbs. On physical examination, he had a diffuse rash at different stages of development, including papules, vesicles, pustules, and crusted vesicles. What is the diagnosis?
\vspace{12pt}

\textbf{Options:}
\begin{enumerate}
\item[A.] Eosinophilic granulomatosis with polyangiitis
\item[B.] Disseminated varicella infection
\item[C.] Disseminated intravascular coagulation
\item[D.] Drug-induced hypersensitivity syndrome
\item[E.] Immune thrombocytopenic purpura
\end{enumerate}

\textbf{Image:}
\begin{center}
\includegraphics[width=0.66\textwidth,height=0.50\textheight,width=0.90\textwidth,keepaspectratio]{images/nejm_20190905.jpg}
\end{center}
\vspace{12pt}
\newpage

\section*{Question 868 (ID: 20190912)}
\textbf{Date: }September 12,2019
\vspace{6pt}

A 60-year-old woman presented to the emergency department with a 2-month history of blurry vision in her left eye. Her medical history was notable for breast cancer treated 3 years earlier with lumpectomy, axillary-node dissection, radiation, and hormonal therapy. Physical exam on upward gaze is demonstrated. What is the diagnosis?
\vspace{12pt}

\textbf{Options:}
\begin{enumerate}
\item[A.] Orbital metastasis
\item[B.] Unilateral Grave’s ophthalmopathy
\item[C.] Partial cranial nerve 3 palsy
\item[D.] Orbital cellulitis
\item[E.] Idiopathic orbital inflammatory disease
\end{enumerate}

\textbf{Image:}
\begin{center}
\includegraphics[width=0.95\textwidth,height=0.50\textheight,width=0.90\textwidth,keepaspectratio]{images/nejm_20190912.jpg}
\end{center}
\vspace{12pt}
\newpage

\section*{Question 869 (ID: 20190919)}
\textbf{Date: }September 19,2019
\vspace{6pt}

A 70-year-old woman presented with a 1-week history of dizziness and generalized rash following the onset of a viral respiratory tract infection. On physical examination, she had a generalized, macular, nonblanching, purplish rash. What is the most likely diagnosis?
\vspace{12pt}

\textbf{Options:}
\begin{enumerate}
\item[A.] Cutaneous larva migrans
\item[B.] Erythema infectiosum
\item[C.] Cutis marmorata
\item[D.] Cold agglutinin disease
\item[E.] Erythema ab igne
\end{enumerate}

\textbf{Image:}
\begin{center}
\includegraphics[width=0.95\textwidth,height=0.50\textheight,width=0.90\textwidth,keepaspectratio]{images/nejm_20190919.jpg}
\end{center}
\vspace{12pt}
\newpage

\section*{Question 870 (ID: 20190926)}
\textbf{Date: }September 26,2019
\vspace{6pt}

A 46-year-old woman presented to the clinic with a 3-month history of walking difficulty due to worsening knee pain. She had received a diagnosis of rheumatoid arthritis 12 years earlier but had received treatment inconsistently. On physical examination, she had nodular swelling and outward bowing of both knees. She had limited range of motion in her left shoulder and in both wrists and both knees. Anteroposterior radiograph of the knees is shown. What is the diagnosis?
\vspace{12pt}

\textbf{Options:}
\begin{enumerate}
\item[A.] Synovial chondromatosis
\item[B.] Synovial chondrosarcoma
\item[C.] Osteochondritis dissecans
\item[D.] Pigmented villonodular synovitis
\item[E.] Disseminated tuberculosis
\end{enumerate}

\textbf{Image:}
\begin{center}
\includegraphics[width=0.79\textwidth,height=0.50\textheight,width=0.90\textwidth,keepaspectratio]{images/nejm_20190926.jpg}
\end{center}
\vspace{12pt}
\newpage

\section*{Question 871 (ID: 20191003)}
\textbf{Date: }October 03,2019
\vspace{6pt}

A 65-year-old woman presented to the rheumatology service with complaints of progressive swelling in both hands and diffuse body pain. She had a history of lung transplantation 8 years earlier, and had received daily antifungal prophylaxis with voriconazole. On examination, there was prominent joint and soft-tissue swelling and pain in both hands, restricted range of motion in both shoulders, and tenderness on palpation of the bony prominences of the shoulders, elbows, neck, lower back, and knees. What is the most likely diagnosis?
\vspace{12pt}

\textbf{Options:}
\begin{enumerate}
\item[A.] Heterotopic ossification
\item[B.] Post-transplant calcinosis cutis
\item[C.] Malignant fibrous histiocytoma
\item[D.] Voriconazole-induced periostitis
\item[E.] Chronic graft versus host disease polyarthritis
\end{enumerate}

\textbf{Image:}
\begin{center}
\includegraphics[width=0.95\textwidth,height=0.50\textheight,width=0.90\textwidth,keepaspectratio]{images/nejm_20191003.jpg}
\end{center}
\vspace{12pt}
\newpage

\section*{Question 872 (ID: 20191010)}
\textbf{Date: }October 10,2019
\vspace{6pt}

A 64-year-old man presented to the oral medicine clinic with a painful, smooth, red tongue and a burning sensation around his lips that had developed 6 months earlier. No deficits were found on neurologic examination. What is the most likely diagnosis?
\vspace{12pt}

\textbf{Options:}
\begin{enumerate}
\item[A.] Oral lichen planus
\item[B.] Sjögren's syndrome
\item[C.] Squamous-cell carcinoma of the tongue
\item[D.] Pernicious anemia
\item[E.] Amyloidosis
\end{enumerate}

\textbf{Image:}
\begin{center}
\includegraphics[width=0.65\textwidth,height=0.50\textheight,width=0.90\textwidth,keepaspectratio]{images/nejm_20191010.jpg}
\end{center}
\vspace{12pt}
\newpage

\section*{Question 873 (ID: 20191017)}
\textbf{Date: }October 17,2019
\vspace{6pt}

A 77-year-old woman presented to a clinic with a 2-year history of slowly progressive, painful swelling of her fingertips. Physical examination of the hands showed thickening of the skin on her right hand as well as soft-tissue swelling of the tips of the fingers of both hands. A plain radiograph of the hands is shown. What is the most likely diagnosis?
\vspace{12pt}

\textbf{Options:}
\begin{enumerate}
\item[A.] Tophaceous gout
\item[B.] Calcinosis
\item[C.] Metastatic chondrosarcoma
\item[D.] Vitamin D toxicity
\item[E.] Myositis ossificans
\end{enumerate}

\textbf{Image:}
\begin{center}
\includegraphics[width=0.95\textwidth,height=0.50\textheight,width=0.90\textwidth,keepaspectratio]{images/nejm_20191017.jpg}
\end{center}
\vspace{12pt}
\newpage

\section*{Question 874 (ID: 20191024)}
\textbf{Date: }October 24,2019
\vspace{6pt}

A previously healthy 36-year-old woman presented to the infectious diseases clinic with a 3-day history of fever, arthralgias, myalgias, and headache. She denied any recent travel or new sexual partners but admitted to having a pet rat. Physical examination revealed swollen and tender joints as well as a maculopapular rash on the feet and hands, with pustule formation. What is the most likely diagnosis?
\vspace{12pt}

\textbf{Options:}
\begin{enumerate}
\item[A.] The plague
\item[B.] Rat bite fever
\item[C.] Leptospirosis
\item[D.] Tularemia
\item[E.] Hantavirus
\end{enumerate}

\textbf{Image:}
\begin{center}
\includegraphics[width=0.64\textwidth,height=0.50\textheight,width=0.90\textwidth,keepaspectratio]{images/nejm_20191024.jpg}
\end{center}
\vspace{12pt}
\newpage

\section*{Question 875 (ID: 20191031)}
\textbf{Date: }October 31,2019
\vspace{6pt}

A 52-year old man presented to the clinic with a bleeding mass on his right great toe. Two months earlier, the patient had noticed a pink nodule on the dorsum of the toe. One month later, his toe was stepped on, after which the nodule grew rapidly and bled intermittently. On examination, the lesion was red and ulcerated with surrounding hyperpigmentation. What is the diagnosis?
\vspace{12pt}

\textbf{Options:}
\begin{enumerate}
\item[A.] Tophaceous gout
\item[B.] Subungual hematoma
\item[C.] Neurofibroma
\item[D.] Subungual exostosis
\item[E.] Acral lentiginous melanoma
\end{enumerate}

\textbf{Image:}
\begin{center}
\includegraphics[width=0.95\textwidth,height=0.50\textheight,width=0.90\textwidth,keepaspectratio]{images/nejm_20191031.jpg}
\end{center}
\vspace{12pt}
\newpage

\section*{Question 876 (ID: 20191107)}
\textbf{Date: }November 07,2019
\vspace{6pt}

A 73-year-old woman presented to the dermatology clinic with a 9-month history of pruritic and painful lesions on the palms of her hands. Physical examination showed sharp demarcation of the folds in the lines of her hands, a velvety appearance of palmar surfaces, and ridging of the skin. What is the most likely diagnosis?
\vspace{12pt}

\textbf{Options:}
\begin{enumerate}
\item[A.] Palmarpsoriasis
\item[B.] Tripe palms
\item[C.] Acromegaly
\item[D.] Palmoplantar keratoderma
\item[E.] Tinea manuum
\end{enumerate}

\textbf{Image:}
\begin{center}
\includegraphics[width=0.95\textwidth,height=0.50\textheight,width=0.90\textwidth,keepaspectratio]{images/nejm_20191107.jpg}
\end{center}
\vspace{12pt}
\newpage

\section*{Question 877 (ID: 20191114)}
\textbf{Date: }November 14,2019
\vspace{6pt}

A previously healthy 22-year-old man presented with a 6-month history of palpitations and chest discomfort. He denied any dyspnea or syncope. On examination, a midsystolic click was heard at the cardiac apex. An image from a transthoracic echocardiogram is shown. What is the most likely diagnosis?
\vspace{12pt}

\textbf{Options:}
\begin{enumerate}
\item[A.] Infective endocarditis valvular vegetation
\item[B.] Intracardiac helminth
\item[C.] Redundant mitral valve
\item[D.] Propagated valvular thrombus
\item[E.] Papillary fibroelastoma
\end{enumerate}

\textbf{Image:}
\begin{center}
\includegraphics[width=0.95\textwidth,height=0.50\textheight,width=0.90\textwidth,keepaspectratio]{images/nejm_20191114.jpg}
\end{center}
\vspace{12pt}
\newpage

\section*{Question 878 (ID: 20191121)}
\textbf{Date: }November 21,2019
\vspace{6pt}

A 6-month-old girl was referred to the hospital for evaluation and treatment of a perirectal mass which had first been noted at 2 weeks of age. Physical examination showed frontal bossing but no rash, hepatosplenomegaly, lymphadenopathy, or limitation of limb movement. A lateral radiograph of the right tibia showed severe periostitis. What is the diagnosis?
\vspace{12pt}

\textbf{Options:}
\begin{enumerate}
\item[A.] Infantile hemangioma
\item[B.] Imperforate hymen
\item[C.] Perianal hamartomatous polyps
\item[D.] Congenital syphilis
\item[E.] Perianal rhabdomyosarcoma
\end{enumerate}

\textbf{Image:}
\begin{center}
\includegraphics[width=0.95\textwidth,height=0.50\textheight,width=0.90\textwidth,keepaspectratio]{images/nejm_20191121.jpg}
\end{center}
\vspace{12pt}
\newpage

\section*{Question 879 (ID: 20191128)}
\textbf{Date: }November 28,2019
\vspace{6pt}

An 84-year-old woman presented with a painless red nodule over the base of the fourth finger on her right hand. The nodule developed rapidly over the course of 1 month and the patient denied trauma to the area or other lesions. What is the most likely diagnosis?
\vspace{12pt}

\textbf{Options:}
\begin{enumerate}
\item[A.] Basal-cell carcinoma
\item[B.] Sporotrichosis
\item[C.] Merkel-cell carcinoma
\item[D.] Furuncle
\item[E.] Pyogenic granuloma
\end{enumerate}

\textbf{Image:}
\begin{center}
\includegraphics[width=0.95\textwidth,height=0.50\textheight,width=0.90\textwidth,keepaspectratio]{images/nejm_20191128.jpg}
\end{center}
\vspace{12pt}
\newpage

\section*{Question 880 (ID: 20191205)}
\textbf{Date: }December 05,2019
\vspace{6pt}

Physical examination of a 1-day-old girl born at 36 weeks of gestation revealed a soft mass protruding from the external genitalia. The mass was noted to increase in size when the infant cried. The physical examination was otherwise normal. What is the diagnosis?
\vspace{12pt}

\textbf{Options:}
\begin{enumerate}
\item[A.] Hydrocolpos
\item[B.] Prolapsed uterus
\item[C.] Interlabial cyst
\item[D.] McKusick-Kaufman syndrome
\item[E.] Rhabdomyosarcoma
\end{enumerate}

\textbf{Image:}
\begin{center}
\includegraphics[width=0.95\textwidth,height=0.50\textheight,width=0.90\textwidth,keepaspectratio]{images/nejm_20191205.jpg}
\end{center}
\vspace{12pt}
\newpage

\section*{Question 881 (ID: 20191212)}
\textbf{Date: }December 12,2019
\vspace{6pt}

A 51-year-old man with history of type 2 diabetes and human immunodeficiency virus infection presented with a 3-year history of worsening swelling of the left wrist associated with pain and numbness in the fingers. Physical examination was notable for an extension deficit in the proximal interphalangeal joint of all four fingers. Magnetic resonance imaging (MRI) showed multiple loose bodies in the ulnar bursal fluid. The patient underwent surgery and this image was taken. What was the diagnosis?
\vspace{12pt}

\textbf{Options:}
\begin{enumerate}
\item[A.] Pigmented villonodular synovitis (PVNS)
\item[B.] Reichel syndrome
\item[C.] Calcium hydroxyapatite deposition disease
\item[D.] Fibrolipomatous hamartoma of ulnar nerve
\item[E.] Mycobacterial tenosynovitis due to Mycobacterium avium complex
\end{enumerate}

\textbf{Image:}
\begin{center}
\includegraphics[width=0.95\textwidth,height=0.50\textheight,width=0.90\textwidth,keepaspectratio]{images/nejm_20191212.jpg}
\end{center}
\vspace{12pt}
\newpage

\section*{Question 882 (ID: 20191219)}
\textbf{Date: }December 19,2019
\vspace{6pt}

A 74-year-old woman was hospitalized with sudden-onset weakness on the left side due to an acute infarct of the right middle cerebral artery. Examination of the oral mucosa and hands demonstrated the following lesions. What is the most likely diagnosis?
\vspace{12pt}

\textbf{Options:}
\begin{enumerate}
\item[A.] Infective endocarditis
\item[B.] Scleroderma
\item[C.] Hereditary hemorrhagic telangiectasia
\item[D.] Levamisole-induced vasculitis
\item[E.] Amyloid angiopathy
\end{enumerate}

\textbf{Image:}
\begin{center}
\includegraphics[width=0.95\textwidth,height=0.50\textheight,width=0.90\textwidth,keepaspectratio]{images/nejm_20191219.jpg}
\end{center}
\vspace{12pt}
\newpage

\section*{Question 883 (ID: 20200102)}
\textbf{Date: }January 02,2020
\vspace{6pt}

A 19-year-old man presented to the dermatology clinic with a 3-month history of an enlarging ulcer on his right hand and multiple tender subcutaneous nodules on the right forearm and elbow. Five months earlier, he had traveled to Ecuador, and worked on an organic produce farm. On physical examination, the ulcer measured 3 cm in diameter and was tender with a firm border. What is the most likely diagnosis?
\vspace{12pt}

\textbf{Options:}
\begin{enumerate}
\item[A.] Syphilitic gumma
\item[B.] Cutaneous leishmaniasis
\item[C.] Basal-cell carcinoma
\item[D.] Tinea manuum
\item[E.] Tropical pyoderma
\end{enumerate}

\textbf{Image:}
\begin{center}
\includegraphics[width=0.95\textwidth,height=0.50\textheight,width=0.90\textwidth,keepaspectratio]{images/nejm_20200102.jpg}
\end{center}
\vspace{12pt}
\newpage

\section*{Question 884 (ID: 20200109)}
\textbf{Date: }January 09,2020
\vspace{6pt}

An 81-year-old man with a history of atrial fibrillation and hypertension presented to the emergency department after a fall. His medications were apixaban, ramipril, bisoprolol, amlodipine, and amiodarone. Physical examination showed discoloration of his nose, cheeks, and forehead. What is the most likely diagnosis?
\vspace{12pt}

\textbf{Options:}
\begin{enumerate}
\item[A.] Methemoglobinemia
\item[B.] Lupus pernio
\item[C.] Silver toxicity
\item[D.] Amiodarone-induced skin discoloration
\item[E.] Eisenmenger syndrome
\end{enumerate}

\textbf{Image:}
\begin{center}
\includegraphics[width=0.73\textwidth,height=0.50\textheight,width=0.90\textwidth,keepaspectratio]{images/nejm_20200109.jpg}
\end{center}
\vspace{12pt}
\newpage

\section*{Question 885 (ID: 20200116)}
\textbf{Date: }January 16,2020
\vspace{6pt}

A 57-year-old man with stage 3b colon cancer on adjuvant chemotherapy noticed progressive darkening of his palms that was not associated with redness, scaling, or pain. The plantar aspects of his feet were minimally affected but the remainder of his skin was normal. What is the diagnosis?
\vspace{12pt}

\textbf{Options:}
\begin{enumerate}
\item[A.] Addison’s disease
\item[B.] Silver toxicity
\item[C.] Ochronosis
\item[D.] Melasma
\item[E.] Fluorouracil side effect
\end{enumerate}

\textbf{Image:}
\begin{center}
\includegraphics[width=0.95\textwidth,height=0.50\textheight,width=0.90\textwidth,keepaspectratio]{images/nejm_20200116.jpg}
\end{center}
\vspace{12pt}
\newpage

\section*{Question 886 (ID: 20200123)}
\textbf{Date: }January 23,2020
\vspace{6pt}

A male neonate with an antenatal diagnosis of Ebstein’s anomaly had respiratory distress with low oxygen saturation immediately after birth at 35 weeks of gestation. He was intubated and received infusions of alprostadil and dobutamine. On the third day of life, a clearly demarcated erythema developed on one side of the body as shown in the image.  During this episode, the physical examination was otherwise unremarkable, with no changes in vital signs. What is the most likely diagnosis?
\vspace{12pt}

\textbf{Options:}
\begin{enumerate}
\item[A.] Nascent hemangioma of infancy
\item[B.] Hypercarbic flush
\item[C.] Infant harlequin syndrome
\item[D.] Reactive hyperemia of heart failure
\item[E.] Klippel-Trénaunay syndrome
\end{enumerate}

\textbf{Image:}
\begin{center}
\includegraphics[width=0.95\textwidth,height=0.50\textheight,width=0.90\textwidth,keepaspectratio]{images/nejm_20200123.jpg}
\end{center}
\vspace{12pt}
\newpage

\section*{Question 887 (ID: 20200130)}
\textbf{Date: }January 30,2020
\vspace{6pt}

A 53-year-old man presented to the emergency department with a 1-month history of relapsing fevers, dyspnea, and rapidly progressive skin lesions. Physical examination revealed painful, indurated nodules with surrounding purple discoloration on his back, trunk, and limbs. 
What is the most likely diagnosis?
\vspace{12pt}

\textbf{Options:}
\begin{enumerate}
\item[A.] Polyarteritis nodosa vasculitis
\item[B.] Lymphoma
\item[C.] Ecthyma gangrenosum
\item[D.] Cutaneous anthrax
\item[E.] Brown recluse spider bites
\end{enumerate}

\textbf{Image:}
\begin{center}
\includegraphics[width=0.95\textwidth,height=0.50\textheight,width=0.90\textwidth,keepaspectratio]{images/nejm_20200130.jpg}
\end{center}
\vspace{12pt}
\newpage

\section*{Question 888 (ID: 20200206)}
\textbf{Date: }February 06,2020
\vspace{6pt}

A 73-year-old woman presented to the emergency department with difficulty in speaking, which had progressively worsened over a period of 4 days. Physical examination showed a large, fluctuant, painless swelling of the anterior floor of the mouth. Contrast-enhanced computed tomography of the neck is shown. What is the diagnosis?
\vspace{12pt}

\textbf{Options:}
\begin{enumerate}
\item[A.] Sublingual epidermoid cyst
\item[B.] Carcinoma of the tongue
\item[C.] Beckwith-Wiedemann syndrome
\item[D.] Diphtheria
\item[E.] Tuberculosis of the tongue
\end{enumerate}

\textbf{Image:}
\begin{center}
\includegraphics[width=0.95\textwidth,height=0.50\textheight,width=0.90\textwidth,keepaspectratio]{images/nejm_20200206.jpg}
\end{center}
\vspace{12pt}
\newpage

\section*{Question 889 (ID: 20200213)}
\textbf{Date: }February 13,2020
\vspace{6pt}

A 74-year-old woman presented to the gastroenterology clinic with a long history of bloating and abdominal distention. On physical examination, the abdomen was found to be distended but was nontender on palpation, with normal bowel sounds. This image shows findings on computed tomography of the abdomen and pelvis. What is the most likely diagnosis?
\vspace{12pt}

\textbf{Options:}
\begin{enumerate}
\item[A.] Diffuse mesenteric hydatid cyst disease
\item[B.] Senile ileus
\item[C.] Pneumatosis cystoides intestinalis
\item[D.] Intestinal lipomatosis
\item[E.] Food bolus obstruction
\end{enumerate}

\textbf{Image:}
\begin{center}
\includegraphics[width=0.95\textwidth,height=0.50\textheight,width=0.90\textwidth,keepaspectratio]{images/nejm_20200213.jpg}
\end{center}
\vspace{12pt}
\newpage

\section*{Question 890 (ID: 20200220)}
\textbf{Date: }February 20,2020
\vspace{6pt}

A 66-year-old man presented to the emergency department with a 4-day history of shortness of breath and swelling in the legs along with recent fatigue and a 15-kg weight loss. Over the course of hospitalization, the patient developed hemodynamic instability and, despite receiving maximal supportive therapy, died in the intensive care unit. Gross pathological examination of the heart is shown. What is the most likely diagnosis?
\vspace{12pt}

\textbf{Options:}
\begin{enumerate}
\item[A.] Staphylococcus aureus septic emboli
\item[B.] Angiosarcoma
\item[C.] Pericardial mesothelioma
\item[D.] Chronic Chagas' heart disease
\item[E.] Cardiac amyloidosis
\end{enumerate}

\textbf{Image:}
\begin{center}
\includegraphics[width=0.79\textwidth,height=0.50\textheight,width=0.90\textwidth,keepaspectratio]{images/nejm_20200220.jpg}
\end{center}
\vspace{12pt}
\newpage

\section*{Question 891 (ID: 20200227)}
\textbf{Date: }February 27,2020
\vspace{6pt}

A 42-year-old man presented to the clinic with a 3-month history of worsening cough, shortness of breath, and fever. Physical examination showed inflamed nasal mucosa and nasal crusting. Wheezes and rales were heard on auscultation. A computed tomographic scan of the face showed extensive destruction of the structural bones of the midface, resulting in a large nasal cavity. What is the diagnosis?
\vspace{12pt}

\textbf{Options:}
\begin{enumerate}
\item[A.] Chronic sinusitis
\item[B.] Cocaine use
\item[C.] Congenital absence of nasal septum
\item[D.] Granulomatosis with polyangiitis
\item[E.] Nasal septal abscess
\end{enumerate}

\textbf{Image:}
\begin{center}
\includegraphics[width=0.95\textwidth,height=0.50\textheight,width=0.90\textwidth,keepaspectratio]{images/nejm_20200227.jpg}
\end{center}
\vspace{12pt}
\newpage

\section*{Question 892 (ID: 20200305)}
\textbf{Date: }March 05,2020
\vspace{6pt}

A 29-year-old man with perinatally acquired human immunodeficiency virus (HIV) infection and intermittent adherence to antiretroviral therapy presented to the hospital with abdominal pain and drenching night sweats. On presentation, his CD4 count was 18 cells per cubic millimeter (reference range, 500 to 1500), and the HIV viral load was undetectable. Physical exam showed severe abdominal distention, splenomegaly, and diffuse abdominal tenderness to palpation. Computed tomography of the abdomen confirmed massive splenomegaly with multifocal infarction of the splenic parenchyma. What is the most likely diagnosis/etiology?
\vspace{12pt}

\textbf{Options:}
\begin{enumerate}
\item[A.] HIV-associated diffuse large B-cell lymphoma
\item[B.] Wandering spleen
\item[C.] Disseminated Mycobacterium avium-intracellulare infection
\item[D.] Chronic hepatitis C coinfection
\item[E.] Multicentric Castleman’s disease
\end{enumerate}

\textbf{Image:}
\begin{center}
\includegraphics[width=0.73\textwidth,height=0.50\textheight,width=0.90\textwidth,keepaspectratio]{images/nejm_20200305.jpg}
\end{center}
\vspace{12pt}
\newpage

\section*{Question 893 (ID: 20200312)}
\textbf{Date: }March 12,2020
\vspace{6pt}

A 19-year-old man presented with a 3-day history of worsening vision in his right eye. He had previously presented at 5 years of age with reduced visual acuity and macular cystic degeneration in both eyes. Examination of the right fundus is shown. Optical coherence tomography revealed a full-thickness retinal detachment. What is the underlying condition?
\vspace{12pt}

\textbf{Options:}
\begin{enumerate}
\item[A.] X-linked retinoschisis
\item[B.] Goldmann-Favre syndrome
\item[C.] Eales disease
\item[D.] Wagner syndrome
\item[E.] Familial exudative vitreoretinopathy
\end{enumerate}

\textbf{Image:}
\begin{center}
\includegraphics[width=0.95\textwidth,height=0.50\textheight,width=0.90\textwidth,keepaspectratio]{images/nejm_20200312.jpg}
\end{center}
\vspace{12pt}
\newpage

\section*{Question 894 (ID: 20200319)}
\textbf{Date: }March 19,2020
\vspace{6pt}

A male neonate, born via cesarean section at 28 weeks of gestation, developed respiratory distress. The infant underwent intubation and surfactant administration through the endotracheal tube. Physical examination showed a giant omphalocele, webbed neck, and deformity of both hands. The pregnancy had been complicated by preeclampsia, and antenatal ultrasonography at 22 weeks of gestation had shown the presence of an omphalocele and polyhydramnios. Chest radiography on the first day of life showed a narrowing of the rib cage with crowding of the ribs in a “coat hanger” appearance. What is the most likely diagnosis/etiology?
\vspace{12pt}

\textbf{Options:}
\begin{enumerate}
\item[A.] Jeune syndrome asphyxiating thoracic dystrophy
\item[B.] Beckwith-Wiedemann syndrome
\item[C.] Respiratory distress syndrome of the newborn
\item[D.] Paternal uniparental disomy 14 (UDP 14 pat) thoracic dysplasia
\item[E.] Short rib polydactyly syndrome
\end{enumerate}

\textbf{Image:}
\begin{center}
\includegraphics[width=0.86\textwidth,height=0.50\textheight,width=0.90\textwidth,keepaspectratio]{images/nejm_20200319.jpg}
\end{center}
\vspace{12pt}
\newpage

\section*{Question 895 (ID: 20200326)}
\textbf{Date: }March 26,2020
\vspace{6pt}

A 31-year-old woman living in Greece presented with a 4-week history of blurred vision and 2 weeks of progressive bulging of her left orbit. On examination there was nontender proptosis of the left eye, paresis of the left abducens nerve, and reduced visual acuity in the left eye. A T2-weighted gadolinium-enhanced magnetic resonance image of the brain is shown. What is the most likely diagnosis?
\vspace{12pt}

\textbf{Options:}
\begin{enumerate}
\item[A.] Retrobulbar abscess
\item[B.] Hydatid cyst
\item[C.] Optic-nerve glioma
\item[D.] Orbital dermoid cyst
\item[E.] Foreign body
\end{enumerate}

\textbf{Image:}
\begin{center}
\includegraphics[width=0.89\textwidth,height=0.50\textheight,width=0.90\textwidth,keepaspectratio]{images/nejm_20200326.jpg}
\end{center}
\vspace{12pt}
\newpage

\section*{Question 896 (ID: 20200402)}
\textbf{Date: }April 02,2020
\vspace{6pt}

A 2-year-old boy who was born at full term and had been delivered with the assistance of forceps presented to the otolaryngology clinic with an 18-month history of flushing on his right cheek that appeared when he ate. Flushing appeared when he ate strawberries in the clinic and was not associated with pain, pruritus, sweating, or respiratory symptoms. The findings from the remainder of the examination were normal. What is the underlying condition?
\vspace{12pt}

\textbf{Options:}
\begin{enumerate}
\item[A.] Frey’s syndrome
\item[B.] Food allergy
\item[C.] Rosacea
\item[D.] Autonomic epilepsy
\item[E.] Carcinoid syndrome
\end{enumerate}

\textbf{Image:}
\begin{center}
\includegraphics[width=0.83\textwidth,height=0.50\textheight,width=0.90\textwidth,keepaspectratio]{images/nejm_20200402.jpg}
\end{center}
\vspace{12pt}
\newpage

\section*{Question 897 (ID: 20200409)}
\textbf{Date: }April 09,2020
\vspace{6pt}

A 45-year-old woman was referred to the outpatient clinic after multiple hepatic lesions were incidentally noted on imaging. She had no related symptoms. The physical examination and laboratory tests were all normal. Enhanced computed tomography of the abdomen had shown multiple small, hypodense, nonenhancing nodules throughout the liver. Magnetic resonance imaging with cholangiopancreatography revealed multiple small, T2-weighted, hyperintense cystic nodules in the liver, without biliary duct communication, creating a “starry sky” appearance. What is the diagnosis?
\vspace{12pt}

\textbf{Options:}
\begin{enumerate}
\item[A.] Acute hepatitis
\item[B.] Multiple biliary hamartomas
\item[C.] Malignant hydatidosis
\item[D.] Multiple liver metastases
\item[E.] Caroli syndrome
\end{enumerate}

\textbf{Image:}
\begin{center}
\includegraphics[width=0.95\textwidth,height=0.50\textheight,width=0.90\textwidth,keepaspectratio]{images/nejm_20200409.jpg}
\end{center}
\vspace{12pt}
\newpage

\section*{Question 898 (ID: 20200416)}
\textbf{Date: }April 16,2020
\vspace{6pt}

A 71-year-old man was hospitalized with altered mental status progressing over the preceding 3 weeks. The patient had a recent diagnosis of adenocarcinoma of the colon with known metastatic lesions in the lung and bones. A gadolinium-enhanced magnetic resonance image of the brain was performed and is shown. What is the most likely diagnosis?
\vspace{12pt}

\textbf{Options:}
\begin{enumerate}
\item[A.] Group D streptococcus abscesses
\item[B.] Neurocysticercosis
\item[C.] Metastatic adenocarcinoma
\item[D.] Tuberculosis
\item[E.] Primary central nervous system lymphoma
\end{enumerate}

\textbf{Image:}
\begin{center}
\includegraphics[width=0.87\textwidth,height=0.50\textheight,width=0.90\textwidth,keepaspectratio]{images/nejm_20200416.jpg}
\end{center}
\vspace{12pt}
\newpage

\section*{Question 899 (ID: 20200423)}
\textbf{Date: }April 23,2020
\vspace{6pt}

An 83-year-old woman presented to the gastroenterology clinic with dysphagia and regurgitation that occurred with every meal associated with postprandial chest pain. For several years she had difficulty swallowing both solids and liquids. Barium esophagram is shown. What is the diagnosis?
\vspace{12pt}

\textbf{Options:}
\begin{enumerate}
\item[A.] Scleroderma
\item[B.] Esophageal stricture
\item[C.] Esophageal web
\item[D.] Esophageal carcinoma
\item[E.] Achalasia
\end{enumerate}

\textbf{Image:}
\begin{center}
\includegraphics[width=0.64\textwidth,height=0.50\textheight,width=0.90\textwidth,keepaspectratio]{images/nejm_20200423.jpg}
\end{center}
\vspace{12pt}
\newpage

\section*{Question 900 (ID: 20200430)}
\textbf{Date: }April 30,2020
\vspace{6pt}

A 34-year-old man with a history of intravenous drug use and hepatitis C infection presented to the ophthalmology clinic with a 1-week history of pain and decreased vision in his right eye. The visual acuity was 20/400 in the right eye and 20/20 in the left eye. Slit-lamp examination of the right eye showed conjunctival injection and inflammation in the anterior chamber. Indirect ophthalmoscopy showed vitreous haze with yellow-white lesions on the retina and optic nerve. Following workup, surgery was performed and a white mass measuring 4 mm by 3 mm by 1 mm was seen adherent to the optic nerve. What is the most likely diagnosis/etiology?
\vspace{12pt}

\textbf{Options:}
\begin{enumerate}
\item[A.] Retinoblastoma
\item[B.] Amelanotic uveal melanoma
\item[C.] Astrocytic hamartoma
\item[D.] Fungal endophthalmitis
\item[E.] Malignant optic nerve prolapse
\end{enumerate}

\textbf{Image:}
\begin{center}
\includegraphics[width=0.8\textwidth,height=0.50\textheight,width=0.90\textwidth,keepaspectratio]{images/nejm_20200430.jpg}
\end{center}
\vspace{12pt}
\newpage

\section*{Question 901 (ID: 20200507)}
\textbf{Date: }May 07,2020
\vspace{6pt}

A 72-year-old woman with a history of esophageal strictures presented to the emergency department with a 3-month history of a rash on both legs. What is the most likely diagnosis?
\vspace{12pt}

\textbf{Options:}
\begin{enumerate}
\item[A.] Scurvy
\item[B.] Immune thrombocytopenic purpura
\item[C.] Covid-19 infection
\item[D.] Rocky mountain spotted fever
\item[E.] Granulomatosis with polyangiitis
\end{enumerate}

\textbf{Image:}
\begin{center}
\includegraphics[width=0.95\textwidth,height=0.50\textheight,width=0.90\textwidth,keepaspectratio]{images/nejm_20200507.jpg}
\end{center}
\vspace{12pt}
\newpage

\section*{Question 902 (ID: 20200514)}
\textbf{Date: }May 14,2020
\vspace{6pt}

An 8-year-old boy receiving treatment for relapsing B-cell leukemia presented with subcutaneous nodules associated with a 1-week history of fever which persisted despite treatment with broad-spectrum intravenous antibiotics and antifungals. Physical examination revealed nodules ranging from 5 to 18 mm in diameter on the chest, back, arms, and legs. What is the diagnosis?
\vspace{12pt}

\textbf{Options:}
\begin{enumerate}
\item[A.] Erythema nodosum
\item[B.] Fusariosis
\item[C.] Dermatofibroma
\item[D.] Leukemia cutis
\item[E.] Disseminated Bacillus Calmette Guerin (BCG) disease
\end{enumerate}

\textbf{Image:}
\begin{center}
\includegraphics[width=0.95\textwidth,height=0.50\textheight,width=0.90\textwidth,keepaspectratio]{images/nejm_20200514.jpg}
\end{center}
\vspace{12pt}
\newpage

\section*{Question 903 (ID: 20200521)}
\textbf{Date: }May 21,2020
\vspace{6pt}

A 69-year-old man with type 2 diabetes mellitus presented to the ophthalmology clinic for a routine ocular examination. He had a history of cataract surgery 3 years ago for a retinal detachment in the right eye. Clinical examination of the right eye showed a white substance covering the upper third of the iris. Visual acuity was 20/400 in the right eye and 20/40 in the left eye. What is the diagnosis?
\vspace{12pt}

\textbf{Options:}
\begin{enumerate}
\item[A.] Loculated hypopyon
\item[B.] Ghost cell glaucoma
\item[C.] Synchysis scintillans
\item[D.] Silicone oil hyperoleon
\item[E.] Iridodialysis
\end{enumerate}

\textbf{Image:}
\begin{center}
\includegraphics[width=0.95\textwidth,height=0.50\textheight,width=0.90\textwidth,keepaspectratio]{images/nejm_20200521.jpg}
\end{center}
\vspace{12pt}
\newpage

\section*{Question 904 (ID: 20200528)}
\textbf{Date: }May 28,2020
\vspace{6pt}

A 5-year-old boy presented to the dermatology clinic with a 2-week history of multiple painless and tense bullae containing clear to slightly hemorrhagic fluid and localized to the scrotum. What is the most likely diagnosis?
\vspace{12pt}

\textbf{Options:}
\begin{enumerate}
\item[A.] Fixed drug eruption
\item[B.] Linear IgA bullous dermatosis
\item[C.] Dermatitis herpetiformis
\item[D.] Cutaneous bullous lupus
\item[E.] Bullous impetigo
\end{enumerate}

\textbf{Image:}
\begin{center}
\includegraphics[width=0.95\textwidth,height=0.50\textheight,width=0.90\textwidth,keepaspectratio]{images/nejm_20200528.jpg}
\end{center}
\vspace{12pt}
\newpage

\section*{Question 905 (ID: 20200604)}
\textbf{Date: }June 04,2020
\vspace{6pt}

A 59-year-old man with metastatic colon cancer presented to the dermatology clinic with a 10-week history of painless and nonpruritic skin lesions coalescing around a large abdominal scar from a hemicolectomy performed 3 years earlier. On examination, the lesions were firm, pink to violaceous in color, and vesicular-appearing. What is the diagnosis?
\vspace{12pt}

\textbf{Options:}
\begin{enumerate}
\item[A.] Herpes zoster
\item[B.] Kaposi’s sarcoma
\item[C.] Cutaneous metastases
\item[D.] Basal cell carcinoma
\item[E.] Cutaneous sarcoidosis
\end{enumerate}

\textbf{Image:}
\begin{center}
\includegraphics[width=0.95\textwidth,height=0.50\textheight,width=0.90\textwidth,keepaspectratio]{images/nejm_20200604.jpg}
\end{center}
\vspace{12pt}
\newpage

\section*{Question 906 (ID: 20200611)}
\textbf{Date: }June 11,2020
\vspace{6pt}

A 78-year-old man presented to the emergency department with weakness on the left side that had developed 90 minutes earlier. An ischemic stroke in the territory of the right middle cerebral artery was diagnosed, and treatment was initiated. After some time, examination showed the following change in the patient’s tongue. What is the most likely diagnosis?
\vspace{12pt}

\textbf{Options:}
\begin{enumerate}
\item[A.] Tissue plasminogen activator-associated angioedema
\item[B.] Lingual artery thrombosis
\item[C.] Lingual dystonia
\item[D.] Unilateral hypoglossal nerve palsy
\item[E.] Melkersson-Rosenthal syndrome
\end{enumerate}

\textbf{Image:}
\begin{center}
\includegraphics[width=0.95\textwidth,height=0.50\textheight,width=0.90\textwidth,keepaspectratio]{images/nejm_20200611.jpg}
\end{center}
\vspace{12pt}
\newpage

\section*{Question 907 (ID: 20200618)}
\textbf{Date: }June 18,2020
\vspace{6pt}

A 26-year-old man presented to the emergency department with a 2-month history of an altered sense of taste associated with malaise, weight loss, and muscle cramps. Physical examination showed white, sharply demarcated, adherent plaques on the sides of the tongue. What is the diagnosis?
\vspace{12pt}

\textbf{Options:}
\begin{enumerate}
\item[A.] Oral hairy leukoplakia
\item[B.] Oral lichen planus
\item[C.] Human papillomavirus infection
\item[D.] Candidiasis
\item[E.] Uremic stomatitis
\end{enumerate}

\textbf{Image:}
\begin{center}
\includegraphics[width=0.61\textwidth,height=0.50\textheight,width=0.90\textwidth,keepaspectratio]{images/nejm_20200618.jpg}
\end{center}
\vspace{12pt}
\newpage

\section*{Question 908 (ID: 20200625)}
\textbf{Date: }June 25,2020
\vspace{6pt}

A 78-year-old man with end-stage renal disease, diabetes mellitus, and microscopic polyangiitis, for which he was taking glucocorticoids, presented to the rheumatology clinic with fever, painful glossitis, and taste abnormalities that had persisted for 1 month. Examination of the tongue showed a large, punched-out, painful ulcer. What is the diagnosis?
\vspace{12pt}

\textbf{Options:}
\begin{enumerate}
\item[A.] Cytomegalovirus infection
\item[B.] Syphilitic chancre
\item[C.] Herpes simplex virus infection
\item[D.] Behcet’s disease
\item[E.] Eosinophilic ulcer
\end{enumerate}

\textbf{Image:}
\begin{center}
\includegraphics[width=0.95\textwidth,height=0.50\textheight,width=0.90\textwidth,keepaspectratio]{images/nejm_20200625.jpg}
\end{center}
\vspace{12pt}
\newpage

\section*{Question 909 (ID: 20200702)}
\textbf{Date: }July 02,2020
\vspace{6pt}

A 41-year-old farmer presented with a 20-year history of slowly growing lesions on the right leg that progressed over time to cause gait abnormality. The lesions were initially painless, pruritic papules on the knee and gradually spread to the dorsum of the foot. Physical examination revealed palpable, coalescent, subcutaneous nodules with verrucous lesions and associated skin changes. What is the diagnosis?
\vspace{12pt}

\textbf{Options:}
\begin{enumerate}
\item[A.] Verruca vulgaris
\item[B.] Hansen’s disease
\item[C.] Phytophotodermatitis
\item[D.] Epidermodysplasia verruciformis
\item[E.] Chromoblastomycosis
\end{enumerate}

\textbf{Image:}
\begin{center}
\includegraphics[width=0.95\textwidth,height=0.50\textheight,width=0.90\textwidth,keepaspectratio]{images/nejm_20200702.jpg}
\end{center}
\vspace{12pt}
\newpage

\section*{Question 910 (ID: 20200709)}
\textbf{Date: }July 09,2020
\vspace{6pt}

A 69-year-old woman presented to the emergency room after experiencing 2 weeks of dizziness. On physical exam she appeared jaundiced, and had hepatomegaly and generalized abdominal tenderness. Rectal exam showed silver-colored stool. She had a hemoglobin level of 7.5 g per deciliter (reference range, 11.0 to 14.5), an elevated total bilirubin level, and an elevated alkaline phosphatase level. What is the diagnosis?
\vspace{12pt}

\textbf{Options:}
\begin{enumerate}
\item[A.] Upper gastrointestinal bleed
\item[B.] Metastatic colon cancer
\item[C.] Heavy-metal toxicity
\item[D.] Liver cirrhosis
\item[E.] Dubin-Johnson syndrome
\end{enumerate}

\textbf{Image:}
\begin{center}
\includegraphics[width=0.71\textwidth,height=0.50\textheight,width=0.90\textwidth,keepaspectratio]{images/nejm_20200709.jpg}
\end{center}
\vspace{12pt}
\newpage

\section*{Question 911 (ID: 20200716)}
\textbf{Date: }July 16,2020
\vspace{6pt}

A 24-year-old woman with a history of HIV, currently on antiretroviral therapy, presented with 2 days of severe, burning leg pain. She had a migraine headache 4 days before presentation for which she was taking ergotamine twice daily. Leg pain extended from toes to midthigh. On physical examination, both legs were cold with absent popliteal and dorsalis pedis pulses. What is the diagnosis?
\vspace{12pt}

\textbf{Options:}
\begin{enumerate}
\item[A.] Atherosclerosis
\item[B.] Ergotism
\item[C.] Herniated disc
\item[D.] Progressive lumbosacral polyradiculopathy
\item[E.] Thromboangitis obliterans
\end{enumerate}

\textbf{Image:}
\begin{center}
\includegraphics[width=0.5\textwidth,height=0.50\textheight,width=0.90\textwidth,keepaspectratio]{images/nejm_20200716.jpg}
\end{center}
\vspace{12pt}
\newpage

\section*{Question 912 (ID: 20200723)}
\textbf{Date: }July 23,2020
\vspace{6pt}

A 50-year-old woman presented to the dermatology clinic with a painless ulcer on her palate and a swollen upper lip. She had an associated history of intermittent fevers and malaise. She described nasal obstruction with purulent discharge for the last 3 years, for which she received treatment with intranasal glucocorticoids and antibiotics. What is the diagnosis?
\vspace{12pt}

\textbf{Options:}
\begin{enumerate}
\item[A.] Mycobacterium ulcerans infection
\item[B.] Glucocorticoid induced ulceration
\item[C.] Extranodal natural killer T-cell lymphoma
\item[D.] Plasmacytoid dendritic cell neoplasm
\item[E.] Necrotizing sialometaplasia
\end{enumerate}

\textbf{Image:}
\begin{center}
\includegraphics[width=0.63\textwidth,height=0.50\textheight,width=0.90\textwidth,keepaspectratio]{images/nejm_20200723.jpg}
\end{center}
\vspace{12pt}
\newpage

\section*{Question 913 (ID: 20200730)}
\textbf{Date: }July 30,2020
\vspace{6pt}

A 48-year-old man presented to the dermatology clinic with a 2-month history of an itchy rash that began in the genital region and then progressed to his torso, hands, and legs. He also noted a 13-kg weight loss over the past few months. On examination, he had erythematous patches and plaques at various stages of healing. What is the diagnosis?
\vspace{12pt}

\textbf{Options:}
\begin{enumerate}
\item[A.] Herpes simplex virus
\item[B.] Acrodermatitis enteropathica
\item[C.] Paraneoplastic pemphigus
\item[D.] Discoid lupus
\item[E.] Necrolytic migratory erythema
\end{enumerate}

\textbf{Image:}
\begin{center}
\includegraphics[width=0.95\textwidth,height=0.50\textheight,width=0.90\textwidth,keepaspectratio]{images/nejm_20200730.jpg}
\end{center}
\vspace{12pt}
\newpage

\section*{Question 914 (ID: 20200806)}
\textbf{Date: }August 06,2020
\vspace{6pt}

A 60-year-old woman presented to an oral surgery clinic with gum swelling. She had associated bleeding while brushing her teeth. She had a history of colorectal cancer treated with surgery and chemotherapy. On examination, a large, nontender, pedunculated mass was noted. What is the diagnosis?
\vspace{12pt}

\textbf{Options:}
\begin{enumerate}
\item[A.] Zimmerman-Laband syndrome
\item[B.] Metastasis of colorectal cancer
\item[C.] Gingival hyperplasia secondary to chemotherapy
\item[D.] Alveolar soft part sarcoma
\item[E.] Irritation fibroma
\end{enumerate}

\textbf{Image:}
\begin{center}
\includegraphics[width=0.94\textwidth,height=0.50\textheight,width=0.90\textwidth,keepaspectratio]{images/nejm_20200806.jpg}
\end{center}
\vspace{12pt}
\newpage

\section*{Question 915 (ID: 20200813)}
\textbf{Date: }August 13,2020
\vspace{6pt}

A 47-year-old man presented to the emergency department with a 3-day history of a pustular rash on both hands. One week before presentation, he had started treatment with penicillin V potassium for pharyngitis. Physical examination demonstrated a rash as shown, with a similar rash being present on the soles of both of his feet. A punch-biopsy specimen of a palmar lesion was obtained, and histopathological analysis revealed subcorneal pustules and a mononuclear-cell infiltrate in the dermis. What is the diagnosis?
\vspace{12pt}

\textbf{Options:}
\begin{enumerate}
\item[A.] Poststreptococcal pustulosis
\item[B.] Palmoplantar psoriasis
\item[C.] Jarisch-Herxheimer reaction
\item[D.] Immunoglobulin A pemphigus
\item[E.] Scabies
\end{enumerate}

\textbf{Image:}
\begin{center}
\includegraphics[width=0.95\textwidth,height=0.50\textheight,width=0.90\textwidth,keepaspectratio]{images/nejm_20200813.jpg}
\end{center}
\vspace{12pt}
\newpage

\section*{Question 916 (ID: 20200820)}
\textbf{Date: }August 20,2020
\vspace{6pt}

A 30-year-old pregnant woman at 34 weeks of gestation presented to the emergency department with a 5-day history of an itchy rash that began on her abdomen and then spread to her thorax and extremities. Physician examination revealed papules on her abdomen and bullous blisters on her arm. What is the diagnosis?
\vspace{12pt}

\textbf{Options:}
\begin{enumerate}
\item[A.] Dermatitis herpetiformis
\item[B.] Pemphigoid gestationis
\item[C.] Bullous pemphigoid
\item[D.] Erythema multiforme
\item[E.] Pruritic urticarial papules and plaques of pregnancy (PUPPP)
\end{enumerate}

\textbf{Image:}
\begin{center}
\includegraphics[width=0.95\textwidth,height=0.50\textheight,width=0.90\textwidth,keepaspectratio]{images/nejm_20200820.jpg}
\end{center}
\vspace{12pt}
\newpage

\section*{Question 917 (ID: 20200827)}
\textbf{Date: }August 27,2020
\vspace{6pt}

An 89-year-old woman presented to the ophthalmology clinic with a 4-month history of redness and a sensation of a foreign body in her right eye. Slit-lamp examination revealed a red, raised lesion on the conjunctiva that extended onto the cornea with accompanying prominent blood vessels. What is the diagnosis?
\vspace{12pt}

\textbf{Options:}
\begin{enumerate}
\item[A.] Conjunctival squamous-cell carcinoma
\item[B.] Pinguecula
\item[C.] Pterygium
\item[D.] Episcleritis
\item[E.] Pyogenic granuloma
\end{enumerate}

\textbf{Image:}
\begin{center}
\includegraphics[width=0.95\textwidth,height=0.50\textheight,width=0.90\textwidth,keepaspectratio]{images/nejm_20200827.jpg}
\end{center}
\vspace{12pt}
\newpage

\section*{Question 918 (ID: 20200903)}
\textbf{Date: }September 03,2020
\vspace{6pt}

A 44-year-old man presented with confusion and a cough a few days after swimming in an indoor pool. At presentation, he had a body temperature of 39.8°C and appeared drowsy. He did not respond to questions or blink to visual threat; however, brainstem reflexes were intact, and he withdrew both arms and both legs from painful stimuli. A lumbar puncture was performed and showed 2083 nucleated cells per cubic millimeter (91\% neutrophils), as well as a glucose level of 87 mg per deciliter (reference range, 40 to 70), and a protein level of 477 mg per deciliter (reference range, 10 to 45). A Wright-Giemsa stain of the cerebrospinal fluid was performed. Which one of the following organisms caused this presentation?
\vspace{12pt}

\textbf{Options:}
\begin{enumerate}
\item[A.] Mycobacterium marinum
\item[B.] Naegleria fowleri
\item[C.] Legionella pneumophila
\item[D.] Sappinia diploidea
\item[E.] Escherichia coli
\end{enumerate}

\textbf{Image:}
\begin{center}
\includegraphics[width=0.95\textwidth,height=0.50\textheight,width=0.90\textwidth,keepaspectratio]{images/nejm_20200903.jpg}
\end{center}
\vspace{12pt}
\newpage

\section*{Question 919 (ID: 20200910)}
\textbf{Date: }September 10,2020
\vspace{6pt}

A 62-year-old man presented to the emergency department with a 1-day history of fever and 3-day history of chest pain. He had a history of coronary artery disease and splenectomy. On examination, he had three dog-bite wounds on his left hand. Laboratory studies revealed a white-cell count of 16,700 per cubic millimeter (reference range, 3900 to 10,200) and a platelet count of 3100 per cubic millimeter (reference range, 15,000 to 37,000). Review of a peripheral-blood smear was as shown. What is the etiology of his illness?
\vspace{12pt}

\textbf{Options:}
\begin{enumerate}
\item[A.] Pasteurella multocida
\item[B.] Capnocytophaga canimorsus
\item[C.] Pasteurella canis
\item[D.] Babesia microti
\item[E.] Bacillus anthracis
\end{enumerate}

\textbf{Image:}
\begin{center}
\includegraphics[width=0.95\textwidth,height=0.50\textheight,width=0.90\textwidth,keepaspectratio]{images/nejm_20200910.jpg}
\end{center}
\vspace{12pt}
\newpage

\section*{Question 920 (ID: 20200917)}
\textbf{Date: }September 17,2020
\vspace{6pt}

An 82-year-old man presented to the emergency department with a 2-week history of generalized weakness and altered mental status. He was started on empiric treatment for meningitis, but he continued to have progressive neurologic decline with the development of seizures, and died 9 days later. Autopsy of the brain showed liquefactive necrosis. Microscopy is as shown. What is the diagnosis?
\vspace{12pt}

\textbf{Options:}
\begin{enumerate}
\item[A.] Granulomatous amebic encephalitis
\item[B.] Gliomatosis cerebri
\item[C.] Rabies
\item[D.] Human polyomavirus 2
\item[E.] Creutzfeldt-Jakob disease
\end{enumerate}

\textbf{Image:}
\begin{center}
\includegraphics[width=0.95\textwidth,height=0.50\textheight,width=0.90\textwidth,keepaspectratio]{images/nejm_20200917.jpg}
\end{center}
\vspace{12pt}
\newpage

\section*{Question 921 (ID: 20200924)}
\textbf{Date: }September 24,2020
\vspace{6pt}

A 62-year-old man presented to the emergency department with a 2-week history of generalized weakness and a diffuse rash. On examination he had a temperature of 38.1°C (100.6°F). Physical examination showed a maculopapular, hyperpigmented, and scaly eruption on his palms, soles, and trunk. Which one of the following is the diagnosis?
\vspace{12pt}

\textbf{Options:}
\begin{enumerate}
\item[A.] Erythema multiforme
\item[B.] Rocky Mountain spotted fever
\item[C.] Secondary syphilis
\item[D.] Pityriasis rosea
\item[E.] Infectious mononucleosis
\end{enumerate}

\textbf{Image:}
\begin{center}
\includegraphics[width=0.95\textwidth,height=0.50\textheight,width=0.90\textwidth,keepaspectratio]{images/nejm_20200924.jpg}
\end{center}
\vspace{12pt}
\newpage

\section*{Question 922 (ID: 20201001)}
\textbf{Date: }October 01,2020
\vspace{6pt}

A 46-year-old woman receiving atezolizumab for bladder cancer presented to the ophthalmology clinic with a 7-day history of pain, photophobia, and blurring of vision in both eyes. Slit-lamp examination showed conjunctival redness, pseudomembrane formation, and corneal epithelial damage. What is the diagnosis?
\vspace{12pt}

\textbf{Options:}
\begin{enumerate}
\item[A.] Autoimmune keratitis
\item[B.] Herpes simplex virus
\item[C.] Pseudomonas keratitis
\item[D.] Acanthamoeba keratitis
\item[E.] Bladder cancer metastases
\end{enumerate}

\textbf{Image:}
\begin{center}
\includegraphics[width=0.95\textwidth,height=0.50\textheight,width=0.90\textwidth,keepaspectratio]{images/nejm_20201001.jpg}
\end{center}
\vspace{12pt}
\newpage

\section*{Question 923 (ID: 20201008)}
\textbf{Date: }October 08,2020
\vspace{6pt}

A 53-year-old woman with metastatic signet-ring appendiceal cancer receiving treatment with panitumumab presents with a 2-month history of trichomegaly. What is the cause of this physical examination finding?
\vspace{12pt}

\textbf{Options:}
\begin{enumerate}
\item[A.] Cornelia de Lange syndrome
\item[B.] Dermatomyositis
\item[C.] Cytokine immunotherapy
\item[D.] Familial trichomegaly
\item[E.] EGFR inhibition
\end{enumerate}

\textbf{Image:}
\begin{center}
\includegraphics[width=0.95\textwidth,height=0.50\textheight,width=0.90\textwidth,keepaspectratio]{images/nejm_20201008.jpg}
\end{center}
\vspace{12pt}
\newpage

\section*{Question 924 (ID: 20201015)}
\textbf{Date: }October 15,2020
\vspace{6pt}

A 64-year-old woman presented with a 5-month history of photophobia with ocular pain and a foreign-body sensation in both eyes. The visual acuity was 20/25 in each eye, and slit-lamp examination showed conjunctival hyperemia, corneal epithelial erosions, and corneal endothelial folds. What is the diagnosis?
\vspace{12pt}

\textbf{Options:}
\begin{enumerate}
\item[A.] Cat scratch disease
\item[B.] Sjögren's syndrome
\item[C.] Trachoma
\item[D.] Behçet's disease
\item[E.] Multiple sclerosis
\end{enumerate}

\textbf{Image:}
\begin{center}
\includegraphics[width=0.95\textwidth,height=0.50\textheight,width=0.90\textwidth,keepaspectratio]{images/nejm_20201015.jpg}
\end{center}
\vspace{12pt}
\newpage

\section*{Question 925 (ID: 20201022)}
\textbf{Date: }October 22,2020
\vspace{6pt}

A 4-month-old boy who was exclusively breast-fed presented to the clinic with a 6-week history of a worsening rash. Physical examination revealed extensive erosive plaques throughout the body. What is the diagnosis?
\vspace{12pt}

\textbf{Options:}
\begin{enumerate}
\item[A.] Atopic dermatitis
\item[B.] Epidermolysis bullosa
\item[C.] Stevens-Johnson Syndrome
\item[D.] Zinc deficiency
\item[E.] Erythema infectiosum
\end{enumerate}

\textbf{Image:}
\begin{center}
\includegraphics[width=0.95\textwidth,height=0.50\textheight,width=0.90\textwidth,keepaspectratio]{images/nejm_20201022.jpg}
\end{center}
\vspace{12pt}
\newpage

\section*{Question 926 (ID: 20201029)}
\textbf{Date: }October 29,2020
\vspace{6pt}

A 73-year-old woman with rheumatoid arthritis presented with a 1-month history of pain in her right eye. She had stopped immunosuppressive treatment a few years earlier. There was no recent history of trauma. Slit-lamp examination showed hyperemia, inflammation, and marked scleral thinning with exposure of the underlying choroid. What is the most likely diagnosis?
\vspace{12pt}

\textbf{Options:}
\begin{enumerate}
\item[A.] Necrotizing anterior scleritis
\item[B.] Posterior scleritis
\item[C.] Hyphema
\item[D.] Conjuctival hemorrhage
\item[E.] Acute angle closure glaucoma
\end{enumerate}

\textbf{Image:}
\begin{center}
\includegraphics[width=0.95\textwidth,height=0.50\textheight,width=0.90\textwidth,keepaspectratio]{images/nejm_20201029.jpg}
\end{center}
\vspace{12pt}
\newpage

\section*{Question 927 (ID: 20201105)}
\textbf{Date: }November 05,2020
\vspace{6pt}

An 83-year-old man presented with food impaction after eating pork. He reported intermittent difficulty with swallowing ongoing for several months. An esophagogastroduodenoscopy was performed to relieve the obstruction. During the endoscopy, the duodenal mucosa was as shown. What is the diagnosis?
\vspace{12pt}

\textbf{Options:}
\begin{enumerate}
\item[A.] Malignant melanoma
\item[B.] Pseudomelanosis duodeni
\item[C.] Hemochromatosis
\item[D.] Chronic bowel ischemia
\item[E.] Peutz-Jeghers syndrome
\end{enumerate}

\textbf{Image:}
\begin{center}
\includegraphics[width=0.95\textwidth,height=0.50\textheight,width=0.90\textwidth,keepaspectratio]{images/nejm_20201105.jpg}
\end{center}
\vspace{12pt}
\newpage

\section*{Question 928 (ID: 20201112)}
\textbf{Date: }November 12,2020
\vspace{6pt}

A 3-day-old male infant presented to the hospital with vomiting and inability to pass stools. Physical exam revealed a distended abdomen without bowel sounds. Exploratory laparotomy was done. What is the diagnosis?
\vspace{12pt}

\textbf{Options:}
\begin{enumerate}
\item[A.] Bowel obstruction
\item[B.] Hirschsprung’s disease
\item[C.] Toxic megacolon
\item[D.] Meckel’s diverticulum
\item[E.] Meconium ileus
\end{enumerate}

\textbf{Image:}
\begin{center}
\includegraphics[width=0.95\textwidth,height=0.50\textheight,width=0.90\textwidth,keepaspectratio]{images/nejm_20201112.jpg}
\end{center}
\vspace{12pt}
\newpage

\section*{Question 929 (ID: 20201119)}
\textbf{Date: }November 19,2020
\vspace{6pt}

A 29-year-old male presented with a 6-month history of progressive painless skin lesions over his face, back, trunk, and limbs. He had no systemic symptoms such as night sweats or weight loss. Skin biopsy showed diffuse dermal infiltrate with blast-like cells extending to the hypodermis. Immunohistochemical analysis showed cells that were positive for CD4, CD56, and CD123. What is the diagnosis?
\vspace{12pt}

\textbf{Options:}
\begin{enumerate}
\item[A.] Lichen planus
\item[B.] Pustular psoriasis
\item[C.] Blastic plasmacytoid dendritic-cell neoplasm
\item[D.] Cutaneous lymphoblastic T-cell lymphoma
\item[E.] Subacute cutaneous lupus erythematosus
\end{enumerate}

\textbf{Image:}
\begin{center}
\includegraphics[width=0.95\textwidth,height=0.50\textheight,width=0.90\textwidth,keepaspectratio]{images/nejm_20201119.jpg}
\end{center}
\vspace{12pt}
\newpage

\section*{Question 930 (ID: 20201126)}
\textbf{Date: }November 26,2020
\vspace{6pt}

A 62-year-old man with chronic obstructive pulmonary disease presented to the emergency department with a 2-day history of dyspnea. He required intubation and ventilation when he was found to be in hypercapnic respiratory failure. Five days after admission to the ICU, his urine became green. Which of the following medications caused this?
\vspace{12pt}

\textbf{Options:}
\begin{enumerate}
\item[A.] Omeprazole
\item[B.] Morphine sulfate
\item[C.] Ipratropium bromide
\item[D.] Propofol
\item[E.] Midazolam
\end{enumerate}

\textbf{Image:}
\begin{center}
\includegraphics[width=0.62\textwidth,height=0.50\textheight,width=0.90\textwidth,keepaspectratio]{images/nejm_20201126.jpg}
\end{center}
\vspace{12pt}
\newpage

\section*{Question 931 (ID: 20201203)}
\textbf{Date: }December 03,2020
\vspace{6pt}

A 46-year-old man presented to the emergency department with a 2-week history of diarrhea, nausea, abdominal pain, and weakness along with a weight loss of 8 kg in the preceding 2 months. An enzyme-linked immunosorbent assay for the human immunodeficiency virus was positive. Laboratory studies revealed a serum creatinine level of 2.4 mg per deciliter. What is the diagnosis?
\vspace{12pt}

\textbf{Options:}
\begin{enumerate}
\item[A.] Focal segmental glomerulosclerosis
\item[B.] Renal cryptococcosis
\item[C.] Diabetic nephropathy
\item[D.] Tubulocystic renal-cell carcinoma
\item[E.] IgA Nephropathy
\end{enumerate}

\textbf{Image:}
\begin{center}
\includegraphics[width=0.69\textwidth,height=0.50\textheight,width=0.90\textwidth,keepaspectratio]{images/nejm_20201203.jpg}
\end{center}
\vspace{12pt}
\newpage

\section*{Question 932 (ID: 20201210)}
\textbf{Date: }December 10,2020
\vspace{6pt}

A 50-year-old man with a history of large-cell neuroendocrince carcinoma of the lung presented with a 5-day history of shortness of breath, chest pain, and a cough. Physical examination noted diminished breath sounds in the right lower lobe and cough after fluid intake. What is the diagnosis?
\vspace{12pt}

\textbf{Options:}
\begin{enumerate}
\item[A.] Bronchoesophageal fistula
\item[B.] Aspiration pneumonia
\item[C.] Pharyngeal pseudodiverticulum
\item[D.] Zenker’s diverticulum
\item[E.] Gastroesophageal reflux disease
\end{enumerate}

\textbf{Image:}
\begin{center}
\includegraphics[width=0.66\textwidth,height=0.50\textheight,width=0.90\textwidth,keepaspectratio]{images/nejm_20201210.jpg}
\end{center}
\vspace{12pt}
\newpage

\section*{Question 933 (ID: 20201217)}
\textbf{Date: }December 17,2020
\vspace{6pt}

A 60-year-old man presented to the hematology clinic with fatigue and dyspnea on exertion. He had a history of sickle cell disease with hemoglobin genotype SS. Laboratory studies showed a hemoglobin level of 4 g per deciliter. Light microscopy of a bone marrow biopsy specimen is as shown. What is the diagnosis?
\vspace{12pt}

\textbf{Options:}
\begin{enumerate}
\item[A.] Sideroblastic anemia
\item[B.] B-cell acute lymphoblastic leukemia
\item[C.] Aplastic anemia
\item[D.] Hemophagocytosis of sickle cells
\item[E.] Multiple myeloma
\end{enumerate}

\textbf{Image:}
\begin{center}
\includegraphics[width=0.95\textwidth,height=0.50\textheight,width=0.90\textwidth,keepaspectratio]{images/nejm_20201217.jpg}
\end{center}
\vspace{12pt}
\newpage

\section*{Question 934 (ID: 20201224)}
\textbf{Date: }December 24,2020
\vspace{6pt}

A 35-year-old man presented with a 6-month history of increased swelling in his right leg. Physical exam revealed firmness in the right lower leg without tenderness to palpation. Magnetic resonance imaging of the area revealed multiple cystic lesions. What is the diagnosis?
\vspace{12pt}

\textbf{Options:}
\begin{enumerate}
\item[A.] Lipomas
\item[B.] Echinococcal cysts
\item[C.] Sebaceous cysts
\item[D.] Metastatic sarcoma
\item[E.] Ganglion cysts
\end{enumerate}

\textbf{Image:}
\begin{center}
\includegraphics[width=0.95\textwidth,height=0.50\textheight,width=0.90\textwidth,keepaspectratio]{images/nejm_20201224.jpg}
\end{center}
\vspace{12pt}
\newpage

\section*{Question 935 (ID: 20201231)}
\textbf{Date: }December 31,2020
\vspace{6pt}

A 45-year-old man presented to the emergency department after a motorcycle accident. Physical examination was notable for a bruise in the pubic area and severe pain in the anterior and posterior pelvic areas with manual compression. What is the diagnosis?
\vspace{12pt}

\textbf{Options:}
\begin{enumerate}
\item[A.] Open-book fracture
\item[B.] Straddle pelvic fracture
\item[C.] Avascular necrosis of the hip
\item[D.] Femoral head fracture
\item[E.] Hip dysplasia
\end{enumerate}

\textbf{Image:}
\begin{center}
\includegraphics[width=0.94\textwidth,height=0.50\textheight,width=0.90\textwidth,keepaspectratio]{images/nejm_20201231.jpg}
\end{center}
\vspace{12pt}
\newpage

\section*{Question 936 (ID: 20210107)}
\textbf{Date: }January 07,2021
\vspace{6pt}

An 83-year-old man presented to the emergency department with a 6-week history of fatigue, loss of appetite, and weight loss. His blood pressure was 115/80 mm Hg while seated but dropped to 95/65 mm Hg on standing. Laboratory investigations demonstrated hyponatremia and hyperkalemia. What is the diagnosis?
\vspace{12pt}

\textbf{Options:}
\begin{enumerate}
\item[A.] Abdominal tuberculosis
\item[B.] Primary adrenal lymphoma
\item[C.] Renal cell carcinoma metastases
\item[D.] Bilateral adrenal hyperplasia
\item[E.] Conn’s syndrome
\end{enumerate}

\textbf{Image:}
\begin{center}
\includegraphics[width=0.5\textwidth,height=0.50\textheight,width=0.90\textwidth,keepaspectratio]{images/nejm_20210107.jpg}
\end{center}
\vspace{12pt}
\newpage

\section*{Question 937 (ID: 20210114)}
\textbf{Date: }January 14,2021
\vspace{6pt}

A 7-year-old girl presented with a 2-week history of fever and painful blisters throughout the body. She completed a course of penicillin for tonsillitis two weeks earlier. Physical examination showed multiple bullae on her extremities and lesions on her lips. Five days after admission, she developed a new skin ulcer on the right hand at the site of an intravenous cannula. What is the diagnosis?
\vspace{12pt}

\textbf{Options:}
\begin{enumerate}
\item[A.] Neutrophilic dermatosis
\item[B.] Dermatitis herpetiformis
\item[C.] Blastomyces dermatitidis
\item[D.] Pemphigus vulgaris
\item[E.] Leukocytoclastic vasculitis
\end{enumerate}

\textbf{Image:}
\begin{center}
\includegraphics[width=0.95\textwidth,height=0.50\textheight,width=0.90\textwidth,keepaspectratio]{images/nejm_20210114.jpg}
\end{center}
\vspace{12pt}
\newpage

\section*{Question 938 (ID: 20210121)}
\textbf{Date: }January 21,2021
\vspace{6pt}

A 21-month-old boy presented with a 1-day history of a pruritic rash on his left leg. On examination of the left lower leg, an edematous erythematous plaque was pulsating with alternating erythema and blanching in synchrony with his pulse. What is the most likely diagnosis?
\vspace{12pt}

\textbf{Options:}
\begin{enumerate}
\item[A.] Insect bite
\item[B.] Atopic dermatitis
\item[C.] Erythema infectiosum
\item[D.] Pityriasis rosea
\item[E.] Tinea corporis
\end{enumerate}

\textbf{Image:}
\begin{center}
\includegraphics[width=0.95\textwidth,height=0.50\textheight,width=0.90\textwidth,keepaspectratio]{images/nejm_20210121.jpg}
\end{center}
\vspace{12pt}
\newpage

\section*{Question 939 (ID: 20210128)}
\textbf{Date: }January 28,2021
\vspace{6pt}

A 17-year-old girl with systemic lupus erythematosus presented to the ophthalmology clinic with a 2-day history of decreased vision in her right eye and a 6-month history of intermittent headaches. She stopped treatment with methylprednisolone and hydroxychloroquine 2 years ago. Examination of the fundus of the right eye is as shown. What is the diagnosis?
\vspace{12pt}

\textbf{Options:}
\begin{enumerate}
\item[A.] Antiphospholipid syndrome
\item[B.] Retinal toxoplasmosis
\item[C.] Lupus retinal vasculitis
\item[D.] Susac's syndrome
\item[E.] Hydroxychloroquine-induced maculopathy
\end{enumerate}

\textbf{Image:}
\begin{center}
\includegraphics[width=0.95\textwidth,height=0.50\textheight,width=0.90\textwidth,keepaspectratio]{images/nejm_20210128.jpg}
\end{center}
\vspace{12pt}
\newpage

\section*{Question 940 (ID: 20210204)}
\textbf{Date: }February 04,2021
\vspace{6pt}

A 37-year-old man presented with a 14-month history of painful, thickened, and cracked skin on the fingertips of both hands. He also had low-grade fevers and muscle weakness. Laboratory testing showed that the creatine kinase level was above 20,000 U per liter (reference range, 51 to 298). What is the diagnosis?
\vspace{12pt}

\textbf{Options:}
\begin{enumerate}
\item[A.] Psoriasis
\item[B.] Antisynthetase syndrome
\item[C.] Rheumatoid arthritis
\item[D.] Overuse from manual labor
\item[E.] Eczema
\end{enumerate}

\textbf{Image:}
\begin{center}
\includegraphics[width=0.88\textwidth,height=0.50\textheight,width=0.90\textwidth,keepaspectratio]{images/nejm_20210204.jpg}
\end{center}
\vspace{12pt}
\newpage

\section*{Question 941 (ID: 20210211)}
\textbf{Date: }February 11,2021
\vspace{6pt}

A 56-year-old man with a new diagnosis of acute monoblastic leukemia received induction chemotherapy and subsequently developed disseminated intravascular coagulation and tumor lysis syndrome. While his condition improved, he began having persistent lower back pain. A pelvic bone marrow specimen is as shown. What is the most likely diagnosis?
\vspace{12pt}

\textbf{Options:}
\begin{enumerate}
\item[A.] Bone marrow necrosis
\item[B.] Megaloblastic anemia
\item[C.] Myelofibrosis
\item[D.] Multiple Myeloma
\item[E.] Essential thrombocythemia
\end{enumerate}

\textbf{Image:}
\begin{center}
\includegraphics[width=0.95\textwidth,height=0.50\textheight,width=0.90\textwidth,keepaspectratio]{images/nejm_20210211.jpg}
\end{center}
\vspace{12pt}
\newpage

\section*{Question 942 (ID: 20210218)}
\textbf{Date: }February 18,2021
\vspace{6pt}

An 80-year-old man presented to the emergency department with abdominal bloating and constipation. He had lost 5 kg during the previous 8 months. Examination of the abdomen was notable for a mass that extended from the epigastrium to the pelvis and was nontender and dull on percussion. What is the diagnosis?
\vspace{12pt}

\textbf{Options:}
\begin{enumerate}
\item[A.] Abdominal aortic aneurysm
\item[B.] Castleman disease
\item[C.] Urachal mucinous cystic tumor
\item[D.] Cysticercosis
\item[E.] Wilm’s tumor
\end{enumerate}

\textbf{Image:}
\begin{center}
\includegraphics[width=0.6\textwidth,height=0.50\textheight,width=0.90\textwidth,keepaspectratio]{images/nejm_20210218.jpg}
\end{center}
\vspace{12pt}
\newpage

\section*{Question 943 (ID: 20210225)}
\textbf{Date: }February 25,2021
\vspace{6pt}

A neonate delivered at 32 weeks developed cyanosis and respiratory distress after delivery. His mother had not received regular antenatal care. Endotracheal intubation and tracheostomy were attempted without success. A barium esophagram was done. What is the diagnosis?
\vspace{12pt}

\textbf{Options:}
\begin{enumerate}
\item[A.] Tracheal agenesis with bronchoesophageal fistula
\item[B.] Squamous cell carcinoma of the trachea
\item[C.] Zenker’s Diverticulum
\item[D.] Pharyngeal pseudodiverticulum
\item[E.] Laryngotracheoesophageal cleft
\end{enumerate}

\textbf{Image:}
\begin{center}
\includegraphics[width=0.77\textwidth,height=0.50\textheight,width=0.90\textwidth,keepaspectratio]{images/nejm_20210225.jpg}
\end{center}
\vspace{12pt}
\newpage

\section*{Question 944 (ID: 20210304)}
\textbf{Date: }March 04,2021
\vspace{6pt}

A 36-year-old woman presented with a 2-month history of cough and worsening shortness of breath. Chest x-ray is as shown. What is the diagnosis?
\vspace{12pt}

\textbf{Options:}
\begin{enumerate}
\item[A.] Pulmonary alveolar microlithiasis
\item[B.] Bronchiolitis obliterans organizing pneumonia
\item[C.] Pulmonary edema
\item[D.] Idiopathic pulmonary fibrosis
\item[E.] Diffuse alveolar hemorrhage
\end{enumerate}

\textbf{Image:}
\begin{center}
\includegraphics[width=0.73\textwidth,height=0.50\textheight,width=0.90\textwidth,keepaspectratio]{images/nejm_20210304.jpg}
\end{center}
\vspace{12pt}
\newpage

\section*{Question 945 (ID: 20210311)}
\textbf{Date: }March 11,2021
\vspace{6pt}

A 63-year-old woman presented with an 8-week history of nodules on her arms. A biopsy was performed and demonstrated multiple granulomas. What is the diagnosis?
\vspace{12pt}

\textbf{Options:}
\begin{enumerate}
\item[A.] Squamous-cell carcinoma
\item[B.] Mycobacterium marinum
\item[C.] Syphilis
\item[D.] Sarcoidosis
\item[E.] Herpes simplex virus infection
\end{enumerate}

\textbf{Image:}
\begin{center}
\includegraphics[width=0.95\textwidth,height=0.50\textheight,width=0.90\textwidth,keepaspectratio]{images/nejm_20210311.jpg}
\end{center}
\vspace{12pt}
\newpage

\section*{Question 946 (ID: 20210318)}
\textbf{Date: }March 18,2021
\vspace{6pt}

A 5-year-old girl presented with 3 days of fevers, sore throat, and pain with swallowing. On examination, she had a temperature of 40°C, fissured lips, and a red tongue with enlarged papillae. What is the diagnosis?
\vspace{12pt}

\textbf{Options:}
\begin{enumerate}
\item[A.] Measles
\item[B.] Henoch-Schönlein purpura
\item[C.] Hand, foot, and mouth disease
\item[D.] Vitamin B12 deficiency
\item[E.] Streptococcal pharyngitis
\end{enumerate}

\textbf{Image:}
\begin{center}
\includegraphics[width=0.95\textwidth,height=0.50\textheight,width=0.90\textwidth,keepaspectratio]{images/nejm_20210318.jpg}
\end{center}
\vspace{12pt}
\newpage

\section*{Question 947 (ID: 20210325)}
\textbf{Date: }March 25,2021
\vspace{6pt}

A 31-year-old man presented to the emergency department with fever, papular skin lesions on his face and body, and weight loss of 10 kg in the past month. Testing for human immunodeficiency virus (HIV) was positive. A blood smear is shown. What is the diagnosis?
\vspace{12pt}

\textbf{Options:}
\begin{enumerate}
\item[A.] Methicillin resistant staphylococcus aureus
\item[B.] Aspergillus fumigatus
\item[C.] Bacillus anthracis
\item[D.] Talaromyces marneffei
\item[E.] Blastomyces dermatitidis
\end{enumerate}

\textbf{Image:}
\begin{center}
\includegraphics[width=0.95\textwidth,height=0.50\textheight,width=0.90\textwidth,keepaspectratio]{images/nejm_20210325.jpg}
\end{center}
\vspace{12pt}
\newpage

\section*{Question 948 (ID: 20210401)}
\textbf{Date: }April 01,2021
\vspace{6pt}

A 5-year-old girl presented with a 4-week history of painful swelling on both sides of her lower abdomen. Six weeks before presentation, her parents removed a tick they found buried in her umbilicus. Five days after this she developed fevers. What is the diagnosis?
\vspace{12pt}

\textbf{Options:}
\begin{enumerate}
\item[A.] Ehrlichiosis
\item[B.] Babesiosis
\item[C.] Tularemia
\item[D.] Lyme disease
\item[E.] Rocky Mountain spotted fever
\end{enumerate}

\textbf{Image:}
\begin{center}
\includegraphics[width=0.95\textwidth,height=0.50\textheight,width=0.90\textwidth,keepaspectratio]{images/nejm_20210401.jpg}
\end{center}
\vspace{12pt}
\newpage

\section*{Question 949 (ID: 20210408)}
\textbf{Date: }April 08,2021
\vspace{6pt}

A 68-year-old woman presented to clinic with a 40-year history of worsening hyperkeratotic non-painful, non-pruritic papules and plaques on her hands and feet. Her mother, son, and granddaughter had similar lesions. What is the diagnosis?
\vspace{12pt}

\textbf{Options:}
\begin{enumerate}
\item[A.] Erythema multiforme
\item[B.] Punctate palmoplantar keratoderma
\item[C.] Incontinentia pigmenti
\item[D.] Tuberous sclerosis
\item[E.] Syphilis
\end{enumerate}

\textbf{Image:}
\begin{center}
\includegraphics[width=0.95\textwidth,height=0.50\textheight,width=0.90\textwidth,keepaspectratio]{images/nejm_20210408.jpg}
\end{center}
\vspace{12pt}
\newpage

\section*{Question 950 (ID: 20210415)}
\textbf{Date: }April 15,2021
\vspace{6pt}

A 27-year-old primigravid woman was admitted with new-onset diabetes and hypertension at 32 weeks of gestation. She had worsening proximal muscle weakness, striae, and facial plethora for several months. Physical examination showed striae over the abdomen, breasts, and dorsal and supraclavicular fat pads. Laboratory studies were notable for a potassium of 2.9 mmol per liter (reference range, 3.5 to 5.5) and a urine protein:creatinine ratio of 24 (reference value, <30). What is the most likely diagnosis?
\vspace{12pt}

\textbf{Options:}
\begin{enumerate}
\item[A.] Eclampsia
\item[B.] Hyperthyroidism
\item[C.] Addison’s disease
\item[D.] Adrenal mass
\item[E.] Parathyroid mass
\end{enumerate}

\textbf{Image:}
\begin{center}
\includegraphics[width=0.86\textwidth,height=0.50\textheight,width=0.90\textwidth,keepaspectratio]{images/nejm_20210415.jpg}
\end{center}
\vspace{12pt}
\newpage

\section*{Question 951 (ID: 20210422)}
\textbf{Date: }April 22,2021
\vspace{6pt}

A 48-year-old man presented with a 2-week history of fever, shortness of breath, and painful skin lesions with blistering. His temperature was 39.2°C. What is the diagnosis?
\vspace{12pt}

\textbf{Options:}
\begin{enumerate}
\item[A.] Weil’s disease
\item[B.] Staphylococcal scalded skin syndrome
\item[C.] Legionnaires' disease
\item[D.] Leprosy
\item[E.] Lyme disease
\end{enumerate}

\textbf{Image:}
\begin{center}
\includegraphics[width=0.62\textwidth,height=0.50\textheight,width=0.90\textwidth,keepaspectratio]{images/nejm_20210422.jpg}
\end{center}
\vspace{12pt}
\newpage

\section*{Question 952 (ID: 20210429)}
\textbf{Date: }April 29,2021
\vspace{6pt}

A 54-year-old man presented to the emergency department with confusion and vomiting. He also reported an unintentional 10 kg weight loss and progressive darkening of his skin over the past 6 months. Laboratory studies showed a serum glucose of 40 mg per deciliter (reference range, 70 to 110 mg per deciliter), sodium of 108 mmol per liter (reference range, 136 to 145), and a potassium of 6.4 mmol per liter (reference range, 3.5 to 5.1). What is the diagnosis?
\vspace{12pt}

\textbf{Options:}
\begin{enumerate}
\item[A.] Primary adrenal insufficiency (Addison’s disease)
\item[B.] Acanthosis nigricans
\item[C.] Cushing’s disease
\item[D.] Smoker’s melanosis
\item[E.] Solar lentigines
\end{enumerate}

\textbf{Image:}
\begin{center}
\includegraphics[width=0.95\textwidth,height=0.50\textheight,width=0.90\textwidth,keepaspectratio]{images/nejm_20210429.jpg}
\end{center}
\vspace{12pt}
\newpage

\section*{Question 953 (ID: 20210506)}
\textbf{Date: }May 06,2021
\vspace{6pt}

A 32-year-old man presented with itchy palmar lesions that appeared over the preceding 3 days. He had a similar episode 6 months earlier that resolved spontaneously. He has a history of recurrent oral herpes simplex infections. What is the diagnosis?
\vspace{12pt}

\textbf{Options:}
\begin{enumerate}
\item[A.] Herpes simplex virus infection
\item[B.] Mycoplasma pneumoniae infection
\item[C.] Chronic urticaria
\item[D.] Stevens-Johnson syndrome
\item[E.] Lupus erythematosus
\end{enumerate}

\textbf{Image:}
\begin{center}
\includegraphics[width=0.93\textwidth,height=0.50\textheight,width=0.90\textwidth,keepaspectratio]{images/nejm_20210506.jpg}
\end{center}
\vspace{12pt}
\newpage

\section*{Question 954 (ID: 20210513)}
\textbf{Date: }May 13,2021
\vspace{6pt}

A 40-year-old man with a long-standing history of smokeless tobacco use presented with a painless, white lesion on the tongue for four months. What is the diagnosis?
\vspace{12pt}

\textbf{Options:}
\begin{enumerate}
\item[A.] Lichen planus
\item[B.] Leukoplakia
\item[C.] Oral candidiasis
\item[D.] Fordyce's spot
\item[E.] Syphilis
\end{enumerate}

\textbf{Image:}
\begin{center}
\includegraphics[width=0.86\textwidth,height=0.50\textheight,width=0.90\textwidth,keepaspectratio]{images/nejm_20210513.jpg}
\end{center}
\vspace{12pt}
\newpage

\section*{Question 955 (ID: 20210520)}
\textbf{Date: }May 20,2021
\vspace{6pt}

A 60-year-old man presented to the emergency room with decreased vision in his right eye 3 days after a fall. Examination showed abrasions on the right supraorbital area and forehead, pain with extraocular movement, and decreased right eye visual acuity. What is the diagnosis?
\vspace{12pt}

\textbf{Options:}
\begin{enumerate}
\item[A.] Orbital wall fracture
\item[B.] Intraocular lens dislocation
\item[C.] Retinal detachment
\item[D.] Vitreous hemorrhage
\item[E.] Posterior uveitis
\end{enumerate}

\textbf{Image:}
\begin{center}
\includegraphics[width=0.95\textwidth,height=0.50\textheight,width=0.90\textwidth,keepaspectratio]{images/nejm_20210520.jpg}
\end{center}
\vspace{12pt}
\newpage

\section*{Question 956 (ID: 20210527)}
\textbf{Date: }May 27,2021
\vspace{6pt}

A 60-year-old woman with a history of rheumatoid arthritis presented with diarrhea, nausea, and anorexia. Endoscopic examination of the colon and biopsy specimens with Congo red staining under polarized light are shown. What is the diagnosis?
\vspace{12pt}

\textbf{Options:}
\begin{enumerate}
\item[A.] Gastrointestinal Kaposi’s sarcoma
\item[B.] Gastrointestinal amyloidosis
\item[C.] Lynch syndrome
\item[D.] Crohn’s disease
\item[E.] Gastrointestinal leiomyosarcoma
\end{enumerate}

\textbf{Image:}
\begin{center}
\includegraphics[width=0.95\textwidth,height=0.50\textheight,width=0.90\textwidth,keepaspectratio]{images/nejm_20210527.jpg}
\end{center}
\vspace{12pt}
\newpage

\section*{Question 957 (ID: 20210603)}
\textbf{Date: }June 03,2021
\vspace{6pt}

A 65-year-old man presented with six months of worsening lesions on his hands and forearms as well as fingernail thickening. His medical history included end-stage renal disease and untreated hepatitis C virus infection. What is the diagnosis?
\vspace{12pt}

\textbf{Options:}
\begin{enumerate}
\item[A.] Buruli ulcers
\item[B.] Peripheral vascular disease
\item[C.] Antineutrophil cytoplasmic antibody (ANCA)-associated vasculitis
\item[D.] Porphyria cutanea tarda
\item[E.] Blastomycosis
\end{enumerate}

\textbf{Image:}
\begin{center}
\includegraphics[width=0.95\textwidth,height=0.50\textheight,width=0.90\textwidth,keepaspectratio]{images/nejm_20210603.jpg}
\end{center}
\vspace{12pt}
\newpage

\section*{Question 958 (ID: 20210610)}
\textbf{Date: }June 10,2021
\vspace{6pt}

A 53-year-old man who worked as a gardener presented to the emergency department with sudden onset of paralysis in the lower legs. He had a 2-month history of intermittent fevers and had undergone a bioprosthetic aortic-valve replacement 6 months prior to presentation. Computed tomographic angiography showed an aortic pseudoaneurysm with emboli. An embolectomy was performed, and a histopathological analysis is as shown. Which organism was the cause of his infection?
\vspace{12pt}

\textbf{Options:}
\begin{enumerate}
\item[A.] Mucormycosis
\item[B.] Aspergillus fumigatus
\item[C.] Curvularia alcornii
\item[D.] Scedosporium apiospermum
\item[E.] Sporothrix schenckii
\end{enumerate}

\textbf{Image:}
\begin{center}
\includegraphics[width=0.95\textwidth,height=0.50\textheight,width=0.90\textwidth,keepaspectratio]{images/nejm_20210610.jpg}
\end{center}
\vspace{12pt}
\newpage

\section*{Question 959 (ID: 20210617)}
\textbf{Date: }June 17,2021
\vspace{6pt}

A 66-year-old woman presented with a pruritic rash. Physical examination showed dyschromic patches, plaques, and poikiloderma on her upper back. Strength testing noted weakness in proximal arms and legs. What is the diagnosis?
\vspace{12pt}

\textbf{Options:}
\begin{enumerate}
\item[A.] Polymyositis
\item[B.] Dermatomyositis
\item[C.] Subacute cutaneous lupus erythematosus
\item[D.] Discoid lupus
\item[E.] Psoriasis
\end{enumerate}

\textbf{Image:}
\begin{center}
\includegraphics[width=0.95\textwidth,height=0.50\textheight,width=0.90\textwidth,keepaspectratio]{images/nejm_20210617.jpg}
\end{center}
\vspace{12pt}
\newpage

\section*{Question 960 (ID: 20210624)}
\textbf{Date: }June 24,2021
\vspace{6pt}

An 83-year-old woman presented to the emergency department with a 1-day history of right-sided tongue swelling. Sensory examination of the tongue was normal. What is the diagnosis?
\vspace{12pt}

\textbf{Options:}
\begin{enumerate}
\item[A.] Thromboembolism
\item[B.] Sjogren's disease
\item[C.] Hematoma
\item[D.] Anti-neutrophil cytoplasm antibodies (ANCA) vasculitis
\item[E.] Hereditary angioedema
\end{enumerate}

\textbf{Image:}
\begin{center}
\includegraphics[width=0.95\textwidth,height=0.50\textheight,width=0.90\textwidth,keepaspectratio]{images/nejm_20210624.jpg}
\end{center}
\vspace{12pt}
\newpage

\section*{Question 961 (ID: 20210701)}
\textbf{Date: }July 01,2021
\vspace{6pt}

A 34-year-old woman presented to the rheumatology clinic with a 3-month history of a rash on her cheeks and an 18-month history of joint pain in her hands and knees. Physical examination showed indurated plaques on her cheeks. What is the diagnosis?
\vspace{12pt}

\textbf{Options:}
\begin{enumerate}
\item[A.] Polymyalgia rheumatica
\item[B.] Rheumatoid arthritis
\item[C.] Systemic lupus erythematosus
\item[D.] Adult-onset Still’s disease
\item[E.] Psoriatic arthritis
\end{enumerate}

\textbf{Image:}
\begin{center}
\includegraphics[width=0.73\textwidth,height=0.50\textheight,width=0.90\textwidth,keepaspectratio]{images/nejm_20210701.jpg}
\end{center}
\vspace{12pt}
\newpage

\section*{Question 962 (ID: 20210708)}
\textbf{Date: }July 08,2021
\vspace{6pt}

A 16-day-old girl was brought to the emergency department with lethargy. Physical exam showed tachypnea and marked hepatomegaly, as well as small hemangiomas on the skin. TSH was elevated. MRI showed numerous hepatic lesions and cardiomegaly. What is the most likely diagnosis?
\vspace{12pt}

\textbf{Options:}
\begin{enumerate}
\item[A.] Hepatic adenoma
\item[B.] Pyogenic liver abscess
\item[C.] Hepatocellular carcinoma
\item[D.] Infantile hepatic hemangiomas
\item[E.] Undifferentiated embryonal sarcoma
\end{enumerate}

\textbf{Image:}
\begin{center}
\includegraphics[width=0.95\textwidth,height=0.50\textheight,width=0.90\textwidth,keepaspectratio]{images/nejm_20210708.jpg}
\end{center}
\vspace{12pt}
\newpage

\section*{Question 963 (ID: 20210715)}
\textbf{Date: }July 15,2021
\vspace{6pt}

Examination of a 68-year-old man with interstitial lung disease was notable for skin changes characterized by alternating areas of hypopigmentation and hyperpigmentation on sclerotic skin. What is the most likely diagnosis?
\vspace{12pt}

\textbf{Options:}
\begin{enumerate}
\item[A.] Amyopathic dermatomyositis
\item[B.] Leprosy
\item[C.] Diffuse cutaneous systemic sclerosis
\item[D.] Subacute cutaneous lupus erythematosus
\item[E.] Adult-onset Still’s disease
\end{enumerate}

\textbf{Image:}
\begin{center}
\includegraphics[width=0.95\textwidth,height=0.50\textheight,width=0.90\textwidth,keepaspectratio]{images/nejm_20210715.jpg}
\end{center}
\vspace{12pt}
\newpage

\section*{Question 964 (ID: 20210722)}
\textbf{Date: }July 22,2021
\vspace{6pt}

A 26-year-old heathy woman developed an axillary mass with white discharge after a spontaneous vaginal delivery. What is the most likely diagnosis?
\vspace{12pt}

\textbf{Options:}
\begin{enumerate}
\item[A.] Eosinophilic folliculitis
\item[B.] Polymastia
\item[C.] Bacterial lymphadenitis
\item[D.] Hidradenitis suppurativa
\item[E.] Lactational mastitis
\end{enumerate}

\textbf{Image:}
\begin{center}
\includegraphics[width=0.95\textwidth,height=0.50\textheight,width=0.90\textwidth,keepaspectratio]{images/nejm_20210722.jpg}
\end{center}
\vspace{12pt}
\newpage

\section*{Question 965 (ID: 20210729)}
\textbf{Date: }July 29,2021
\vspace{6pt}

A 42-year-old woman receiving induction chemotherapy for acute monocytic leukemia developed edematous, erythematous plaques on her face. Biopsy showed perieccrine neutrophilic infiltration with eccrine duct necrosis. What is the most likely diagnosis?
\vspace{12pt}

\textbf{Options:}
\begin{enumerate}
\item[A.] Eccrine miliaria rubra
\item[B.] Leukemia cutis
\item[C.] Erysipelas
\item[D.] Drug-induced lupus
\item[E.] Neutrophilic eccrine hidradenitis
\end{enumerate}

\textbf{Image:}
\begin{center}
\includegraphics[width=0.72\textwidth,height=0.50\textheight,width=0.90\textwidth,keepaspectratio]{images/nejm_20210729.jpg}
\end{center}
\vspace{12pt}
\newpage

\section*{Question 966 (ID: 20210805)}
\textbf{Date: }August 05,2021
\vspace{6pt}

A 62-year-old woman with recent travel to Colombia presented with pruritic, draining nodules with a central punctum on her back and buttocks. The lesions were unchanged after a trial of antibiotics. What is the most likely diagnosis?
\vspace{12pt}

\textbf{Options:}
\begin{enumerate}
\item[A.] Cutaneous leishmaniasis
\item[B.] Epidermoid cyst
\item[C.] Furunculosis
\item[D.] Myiasis
\item[E.] Tungiasis
\end{enumerate}

\textbf{Image:}
\begin{center}
\includegraphics[width=0.95\textwidth,height=0.50\textheight,width=0.90\textwidth,keepaspectratio]{images/nejm_20210805.jpg}
\end{center}
\vspace{12pt}
\newpage

\section*{Question 967 (ID: 20210812)}
\textbf{Date: }August 12,2021
\vspace{6pt}

A 26-year-old woman with the Peutz-Jeghers syndrome presented to the emergency department with abdominal pain and nonbilious vomiting. A tender mass was palpable in the upper abdomen. What is the most likely diagnosis?
\vspace{12pt}

\textbf{Options:}
\begin{enumerate}
\item[A.] Acute necrotizing pancreatitis
\item[B.] Cholangiocarcinoma
\item[C.] Gastric leiomyosarcoma
\item[D.] Gastrogastric intussusception
\item[E.] Pancreatic adenocarcinoma with gastric invasion
\end{enumerate}

\textbf{Image:}
\begin{center}
\includegraphics[width=0.95\textwidth,height=0.50\textheight,width=0.90\textwidth,keepaspectratio]{images/nejm_20210812.jpg}
\end{center}
\vspace{12pt}
\newpage

\section*{Question 968 (ID: 20210819)}
\textbf{Date: }August 19,2021
\vspace{6pt}

A 30-year-old man presented with a rash on his face, hands, and feet. He had a history of HIV with a CD4+ T-cell count of 374 per cubic millimeter. Physical examination showed circinate lesions on the palms, soles, and face and patchy alopecia. What is the likely diagnosis?
\vspace{12pt}

\textbf{Options:}
\begin{enumerate}
\item[A.] Disseminated granuloma annulare
\item[B.] Erythema annulare centrifugum
\item[C.] Syphilis
\item[D.] Tinea corporis
\item[E.] Tuberculoid leprosy
\end{enumerate}

\textbf{Image:}
\begin{center}
\includegraphics[width=0.95\textwidth,height=0.50\textheight,width=0.90\textwidth,keepaspectratio]{images/nejm_20210819.jpg}
\end{center}
\vspace{12pt}
\newpage

\section*{Question 969 (ID: 20210826)}
\textbf{Date: }August 26,2021
\vspace{6pt}

A 54-year-old woman presented with abdominal pain and difficulty walking. Abdominal imaging showed multiple abscesses surrounding an intrauterine device (IUD). Aspiration with gram stain identified branching, filamentous, gram-positive rods. What is the most likely diagnosis?
\vspace{12pt}

\textbf{Options:}
\begin{enumerate}
\item[A.] Actinomycosis
\item[B.] Aspergillosis
\item[C.] Crohn’s disease
\item[D.] Listeriosis
\item[E.] Nocardiosis
\end{enumerate}

\textbf{Image:}
\begin{center}
\includegraphics[width=0.95\textwidth,height=0.50\textheight,width=0.90\textwidth,keepaspectratio]{images/nejm_20210826.jpg}
\end{center}
\vspace{12pt}
\newpage

\section*{Question 970 (ID: 20210902)}
\textbf{Date: }September 02,2021
\vspace{6pt}

A 7-year-old girl presented with an itchy rash while taking amoxicillin for pharyngitis. Physical exam showed purulent tonsils, along with a maculopapular rash involving her face, trunk, arms, and legs. What is the most likely diagnosis?
\vspace{12pt}

\textbf{Options:}
\begin{enumerate}
\item[A.] Amoxicillin rash in infectious mononucleosis
\item[B.] Guttate psoriasis
\item[C.] Hand, foot, and mouth disease
\item[D.] Scarlet fever
\item[E.] Serum sickness-like reaction
\end{enumerate}

\textbf{Image:}
\begin{center}
\includegraphics[width=0.95\textwidth,height=0.50\textheight,width=0.90\textwidth,keepaspectratio]{images/nejm_20210902.jpg}
\end{center}
\vspace{12pt}
\newpage

\section*{Question 971 (ID: 20210909)}
\textbf{Date: }September 09,2021
\vspace{6pt}

A 50-year-old man presented with a lesion on his eye that developed over the preceding month. He also had violaceous plaques on his back and lower limbs. Testing for HIV was positive. What is the diagnosis?
\vspace{12pt}

\textbf{Options:}
\begin{enumerate}
\item[A.] Angiosarcoma
\item[B.] Bacillary angiomatosis
\item[C.] Herpes simplex keratoconjunctivitis
\item[D.] Kaposi’s sarcoma
\item[E.] Ocular surface squamous neoplasia
\end{enumerate}

\textbf{Image:}
\begin{center}
\includegraphics[width=0.95\textwidth,height=0.50\textheight,width=0.90\textwidth,keepaspectratio]{images/nejm_20210909.jpg}
\end{center}
\vspace{12pt}
\newpage

\section*{Question 972 (ID: 20210916)}
\textbf{Date: }September 16,2021
\vspace{6pt}

A 73-year-old man presented with a 5-month history of blurry vision, photophobia, and burning in both eyes. On examination, he was found to have filmlike adhesions between the bulbar and palpebral conjunctivae, as well as scattered vesicles and bullae on his arms and back. What is the eye exam finding?
\vspace{12pt}

\textbf{Options:}
\begin{enumerate}
\item[A.] Conjunctivitis
\item[B.] Episcleritis
\item[C.] Pinguecula
\item[D.] Pterygium
\item[E.] Symblepharon
\end{enumerate}

\textbf{Image:}
\begin{center}
\includegraphics[width=0.95\textwidth,height=0.50\textheight,width=0.90\textwidth,keepaspectratio]{images/nejm_20210916.jpg}
\end{center}
\vspace{12pt}
\newpage

\section*{Question 973 (ID: 20210923)}
\textbf{Date: }September 23,2021
\vspace{6pt}

A 65-year-old man presented with painful, erythematous, and purplish plaques on the extensor surfaces of his hands and elbows. Laboratory examinations showed leukocytosis and thrombocytopenia. What is the most likely diagnosis?
\vspace{12pt}

\textbf{Options:}
\begin{enumerate}
\item[A.] Gottron’s papules
\item[B.] Guttate psoriasis
\item[C.] Immune thrombocytopenia purpura (ITP)
\item[D.] Leukemia cutis
\item[E.] Solar purpura
\end{enumerate}

\textbf{Image:}
\begin{center}
\includegraphics[width=0.95\textwidth,height=0.50\textheight,width=0.90\textwidth,keepaspectratio]{images/nejm_20210923.jpg}
\end{center}
\vspace{12pt}
\newpage

\section*{Question 974 (ID: 20210930)}
\textbf{Date: }September 30,2021
\vspace{6pt}

A healthy 15-year-old boy presented with a 3-month history of asymptomatic papules on his glans penis. What is the diagnosis?
\vspace{12pt}

\textbf{Options:}
\begin{enumerate}
\item[A.] Condyloma acuminatum
\item[B.] Fordyce spots
\item[C.] Molluscum contagiosum
\item[D.] Pearly penile papules
\item[E.] Sebaceous hyperplasia
\end{enumerate}

\textbf{Image:}
\begin{center}
\includegraphics[width=0.95\textwidth,height=0.50\textheight,width=0.90\textwidth,keepaspectratio]{images/nejm_20210930.jpg}
\end{center}
\vspace{12pt}
\newpage

\section*{Question 975 (ID: 20211007)}
\textbf{Date: }October 07,2021
\vspace{6pt}

A 80-year-old man undergoing treatment for multiple myeloma presented with fevers and confusion. Magnetic resonance imaging (MRI) of the head showed ring-enhancing lesions. Gram stain of cerebrospinal fluid (CSF) showed gram-positive bacilli. What is the most likely diagnosis?
\vspace{12pt}

\textbf{Options:}
\begin{enumerate}
\item[A.] Clostridium septicum
\item[B.] Listeriosis
\item[C.] Nocardiosis
\item[D.] Peptostreptococcus species
\item[E.] Propionibacterium acnes
\end{enumerate}

\textbf{Image:}
\begin{center}
\includegraphics[width=0.95\textwidth,height=0.50\textheight,width=0.90\textwidth,keepaspectratio]{images/nejm_20211007.jpg}
\end{center}
\vspace{12pt}
\newpage

\section*{Question 976 (ID: 20211014)}
\textbf{Date: }October 14,2021
\vspace{6pt}

An 82-year-old man with diabetes presented to the emergency department with fever and confusion. Liver function test showed aspartate aminotransferase level of 1380 u/l (reference range, 5 to 40) and alanine aminotransferase level of 1121 u/l (reference range, 5 to 40). A computed tomographic scan of the abdomen was performed. What is the most likely diagnosis?
\vspace{12pt}

\textbf{Options:}
\begin{enumerate}
\item[A.] Acute hepatic necrosis
\item[B.] Cholecystoduodenal fistula
\item[C.] Emphysematous hepatitis
\item[D.] Pyogenic liver abscess
\item[E.] Ruptured hydatid cyst
\end{enumerate}

\textbf{Image:}
\begin{center}
\includegraphics[width=0.95\textwidth,height=0.50\textheight,width=0.90\textwidth,keepaspectratio]{images/nejm_20211014.jpg}
\end{center}
\vspace{12pt}
\newpage

\section*{Question 977 (ID: 20211021)}
\textbf{Date: }October 21,2021
\vspace{6pt}

A 49-year-old woman, who had undergone kidney transplantation 4 months earlier, presented to the emergency department with a 2-week history of headache, dizziness, and rash. On exam she had diffuse, umbilicated papules. What is the most likely pathogen?
\vspace{12pt}

\textbf{Options:}
\begin{enumerate}
\item[A.] Blastomyces dermatitidis
\item[B.] Cryptococcus neoformans
\item[C.] Histoplasma capsulatum
\item[D.] Molluscum contagiosum
\item[E.] Mycobacterium tuberculosis
\end{enumerate}

\textbf{Image:}
\begin{center}
\includegraphics[width=0.95\textwidth,height=0.50\textheight,width=0.90\textwidth,keepaspectratio]{images/nejm_20211021.jpg}
\end{center}
\vspace{12pt}
\newpage

\section*{Question 978 (ID: 20211028)}
\textbf{Date: }October 28,2021
\vspace{6pt}

A 26-year-old woman presented with recurrent hemoptysis coinciding with her menstrual cycles. Noncontrast computed tomography (CT) imaging of the lungs showed a cavitary nodule with ground-glass appearance in the right lower lobe. What is the most likely diagnosis?
\vspace{12pt}

\textbf{Options:}
\begin{enumerate}
\item[A.] Accessory breast tissue
\item[B.] Catamenial pneumothorax
\item[C.] Ectopic pregnancy
\item[D.] Progestogen hypersensitivity
\item[E.] Thoracic endometriosis
\end{enumerate}

\textbf{Image:}
\begin{center}
\includegraphics[width=0.95\textwidth,height=0.50\textheight,width=0.90\textwidth,keepaspectratio]{images/nejm_20211028.jpg}
\end{center}
\vspace{12pt}
\newpage

\section*{Question 979 (ID: 20211104)}
\textbf{Date: }November 04,2021
\vspace{6pt}

A 53-year-old woman presented with pain, swelling, and discoloration of her right fifth finger. Physical examination showed a demarcated, brightly erythematous plaque with a dusky center. What is the most likely diagnosis?
\vspace{12pt}

\textbf{Options:}
\begin{enumerate}
\item[A.] Dactylitis
\item[B.] Erythema multiforme
\item[C.] Herpetic whitlow
\item[D.] Spider bite
\item[E.] Sporotrichosis
\end{enumerate}

\textbf{Image:}
\begin{center}
\includegraphics[width=0.53\textwidth,height=0.50\textheight,width=0.90\textwidth,keepaspectratio]{images/nejm_20211104.jpg}
\end{center}
\vspace{12pt}
\newpage

\section*{Question 980 (ID: 20211111)}
\textbf{Date: }November 11,2021
\vspace{6pt}

A 61-year-old woman presented with discoloration along her gums that had rapidly expanded over the past year. What is the diagnosis?
\vspace{12pt}

\textbf{Options:}
\begin{enumerate}
\item[A.] Amalgam tattoo
\item[B.] Gingival melanoma
\item[C.] Kaposi’s sarcoma
\item[D.] Oral melanoacanthoma
\item[E.] Physiologic pigmentation
\end{enumerate}

\textbf{Image:}
\begin{center}
\includegraphics[width=0.95\textwidth,height=0.50\textheight,width=0.90\textwidth,keepaspectratio]{images/nejm_20211111.jpg}
\end{center}
\vspace{12pt}
\newpage

\section*{Question 981 (ID: 20211118)}
\textbf{Date: }November 18,2021
\vspace{6pt}

A 50-year-old woman with a history of Crohn’s disease presented with 10 days of tongue and inner cheek pain. Laboratory studies showed an absolute eosinophil count of 870 per cubic millimeter (reference range, 50 to 500). What is the most likely diagnosis?
\vspace{12pt}

\textbf{Options:}
\begin{enumerate}
\item[A.] Herpes simplex gingivostomatitis
\item[B.] Herpetic geometric glossitis
\item[C.] Lichen planus
\item[D.] Oral candidiasis
\item[E.] Pyostomatitis vegetans
\end{enumerate}

\textbf{Image:}
\begin{center}
\includegraphics[width=0.86\textwidth,height=0.50\textheight,width=0.90\textwidth,keepaspectratio]{images/nejm_20211118.jpg}
\end{center}
\vspace{12pt}
\newpage

\section*{Question 982 (ID: 20211125)}
\textbf{Date: }November 25,2021
\vspace{6pt}

A 38-year-old man presented to the otolaryngology clinic with chronic difficulty breathing through his right nostril. Physical examination showed nasal septal deviation, calcified septal spurs, and a 2-cm perforation in the posterior septum. On rhinoscopy, a hard, nontender, white mass was observed in the floor of the right nostril. CT of the paranasal sinuses showed a well-defined, radiodense mass. Which of the following is the most likely etiology of the nasal mass?
\vspace{12pt}

\textbf{Options:}
\begin{enumerate}
\item[A.] Calcified polyp
\item[B.] Enchondroma
\item[C.] Osteoma
\item[D.] Rhinolith
\end{enumerate}

\textbf{Image:}
\begin{center}
\includegraphics[width=0.95\textwidth,height=0.50\textheight,width=0.90\textwidth,keepaspectratio]{images/nejm_20211125.jpg}
\end{center}
\vspace{12pt}
\newpage

\section*{Question 983 (ID: 20211202)}
\textbf{Date: }December 02,2021
\vspace{6pt}

A female infant delivered at term had a “blueberry muffin” rash at birth. Laboratory tests and imaging studies were normal. Skin biopsy showed a dense infiltrate of cells with kidney-shaped nuclei and positive S100+ and CD1a+ on immunohistochemistry. What is the most likely diagnosis?
\vspace{12pt}

\textbf{Options:}
\begin{enumerate}
\item[A.] Congenital cytomegalovirus infection
\item[B.] Congenital rubella syndrome
\item[C.] Langerhans-cell histiocytosis
\item[D.] Leukemia cutis
\item[E.] Transient myeloproliferative disorder of Down syndrome
\end{enumerate}

\textbf{Image:}
\begin{center}
\includegraphics[width=0.95\textwidth,height=0.50\textheight,width=0.90\textwidth,keepaspectratio]{images/nejm_20211202.jpg}
\end{center}
\vspace{12pt}
\newpage

\section*{Question 984 (ID: 20211209)}
\textbf{Date: }December 09,2021
\vspace{6pt}

A 74-year-old woman presented with a 3-day history of headaches and tongue swelling and several hours of blindness in the right eye. On examination, the tongue was discolored and ulcerated. What is the diagnosis?
\vspace{12pt}

\textbf{Options:}
\begin{enumerate}
\item[A.] Carcinoma of the tongue
\item[B.] Carotid artery stenosis
\item[C.] Giant-cell arteritis
\item[D.] Oral lichen planus
\item[E.] Thromboembolic disease
\end{enumerate}

\textbf{Image:}
\begin{center}
\includegraphics[width=0.95\textwidth,height=0.50\textheight,width=0.90\textwidth,keepaspectratio]{images/nejm_20211209.jpg}
\end{center}
\vspace{12pt}
\newpage

\section*{Question 985 (ID: 20211216)}
\textbf{Date: }December 16,2021
\vspace{6pt}

A 37-year-old man presented to the neurosurgery clinic with blurry vision, headache, and difficulty looking upward. On exam, he had diplopia, upward gaze palsy, and vertical misalignment of the eyes. His pupils only minimally constricted to light but constricted to near objects. What is the name of this pupillary exam finding?
\vspace{12pt}

\textbf{Options:}
\begin{enumerate}
\item[A.] Adie’s pupil
\item[B.] Anisocoria
\item[C.] Argyll Robertson pupil
\item[D.] Light-near dissociation
\item[E.] Relative afferent pupillary defect (e.g., Marcus Gunn pupil)
\end{enumerate}

\textbf{Image:}
\begin{center}
\includegraphics[width=0.95\textwidth,height=0.50\textheight,width=0.90\textwidth,keepaspectratio]{images/nejm_20211216.jpg}
\end{center}
\vspace{12pt}
\newpage

\section*{Question 986 (ID: 20211223)}
\textbf{Date: }December 23,2021
\vspace{6pt}

A 65-year-old man presented to the dermatology clinic with a 7-month history of painless skin thickening over the left side of his chest and on his left arm. Physical examination showed sclerotic skin over his left chest and nipple, multiple erythematous nodules over the left arm, axilla, and posterior trunk, left axillary adenopathy, and left arm lymphedema. A skin biopsy showed metastatic carcinoma suggestive of primary breast cancer. What is the diagnosis?
\vspace{12pt}

\textbf{Options:}
\begin{enumerate}
\item[A.] Carcinoma en cuirasse
\item[B.] Carcinoma erysipeloides
\item[C.] Inflammatory breast cancer
\item[D.] Morphea
\item[E.] Paget's disease
\end{enumerate}

\textbf{Image:}
\begin{center}
\includegraphics[width=0.62\textwidth,height=0.50\textheight,width=0.90\textwidth,keepaspectratio]{images/nejm_20211223.jpg}
\end{center}
\vspace{12pt}
\newpage

\section*{Question 987 (ID: 20211230)}
\textbf{Date: }December 30,2021
\vspace{6pt}

A 48-year-old woman presented to the emergency department with a 1-week history of fatigue and painful, swollen gums. On examination, she was febrile and tachycardic with a well-demarcated region of gingival infiltration and whitening and submandibular lymphadenopathy. A white-cell count was 225,000 per cubic millimeter. What is the diagnosis?
\vspace{12pt}

\textbf{Options:}
\begin{enumerate}
\item[A.] Gingival candidiasis
\item[B.] Leukemic infiltration of the gingiva
\item[C.] Oral lichen planus
\item[D.] Proliferative verrucous leukoplakia
\item[E.] White sponge nevus
\end{enumerate}

\textbf{Image:}
\begin{center}
\includegraphics[width=0.9\textwidth,height=0.50\textheight,width=0.90\textwidth,keepaspectratio]{images/nejm_20211230.jpg}
\end{center}
\vspace{12pt}
\newpage

\section*{Question 988 (ID: 20191226)}
\textbf{Date: }December 26,2019
\vspace{6pt}

A 19-year-old woman presented to the emergency department with severe abdominal pain that had started 9 hours earlier. Rebound tenderness was present in the suprapubic and left iliac fossa regions on physical examination. A urine test for beta human chorionic gonadotropin was negative. A diagnostic laparoscopy was performed. What is the diagnosis?
\vspace{12pt}

\textbf{Options:}
\begin{enumerate}
\item[A.] Ovarian torsion
\item[B.] Paraovarian cyst torsion
\item[C.] Fallopian tube torsion
\item[D.] Broad ligament torsion
\item[E.] Uterine torsion
\end{enumerate}

\textbf{Image:}
\begin{center}
\includegraphics[width=0.95\textwidth,height=0.50\textheight,width=0.90\textwidth,keepaspectratio]{images/nejm_20191226.jpg}
\end{center}
\vspace{12pt}
\newpage

\section*{Question 989 (ID: 20220106)}
\textbf{Date: }January 06,2022
\vspace{6pt}

A 59-year-old tile layer presented with fatigue and one year of progressive skin darkening on his palms and soles. Physical examination showed a smooth tongue with patchy areas of mucosal darkening, as well as hyperpigmented palms and soles. Laboratory studies showed a hemoglobin level of 9.4 g per deciliter (reference range, 14 to 18), a mean corpuscular volume of 117 femtoliters (reference range, 80 to 94), and mild leukopenia and thrombocytopenia. Which of the following is the most likely diagnosis?
\vspace{12pt}

\textbf{Options:}
\begin{enumerate}
\item[A.] Adrenal insufficiency
\item[B.] Contact dermatitis
\item[C.] Hyperthyroidism
\item[D.] Melanoma
\item[E.] Vitamin B12 deficiency
\end{enumerate}

\textbf{Image:}
\begin{center}
\includegraphics[width=0.95\textwidth,height=0.50\textheight,width=0.90\textwidth,keepaspectratio]{images/nejm_20220106.jpg}
\end{center}
\vspace{12pt}
\newpage

\section*{Question 990 (ID: 20220113)}
\textbf{Date: }January 13,2022
\vspace{6pt}

A 35-year-old woman with idiopathic pulmonary arterial hypertension and a pulmonary aneurysm presented with chest pain. Computed tomography (CT) of the chest is shown. What is the most likely diagnosis?
\vspace{12pt}

\textbf{Options:}
\begin{enumerate}
\item[A.] Aortic dissection
\item[B.] Aortic pseudoaneurysm
\item[C.] Intramural hematoma
\item[D.] Pulmonary-artery dissection
\item[E.] Pulmonary-artery rupture
\end{enumerate}

\textbf{Image:}
\begin{center}
\includegraphics[width=0.95\textwidth,height=0.50\textheight,width=0.90\textwidth,keepaspectratio]{images/nejm_20220113.jpg}
\end{center}
\vspace{12pt}
\newpage

\section*{Question 991 (ID: 20220120)}
\textbf{Date: }January 20,2022
\vspace{6pt}

A 48-year-old woman with advanced cirrhosis presented with fatigue, jaundice, and abdominal distention. Laboratory studies showed a hemoglobin level of 5.8 g per deciliter (reference range, 12.0 to 15.5), a lactate dehydrogenase level of 1219 U per liter (reference range, 125 to 220), a low haptoglobin level, and an elevated reticulocyte count. Coombs’ antiglobulin testing was normal. The result of a peripheral-blood smear is shown. What is the diagnosis?
\vspace{12pt}

\textbf{Options:}
\begin{enumerate}
\item[A.] Autoimmune hemolytic anemia
\item[B.] Babesiosis
\item[C.] Hypersplenism
\item[D.] Microangiopathic hemolytic anemia
\item[E.] Spur-cell hemolytic anemia
\end{enumerate}

\textbf{Image:}
\begin{center}
\includegraphics[width=0.95\textwidth,height=0.50\textheight,width=0.90\textwidth,keepaspectratio]{images/nejm_20220120.jpg}
\end{center}
\vspace{12pt}
\newpage

\section*{Question 992 (ID: 20220127)}
\textbf{Date: }January 27,2022
\vspace{6pt}

A 40-year-old man with neurofibromatosis type 1 presented for a yearly eye exam and was found to have multiple well-defined, light-brown nodules elevated above the iris bilaterally. What is the most likely diagnosis?
\vspace{12pt}

\textbf{Options:}
\begin{enumerate}
\item[A.] Brushfield spots
\item[B.] Iris melanoma
\item[C.] Lisch nodules
\item[D.] Multiple iris nevi
\item[E.] Sarcoid-associated uveitis
\end{enumerate}

\textbf{Image:}
\begin{center}
\includegraphics[width=0.85\textwidth,height=0.50\textheight,width=0.90\textwidth,keepaspectratio]{images/nejm_20220127.jpg}
\end{center}
\vspace{12pt}
\newpage

\section*{Question 993 (ID: 20220203)}
\textbf{Date: }February 03,2022
\vspace{6pt}

A 60-year-old man with HIV presented to the ophthalmology clinic with eyelid and facial lesions. His CD4 cell count was 20 per cubic millimeter and his HIV viral load was 120,000 copies/ml. Physical exam showed numerous dome-shaped papules with central umbilication. What is the most likely diagnosis?
\vspace{12pt}

\textbf{Options:}
\begin{enumerate}
\item[A.] Basal-cell carcinoma
\item[B.] Cryptococcosis
\item[C.] Histoplasmosis
\item[D.] Molluscum contagiosum
\item[E.] Verruca vulgaris
\end{enumerate}

\textbf{Image:}
\begin{center}
\includegraphics[width=0.95\textwidth,height=0.50\textheight,width=0.90\textwidth,keepaspectratio]{images/nejm_20220203.jpg}
\end{center}
\vspace{12pt}
\newpage

\section*{Question 994 (ID: 20240627)}
\textbf{Date: }June 27,2024
\vspace{6pt}

A 67-year-old woman with a history of chronic rhinosinusitis presented to the emergency
department with a 1-week history of dyspnea and cough. On physical examination, all 10 fingernails were yellow and thickened (upper image). Pitting edema was also present in the legs (lower image), and decreased breath sounds were detected at the left lung base. A chest radiograph showed a pleural effusion on the left side. A thoracentesis was performed, and 1.1 liters of milky, yellow fluid were removed. Findings from analysis of the pleural effusion were consistent with a chylous effusion. What single diagnosis could explain all of this patient’s findings?
\vspace{12pt}

\textbf{Options:}
\begin{enumerate}
\item[A.] Autoimmune hepatitis
\item[B.] Graves’ disease
\item[C.] Lymphoma
\item[D.] Psoriasis
\item[E.] Yellow nail syndrome
\end{enumerate}

\textbf{Image:}
\begin{center}
\includegraphics[width=0.43\textwidth,height=0.50\textheight,width=0.90\textwidth,keepaspectratio]{images/nejm_20240627.jpg}
\end{center}
\vspace{12pt}
\newpage

\section*{Question 995 (ID: 20240704)}
\textbf{Date: }July 04,2024
\vspace{6pt}

A 29-year-old man presented to the outpatient clinic with 2 months of bright red blood from the rectum. He had no fever, weight loss, diarrhea, or hematuria. He lived in the interior of northeastern Brazil and regularly bathed in rivers. On physical examination, he had slight tenderness to palpation of the left flank. Laboratory testing was notable for an absolute eosinophil count of 470 per cubic millimeter (reference range, 34 to 420). A colonoscopy identified a reddish, polypoid lesion in the distal rectum. Biopsy of the lesion revealed a dense inflammatory infiltrate containing eosinophils and schistosome eggs. What is the appropriate treatment for this condition?
\vspace{12pt}

\textbf{Options:}
\begin{enumerate}
\item[A.] Itraconazole
\item[B.] Ivermectin
\item[C.] Metronidazole
\item[D.] Nitazoxanide
\item[E.] Praziquantel
\end{enumerate}

\textbf{Image:}
\begin{center}
\includegraphics[width=0.95\textwidth,height=0.50\textheight,width=0.90\textwidth,keepaspectratio]{images/nejm_20240704.jpg}
\end{center}
\vspace{12pt}
\newpage

\section*{Question 996 (ID: 20240711)}
\textbf{Date: }July 11,2024
\vspace{6pt}

A 26-year-old man presented to the ophthalmology clinic with a 2-month history of a painless, reddish growth on his left eye. A physical examination was notable for the presence of a sessile mass protruding just above the lower eyelid. When the eyelid was lowered, the entire mass - including its vascular fronds and episcleral feeder vessel - was visible on the surface of the inferior bulbar conjunctiva. What is the most likely diagnosis?
\vspace{12pt}

\textbf{Options:}
\begin{enumerate}
\item[A.] Capillary hemangioma
\item[B.] Conjunctival papilloma
\item[C.] Pinguecula
\item[D.] Pterygium
\item[E.] Squamous cell carcinoma
\end{enumerate}

\textbf{Image:}
\begin{center}
\includegraphics[width=0.95\textwidth,height=0.50\textheight,width=0.90\textwidth,keepaspectratio]{images/nejm_20240711.jpg}
\end{center}
\vspace{12pt}
\newpage

\section*{Question 997 (ID: 20240718)}
\textbf{Date: }July 18,2024
\vspace{6pt}

A 29-year-old woman with a history of systemic lupus erythematosus presented to the dermatology clinic with a 2-week history of an itchy, painful rash on her nose and hands. The rash had first appeared 1 day after the weather had turned cold. The patient reported no sun exposure. The physical examination was notable for erythematosus macules and papules with punched-out ulcers on the nose. Scattered papules were seen on the palms, and edematous erythrocyanosis of the fingertips with ulcerations on the lateral aspects was noted. Blood tests were positive for antinuclear antibodies, anti-double-stranded DNA antibodies, rheumatoid factor, anti-Ro antibodies, and antiphospholipid antibodies, and hypergammaglobulinemia was present. Cryoglobulin and cold agglutinin testing was negative. Histopathological examination of a biopsy specimen of the right nasal sidewall revealed vacuolar interface dermatitis and perivascular lymphocytic infiltrate. Which of the following is an important intervention to prevent this finding?
\vspace{12pt}

\textbf{Options:}
\begin{enumerate}
\item[A.] Avoidance of exacerbating drugs
\item[B.] Cold avoidance
\item[C.] Photoprotection
\item[D.] Smoking abstinence
\item[E.] Vitamin D supplementation
\end{enumerate}

\textbf{Image:}
\begin{center}
\includegraphics[width=0.95\textwidth,height=0.50\textheight,width=0.90\textwidth,keepaspectratio]{images/nejm_20240718.jpg}
\end{center}
\vspace{12pt}
\newpage

\section*{Question 998 (ID: 20240725)}
\textbf{Date: }July 25,2024
\vspace{6pt}

A 73-year-old woman had sudden chest pain during pulmonary-function testing. She had a history of breast cancer, for which a double mastectomy with breast reconstruction had been performed 23 years earlier, followed by the insertion of silicone breast implants 12 years later. She also had a history of non-small-cell lung cancer, for which a superior segmentectomy of the right lower lung had been performed by means of open thoracotomy 3 years before presentation. To evaluate the patient’s chest pain, a computed tomographic scan of the chest was performed. What radiographic finding is shown?
\vspace{12pt}

\textbf{Options:}
\begin{enumerate}
\item[A.] Intrathoracic migration of a breast implant
\item[B.] Lobar consolidation
\item[C.] Loculated pleural effusion
\item[D.] Pneumothorax
\item[E.] Segmental atelectasis
\end{enumerate}

\textbf{Image:}
\begin{center}
\includegraphics[width=0.95\textwidth,height=0.50\textheight,width=0.90\textwidth,keepaspectratio]{images/nejm_20240725.jpg}
\end{center}
\vspace{12pt}
\newpage

\section*{Question 999 (ID: 20240801)}
\textbf{Date: }August 01,2024
\vspace{6pt}

A 70-year-old woman with a history of hypothyroidism presented to the dermatology clinic with a 1-year history of progressive indentation and darkening of the skin of her forehead. The skin changes had occurred along a wrinkle that had been present for 10 years. On physical examination, a paramedian, linear, atrophic depression with a violaceous border was observed on the right side of the forehead. The depression extended from the orbital rim to the frontal scalp and was associated with cicatricial alopecia. Examination of a skin-biopsy specimen of the depression revealed a perivascular infiltrate of lymphocytes, eosinophils, and plasma cells, along with vacuolar interface dermatitis and dermal scarring. What is the diagnosis?
\vspace{12pt}

\textbf{Options:}
\begin{enumerate}
\item[A.] Alopecia areata
\item[B.] Carcinoma en cuirasse
\item[C.] En Coup de Sabre
\item[D.] Lupus panniculitis
\item[E.] Progressive hemifacial atrophy (Parry-Romberg syndrome)
\end{enumerate}

\textbf{Image:}
\begin{center}
\includegraphics[width=0.95\textwidth,height=0.50\textheight,width=0.90\textwidth,keepaspectratio]{images/nejm_20240801.jpg}
\end{center}
\vspace{12pt}
\newpage

\section*{Question 1000 (ID: 20240808)}
\textbf{Date: }August 08,2024
\vspace{6pt}

A 58-year-old man presented to the dermatology clinic with a 2-year history of eyelid lesions and with weight loss and fatigue over the past several months. On physical examination, scattered periorbital petechiae and purpura were noted, as well as coalescing, waxy papules on the eyelids. Macroglossia was also present. Serum coagulation tests were normal. A biopsy of the left upper eyelid resulted in ecchymosis. What is the most likely explanation for the spontaneous periorbital petechiae and significant bruising after the biopsy in this case?
\vspace{12pt}

\textbf{Options:}
\begin{enumerate}
\item[A.] Aspirin use
\item[B.] Chronic alcohol use
\item[C.] Factor X deficiency
\item[D.] Vascular fragility owing to infiltration of vessel walls
\item[E.] Vitamin C deficiency
\end{enumerate}

\textbf{Image:}
\begin{center}
\includegraphics[width=0.95\textwidth,height=0.50\textheight,width=0.90\textwidth,keepaspectratio]{images/nejm_20240808.jpg}
\end{center}
\vspace{12pt}
\newpage

\section*{Question 1001 (ID: 20240815)}
\textbf{Date: }August 15,2024
\vspace{6pt}

A 61-year-old woman presented to the emergency department with a 2-day history of dull, intermittent lower abdominal pain. She reported no vomiting, diarrhea, bloody stools, or hematuria. The vital signs were normal with the exception of the blood pressure, which was 174/107 mm Hg. On physical examination, tenderness was present in the left lower abdomen without rebound or guarding. The white-cell count and C-reactive protein level were normal. Computed tomography of the abdomen showed an ovoid lesion adjacent to the descending colon with ring enhancement, density of fat, and surrounding fat stranding. What is the most appropriate initial management?
\vspace{12pt}

\textbf{Options:}
\begin{enumerate}
\item[A.] Analgesia only
\item[B.] Elective surgical removal
\item[C.] Intravenous antibacterial agents
\item[D.] Percutaneous needle biopsy
\item[E.] Urgent surgical removal
\end{enumerate}

\textbf{Image:}
\begin{center}
\includegraphics[width=0.95\textwidth,height=0.50\textheight,width=0.90\textwidth,keepaspectratio]{images/nejm_20240815.jpg}
\end{center}
\vspace{12pt}
\newpage

\section*{Question 1002 (ID: 20240822)}
\textbf{Date: }August 22,2024
\vspace{6pt}

A 57-year-old man called emergency medical services to report a 1-hour history of chest pain. When he arrived at the hospital, his heart rate was 125 beats per minute, the blood pressure 128/84 mm Hg, and the oxygen saturation 99\% while he was breathing ambient air. The physical examination was normal. An initial serum troponin I level was 35 ng per liter (reference range, 0 to 34). What does the electrocardiogram most likely represent?
\vspace{12pt}

\textbf{Options:}
\begin{enumerate}
\item[A.] Anteroseptal ST-elevation myocardial infarction
\item[B.] Diffuse subendocardial ischemia
\item[C.] Pericarditis
\item[D.] Posterior myocardial infarction
\item[E.] Wellens syndrome
\end{enumerate}

\textbf{Image:}
\begin{center}
\includegraphics[width=0.95\textwidth,height=0.50\textheight,width=0.90\textwidth,keepaspectratio]{images/nejm_20240822.jpg}
\end{center}
\vspace{12pt}
\newpage

\section*{Question 1003 (ID: 20240829)}
\textbf{Date: }August 29,2024
\vspace{6pt}

A 9-year-old girl presented to the dermatology department with a 7-day history of a lesion on the left cheek and a 4-day history of fever. On physical examination, a round plaque 3 cm in diameter on an erythematous base was observed on the patient’s left cheek, with overlying crusting and a single intact vesicle. Satellite vesicles and ipsilateral cervical lymphadenopathy were also present. There were no mucosal lesions. What is the most likely diagnosis?
\vspace{12pt}

\textbf{Options:}
\begin{enumerate}
\item[A.] Cutaneous non-tuberculous mycobacterial infection
\item[B.] Impetigo
\item[C.] Primary cutaneous HSV-1 infection
\item[D.] Syphilitic chancre
\item[E.] Tinea corporis
\end{enumerate}

\textbf{Image:}
\begin{center}
\includegraphics[width=0.95\textwidth,height=0.50\textheight,width=0.90\textwidth,keepaspectratio]{images/nejm_20240829.jpg}
\end{center}
\vspace{12pt}
\newpage

\section*{Question 1004 (ID: 20240905)}
\textbf{Date: }September 05,2024
\vspace{6pt}

A previously healthy 53-year-old woman presented with a recurrent episode of severe, diffuse abdominal pain. During the previous month, she had been having sudden, self-limited bouts of abdominal pain twice per week. She had no history of swelling of the face or limbs and had not been taking an angiotensin-converting-enzyme inhibitor. There was no family history of angioedema. An abdominal examination was normal. CT of the abdomen revealed segmental thickening of the walls of the colon and rectum and mesenteric edema. A colonoscopy revealed edema of the entire colon. No abnormalities in the colonic mucosa were detected on biopsy. Laboratory testing showed low serum levels of C4, C1 inhibitor antigen, and C1q on repeated measurements, as well as decreased C1 esterase function. A diagnosis of angioedema of the intestines was made. Which of the following is the most likely cause?
\vspace{12pt}

\textbf{Options:}
\begin{enumerate}
\item[A.] Acquired C1 inhibitor deficiency
\item[B.] Drug-induced angioedema (ACE-I or NSAID)
\item[C.] Hereditary angioedema with low functional C1 inhibitor
\item[D.] Hereditary angioedema with normal C1 inhibitor
\item[E.] Idiopathic nonhistaminergic angioedema
\end{enumerate}

\textbf{Image:}
\begin{center}
\includegraphics[width=0.95\textwidth,height=0.50\textheight,width=0.90\textwidth,keepaspectratio]{images/nejm_20240905.jpg}
\end{center}
\vspace{12pt}
\newpage

\section*{Question 1005 (ID: 20240912)}
\textbf{Date: }September 12,2024
\vspace{6pt}

A 44-year-old man presented with a 5-year history of heat intolerance and burning pain in his hands and feet. He also had exertional dyspnea, decreased perspiration, sinus tachycardia, and proteinuria of unclear cause. On physical examination, there were periumbilical vascular skin lesions that had been present for the previous 20 years. A sample from a 24-hour urine collection showed non-nephrotic proteinuria. What is the most likely diagnosis?
\vspace{12pt}

\textbf{Options:}
\begin{enumerate}
\item[A.] Diabetes Mellitus
\item[B.] Fabry’s disease
\item[C.] IgG4-related disease
\item[D.] Porphyria
\item[E.] Systemic AL amyloidosis
\end{enumerate}

\textbf{Image:}
\begin{center}
\includegraphics[width=0.49\textwidth,height=0.50\textheight,width=0.90\textwidth,keepaspectratio]{images/nejm_20240912.jpg}
\end{center}
\vspace{12pt}
\newpage

\section*{Question 1006 (ID: 20240919)}
\textbf{Date: }September 19,2024
\vspace{6pt}

A 42-year-old woman with a 20-pack-year smoking history presented to the dermatology clinic with a 2-year history of a facial rash. One year before presentation, the patient’s rash had been evaluated without a request for removal of her makeup, and treatment for possible acne had been recommended. At the current presentation, a skin examination was performed after removal of her makeup. An indurated plaque with central hypopigmentation, dilated follicular ostia, and alopecia over the right eyebrow were observed, along with a plaque with scattered areas of hyperpigmentation and hypopigmentation on the right cheek. On the left cheek, there were scattered nodules, open comedones, and areas of hyperpigmentation and hypopigmentation. What is the most likely diagnosis?
\vspace{12pt}

\textbf{Options:}
\begin{enumerate}
\item[A.] Cutaneous sarcoidosis
\item[B.] Discoid lupus erythematosus
\item[C.] Granuloma faciale
\item[D.] Lupus vulgaris
\item[E.] Rosacea
\end{enumerate}

\textbf{Image:}
\begin{center}
\includegraphics[width=0.95\textwidth,height=0.50\textheight,width=0.90\textwidth,keepaspectratio]{images/nejm_20240919.jpg}
\end{center}
\vspace{12pt}
\newpage

\section*{Question 1007 (ID: 20240926)}
\textbf{Date: }September 26,2024
\vspace{6pt}

A 26-year-old man presented with a 1-week history of fevers and bloody stools. On physical examination, there was a small external hemorrhoid but no abdominal pain, genital lesions, or inguinal lymphadenopathy. Computed tomography of the pelvis showed thickening of the wall of the rectum (left) and perirectal lymphadenopathy. A subsequent flexible sigmoidoscopy showed nodular mucosa with erythema and ulceration in the distal rectum (right). After a tissue biopsy of the lesion, which of the following is the most appropriate next step?
\vspace{12pt}

\textbf{Options:}
\begin{enumerate}
\item[A.] CT of the chest
\item[B.] Rectal swab for sexually transmitted infections
\item[C.] Fecal calprotectin
\item[D.] Upper endoscopy
\item[E.] Whole body PET-CT
\end{enumerate}

\textbf{Image:}
\begin{center}
\includegraphics[width=0.95\textwidth,height=0.50\textheight,width=0.90\textwidth,keepaspectratio]{images/nejm_20240926.jpg}
\end{center}
\vspace{12pt}
\newpage

\section*{Question 1008 (ID: 20241003)}
\textbf{Date: }October 03,2024
\vspace{6pt}

A 25-year-old woman with systemic lupus erythematosus (SLE) presented with a 3-month history of a rash on her arms, legs, and groin. She had been having condomless sexual intercourse with one male partner during the year preceding presentation. During the 9 months before presentation, she had been taking hydroxychloroquine at a dose of 200 mg daily and prednisolone at a dose of 15 to 40 mg daily to control her SLE. Physical examination was notable for erosive, violaceous plaques in the antecubital fossae, popliteal fossae, and inguinal regions. There were also scaly erythematous patches on both palms. What is the most likely diagnosis?
\vspace{12pt}

\textbf{Options:}
\begin{enumerate}
\item[A.] Discoid lupus
\item[B.] Extramammary Paget disease
\item[C.] Pemphigus vulgaris
\item[D.] Symmetric drug-related intertriginous and flexural exanthema
\item[E.] Syphilis
\end{enumerate}

\textbf{Image:}
\begin{center}
\includegraphics[width=0.95\textwidth,height=0.50\textheight,width=0.90\textwidth,keepaspectratio]{images/nejm_20241003.jpg}
\end{center}
\vspace{12pt}
\newpage

\section*{Question 1009 (ID: 20241010)}
\textbf{Date: }October 10,2024
\vspace{6pt}

A 19-year-old man with a history of mild acne vulgaris presented with a 10-day history of rapidly worsening acne, along with fever, muscle aches, and knee pain. His temperature was 38.5°C. On physical examination, diffuse papulonodular and pustular lesions with areas of overlying crusting were noted across the forehead, nose, cheeks, and chin. There were similar lesions on the neck, shoulders, chest, back, and thighs. Laboratory studies were notable for neutrophilic leukocytosis and an elevated erythrocyte sedimentation rate and C-reactive protein level. A culture of a skin swab grew only Cutibacterium acnes. Histopathological examination of a skin-biopsy specimen taken from behind the left ear showed suppurative folliculitis with adjacent dermal edema. What is the most likely diagnosis?
\vspace{12pt}

\textbf{Options:}
\begin{enumerate}
\item[A.] Acne fulminans
\item[B.] Acute febrile neutrophilic dermatosis
\item[C.] Hidradenitis suppurativa
\item[D.] Pustular psoriasis
\item[E.] Rosacea fulminans
\end{enumerate}

\textbf{Image:}
\begin{center}
\includegraphics[width=0.95\textwidth,height=0.50\textheight,width=0.90\textwidth,keepaspectratio]{images/nejm_20241010.jpg}
\end{center}
\vspace{12pt}
\newpage

\section*{Question 1010 (ID: 20241017)}
\textbf{Date: }October 17,2024
\vspace{6pt}

A 2-year-old boy from a rural village was brought to the pediatric clinic with a 6-month history of diarrhea and poor weight gain. His body weight was 12.1 kg (below the 25th percentile for his age) and height 90 cm (1 SD below the median for his age). Physical examination revealed dry mucous membranes and decreased skin turgor. Laboratory tests showed iron-deficiency anemia, eosinophilia, and occult blood in the stool.  Stool samples examined by direct microscopy for ova and parasites were negative. A colonoscopy showed numerous mobile, white worms adherent to the colon wall. What is the most likely culprit organism?
\vspace{12pt}

\textbf{Options:}
\begin{enumerate}
\item[A.] Ascaris lumbricoides
\item[B.] Enterobius vermicularis (pinworm)
\item[C.] Hymenolepis nana (dwarf tapeworm)
\item[D.] Strongyloides stercoralis
\item[E.] Trichuris trichiura (whipworm)
\end{enumerate}

\textbf{Image:}
\begin{center}
\includegraphics[width=0.95\textwidth,height=0.50\textheight,width=0.90\textwidth,keepaspectratio]{images/nejm_20241017.jpg}
\end{center}
\vspace{12pt}
\newpage

\section*{Question 1011 (ID: 20241024)}
\textbf{Date: }October 24,2024
\vspace{6pt}

A 44-year-old man with Crohn’s disease that was being treated with infliximab presented with a 2-day history of a facial rash. In the previous week, his daughter had had a sore throat, and his mother had developed a similar rash on her face. His heart rate was 96 beats per minute, and his temperature was 36.6°C (97.9°F). Physical examination was notable for well-demarcated, warm, erythematous, confluent plaques on the cheeks, nose, and glabella. The pharynx was normal, and no cervical lymphadenopathy was observed. Which of the following is the most likely culprit organism for the underlying diagnosis?
\vspace{12pt}

\textbf{Options:}
\begin{enumerate}
\item[A.] Escherichia coli
\item[B.] Pseudomonas aeruginosa
\item[C.] Staphylococcus aureus
\item[D.] Staphylococcus epidermidis
\item[E.] Streptococcus pyogenes
\end{enumerate}

\textbf{Image:}
\begin{center}
\includegraphics[width=0.55\textwidth,height=0.50\textheight,width=0.90\textwidth,keepaspectratio]{images/nejm_20241024.jpg}
\end{center}
\vspace{12pt}
\newpage

\section*{Question 1012 (ID: 20241031)}
\textbf{Date: }October 31,2024
\vspace{6pt}

A 3-year-old boy was brought to the endocrinology department with an 18-month history of bowing of the left leg that had resulted in regression of his ability to walk. On physical examination, there was lateral bowing of the left femur and anterior bowing of the left tibia, as well as testicular enlargement. Café au lait spots were also noted on the lower back, cheek, and neck. Radiographs of the wrist, femur, and tibia on the left side showed fibrous dysplasia (arrows) and rickets. What is the most likely diagnosis in this case?
\vspace{12pt}

\textbf{Options:}
\begin{enumerate}
\item[A.] Carney complex
\item[B.] Fanconi anemia
\item[C.] McCune-Albright syndrome
\item[D.] Neurofibromatosis type 1 (NF1)
\item[E.] Osteofibrous dysplasia
\end{enumerate}

\textbf{Image:}
\begin{center}
\includegraphics[width=0.95\textwidth,height=0.50\textheight,width=0.90\textwidth,keepaspectratio]{images/nejm_20241031.jpg}
\end{center}
\vspace{12pt}
\newpage

\section*{Question 1013 (ID: 20241107)}
\textbf{Date: }November 07,2024
\vspace{6pt}

A 30-year-old man presented with a 6-month history of depressed skin lesions on his back. On physical examination, numerous well-defined, erythematous, atrophic plaques were noted across the back. A fluorescence staining preparation of skin scrapings showed thick fungal hyphae and yeast cells with a “spaghetti and meatballs” pattern. What is the most likely causative organism for this skin infection?
\vspace{12pt}

\textbf{Options:}
\begin{enumerate}
\item[A.] Candida species
\item[B.] Epidermophyton species
\item[C.] Malassezia species
\item[D.] Microsporum species
\item[E.] Trichophyton species
\end{enumerate}

\textbf{Image:}
\begin{center}
\includegraphics[width=0.95\textwidth,height=0.50\textheight,width=0.90\textwidth,keepaspectratio]{images/nejm_20241107.jpg}
\end{center}
\vspace{12pt}
\newpage

\section*{Question 1014 (ID: 20241114)}
\textbf{Date: }November 14,2024
\vspace{6pt}

A previously healthy 18-month-old girl was brought to the emergency department with sudden-onset abdominal distention that had been preceded by 3 days of diarrhea and 1 day of vomiting. On physical examination, the patient appeared lethargic and dehydrated. The abdomen was markedly distended with decreased bowel sounds, but there was no tenderness or guarding. An abdominal radiograph, obtained with the patient in the supine position, showed three circular radiopaque objects in the intestines, along with dilated loops of bowel. What is the most appropriate next step in this case?
\vspace{12pt}

\textbf{Options:}
\begin{enumerate}
\item[A.] Colonoscopy retrieval
\item[B.] Emergency exploratory laparotomy
\item[C.] Laxatives
\item[D.] Observation and monitoring
\item[E.] Serial imaging
\end{enumerate}

\textbf{Image:}
\begin{center}
\includegraphics[width=0.87\textwidth,height=0.50\textheight,width=0.90\textwidth,keepaspectratio]{images/nejm_20241114.jpg}
\end{center}
\vspace{12pt}
\newpage

\section*{Question 1015 (ID: 20241121)}
\textbf{Date: }November 21,2024
\vspace{6pt}

A 40-year-old man presented with a 2-day history of a burning rash on both hands. Physical examination was notable for a confluent region of erythema extending from the dorsal aspect of the thumbs to the medial aspect of the second finger. Scattered patches of erythema were observed on the knuckles and other fingers, and a small blister was noted on the base of the left thumb. What is the most likely diagnosis?
\vspace{12pt}

\textbf{Options:}
\begin{enumerate}
\item[A.] Atopic dermatitis
\item[B.] Hand, foot, and mouth disease
\item[C.] Irritant contact dermatitis
\item[D.] Phytophotodermatitis
\item[E.] Porphyria cutanea tarda
\end{enumerate}

\textbf{Image:}
\begin{center}
\includegraphics[width=0.95\textwidth,height=0.50\textheight,width=0.90\textwidth,keepaspectratio]{images/nejm_20241121.jpg}
\end{center}
\vspace{12pt}
\newpage

\section*{Question 1016 (ID: 20241128)}
\textbf{Date: }November 28,2024
\vspace{6pt}

A previously healthy 32-year-old woman presented with a 2-hour history of severe chest pain at rest. An electrocardiogram showed ST-segment elevations in leads II, III, and aVF with reciprocal ST-segment depressions in leads I and aVL. An initial troponin I level was normal. Owing to concern for acute coronary syndrome, emergency coronary angiography was performed, which showed an abrupt caliber change extending from the proximal to the distal right coronary artery. What is the most likely etiology of arterial injury?
\vspace{12pt}

\textbf{Options:}
\begin{enumerate}
\item[A.] Atherosclerotic plaque rupture and distal occlusion
\item[B.] Atherosclerotic stenosis
\item[C.] Coronary arteritis
\item[D.] Dissection
\item[E.] Vasospasm
\end{enumerate}

\textbf{Image:}
\begin{center}
\includegraphics[width=0.95\textwidth,height=0.50\textheight,width=0.90\textwidth,keepaspectratio]{images/nejm_20241128.jpg}
\end{center}
\vspace{12pt}
\newpage

\section*{Question 1017 (ID: 20241205)}
\textbf{Date: }December 05,2024
\vspace{6pt}

A previously healthy 10-year-old boy was brought to the pediatric emergency department with a 4-day history of progressively worsening bruising around the eyelids. Four weeks prior, he developed a dry cough, which had intensified over the previous week. The patient was up to date on routine vaccinations and had no history of trauma or bleeding disorders. On physical examination, ecchymosis was noted on the eyelids, along with subconjunctival hemorrhages in both eyes. A complete ophthalmologic examination was otherwise normal. Which of the following is the most likely etiology of the patient’s symptoms and findings?
\vspace{12pt}

\textbf{Options:}
\begin{enumerate}
\item[A.] Acute lymphoblastic leukemia
\item[B.] Influenza
\item[C.] Inhaled foreign body
\item[D.] Leptospirosis
\item[E.] Pertussis
\end{enumerate}

\textbf{Image:}
\begin{center}
\includegraphics[width=0.95\textwidth,height=0.50\textheight,width=0.90\textwidth,keepaspectratio]{images/nejm_20241205.jpg}
\end{center}
\vspace{12pt}
\newpage

\section*{Question 1018 (ID: 20241212)}
\textbf{Date: }December 12,2024
\vspace{6pt}

A 63-year-old man with a history of coronary-artery bypass grafting presented with a 2-month history of angina that occurred during exertion of the left arm. Five years before presentation, he had undergone grafting of the left internal thoracic artery (LITA) to the left anterior descending (LAD) coronary artery. The blood pressure was 128/84 mm Hg in the right arm and unmeasurable in the left arm. On physical examination, the brachial and radial pulses on the left side were feeble. A computed tomographic angiogram of the aorta and neck vessels showed total occlusion of the subclavian artery on the left side, proximal to the origin of the LITA.
Which of the following is the most likely diagnosis?
\vspace{12pt}

\textbf{Options:}
\begin{enumerate}
\item[A.] Aortic Arch Syndrome
\item[B.] Congenital Atretic Subclavian Artery
\item[C.] Coronary-Artery Bypass Graft Failure
\item[D.] Coronary-Subclavian Steal Syndrome
\item[E.] Thoracic Outlet Syndrome
\end{enumerate}

\textbf{Image:}
\begin{center}
\includegraphics[width=0.49\textwidth,height=0.50\textheight,width=0.90\textwidth,keepaspectratio]{images/nejm_20241212.jpg}
\end{center}
\vspace{12pt}
\newpage

\section*{Question 1019 (ID: 20241219)}
\textbf{Date: }December 19,2024
\vspace{6pt}

A newborn boy was admitted to the neonatal intensive care unit for management of a congenital abnormality, which had first been identified on an antenatal ultrasound image at 20 weeks’ gestation. A physical examination was notable for the presence of a red, fluid-filled sac, measuring 7.7 cm by 7.1 cm by 5.3 cm, that protruded through a lumbosacral defect. Subsequent magnetic resonance imaging of the spine confirmed the finding. What is the most likely diagnosis?
\vspace{12pt}

\textbf{Options:}
\begin{enumerate}
\item[A.] Dandy-Walker syndrome
\item[B.] Hemangioma
\item[C.] Meningocele
\item[D.] Presacral neurenteric cyst
\item[E.] Spina bifida occulta
\end{enumerate}

\textbf{Image:}
\begin{center}
\includegraphics[width=0.95\textwidth,height=0.50\textheight,width=0.90\textwidth,keepaspectratio]{images/nejm_20241219.jpg}
\end{center}
\vspace{12pt}
\newpage

\section*{Question 1020 (ID: 20241226)}
\textbf{Date: }December 26,2024
\vspace{6pt}

A 90-year-old man with atrial fibrillation and dementia presented to the hospital with sudden onset of dyspnea that had begun 1 hour earlier. Before admission, he had not been taking anticoagulation therapy on the basis of discussions of his preferences with his primary care physician. Physical examination was notable for tachypnea and tachycardia. On chest radiography, the pulmonary vasculature was not visible in the right lung fields. In addition, the right descending pulmonary artery was enlarged. What diagnosis is suggested by these chest radiograph appearances?
\vspace{12pt}

\textbf{Options:}
\begin{enumerate}
\item[A.] Ascending Aortic Aneurysm
\item[B.] Atrial septal defect
\item[C.] Bullous emphysema
\item[D.] Pulmonary embolism
\item[E.] Pulmonary arterial hypertension
\end{enumerate}

\textbf{Image:}
\begin{center}
\includegraphics[width=0.89\textwidth,height=0.50\textheight,width=0.90\textwidth,keepaspectratio]{images/nejm_20241226.jpg}
\end{center}
\vspace{12pt}
\newpage

\section*{Question 1021 (ID: 20250102)}
\textbf{Date: }January 02,2025
\vspace{6pt}

An 84-year-old man with benign prostatic hyperplasia who had been admitted to the hospital with acute kidney injury due to obstructive uropathy was noted to have gray skin. The skin changes had been present for 5 years. On physical examination, diffuse slate-gray pigmentation of the skin, particularly on the face, hands and nails, and sclera, was seen. A skin biopsy revealed small, dark granules in the basement membrane of sweat glands and in pilosebaceous units, blood vessels, and elastic fibers in the dermis. What is the diagnosis?
\vspace{12pt}

\textbf{Options:}
\begin{enumerate}
\item[A.] Amiodarone-induced hyperpigmentation
\item[B.] Argyria
\item[C.] Hemochromatosis
\item[D.] Minocycline-induced hyperpigmentation
\item[E.] Wilson’s disease
\end{enumerate}

\textbf{Image:}
\begin{center}
\includegraphics[width=0.95\textwidth,height=0.50\textheight,width=0.90\textwidth,keepaspectratio]{images/nejm_20250102.jpg}
\end{center}
\vspace{12pt}
\newpage

\section*{Question 1022 (ID: 20250109)}
\textbf{Date: }January 09,2025
\vspace{6pt}

A 66-year-old woman with a history of metastatic breast cancer presented with a 1-month history of an asymptomatic, lacy rash on her left thigh. The rash did not vary with changes in environmental temperature. Her breast cancer treatment had been interrupted 3 months before presentation by the coronavirus disease 2019 pandemic. Physical examination was notable for an irregular, broken, netlike pattern of mottling on the left thigh that extended to the knee and lower leg. Testing for autoimmune conditions, thrombophilia, and coagulation disorders was negative. A skin-biopsy sample obtained from the left thigh showed atypical large cells within the vasculature of the reticular dermis that were positive for cytokeratin 7 and GATA-binding protein 3 on immunohistochemical staining. What is the diagnosis?
\vspace{12pt}

\textbf{Options:}
\begin{enumerate}
\item[A.] Cutis marmorata
\item[B.] Erythema ab igne
\item[C.] Intravascular lymphoma
\item[D.] Livedo racemosa
\item[E.] Livedo reticularis
\end{enumerate}

\textbf{Image:}
\begin{center}
\includegraphics[width=0.95\textwidth,height=0.50\textheight,width=0.90\textwidth,keepaspectratio]{images/nejm_20250109.jpg}
\end{center}
\vspace{12pt}
\newpage

\section*{Question 1023 (ID: 20250116)}
\textbf{Date: }January 16,2025
\vspace{6pt}

A 34-year-old woman with human immunodeficiency virus (HIV) infection presented with a 3-day history of a blind spot in her right eye. One week before the current presentation, the HIV viral load was 57 copies per milliliter (reference value, <20) and the CD4 cell count was 81 per cubic millimeter (reference range, 500 to 1500). On eye examination, visual acuity was 20/50 in the right eye and 20/20 in the left eye. Fundoscopy of the right eye revealed fulminant retinitis with dense areas of retinal necrosis and hemorrhage. The left eye, which had been asymptomatic, had similar changes. What is the diagnosis?
\vspace{12pt}

\textbf{Options:}
\begin{enumerate}
\item[A.] Cytomegalovirus retinitis
\item[B.] Herpes simplex retinitis
\item[C.] HIV retinopathy
\item[D.] Retinal vasculitis from ocular syphilis
\item[E.] Toxoplasma retinitis
\end{enumerate}

\textbf{Image:}
\begin{center}
\includegraphics[width=0.95\textwidth,height=0.50\textheight,width=0.90\textwidth,keepaspectratio]{images/nejm_20250116.jpg}
\end{center}
\vspace{12pt}
\newpage

\section*{Question 1024 (ID: 20250123)}
\textbf{Date: }January 23,2025
\vspace{6pt}

A previously healthy 10-month-old girl was referred to the pediatric dermatology clinic with a 6-month history of an intermittently itchy and swollen skin lesion on her neck that had gradually increased in size. On physical examination, a thin, well-demarcated, tan plaque was seen on the neck. The lesion was rubbed firmly with a tongue depressor for 5 seconds, and within a few minutes it became edematous with a flare of surrounding erythema - a finding known as Darier’s sign. What is the most likely diagnosis?
\vspace{12pt}

\textbf{Options:}
\begin{enumerate}
\item[A.] Dermatofibroma
\item[B.] Juvenile Xanthogranuloma (JXG)
\item[C.] Langerhans Cell Histiocytosis (LCH)
\item[D.] Solitary Mastocytoma
\item[E.] Spitz Nevus
\end{enumerate}

\textbf{Image:}
\begin{center}
\includegraphics[width=0.95\textwidth,height=0.50\textheight,width=0.90\textwidth,keepaspectratio]{images/nejm_20250123.jpg}
\end{center}
\vspace{12pt}
\newpage

\section*{Question 1025 (ID: 20250130)}
\textbf{Date: }January 30,2025
\vspace{6pt}

A 29-year-old man with well-controlled human immunodeficiency virus infection presented with a 1-week history of a painful, whitish discoloration on his tongue. He also reported a 4-week history of painful ulcers on the scrotum. Physical examination showed smooth, pink, guttate macules on the tongue that were within a white coating that did not wipe off. No rash, ulcer, or lymphadenopathy was seen on genital examination; the reported lesions were presumed to have already resolved. Which of the following is the most likely cause of this patient’s tongue changes?
\vspace{12pt}

\textbf{Options:}
\begin{enumerate}
\item[A.] Geographic tongue
\item[B.] Lichen Planus
\item[C.] Oral leukoplakia
\item[D.] Secondary syphilis
\item[E.] Squamous Cell Carcinoma
\end{enumerate}

\textbf{Image:}
\begin{center}
\includegraphics[width=0.94\textwidth,height=0.50\textheight,width=0.90\textwidth,keepaspectratio]{images/nejm_20250130.jpg}
\end{center}
\vspace{12pt}
\newpage

\section*{Question 1026 (ID: 20250206)}
\textbf{Date: }February 06,2025
\vspace{6pt}

An 84-year-old man with a history of melanoma of the right forehead complicated by in-transit metastases to the scalp presented to the dermatology clinic. Seven months prior, treatment with two immunotherapy agents had been started, which had resulted in shrinking of the metastatic scalp nodules. On physical examination, there were multiple blue-gray macules on the scalp, as well as an unhealed ulcer from trauma. Dermoscopy-guided biopsy of four macules revealed scattered melanin-laden macrophages within the superficial and middle dermis without any evidence of residual melanoma. Immunohistochemical staining of a biopsy sample for markers of melanoma was also negative. What is the most likely diagnosis?
\vspace{12pt}

\textbf{Options:}
\begin{enumerate}
\item[A.] Lentigo maligna
\item[B.] Metastatic melanoma
\item[C.] Post inflammatory hyperpigmentation
\item[D.] Solar lentigo
\item[E.] Tumoral melanosis
\end{enumerate}

\textbf{Image:}
\begin{center}
\includegraphics[width=0.95\textwidth,height=0.50\textheight,width=0.90\textwidth,keepaspectratio]{images/nejm_20250206.jpg}
\end{center}
\vspace{12pt}
\newpage

\section*{Question 1027 (ID: 20250213)}
\textbf{Date: }February 13,2025
\vspace{6pt}

A 19-year-old man presented with a 3-week history of an asymptomatic rash on his neck. He worked as a beach lifeguard in southern California and reported no recent international travel. On physical examination, two erythematous, raised, serpiginous eruptions were seen on the neck - one on the posterior surface and another on the right lateral surface. A punch biopsy showed nonspecific inflammation. A potassium hydroxide scraping was negative for fungal elements. What is the diagnosis?
\vspace{12pt}

\textbf{Options:}
\begin{enumerate}
\item[A.] Cutaneous larva migrans
\item[B.] Larva currens
\item[C.] Loiasis
\item[D.] Tinea corporis
\item[E.] Scabies
\end{enumerate}

\textbf{Image:}
\begin{center}
\includegraphics[width=0.95\textwidth,height=0.50\textheight,width=0.90\textwidth,keepaspectratio]{images/nejm_20250213.jpg}
\end{center}
\vspace{12pt}
\newpage

\section*{Question 1028 (ID: 20250220)}
\textbf{Date: }February 20,2025
\vspace{6pt}

A previously healthy 44-year-old woman presented to the oral medicine clinic with an 11-month history of mouth ulcers and blistered lips. The lesions had caused pain while eating, with resultant unintentional weight loss. She also reported a 6-month history of skin blisters on her arms and legs. On physical examination, the lips were swollen with bleeding erosions and yellow pustules. Erosions were seen on the hard palate, as well as on the floor of the mouth, tongue, and buccal mucosa. Gingivitis and halitosis were also present. Flaccid bullae and blisters were seen on the skin of the arms and legs. Biopsy of the gingiva revealed intraepithelial separation and retention of basal cells along the basement membrane. Direct immunofluorescence of the biopsy specimen showed IgG deposition along epithelial-cell membranes, resulting in a “chicken wire” pattern. What is the most likely diagnosis?
\vspace{12pt}

\textbf{Options:}
\begin{enumerate}
\item[A.] Behcet’s disease
\item[B.] Bullous pemphigoid
\item[C.] Linear IgA bullous dermatosis
\item[D.] Pemphigus foliaceus
\item[E.] Pemphigus vulgaris
\end{enumerate}

\textbf{Image:}
\begin{center}
\includegraphics[width=0.95\textwidth,height=0.50\textheight,width=0.90\textwidth,keepaspectratio]{images/nejm_20250220.jpg}
\end{center}
\vspace{12pt}
\newpage

\section*{Question 1029 (ID: 20250227)}
\textbf{Date: }February 27,2025
\vspace{6pt}

A 62-year-old man presented with a 5-day history of low back pain and a 1-day history of lethargy. Physical examination was notable for tenderness over the lumbar spine and a normal neurologic examination. Radiographs of the lumbar spine showed gas within L4. Computed tomography of the lumbar spine confirmed the presence of intramedullary gas within L4, and also showed that the gas extended into the surrounding soft tissue, epidural space, and psoas muscles. What is the most likely etiology of this skeletal finding?
\vspace{12pt}

\textbf{Options:}
\begin{enumerate}
\item[A.] Air embolism
\item[B.] Calcium oxalate crystal deposition
\item[C.] Emphysematous osteomyelitis
\item[D.] Iatrogenic
\item[E.] Osteonecrosis
\end{enumerate}

\textbf{Image:}
\begin{center}
\includegraphics[width=0.95\textwidth,height=0.50\textheight,width=0.90\textwidth,keepaspectratio]{images/nejm_20250227.jpg}
\end{center}
\vspace{12pt}
\newpage

\section*{Question 1030 (ID: 20250306)}
\textbf{Date: }March 06,2025
\vspace{6pt}

A previously healthy 2-year-old boy was brought to clinic with a 1-week history of itchy, red spots. Three days before the onset of the rash, the child had had an upper respiratory infection for which he had been given ibuprofen. On physical examination, tense vesicles and edematous pink plaques with central erosions and crust were seen on the patient’s legs, arms, and back. Dense clustering of skin lesions was observed in the axillae and inguinal folds. No mucosal involvement was noted. Histopathological analysis of a skin-biopsy sample obtained from the right lower back showed a subepidermal blister with robust neutrophilic infiltration. Direct immunofluorescence revealed a linear band of IgA along the dermoepidermal junction. What is the most likely diagnosis?
\vspace{12pt}

\textbf{Options:}
\begin{enumerate}
\item[A.] Acute febrile neutrophilic dermatosis
\item[B.] Dermatitis herpetiformis
\item[C.] Erythema multiforme
\item[D.] Henoch-Schonlein purpura
\item[E.] Linear IgA bullous dermatosis of childhood
\end{enumerate}

\textbf{Image:}
\begin{center}
\includegraphics[width=0.95\textwidth,height=0.50\textheight,width=0.90\textwidth,keepaspectratio]{images/nejm_20250306.jpg}
\end{center}
\vspace{12pt}
\newpage

\section*{Question 1031 (ID: 20250313)}
\textbf{Date: }March 13,2025
\vspace{6pt}

A 40-year-old man in the intensive care unit (ICU) was noted to have red urine. Eight hours earlier, he had been found unconscious in a fume-filled car. Cardiopulmonary resuscitation had been initiated for asystole, which resulted in the return of spontaneous circulation. Endotracheal intubation was performed and he received treatment for inhalation of car fumes. Which of the following is the most likely cause of this patient’s red urine?
\vspace{12pt}

\textbf{Options:}
\begin{enumerate}
\item[A.] Hematuria
\item[B.] Hydroxocobalamin-induced discoloration
\item[C.] Myoglobinuria
\item[D.] Porphyria
\item[E.] Rifampin-induced discoloration
\end{enumerate}

\textbf{Image:}
\begin{center}
\includegraphics[width=0.73\textwidth,height=0.50\textheight,width=0.90\textwidth,keepaspectratio]{images/nejm_20250313.jpg}
\end{center}
\vspace{12pt}
\newpage

\section*{Question 1032 (ID: 20250320)}
\textbf{Date: }March 20,2025
\vspace{6pt}

A full-term baby boy was transferred to the neonatal intensive care unit (NICU) at 30 hours of age owing to abdominal distention and failure to pass meconium. Before the transfer, the baby had fed poorly and regurgitated colostrum. A nasogastric tube had drained yellow fluid. An abdominal radiograph had shown dilated loops of small intestine. At the NICU, a bedside abdominal ultrasonogram showed possible reversal of the superior mesenteric vessels, and an urgent laparotomy was performed. Considerable distention of the small bowel by thick meconium was identified. Which of the following is the most likely underlying cause of this patient’s symptoms?
\vspace{12pt}

\textbf{Options:}
\begin{enumerate}
\item[A.] Congenital chloride diarrhea
\item[B.] Cystic fibrosis
\item[C.] Hirschsprung’s disease
\item[D.] Hypothyroidism
\item[E.] Intestinal atresia
\end{enumerate}

\textbf{Image:}
\begin{center}
\includegraphics[width=0.95\textwidth,height=0.50\textheight,width=0.90\textwidth,keepaspectratio]{images/nejm_20250320.jpg}
\end{center}
\vspace{12pt}
\newpage

\section*{Question 1033 (ID: 20250327)}
\textbf{Date: }March 27,2025
\vspace{6pt}

A previously healthy, full-term, 2-month-old boy was brought to the dermatology clinic with a 2-week history of annular plaques with edematous borders and central crusting on the face, scalp, and trunk. Results of a complete blood count and comprehensive metabolic panel were normal. Examination of a skin-biopsy sample obtained from the baby’s right forehead showed vacuolar interface dermatitis and perivascular and periadnexal lymphocytic infiltrates. Which of the following is the most likely underlying cause of this baby’s findings?
\vspace{12pt}

\textbf{Options:}
\begin{enumerate}
\item[A.] Cutaneous Langerhans cell histiocytosis
\item[B.] Impetigo
\item[D.] Neonatal lupus erythematosus
\item[E.] Tinea corporis
\end{enumerate}

\textbf{Image:}
\begin{center}
\includegraphics[width=0.95\textwidth,height=0.50\textheight,width=0.90\textwidth,keepaspectratio]{images/nejm_20250327.jpg}
\end{center}
\vspace{12pt}
\newpage

\section*{Question 1034 (ID: 20250403)}
\textbf{Date: }April 03,2025
\vspace{6pt}

A 22-year-old man presented to the dermatology clinic with a 1-year history of a red, scaly rash on his face and body. On physical examination, widespread erythroderma with overlying erosions, scaling, and crusting was observed. The oral mucosa was spared. A skin biopsy from the back showed superficial acantholysis (the loss of adhesion between keratinocytes). Direct immunofluorescence of the biopsy specimen showed intercellular IgG antibodies against desmoglein-1 in the superficial layers of the epidermis. What is the most likely diagnosis?
\vspace{12pt}

\textbf{Options:}
\begin{enumerate}
\item[A.] Bullous pemphigoid
\item[B.] Dermatitis herpetiformis
\item[C.] IgA pemphigus
\item[D.] Pemphigus foliaceus
\item[E.] Pemphigus vulgaris
\end{enumerate}

\textbf{Image:}
\begin{center}
\includegraphics[width=0.95\textwidth,height=0.50\textheight,width=0.90\textwidth,keepaspectratio]{images/nejm_20250403.jpg}
\end{center}
\vspace{12pt}
\newpage

\section*{Question 1035 (ID: 20250410)}
\textbf{Date: }April 10,2025
\vspace{6pt}

A previously healthy 4-year-old girl presented to the emergency department with a 3-day history of dry cough and fever. On physical examination, intercostal retractions and decreased breath sounds over the left lung were observed. A chest radiograph is shown. What is the most likely diagnosis?
\vspace{12pt}

\textbf{Options:}
\begin{enumerate}
\item[A.] Bronchogenic cyst
\item[B.] Pulmonary abscess
\item[C.] Pulmonary infarct
\item[D.] Pulmonary metastasis
\item[E.] Round Pneumonia
\end{enumerate}

\textbf{Image:}
\begin{center}
\includegraphics[width=0.91\textwidth,height=0.50\textheight,width=0.90\textwidth,keepaspectratio]{images/nejm_20250410.jpg}
\end{center}
\vspace{12pt}
\newpage

\section*{Question 1036 (ID: 20250417)}
\textbf{Date: }April 17,2025
\vspace{6pt}

A previously healthy 3-year-old boy was brought to the emergency department with a 6-week history of painless bumps and ulcers on his left leg. He initially had itchy insect bites on his foot and then bumps and ulcers had developed slowly on his leg. The boy had recently immigrated from Venezuela after a months-long journey by land with his family. On physical examination, ulcerated plaques with rolled borders and satellite papules were observed on the left lower leg, buttocks, and back. Three biopsy samples were obtained from one of the lesions. Histopathological analysis was notable for severe lymphohistiocytic infiltration with no organisms identified by routine microbiologic stains. What is the most likely diagnosis?
\vspace{12pt}

\textbf{Options:}
\begin{enumerate}
\item[A.] Brown recluse spider bites
\item[B.] Cutaneous Leishmaniasis
\item[C.] Leprosy
\item[D.] Phaeohyphomycosis
\item[E.] Pyoderma gangrenosum
\end{enumerate}

\textbf{Image:}
\begin{center}
\includegraphics[width=0.71\textwidth,height=0.50\textheight,width=0.90\textwidth,keepaspectratio]{images/nejm_20250417.jpg}
\end{center}
\vspace{12pt}
\newpage

\section*{Question 1037 (ID: 20250424)}
\textbf{Date: }April 24,2025
\vspace{6pt}

A 26-year-old man presented with sudden onset of severe pain in the legs and inability to move the left leg. On physical examination, he had complete loss of motor function in the left leg. Bedside ultrasonographic examination with color Doppler showed no blood flow in the distal aorta. Computed tomographic angiography of the abdomen revealed a saddle embolus at the aortoiliac junction (left). Emergency aortoiliac embolectomy was performed, and a gelatinous mass was removed. A subsequent transthoracic echocardiogram identified a heterogeneous mass in the left atrium (middle). On hospital day 2, cardiothoracic surgery was performed to remove the left atrial mass, and a villous, friable lesion was excised (right). Histopathology of the cardiac mass showed abundant mucopolysaccharide matrix with scattered nests of lepidic cells. What is the diagnosis?
\vspace{12pt}

\textbf{Options:}
\begin{enumerate}
\item[A.] Cardiac myxoma
\item[B.] Cardiac sarcoma
\item[C.] Intracardiac thrombus
\item[D.] Marantic endocarditis
\item[E.] Papillary fibroelastoma
\end{enumerate}

\textbf{Image:}
\begin{center}
\includegraphics[width=0.95\textwidth,height=0.50\textheight,width=0.90\textwidth,keepaspectratio]{images/nejm_20250424.jpg}
\end{center}
\vspace{12pt}
\newpage

\section*{Question 1038 (ID: 20250501)}
\textbf{Date: }May 01,2025
\vspace{6pt}

A 28-year-old woman presented with a painful rash on her right thigh. Five days before presentation, she had been walking on sea rocks on an island in the Cyclades region of Greece when she fell, landing on her buttocks. She immediately felt a severe pain in her right thigh, and an itchy, red lesion developed. On physical examination at the current presentation, innumerable fine, erythematous, linear lesions in a stellate distribution, with central clearing, on the posterior right thigh were seen. By what marine animal did the patient most likely get stung?
\vspace{12pt}

\textbf{Options:}
\begin{enumerate}
\item[B.] Jellyfish
\item[C.] Sea anemone
\item[D.] Sea urchin
\item[E.] Stingray
\end{enumerate}

\textbf{Image:}
\begin{center}
\includegraphics[width=0.76\textwidth,height=0.50\textheight,width=0.90\textwidth,keepaspectratio]{images/nejm_20250501.jpg}
\end{center}
\vspace{12pt}
\newpage

\section*{Question 1039 (ID: 20250508)}
\textbf{Date: }May 08,2025
\vspace{6pt}

A 6-month-old girl was brought to the emergency department with a 3-day history of redness and swelling of a toe. A physical examination is shown. Which of the following is the most likely underlying cause of the findings?
\vspace{12pt}

\textbf{Options:}
\begin{enumerate}
\item[A.] Allergic reaction
\item[B.] Cellulitis
\item[C.] Hair tourniquet
\item[D.] Non-accidental trauma
\item[E.] Insect bite
\end{enumerate}

\textbf{Image:}
\begin{center}
\includegraphics[width=0.84\textwidth,height=0.50\textheight,width=0.90\textwidth,keepaspectratio]{images/nejm_20250508.jpg}
\end{center}
\vspace{12pt}
\newpage

\section*{Question 1040 (ID: 20250515)}
\textbf{Date: }May 15,2025
\vspace{6pt}

A 68-year-old man with cirrhosis presented to the emergency department with a 2-day history of severe abdominal pain. On physical examination, hypoactive bowel sounds, pain with palpation, and rebound tenderness throughout the abdomen were noted. Computed tomography of the abdomen with administration of contrast material showed thrombosis of the superior mesenteric vein (left, arrows). What complication of superior mesenteric vein thrombosis was also seen on imaging?
\vspace{12pt}

\textbf{Options:}
\begin{enumerate}
\item[A.] Ascites
\item[B.] Ischemic Colitis
\item[C.] Portal hypertension
\item[D.] Small-Bowel Perforation
\item[E.] Small-Bowel Infarction
\end{enumerate}

\textbf{Image:}
\begin{center}
\includegraphics[width=0.95\textwidth,height=0.50\textheight,width=0.90\textwidth,keepaspectratio]{images/nejm_20250515.jpg}
\end{center}
\vspace{12pt}
\newpage

\section*{Question 1041 (ID: 20250522)}
\textbf{Date: }May 22,2025
\vspace{6pt}

A 47-year-old man with type 2 diabetes presented to the endocrinology clinic owing to several years of progressive growth of skin lesions on his lower abdominal wall where he had repeatedly injected insulin. He also reported unpredictable episodes of hypoglycemia. A physical examination was notable for two pendulous skin masses on the lower abdominal wall. The glycated hemoglobin level was 9.2\% (reference value, <7.1). What is the most likely diagnosis?
\vspace{12pt}

\textbf{Options:}
\begin{enumerate}
\item[A.] Inguinal hernias
\item[B.] Injection site granulomas
\item[C.] Insulin-derived amyloidosis
\item[D.] Lipoatrophy
\item[E.] Lipomas
\end{enumerate}

\textbf{Image:}
\begin{center}
\includegraphics[width=0.95\textwidth,height=0.50\textheight,width=0.90\textwidth,keepaspectratio]{images/nejm_20250522.jpg}
\end{center}
\vspace{12pt}
\newpage

\section*{Question 1042 (ID: 20250529)}
\textbf{Date: }May 29,2025
\vspace{6pt}

A 7-year-old girl was referred to the emergency department for evaluation of abnormal blood work. Her family had recently made an immigration journey to the United States. Physical examination was notable for dental caries, mild conjunctival pallor, and no signs of neurodevelopmental delay. Laboratory studies showed a hemoglobin level of 10.5 g per deciliter (reference range, 11.3 to 14.6), a mean corpuscular volume of 64.4 fl (reference range, 77.8 to 86.5), and a ferritin level of 8 ng per milliliter (reference range, 10 to 320). An abdominal radiograph showed intraluminal radiodensities throughout the colon. Which of the following diagnoses best explains the patient’s laboratory and radiographic abnormalities?
\vspace{12pt}

\textbf{Options:}
\begin{enumerate}
\item[A.] Bismuth poisoning
\item[B.] Copper deficiency
\item[C.] Lead poisoning
\item[D.] Inflammatory bowel disease
\item[E.] Thalassemia
\end{enumerate}

\textbf{Image:}
\begin{center}
\includegraphics[width=0.71\textwidth,height=0.50\textheight,width=0.90\textwidth,keepaspectratio]{images/nejm_20250529.jpg}
\end{center}
\vspace{12pt}
\newpage

\section*{Question 1043 (ID: 20250605)}
\textbf{Date: }June 05,2025
\vspace{6pt}

A previously healthy 48-year-old woman presented to the dermatology clinic with a 1-day history of a rash on her right ear that was preceded by 2 days of ear pain and an inability to move the right side of her face. Physical examination was notable for facial-nerve palsy on the right side and erythema with a few vesicles and some crusting on the pinna of the right ear. An oral examination revealed vesicles on the right side of the hard palate, patchy erythema of the anterior tongue, and an ulceration on the right lower lip. Her hearing was intact. What is the most likely diagnosis?
\vspace{12pt}

\textbf{Options:}
\begin{enumerate}
\item[A.] Aural herpes simplex
\item[B.] Bell’s palsy
\item[C.] Herpes zoster oticus
\item[D.] Malignant otitis externa
\item[E.] Sarcoidosis (Heerfordt’s syndrome)
\end{enumerate}

\textbf{Image:}
\begin{center}
\includegraphics[width=0.95\textwidth,height=0.50\textheight,width=0.90\textwidth,keepaspectratio]{images/nejm_20250605.jpg}
\end{center}
\vspace{12pt}
\newpage

\section*{Question 1044 (ID: 20250612)}
\textbf{Date: }June 12,2025
\vspace{6pt}

A 54-year-old woman with invasive ductal carcinoma of the breast and well-controlled chronic plaque psoriasis presented to the emergency department with a 2-day history of painful rash and fever. Five days before the onset of the rash, the patient had completed a 2-week course of systemic glucocorticoids to treat side effects of chemotherapy. The body temperature was 38.2°C, and heart rate 137 beats per minute. On physical examination, widespread erythematous patches with overlying, coalescing pustules were seen on the torso, arms, legs, face, and scalp, with sparing of mucosal surfaces. Laboratory testing was notable for neutrophilic leukocytosis and an elevated C-reactive protein level. Biopsy of the rash in the periumbilical region revealed neutrophil-rich pustules in the epidermis on histopathological testing. What is the most likely diagnosis?
\vspace{12pt}

\textbf{Options:}
\begin{enumerate}
\item[A.] Acute generalized exanthematous pustulosis
\item[B.] Generalized pustular psoriasis
\item[C.] IgA Pemphigus
\item[D.] Staphylococcal scalded skin syndrome
\item[E.] Subcorneal pustular dermatosis
\end{enumerate}

\textbf{Image:}
\begin{center}
\includegraphics[width=0.95\textwidth,height=0.50\textheight,width=0.90\textwidth,keepaspectratio]{images/nejm_20250612.jpg}
\end{center}
\vspace{12pt}
\newpage

\section*{Question 1045 (ID: 20250619)}
\textbf{Date: }June 19,2025
\vspace{6pt}

A previously healthy 59-year-old man presented to the emergency department with a 5-hour history of severe, pleuritic chest pain. Half an hour before the onset of symptoms, he vomited a large amount of gastric contents after eating street food. On physical examination, his breathing was found to be rapid and shallow. Palpable crepitus was noted in the neck, and breath sounds at the base of the right lung were diminished. Computed tomography of the chest is shown here. What is the most likely diagnosis?
\vspace{12pt}

\textbf{Options:}
\begin{enumerate}
\item[A.] Aortoesophageal fistula
\item[B.] Aspiration pneumonitis
\item[C.] Esophageal rupture
\item[D.] Perforated Peptic Ulcer
\item[E.] Tension Pneumothorax with subcutaneous emphysema
\end{enumerate}

\textbf{Image:}
\begin{center}
\includegraphics[width=0.95\textwidth,height=0.50\textheight,width=0.90\textwidth,keepaspectratio]{images/nejm_20250619.jpg}
\end{center}
\vspace{12pt}
\newpage

\section*{Question 1046 (ID: 20250626)}
\textbf{Date: }June 26,2025
\vspace{6pt}

A 23-year-old man presented to an outpatient health center with a 2-day history of painful swallowing and hoarseness. The symptoms developed after he had inhaled nitrous oxide through his mouth from a hand-held canister for recreation. Physical examination was notable for erythema, swelling, and sloughing of the mucosa on the soft palate, uvula, and posterior oropharynx. There was no increased work of breathing or stridor. Flexible nasolaryngoscopy revealed a small area of ulceration and swelling on the right vocal fold. What is the most likely diagnosis?
\vspace{12pt}

\textbf{Options:}
\begin{enumerate}
\item[A.] Chemical mucositis
\item[B.] Frostbite injury
\item[C.] Hypersensitivity reaction
\item[D.] Oropharyngeal candidiasis
\item[E.] Thermal injury
\end{enumerate}

\textbf{Image:}
\begin{center}
\includegraphics[width=0.95\textwidth,height=0.50\textheight,width=0.90\textwidth,keepaspectratio]{images/nejm_20250626.jpg}
\end{center}
\vspace{12pt}
\newpage

\section*{Question 1047 (ID: 20250703)}
\textbf{Date: }July 03,2025
\vspace{6pt}

A 77-year-old woman presented to the emergency department with a 2-day history of abdominal pain, nausea, and vomiting. On physical examination, there was abdominal distension with tenderness in the left lower quadrant. Computed tomography of the abdomen showed herniation of a loop of ileum on the left side inferior to the inguinal ligament, lateral to the pubic tubercle, and medial to the common femoral vessels with compression the femoral vein. Small bowel loops were dilated proximal to the herniation and collapsed distal to it. What is the most appropriate management for this patient?
\vspace{12pt}

\textbf{Options:}
\begin{enumerate}
\item[A.] Bowel regimen
\item[B.] Elective surgical repair
\item[C.] Emergent surgical repair
\item[D.] Manual reduction
\item[E.] Nasogastric tube insertion and nil per os
\end{enumerate}

\textbf{Image:}
\begin{center}
\includegraphics[width=0.95\textwidth,height=0.50\textheight,width=0.90\textwidth,keepaspectratio]{images/nejm_20250703.jpg}
\end{center}
\vspace{12pt}
\newpage

\section*{Question 1048 (ID: 20250710)}
\textbf{Date: }July 10,2025
\vspace{6pt}

A 55-year-old man with metastatic squamous-cell lung cancer presented to the hospital with a 6-week history of pain and swelling of the right great toe and tip of the right middle finger. He reported no history of fever. On physical examination, swelling and erythema of the distal phalanx of the right third finger and right great toe. There was also an area of ulceration adjacent to the great toenail. The lesions were firm and tender to palpation. Radiographs of the hands and feet were taken. Which of the following is the most likely underlying cause of the findings?
\vspace{12pt}

\textbf{Options:}
\begin{enumerate}
\item[A.] Acrometastases
\item[C.] Osteomyelitis
\item[D.] Psoriatic arthritis
\item[E.] Trauma-induced osteonecrosis
\end{enumerate}

\textbf{Image:}
\begin{center}
\includegraphics[width=0.76\textwidth,height=0.50\textheight,width=0.90\textwidth,keepaspectratio]{images/nejm_20250710.jpg}
\end{center}
\vspace{12pt}
\newpage

\section*{Question 1049 (ID: 20250717)}
\textbf{Date: }July 17,2025
\vspace{6pt}

A 26-year-old man who had emigrated from Nepal 7 years earlier was referred to the pulmonology clinic with several months of ankle swelling, a 2-month history of cough and unintentional weight loss, and a 3-day history of leg rash. A chest radiograph showed hilar lymphadenopathy (left). Physical examination was notable for erythematous, tender nodules and plaques on the anterior shins (middle). There was also swelling and tenderness of the ankles. Computed tomography of the chest showed enlarged mediastinal and hilar lymph nodes (right) and normal lung parenchyma. Transbronchial biopsy of the right paratracheal and hilar lymph nodes was performed. Histopathological analysis showed noncaseating granulomas. Microbiological studies for fungal and mycobacterial infections were negative, including a nucleic acid amplification test for Mycobacterium tuberculosis. What is the most likely diagnosis?
\vspace{12pt}

\textbf{Options:}
\begin{enumerate}
\item[A.] Extra-intestinal Crohn’s disease
\item[B.] Löfgren’s syndrome
\item[C.] Non-Hodgkin lymphoma
\item[D.] Non-tuberculous mycobacterial infection
\item[E.] Reactive arthritis
\end{enumerate}

\textbf{Image:}
\begin{center}
\includegraphics[width=0.95\textwidth,height=0.50\textheight,width=0.90\textwidth,keepaspectratio]{images/nejm_20250717.jpg}
\end{center}
\vspace{12pt}
\end{document}